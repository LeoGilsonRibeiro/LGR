% Options for packages loaded elsewhere
\PassOptionsToPackage{unicode}{hyperref}
\PassOptionsToPackage{hyphens}{url}
%
\documentclass[
  a4paper,
  oneside]{scrbook}

\usepackage{amsmath,amssymb}
\usepackage{iftex}
\ifPDFTeX
  \usepackage[T1]{fontenc}
  \usepackage[utf8]{inputenc}
  \usepackage{textcomp} % provide euro and other symbols
\else % if luatex or xetex
  \usepackage{unicode-math}
  \defaultfontfeatures{Scale=MatchLowercase}
  \defaultfontfeatures[\rmfamily]{Ligatures=TeX,Scale=1}
\fi
\usepackage{lmodern}
\ifPDFTeX\else  
    % xetex/luatex font selection
  \setmainfont[]{QT Garomand}
  \setsansfont[]{Alegreya Sans}
\fi
% Use upquote if available, for straight quotes in verbatim environments
\IfFileExists{upquote.sty}{\usepackage{upquote}}{}
\IfFileExists{microtype.sty}{% use microtype if available
  \usepackage[]{microtype}
  \UseMicrotypeSet[protrusion]{basicmath} % disable protrusion for tt fonts
}{}
\makeatletter
\@ifundefined{KOMAClassName}{% if non-KOMA class
  \IfFileExists{parskip.sty}{%
    \usepackage{parskip}
  }{% else
    \setlength{\parindent}{0pt}
    \setlength{\parskip}{6pt plus 2pt minus 1pt}}
}{% if KOMA class
  \KOMAoptions{parskip=half}}
\makeatother
\usepackage{xcolor}
\setlength{\emergencystretch}{3em} % prevent overfull lines
\setcounter{secnumdepth}{2}
% Make \paragraph and \subparagraph free-standing
\ifx\paragraph\undefined\else
  \let\oldparagraph\paragraph
  \renewcommand{\paragraph}[1]{\oldparagraph{#1}\mbox{}}
\fi
\ifx\subparagraph\undefined\else
  \let\oldsubparagraph\subparagraph
  \renewcommand{\subparagraph}[1]{\oldsubparagraph{#1}\mbox{}}
\fi


\providecommand{\tightlist}{%
  \setlength{\itemsep}{0pt}\setlength{\parskip}{0pt}}\usepackage{longtable,booktabs,array}
\usepackage{calc} % for calculating minipage widths
% Correct order of tables after \paragraph or \subparagraph
\usepackage{etoolbox}
\makeatletter
\patchcmd\longtable{\par}{\if@noskipsec\mbox{}\fi\par}{}{}
\makeatother
% Allow footnotes in longtable head/foot
\IfFileExists{footnotehyper.sty}{\usepackage{footnotehyper}}{\usepackage{footnote}}
\makesavenoteenv{longtable}
\usepackage{graphicx}
\makeatletter
\def\maxwidth{\ifdim\Gin@nat@width>\linewidth\linewidth\else\Gin@nat@width\fi}
\def\maxheight{\ifdim\Gin@nat@height>\textheight\textheight\else\Gin@nat@height\fi}
\makeatother
% Scale images if necessary, so that they will not overflow the page
% margins by default, and it is still possible to overwrite the defaults
% using explicit options in \includegraphics[width, height, ...]{}
\setkeys{Gin}{width=\maxwidth,height=\maxheight,keepaspectratio}
% Set default figure placement to htbp
\makeatletter
\def\fps@figure{htbp}
\makeatother

\usepackage{scalerel}
\usepackage{ccicons}
\usepackage{xltxtra}
\usepackage{tikz}
\usepackage{fontawesome}
\makeatletter
\@ifpackageloaded{caption}{}{\usepackage{caption}}
\AtBeginDocument{%
\ifdefined\contentsname
  \renewcommand*\contentsname{Índice}
\else
  \newcommand\contentsname{Índice}
\fi
\ifdefined\listfigurename
  \renewcommand*\listfigurename{Lista de Figuras}
\else
  \newcommand\listfigurename{Lista de Figuras}
\fi
\ifdefined\listtablename
  \renewcommand*\listtablename{Lista de Tabelas}
\else
  \newcommand\listtablename{Lista de Tabelas}
\fi
\ifdefined\figurename
  \renewcommand*\figurename{Figura}
\else
  \newcommand\figurename{Figura}
\fi
\ifdefined\tablename
  \renewcommand*\tablename{Tabela}
\else
  \newcommand\tablename{Tabela}
\fi
}
\@ifpackageloaded{float}{}{\usepackage{float}}
\floatstyle{ruled}
\@ifundefined{c@chapter}{\newfloat{codelisting}{h}{lop}}{\newfloat{codelisting}{h}{lop}[chapter]}
\floatname{codelisting}{Listagem}
\newcommand*\listoflistings{\listof{codelisting}{Lista de Listagens}}
\makeatother
\makeatletter
\makeatother
\makeatletter
\@ifpackageloaded{caption}{}{\usepackage{caption}}
\@ifpackageloaded{subcaption}{}{\usepackage{subcaption}}
\makeatother
\ifLuaTeX
\usepackage[bidi=basic]{babel}
\else
\usepackage[bidi=default]{babel}
\fi
\babelprovide[main,import]{brazilian}
\ifPDFTeX
\else
\babelfont{rm}[]{QT Garomand}
\fi
% get rid of language-specific shorthands (see #6817):
\let\LanguageShortHands\languageshorthands
\def\languageshorthands#1{}
\ifLuaTeX
  \usepackage{selnolig}  % disable illegal ligatures
\fi
\usepackage{bookmark}

\IfFileExists{xurl.sty}{\usepackage{xurl}}{} % add URL line breaks if available
\urlstyle{same} % disable monospaced font for URLs
\hypersetup{
  pdftitle={Apresentação aos textos reunidos de Leo Gilson Ribeiro},
  pdfauthor={Fernando Rey Puente},
  pdflang={pt-BR},
  hidelinks,
  pdfcreator={LaTeX via pandoc}}

\title{Apresentação aos textos reunidos de Leo Gilson Ribeiro}
\author{Fernando Rey Puente}
\date{}

\begin{document}
\frontmatter
\begin{titlepage}
\begin{tikzpicture}[remember picture,overlay]

\node [align=center] at (current page.north) % upper
    {\begin{tikzpicture}[remember picture, overlay]
  \fill[fill=] (-10,0) rectangle (10,-0.5);
  \node[anchor=north] at (0,-0.5) {\LARGE\textit{
        Fernando Rey Puente (org.)
    }};
    \end{tikzpicture}};

\node [yshift=+6cm] at (current page.center) % title
    {\begin{tikzpicture}[remember picture, overlay]
    \fill[fill=] (-10,-5) rectangle (10 ,1);
    \node[align=center] at (0,-2)
        {\resizebox{1.3\linewidth}{!}{\Huge\textbf{\textsc{
        
        }}}};
    %          \node[anchor=east,align=right] at (10,-5.5) {\Huge\color{black}\textit{}};
    \end{tikzpicture}};

    \node[align=center] at (8,-16){\resizebox{0.65\linewidth}{!}{\Huge\textbf{\textsc{
        Leo Gilson Ribeiro
    }}}};

\node [shift={(-5cm,-5cm)}] at (current page.north east) % Revision 2
    {\begin{tikzpicture}[remember picture, overlay]
    \draw[fill=black] (0,5) -- (1.5,5) -- (5,1.5) -- (5,0) -- cycle ;
    \node[inner sep=0pt,rotate=-45] (rev) at (2.9,2.9) {\huge\color{white}\textbf{
            Volume 
        }};
    \end{tikzpicture}};

\node [align=center] at (current page.south) % bottom
    {\begin{tikzpicture}[remember picture, overlay]
    \node[anchor=south west, align=left] at (-10,0.5)
        {\resizebox{.3\linewidth}{!}
          {\Huge\color{black}
            \textbf{\color{}\ccbysa \color{black}}}};
%            \node[anchor=south east, align=right] at (10,0.5) {\Huge\color{black}\textbf{
%
%                }};
    \end{tikzpicture}};
\end{tikzpicture}
\end{titlepage}

\newpage

\thispagestyle{empty}
\bigskip

{\large{\textbf{Volume  dos Textos Reunidos de Leo Gilson Ribeiro}}}

\bigskip

{\large{Textos Reunidos de Leo Gilson Ribeiro}}

\vfill

\textbf{Transcrito e organizado por}\smallskip

Fernando Rey Puente

\smallskip

\href{mailto:ferey99@yahoo.com.br}{ferey99@yahoo.com.br}

\smallskip

\scalerel*{\includegraphics{./icons/orcid_icon.pdf}}{O} \href{https://orcid.org/0000-0001-8862-4077}{0000-0001-8862-4077}

\vfill

\textbf{Editado por}

\smallskip

Bernardo C. D. A. Vasconcelos

\smallskip

\href{mailto:bernardovasconcelos@gmail.com}{bernardovasconcelos@gmail.com}

\smallskip

\scalerel*{\includegraphics{./icons/orcid_icon.pdf}}{O} \href{https://orcid.org/0000-0002-3357-1710}{0000-0002-3357-1710}

\vfill

\textbf{Digital Object Identifier (DOI)}

\smallskip

\scalerel*{\includegraphics{./icons/doi_icon.pdf}} {O} \href{https://doi.org/10.5281/zenodo.8368806}{10.5281/zenodo.8368806}
\scalerel*{\includegraphics{./icons/open_access_icon.pdf}}{O}

\vfill

\textbf{Versão}

\smallskip



\vfill

\textbf{Licença}

\smallskip

\ccbysa

\smallskip

\href{https://creativecommons.org/licenses/by-sa/4.0/}{Creative Commons Attribution-ShareAlike 4.0 International}

\cleardoublepage

\hypertarget{apresentauxe7uxe3o-aos-textos-reunidos-de-leo-gilson-ribeiro}{%
\section*{Apresentação aos textos reunidos de Leo Gilson
Ribeiro}\label{apresentauxe7uxe3o-aos-textos-reunidos-de-leo-gilson-ribeiro}}
\addcontentsline{toc}{section}{Apresentação aos textos reunidos de Leo
Gilson Ribeiro}

Há muitos anos eu estava guardando um material que eu obtive de Leo
Gilson Ribeiro. Tratava-se de inúmeros recortes de jornal e de revistas,
bem como um bom número de textos datilografados que ele preservava, mas
de modo bastante desordenado, amontoados em prateleiras em um pequeno
quarto nos fundos de sua casa. Nestes últimos anos de colapso cultural
que estamos vivendo com recorrentes ataques do governo às universidades
e aos centros de pesquisa, nada mais importante para um professor
universitário do que procurar resgatar parte de nosso passado cultural
do esquecimento e torná-lo público.

Foi por ocasião da pandemia e em função da decorrente paralização das
universidades que eu tive finalmente algum tempo livre para ordenar esse
vasto, rico, mas caótico material. Um dia espalhei-o no chão de minha
biblioteca e fui separando dia após dia pilhas e pilhas de recortes
amarelados de jornais e de revistas procurando organizar tematicamente
esse riquíssimo acervo de quase cinco décadas de produção cultural que,
infelizmente, como sói acontecer no Brasil, é tão frequentemente
perdido.

Devido ao fato de que, ao longo de diversos anos de amizade, Leo e eu
conversávamos sempre sobre a publicação em forma de livros de seus
inúmeros textos dispersos em jornais e revistas, acreditei que poderia
levar adiante esse projeto com um espírito próximo ao dele e, em alguns
casos, até mesmo seguindo algumas indicações que ele próprio havia feito
oralmente em nossas inúmeras tertúlias ou deixado em anotações em papeis
avulsos ou nos próprios recortes de jornal. Foram somente dois livros
que Leo Gilson Ribeiro publicou em vida -- \emph{Os Cronistas do
Absurdo} (José Álvaro editor, Rio de Janeiro, 1964) e \emph{O Continente
Submerso} (Editora Nova Cultural, São Paulo, 1988) - pois se recusou a
publicar outros livros durante a ditadura militar. Pude verificar com o
apoio dos textos que tinha em mãos, que ambos esses livros foram
constituídos precisamente com os artigos que ele havia redigido, com as
entrevistas que ele havia feito e, por fim, com os depoimentos que havia
colhido junto a escritoras e escritores para os diversos veículos de
imprensa nos quais trabalhava. Isso me animou a prosseguir com esse
projeto, pois vi que minha interferência nesse imenso acervo literário
seria mínima e, mais importante, que esses textos não estariam
simplesmente fadados ao esquecimento, o que estava acontecendo desde a
morte de Leo Gilson Ribeiro em 2007.

Todavia, um grande obstáculo com que me deparei então era o fato de que
muitos desses textos de jornal estavam recortados sem a anotação exata
da data em que foram publicados. Tentei recorrer aos arquivos digitais,
mas infelizmente o arquivo do \emph{Jornal da Tarde}, um dos veículos
para o qual Leo Gilson Ribeiro mais escreveu, não está digitalizado,
razão pela qual alguns dos textos extraídos desse jornal e aproveitados
nos livros aqui reunidos não possuem datas precisas ou em casos mais
raros não possuem datas. O mesmo ocorre com muitos artigos extraídos de
diversas revistas que não pude datar corretamente ou aos quais pura e
simplesmente não pude ter acesso. A ausência de uma digitalização da
revista \emph{Caros Amigos} constitui igualmente um caso parecido.
Consegui adquirir diversos exemplares dessa revista em sebos, e tive o
apoio do escritor Guilherme Scalzilli que me enviou fotografias de
vários números da revista \emph{Caros Amigos} onde foi publicada a seção
``Janelas Abertas'' de autoria de nosso crítico, mas continuei sem
acesso a alguns números da \emph{Caros Amigos}.

Para levar a cabo esse projeto eu infelizmente não contei com o apoio de
mais ninguém, de modo que eu mesmo comecei a transcrever esse vasto
material e consegui produzir até agora seis livros que organizei com uma
parte desse acervo. Os artigos e ensaios remanescentes já foram por mim
ordenados em distintas pastas temáticas, mas isso significa dizer também
que ainda resta um imenso trabalho de transcrição pela frente (não foi
possível fazer uma transcrição direta para o Word a partir de uma
digitalização prévia de artigos amarelados de jornais). O retorno às
atividades acadêmicas, primeiro em modo remoto e depois em modo
presencial, que consome a maior parte do meu tempo dedicado à pesquisa
em minha própria área de trabalho que não é a literatura, mas sim a
filosofia, dificulta e atrasa ainda mais esse empreendimento, mas ele
segue em curso.

Tendo organizado seis livros pensei então que finalmente poderia me
dirigir às editoras com esse representativo material e que conseguiria
certamente despertar o interesse de alguma editora disposta a publicar
esses livros. Qual não foi minha surpresa ao constatar que das inúmeras
editoras para as quais eu escrevi pouquíssimas foram aquelas que tiveram
ao menos a delicadeza de me responderem dizendo não estarem interessadas
na publicação dos livros. Todavia, com a ajuda de um jovem amigo,
recém-doutor em filosofia e com um ótimo conhecimento em informática,
Bernardo Vasconcelos, consegui realizar meu desejo de manter viva a
palavra aliciante de meu pranteado amigo Leo Gilson Ribeiro, que fez da
literatura a sua vida. Uma palavra que será capaz, creio eu, de fecundar
por esse meio digital aberto e democrático novos corações e mentes
desejosos de se enveredarem nessa arte tão fascinante que é a arte da
escrita e sobre a qual Leo Gilson Ribeiro refletiu e produziu durante
toda a sua vida procurando sempre colmar o hiato entre essas obras, às
vezes difíceis e complexas, e o público leigo, porém, interessado em
adentrar no universo dessas escritoras e desses escritores do Brasil e
do mundo.

Talvez seja útil dizer ainda, nessa breve introdução ao projeto que aqui
se apresenta materializado virtualmente, qual a razão de eu ter
organizado esses seis livros para iniciar o processo de resgate desses
inúmeros textos de Leo Gilson Ribeiro.

Em alguns casos, deveu-se a uma surpresa que eu mesmo tive com a grande
quantidade de textos sobre um determinado assunto cuja atualidade é
crescente. Isso ocorreu, por exemplo, com o primeiro livro,
\emph{Racismo e a Literatura Negra}. Sabia do interesse de nosso crítico
pelo assunto, pois eu mesmo o havia escutado em conferências tratando
desse tema nos anos oitenta em São Paulo, mas ignorava a imensa
quantidade de textos que ele já havia escrito sobre o tema desde os anos
sessenta. Isso somado ao esquecimento que o nome de nosso crítico padece
hodiernamente tanto nas editoras quanto nos grupos de pesquisa que
publicam sobre e pesquisam esse tema me fizeram perceber a urgência de
publicizar esse material tão variegado e abundante e que aborda com
antecipação de décadas um assunto tão importante e atual para todos nós
brasileiros e brasileiras.

No caso do segundo volume, \emph{Os Escritores Aquém e Além da
Literatura}, a sua organização foi devida à somatória do meu interesse
pessoal (afinal acabei falando com Leo Gilson Ribeiro, pois nos anos
oitenta quis encontrar a escritora Hilda Hilst que, obviamente, conheci
por uma bela resenha de nosso crítico sobre a autora então quase
desconhecida e hoje justamente tornada célebre), da importância que ele
mesmo conferia a esses três autores com os quais conviveu (Guimarães
Rosa, Clarice Lispector e Hilda Hilst) e da constatação de outra
injustiça feita em relação ao nosso crítico, a saber: nas raras e
ocasionais referências a ele, o mesmo era quase sempre visto como sendo
apenas aquele crítico que desde o início da carreira de Hilda Hilst
chamou a atenção para a sua obra. Diante do grande volume de textos que
estavam sob meus olhos senti igualmente a urgência de mostrar ao público
que a obra dele não se resumia de modo algum apenas a isso, mas que ele
havia escrito por décadas, e com muita competência e discernimento,
sobre inúmeros outros autores e temas. Além disso, uma entrevista
inédita com Guimarães Rosa, um depoimento que ele fez sobre Clarice
Lispector em uma carta a e as várias entrevistas e depoimentos com Hilda
Hilst que ele realizou não me parecia que merecessem continuar ignorados
ou de difícil acesso.

O terceiro livro que organizei foi uma total surpresa para mim mesmo,
pois descobri em meio ao volumoso material que guardava uma pasta com
indicações sobre um curso, ``Testemunhos Literários do século XX'', que
nosso crítico ofertou nos anos sessenta no Rio de Janeiro e decidi então
reconstruir esse material com o acréscimo de outros textos sobre os
autores por ele ali estudados. Pareceu-me uma bela introdução à
literatura contemporânea que valeria à pena apresentar às jovens e aos
jovens leitores de nossos dias.

Os artigos de Leo Gilson Ribeiro sobre a poesia brasileira chamaram a
minha atenção pela sua clareza, abrangência e profundidade e me
pareceram compor um painel bastante rico e interessante sobre diversos
poetas brasileiros, alguns já consagrados e outros menos conhecidos na
época, e ainda hoje, e resolvi assim compor com esse material o quarto
volume deste projeto.

Tendo ouvido Leo Gilson Ribeiro falar durante o ano de 1992 com
entusiasmo do curso que estava ministrando em algumas unidades do SESC
no Estado de São Paulo sobre a Semana de Arte de 1922, foi com alegria
que encontrei entre seus papeis ao menos as anotações da parte de seu
curso relativa a Mário de Andrade. Descobrindo igualmente entre seus
papéis entrevistas com artistas envolvidos na Semana de Arte de 1922 e
alguns artigos prévios de nosso crítico para a grande imprensa sobre
esse evento - divisor de águas em nossa cultura - achei que dada a
coincidência do centenário de comemoração desse evento seria importante
tornar esse material público ainda neste ano.

O sexto e último livro que eu escolhi organizar foi dedicado à relação
literária entre Portugal e o Brasil, um assunto pelo qual Leo Gilson
Ribeiro sempre se interessou e sobre o qual escreveu muitos textos e
realizou diversas entrevistas importantes. Nunca é demais chamar a
atenção dos brasileiros para Portugal, não o país que agora parece ser o
destino preferencial das viagens da classe média abastada brasileira,
mas sim o imorredouro Portugal da tradição literária plurissecular,
particularmente poética, que fundou nosso idioma e a cuja riquíssima
tradição nós temos acesso direto sem ter de passar pela mediação tantas
vezes deveras problemática das traduções.

Desejo então que as leitoras e os leitores desses livros virtuais por
mim organizados e aqui reunidos digitalmente possam usufruir da escrita
aliciante e envolvente de Leo Gilson Ribeiro que, espero eu, possa
conduzir a todas e todos pelo universo labiríntico, mágico e encantado
que nos é desvelado pelas literaturas de vários países, e em especial do
Brasil, em suas inúmeras formas e manifestações ao longo do tempo.

Boa leitura.

\cleardoublepage
%\thispagestyle{empty}
%\mbox{}
%\clearpage
\renewcommand*\contentsname{Índice}
{
\setcounter{tocdepth}{0}
\tableofcontents
}
Há muitos anos eu estava guardando um material que eu obtive de Leo
Gilson Ribeiro. Tratava-se de inúmeros recortes de jornal e de revistas,
bem como um bom número de textos datilografados que ele preservava, mas
de modo bastante desordenado, amontoados em prateleiras em um pequeno
quarto nos fundos de sua casa. Nestes últimos anos de colapso cultural
que estamos vivendo com recorrentes ataques do governo às universidades
e aos centros de pesquisa, nada mais importante para um professor
universitário do que procurar resgatar parte de nosso passado cultural
do esquecimento e torná-lo público.

Foi por ocasião da pandemia e em função da decorrente paralização das
universidades que eu tive finalmente algum tempo livre para ordenar esse
vasto, rico, mas caótico material. Um dia espalhei-o no chão de minha
biblioteca e fui separando dia após dia pilhas e pilhas de recortes
amarelados de jornais e de revistas procurando organizar tematicamente
esse riquíssimo acervo de quase cinco décadas de produção cultural que,
infelizmente, como sói acontecer no Brasil, é tão frequentemente
perdido.

Devido ao fato de que, ao longo de diversos anos de amizade, Leo e eu
conversávamos sempre sobre a publicação em forma de livros de seus
inúmeros textos dispersos em jornais e revistas, acreditei que poderia
levar adiante esse projeto com um espírito próximo ao dele e, em alguns
casos, até mesmo seguindo algumas indicações que ele próprio havia feito
oralmente em nossas inúmeras tertúlias ou deixado em anotações em papeis
avulsos ou nos próprios recortes de jornal. Foram somente dois livros
que Leo Gilson Ribeiro publicou em vida -- \emph{Os Cronistas do
Absurdo} (José Álvaro editor, Rio de Janeiro, 1964) e \emph{O Continente
Submerso} (Editora Nova Cultural, São Paulo, 1988) - pois se recusou a
publicar outros livros durante a ditadura militar. Pude verificar com o
apoio dos textos que tinha em mãos, que ambos esses livros foram
constituídos precisamente com os artigos que ele havia redigido, com as
entrevistas que ele havia feito e, por fim, com os depoimentos que havia
colhido junto a escritoras e escritores para os diversos veículos de
imprensa nos quais trabalhava. Isso me animou a prosseguir com esse
projeto, pois vi que minha interferência nesse imenso acervo literário
seria mínima e, mais importante, que esses textos não estariam
simplesmente fadados ao esquecimento, o que estava acontecendo desde a
morte de Leo Gilson Ribeiro em 2007.

Todavia, um grande obstáculo com que me deparei então era o fato de que
muitos desses textos de jornal estavam recortados sem a anotação exata
da data em que foram publicados. Tentei recorrer aos arquivos digitais,
mas infelizmente o arquivo do \emph{Jornal da Tarde}, um dos veículos
para o qual Leo Gilson Ribeiro mais escreveu, não está digitalizado,
razão pela qual alguns dos textos extraídos desse jornal e aproveitados
nos livros aqui reunidos não possuem datas precisas ou em casos mais
raros não possuem datas. O mesmo ocorre com muitos artigos extraídos de
diversas revistas que não pude datar corretamente ou aos quais pura e
simplesmente não pude ter acesso. A ausência de uma digitalização da
revista \emph{Caros Amigos} constitui igualmente um caso parecido.
Consegui adquirir diversos exemplares dessa revista em sebos, e tive o
apoio do escritor Guilherme Scalzilli que me enviou fotografias de
vários números da revista \emph{Caros Amigos} onde foi publicada a seção
``Janelas Abertas'' de autoria de nosso crítico, mas continuei sem
acesso a alguns números da \emph{Caros Amigos}.

Para levar a cabo esse projeto eu infelizmente não contei com o apoio de
mais ninguém, de modo que eu mesmo comecei a transcrever esse vasto
material e consegui produzir até agora seis livros que organizei com uma
parte desse acervo. Os artigos e ensaios remanescentes já foram por mim
ordenados em distintas pastas temáticas, mas isso significa dizer também
que ainda resta um imenso trabalho de transcrição pela frente (não foi
possível fazer uma transcrição direta para o Word a partir de uma
digitalização prévia de artigos amarelados de jornais). O retorno às
atividades acadêmicas, primeiro em modo remoto e depois em modo
presencial, que consome a maior parte do meu tempo dedicado à pesquisa
em minha própria área de trabalho que não é a literatura, mas sim a
filosofia, dificulta e atrasa ainda mais esse empreendimento, mas ele
segue em curso.

Tendo organizado seis livros pensei então que finalmente poderia me
dirigir às editoras com esse representativo material e que conseguiria
certamente despertar o interesse de alguma editora disposta a publicar
esses livros. Qual não foi minha surpresa ao constatar que das inúmeras
editoras para as quais eu escrevi pouquíssimas foram aquelas que tiveram
ao menos a delicadeza de me responderem dizendo não estarem interessadas
na publicação dos livros. Todavia, com a ajuda de um jovem amigo,
recém-doutor em filosofia e com um ótimo conhecimento em informática,
Bernardo Vasconcelos, consegui realizar meu desejo de manter viva a
palavra aliciante de meu pranteado amigo Leo Gilson Ribeiro, que fez da
literatura a sua vida. Uma palavra que será capaz, creio eu, de fecundar
por esse meio digital aberto e democrático novos corações e mentes
desejosos de se enveredarem nessa arte tão fascinante que é a arte da
escrita e sobre a qual Leo Gilson Ribeiro refletiu e produziu durante
toda a sua vida procurando sempre colmar o hiato entre essas obras, às
vezes difíceis e complexas, e o público leigo, porém, interessado em
adentrar no universo dessas escritoras e desses escritores do Brasil e
do mundo.

Talvez seja útil dizer ainda, nessa breve introdução ao projeto que aqui
se apresenta materializado virtualmente, qual a razão de eu ter
organizado esses seis livros para iniciar o processo de resgate desses
inúmeros textos de Leo Gilson Ribeiro.

Em alguns casos, deveu-se a uma surpresa que eu mesmo tive com a grande
quantidade de textos sobre um determinado assunto cuja atualidade é
crescente. Isso ocorreu, por exemplo, com o primeiro livro,
\emph{Racismo e a Literatura Negra}. Sabia do interesse de nosso crítico
pelo assunto, pois eu mesmo o havia escutado em conferências tratando
desse tema nos anos oitenta em São Paulo, mas ignorava a imensa
quantidade de textos que ele já havia escrito sobre o tema desde os anos
sessenta. Isso somado ao esquecimento que o nome de nosso crítico padece
hodiernamente tanto nas editoras quanto nos grupos de pesquisa que
publicam sobre e pesquisam esse tema me fizeram perceber a urgência de
publicizar esse material tão variegado e abundante e que aborda com
antecipação de décadas um assunto tão importante e atual para todos nós
brasileiros e brasileiras.

No caso do segundo volume, \emph{Os Escritores Aquém e Além da
Literatura}, a sua organização foi devida à somatória do meu interesse
pessoal (afinal acabei falando com Leo Gilson Ribeiro, pois nos anos
oitenta quis encontrar a escritora Hilda Hilst que, obviamente, conheci
por uma bela resenha de nosso crítico sobre a autora então quase
desconhecida e hoje justamente tornada célebre), da importância que ele
mesmo conferia a esses três autores com os quais conviveu (Guimarães
Rosa, Clarice Lispector e Hilda Hilst) e da constatação de outra
injustiça feita em relação ao nosso crítico, a saber: nas raras e
ocasionais referências a ele, o mesmo era quase sempre visto como sendo
apenas aquele crítico que desde o início da carreira de Hilda Hilst
chamou a atenção para a sua obra. Diante do grande volume de textos que
estavam sob meus olhos senti igualmente a urgência de mostrar ao público
que a obra dele não se resumia de modo algum apenas a isso, mas que ele
havia escrito por décadas, e com muita competência e discernimento,
sobre inúmeros outros autores e temas. Além disso, uma entrevista
inédita com Guimarães Rosa, um depoimento que ele fez sobre Clarice
Lispector em uma carta a e as várias entrevistas e depoimentos com Hilda
Hilst que ele realizou não me parecia que merecessem continuar ignorados
ou de difícil acesso.

O terceiro livro que organizei foi uma total surpresa para mim mesmo,
pois descobri em meio ao volumoso material que guardava uma pasta com
indicações sobre um curso, ``Testemunhos Literários do século XX'', que
nosso crítico ofertou nos anos sessenta no Rio de Janeiro e decidi então
reconstruir esse material com o acréscimo de outros textos sobre os
autores por ele ali estudados. Pareceu-me uma bela introdução à
literatura contemporânea que valeria à pena apresentar às jovens e aos
jovens leitores de nossos dias.

Os artigos de Leo Gilson Ribeiro sobre a poesia brasileira chamaram a
minha atenção pela sua clareza, abrangência e profundidade e me
pareceram compor um painel bastante rico e interessante sobre diversos
poetas brasileiros, alguns já consagrados e outros menos conhecidos na
época, e ainda hoje, e resolvi assim compor com esse material o quarto
volume deste projeto.

Tendo ouvido Leo Gilson Ribeiro falar durante o ano de 1992 com
entusiasmo do curso que estava ministrando em algumas unidades do SESC
no Estado de São Paulo sobre a Semana de Arte de 1922, foi com alegria
que encontrei entre seus papeis ao menos as anotações da parte de seu
curso relativa a Mário de Andrade. Descobrindo igualmente entre seus
papéis entrevistas com artistas envolvidos na Semana de Arte de 1922 e
alguns artigos prévios de nosso crítico para a grande imprensa sobre
esse evento - divisor de águas em nossa cultura - achei que dada a
coincidência do centenário de comemoração desse evento seria importante
tornar esse material público ainda neste ano.

O sexto e último livro que eu escolhi organizar foi dedicado à relação
literária entre Portugal e o Brasil, um assunto pelo qual Leo Gilson
Ribeiro sempre se interessou e sobre o qual escreveu muitos textos e
realizou diversas entrevistas importantes. Nunca é demais chamar a
atenção dos brasileiros para Portugal, não o país que agora parece ser o
destino preferencial das viagens da classe média abastada brasileira,
mas sim o imorredouro Portugal da tradição literária plurissecular,
particularmente poética, que fundou nosso idioma e a cuja riquíssima
tradição nós temos acesso direto sem ter de passar pela mediação tantas
vezes deveras problemática das traduções.

Desejo então que as leitoras e os leitores desses livros virtuais por
mim organizados e aqui reunidos digitalmente possam usufruir da escrita
aliciante e envolvente de Leo Gilson Ribeiro que, espero eu, possa
conduzir a todas e todos pelo universo labiríntico, mágico e encantado
que nos é desvelado pelas literaturas de vários países, e em especial do
Brasil, em suas inúmeras formas e manifestações ao longo do tempo.

Boa leitura.



\end{document}
