\clearpage
\thispagestyle{empty}

Que é mesmo para valer, já o título o diz. Mas vale por grata madrugada este livro de ensaios, e consola bem à gente haver um Leo Gilson.

Ágil nos cimos raros ou nos obscuros fundos, ele me parece um gato – de lucidez e agudez. Espreita só o vivo, o que nos textos e autores verdadeiramente se mexe. Desdeixa de caroços e bagaços. Pega, capta, mede, pesa.

E nele sente-se um mestre do entendimento, já se libertando na Caverna – consoante o mito platônico. Toma os símbolos pelo que devem ser: sombras reveladoras, janelas de plena transparência. Não tem medo de ver, de pensar, de olhar para cima.

Rio adiante, assim é que – como numa das margens “de seu púlpito de nuvens”, Sören Kierkegaard, e, na outra, Novalis, o “idealista mágico” – ele nos descreve, entre mais três ou dois, menores, Kafka, tonto de crer e querer o inatingível e a milenaríssima sabedoria de Ionesco.

Sou-lhe reconhecido pelo que nos traz: uma informação sob nítida luz e uma interpretação sempre redimidora.

-- João Guimarães Rosa

\clearpage
