% Options for packages loaded elsewhere
\PassOptionsToPackage{unicode}{hyperref}
\PassOptionsToPackage{hyphens}{url}
\PassOptionsToPackage{dvipsnames,svgnames,x11names}{xcolor}
%
\documentclass[
  letterpaper,
  DIV=11,
  numbers=noendperiod]{scrartcl}

\usepackage{amsmath,amssymb}
\usepackage{iftex}
\ifPDFTeX
  \usepackage[T1]{fontenc}
  \usepackage[utf8]{inputenc}
  \usepackage{textcomp} % provide euro and other symbols
\else % if luatex or xetex
  \usepackage{unicode-math}
  \defaultfontfeatures{Scale=MatchLowercase}
  \defaultfontfeatures[\rmfamily]{Ligatures=TeX,Scale=1}
\fi
\usepackage{lmodern}
\ifPDFTeX\else  
    % xetex/luatex font selection
\fi
% Use upquote if available, for straight quotes in verbatim environments
\IfFileExists{upquote.sty}{\usepackage{upquote}}{}
\IfFileExists{microtype.sty}{% use microtype if available
  \usepackage[]{microtype}
  \UseMicrotypeSet[protrusion]{basicmath} % disable protrusion for tt fonts
}{}
\makeatletter
\@ifundefined{KOMAClassName}{% if non-KOMA class
  \IfFileExists{parskip.sty}{%
    \usepackage{parskip}
  }{% else
    \setlength{\parindent}{0pt}
    \setlength{\parskip}{6pt plus 2pt minus 1pt}}
}{% if KOMA class
  \KOMAoptions{parskip=half}}
\makeatother
\usepackage{xcolor}
\setlength{\emergencystretch}{3em} % prevent overfull lines
\setcounter{secnumdepth}{-\maxdimen} % remove section numbering
% Make \paragraph and \subparagraph free-standing
\ifx\paragraph\undefined\else
  \let\oldparagraph\paragraph
  \renewcommand{\paragraph}[1]{\oldparagraph{#1}\mbox{}}
\fi
\ifx\subparagraph\undefined\else
  \let\oldsubparagraph\subparagraph
  \renewcommand{\subparagraph}[1]{\oldsubparagraph{#1}\mbox{}}
\fi


\providecommand{\tightlist}{%
  \setlength{\itemsep}{0pt}\setlength{\parskip}{0pt}}\usepackage{longtable,booktabs,array}
\usepackage{calc} % for calculating minipage widths
% Correct order of tables after \paragraph or \subparagraph
\usepackage{etoolbox}
\makeatletter
\patchcmd\longtable{\par}{\if@noskipsec\mbox{}\fi\par}{}{}
\makeatother
% Allow footnotes in longtable head/foot
\IfFileExists{footnotehyper.sty}{\usepackage{footnotehyper}}{\usepackage{footnote}}
\makesavenoteenv{longtable}
\usepackage{graphicx}
\makeatletter
\def\maxwidth{\ifdim\Gin@nat@width>\linewidth\linewidth\else\Gin@nat@width\fi}
\def\maxheight{\ifdim\Gin@nat@height>\textheight\textheight\else\Gin@nat@height\fi}
\makeatother
% Scale images if necessary, so that they will not overflow the page
% margins by default, and it is still possible to overwrite the defaults
% using explicit options in \includegraphics[width, height, ...]{}
\setkeys{Gin}{width=\maxwidth,height=\maxheight,keepaspectratio}
% Set default figure placement to htbp
\makeatletter
\def\fps@figure{htbp}
\makeatother

\KOMAoption{captions}{tableheading}
\makeatletter
\@ifpackageloaded{caption}{}{\usepackage{caption}}
\AtBeginDocument{%
\ifdefined\contentsname
  \renewcommand*\contentsname{Table of contents}
\else
  \newcommand\contentsname{Table of contents}
\fi
\ifdefined\listfigurename
  \renewcommand*\listfigurename{List of Figures}
\else
  \newcommand\listfigurename{List of Figures}
\fi
\ifdefined\listtablename
  \renewcommand*\listtablename{List of Tables}
\else
  \newcommand\listtablename{List of Tables}
\fi
\ifdefined\figurename
  \renewcommand*\figurename{Figure}
\else
  \newcommand\figurename{Figure}
\fi
\ifdefined\tablename
  \renewcommand*\tablename{Table}
\else
  \newcommand\tablename{Table}
\fi
}
\@ifpackageloaded{float}{}{\usepackage{float}}
\floatstyle{ruled}
\@ifundefined{c@chapter}{\newfloat{codelisting}{h}{lop}}{\newfloat{codelisting}{h}{lop}[chapter]}
\floatname{codelisting}{Listing}
\newcommand*\listoflistings{\listof{codelisting}{List of Listings}}
\makeatother
\makeatletter
\makeatother
\makeatletter
\@ifpackageloaded{caption}{}{\usepackage{caption}}
\@ifpackageloaded{subcaption}{}{\usepackage{subcaption}}
\makeatother
\ifLuaTeX
  \usepackage{selnolig}  % disable illegal ligatures
\fi
\usepackage{bookmark}

\IfFileExists{xurl.sty}{\usepackage{xurl}}{} % add URL line breaks if available
\urlstyle{same} % disable monospaced font for URLs
\hypersetup{
  pdftitle={Uma história fantástica, contada por um mestre},
  pdfauthor={Leo Gilson Ribeiro},
  colorlinks=true,
  linkcolor={blue},
  filecolor={Maroon},
  citecolor={Blue},
  urlcolor={Blue},
  pdfcreator={LaTeX via pandoc}}

\title{Uma história fantástica, contada por um mestre}
\author{Leo Gilson Ribeiro}
\date{}

\begin{document}
\maketitle
\begin{abstract}
Jornal da Tarde, 1985-8-10. Aguardando revisão.
\end{abstract}

José J. Veiga escreve sempre fora do círculo de relações pessoais de
obrigatória admiração mútua longe dos ambientes em que os ditos
literatos e toda uma indústria de prêmios e louvações se auto ungem sem
cessar. Alheio aos favores, às concessões, aos conchavos em torno dos
quais pulula uma grande parte dos que julgam que escrevem e julgam criar
poesia entre nós, o seu recolhimento não é fruto de provincianismo.
Goiano, nunca precisou de se arvorar em grande intérprete da
criatividade goiana, ao contrário de tantos outros escritores
bairristas, amarrados à sua origem estadual como ao logotipo de uma
mediocridade intocável: nasceu em tal ou tal Estado! Culto, leitor de
Kafka, de Orwell, de Swift, de Karel Kapek, domina o inglês com rara
fluência, mas nunca se jactou de seus profundos conhecimentos da melhor
literatura estrangeira. Seus contos e romances já o tornaram,
involuntariamente, citado em excelentes resenhas do \emph{New York Times
Book Review}, do jornal alemão \emph{Frankfurt Allgemeine Zeitung} e
divulgado por editoras atentas e seletivas como a Bruguera, de
Barcelona, ou a Alfred Knopf, de Nova York. Em \emph{Os Cavalinhos de
Platiplanto}, \emph{A Hora dos Ruminantes}, \emph{A Estranha Máquina
Extraviada} sobretudo no admirável \emph{Sombras de Reis Barbudos}, José
J. Veiga já construíra todo um mundo de fortíssimas alegorias.

Difícil seria rotular o seu mundo ficcional. Aquela estranha máquina que
chega em três caminhões e é armada na praça principal da cidadezinha do
Interior e ninguém sabe de onde veio nem para que serve seria uma
metáfora do poder opressivo da técnica que esmaga os cidadãos ou os faz
curvarem-se, submissos, diante do poder tirânico? As cidadezinhas
modorrentas que são invadidas por ruminantes ou por burocratas que tudo
medem, tudo multam, tudo castram são uma forma oblíqua de representar a
opressão dos governos totalitários, a esmagar os fracos e a arrebentar
vontades que se anteponham ao seu desmando tão absurdo quanto imbatível?

José J. Veiga não fala de si mesmo. Guarda, como Dalton Trevisan, um
silêncio absoluto sobre o que pensa da literatura, do mundo, das
receitas mágicas para endireitar o viver. O que escreve fala por si. O
leitor não tem, no entanto, a impressão propriamente de uma incursão
pelo pesadelo, mas sim de um relato que descreva os acontecimentos sem
adjetivá-los: narra-se quase que impessoalmente. Naturalmente, vários
personagens interpretam o que acontece: alguns se conformam, temerosos,
outros ousam-se insurgir contra regimes autoritários e grotescos, os
padres frequentemente se refugiam num mutismo impenetrável. Estão presos
a um maniqueísmo que parece obsoleto, o de esperar que qualquer ato ou
fato seja previamente aprovado pelas autoridades eclesiásticas do
Vaticano ou sejam lançados no \emph{index} que já instituiu a Inquisição
queimou em fogueiras seres humanos vivos, aliou-se aos cambiantes
poderes vigentes ou encarnou, como a Igreja medieval, um poder temporal
de temíveis proporções.

\emph{Torvelinho Dia e Noite}, publicado pela editora Difel 206 páginas,
revela um José J. Veiga menos abstrato, menos impassível, mais
participante, se possível, diante de tudo aquilo que a parapsicologia
moderna pesquisa: fantasmas, contatos imediatos de terceiro grau com
extraterrestres, flores que falam, auras boas ou más de cidadãos que
tanto podem ser prosaicos, terra a terra, como suspeitos de pertencerem
a outras galáxias. Torvelinho, a cidadezinha interiorana, tem nas
crianças (como geralmente sucede nas narrativas do autor goiano) os seus
médiuns, os seus melhores sensores de mundos que fogem à lógica adulta e
estreita. Através da sensibilidade dos que estão plenamente dispostos a
aceitar outras realidades, todos os entes inclassificáveis entram mais
facilmente nas vidas ``quadradas'' que vêm mudar. Como epígrafe, José J.
Veiga usa de uma citação ardilosa em que se afirma que

``Tremer de medo de fantasma é um comportamento irracional. Eles não
fazem mal a ninguém e gostam de ser úteis. Se às vezes nos pregam peças,
isso só acontece quando são ofendidos em seus brios. A pior ofensa que
se pode fazer a um fantasma é tratá-lo de assombração. Outra ofensa
grave é usar o Credo como arma contra eles, arma aliás tão inócua como a
ameaça de prisão para banqueiro e ministro de Estado.''

Verídico ou inventado, este tratado sobre fantasmas colocado como
epígrafe do livro é uma típica ironia ferina e jocosa do autor contra os
Delfins Neto de gorda e indigesta memória ou dos afoitos políticos que,
ao mesmo tempo que deixam seus operários sem salários meses a fio, na
imprensa querem ``brilhar'' como democratas lídimos e levemente
``esclarecidos'', isto é, a favor da esquerda moscovita, à maneira do
ilustre senador Severo Gomes e sua indústria de cobertores do Vale do
Paraíba\ldots{}

José J. Veiga entra de chofre na história que vai relatar, pois uma de
suas melhores e mais inesperadas surpresas é a de fazer brotar do
cotidiano o incomum, o apavorante. Aqui o foco central é uma criança, o
menino Nilo, que na noite anterior literalmente viu um fantasma e hesita
em participar o fato fantástico aos pais: eles não compreenderiam. Quase
sempre as gerações não se entendem nos mundos de José J. Veiga: não só a
linguagem de gíria, de americanismos impostos pela máquina dos enlatados
culturais, distingue os mais novos dos mais velhos. Os mais moços são
muito mais pragmáticos também e aptos a mudar o ramerrão de tudo que foi
aceito sem questionamento e muito menos abertos aos preconceitos
imemoriais. Felizmente, o autor de \emph{Aquele Mundo de Vasabarros}
nunca fabrica estereótipos: há igualmente adultos inteligentes de
espírito aberto, como há, embora raramente, crianças obtusas. Há padres
temerosos de qualquer mudança assim como há padres que são honestos,
retos, de bom coração. Nada de esquematismos idiotas.

\emph{Torvelinho Dia e Noite} capta toda a oralidade da fala brasileira,
com ``Peraí'', ``Corta essa'' etc. Uma linguagem que mistura elementos
atuais com elementos de gíria, frases de um português correto e
escorreito com novidades científicas e tecnológicas como ``clone'',
``computador'', ``buracos negros''. A incerteza do que é a realidade
objetiva -- que o autor deixa transparecer ser uma coisa inexistente,
porque é captada e interpretada por cada pessoa individualmente -- dá
vazão a cenas de sonho ou de alucinação ou de uma tranquilidade que as
limitações humanas não permitem admitir.

``Quando passou para a parte sombreada, Nilo ouviu um zumbido, um
chiado, ou sopro e sentiu-se leve de corpo e supôs que estivesse
levitando. Parou para entender: olhou em volta e viu o largo cheio de
gente estranha, como ficava em dias de festa. Quem sabe era de noite, e
ele estava em casa dormindo e sonhando? Se era isso, como explicar a
tigela de doce na palma da mão? Nilo fechou os olhos, pensando: e quando
os abriu e olhou de novo, o largo já estava normal, os estranhos tinham
evaporado, só ficaram as pessoas que normalmente estariam ou passariam
lá àquela hora. Que coisa! Como explicar isso?''

Sub-repticiamente as farpas certeiras do romancista espetam aqui e ali:

``Durante o jantar dr. Gumercindo falou da viagem, do clima de Brasília,
tão seco que racha os lábios das pessoas. Falou das grandes distâncias,
da uniformidade monótona das quadras, da aberração das cidades-satélites
que não estavam previstas no plano urbanístico porque os planejadores se
esqueceram de incluir a pobreza.''

Em escala diferente mas com intenção semelhante, os escritor brasileiro
e a escritora inglesa Doris Lessing falam, ambos, de seres vindos de
outras constelações que disputam o planeta Terra numa luta cósmica entre
o Bem e o Mal, como a descrita em \emph{Shikasta}. Ou como \emph{Eu,
Robô} de Isaac Asimov, as pessoas são indiferenciáveis dos robôs
aperfeiçoadíssimos: não seria obviamente o caso de Abreuciano e
d.~Cyannara, o apicultor e a rendeira que vieram para salvar a cidade e
trazer um mundo novo para seus habitantes, sem a opressão de passaportes
internos como no regime da Rússia soviética nem ditaduras fascistas?

Naquela ``epidemia de fantasmas'', em que cada pessoa tem o seu, surgem
versões monstruosas também de espectros cruéis, os Tora-pés, que à noite
serram os pés de suas vítimas, como os que se dizia que tinham chegado
até Varginha, decepando pobres bípedes que dormiam até de botas para
afastar o perigo. O prefeito maligno muda de solertes intenções
políticas de mando e se torna um pacifista manso, infenso a qualquer
vertigem de poder ou ascensão no partido a que pertencera. A liberdade é
um dos temas eternos dos satiristas -- Kafka, Orwell, Swift -- e José J.
Veiga a coloca sempre como uma das preocupações fundamentais do ser em
si: nada justifica a escravidão de outrem nem a existência de tiranias,
antigas ou atuais. Essa noção de liberdade abrange a libertação das
ideias mortas, codificadas em leis caducas por interesses que
acorrentaram a mulher e os dissidentes a algemas milenares. E
estranhamente, talvez pela primeira vez em seu excelente percurso
literário, aqui se verifica uma mudança: José J. Veiga demonstra crer
num mundo além do que pode ser medido, pesado, tocado:

``O vento que agora sopra é diferente, é áspero e desconfortável. Pela
hora, o melhor abrigo contra ele são as cobertas da cama. Mas ninguém se
iluda. As cobertas só protegem do lado de cá. Do lado de lá ficamos
expostos aos ventos do desconhecido. Exatamente como do lado de cá''

Quem sabe, afinal, os fantasmas somos nós ou são todos os que creem numa
realidade fantástica e transformadora? A esperança, parece dizer José J.
Veiga, fugiu dos laboratórios políticos e assumiu todas as formas do
chamado irreal. Como nos filmes do magistral diretor russo Andrei
Tarkovsky (\emph{Solaris}, \emph{Stalker}) que negam todo e qualquer
realismo, socialista ou não, a esperança se adentra por uma pluralidade
de realidades que nada têm do mito, mas sim do concreto: a realidade da
consciência e da espiritualidade humanas.



\end{document}
