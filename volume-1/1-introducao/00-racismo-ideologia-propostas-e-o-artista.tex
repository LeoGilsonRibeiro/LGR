% Options for packages loaded elsewhere
\PassOptionsToPackage{unicode}{hyperref}
\PassOptionsToPackage{hyphens}{url}
\PassOptionsToPackage{dvipsnames,svgnames,x11names}{xcolor}
%
\documentclass[
  letterpaper,
  DIV=11,
  numbers=noendperiod]{scrreprt}

\usepackage{amsmath,amssymb}
\usepackage{iftex}
\ifPDFTeX
  \usepackage[T1]{fontenc}
  \usepackage[utf8]{inputenc}
  \usepackage{textcomp} % provide euro and other symbols
\else % if luatex or xetex
  \usepackage{unicode-math}
  \defaultfontfeatures{Scale=MatchLowercase}
  \defaultfontfeatures[\rmfamily]{Ligatures=TeX,Scale=1}
\fi
\usepackage{lmodern}
\ifPDFTeX\else  
    % xetex/luatex font selection
\fi
% Use upquote if available, for straight quotes in verbatim environments
\IfFileExists{upquote.sty}{\usepackage{upquote}}{}
\IfFileExists{microtype.sty}{% use microtype if available
  \usepackage[]{microtype}
  \UseMicrotypeSet[protrusion]{basicmath} % disable protrusion for tt fonts
}{}
\makeatletter
\@ifundefined{KOMAClassName}{% if non-KOMA class
  \IfFileExists{parskip.sty}{%
    \usepackage{parskip}
  }{% else
    \setlength{\parindent}{0pt}
    \setlength{\parskip}{6pt plus 2pt minus 1pt}}
}{% if KOMA class
  \KOMAoptions{parskip=half}}
\makeatother
\usepackage{xcolor}
\setlength{\emergencystretch}{3em} % prevent overfull lines
\setcounter{secnumdepth}{5}
% Make \paragraph and \subparagraph free-standing
\ifx\paragraph\undefined\else
  \let\oldparagraph\paragraph
  \renewcommand{\paragraph}[1]{\oldparagraph{#1}\mbox{}}
\fi
\ifx\subparagraph\undefined\else
  \let\oldsubparagraph\subparagraph
  \renewcommand{\subparagraph}[1]{\oldsubparagraph{#1}\mbox{}}
\fi


\providecommand{\tightlist}{%
  \setlength{\itemsep}{0pt}\setlength{\parskip}{0pt}}\usepackage{longtable,booktabs,array}
\usepackage{calc} % for calculating minipage widths
% Correct order of tables after \paragraph or \subparagraph
\usepackage{etoolbox}
\makeatletter
\patchcmd\longtable{\par}{\if@noskipsec\mbox{}\fi\par}{}{}
\makeatother
% Allow footnotes in longtable head/foot
\IfFileExists{footnotehyper.sty}{\usepackage{footnotehyper}}{\usepackage{footnote}}
\makesavenoteenv{longtable}
\usepackage{graphicx}
\makeatletter
\def\maxwidth{\ifdim\Gin@nat@width>\linewidth\linewidth\else\Gin@nat@width\fi}
\def\maxheight{\ifdim\Gin@nat@height>\textheight\textheight\else\Gin@nat@height\fi}
\makeatother
% Scale images if necessary, so that they will not overflow the page
% margins by default, and it is still possible to overwrite the defaults
% using explicit options in \includegraphics[width, height, ...]{}
\setkeys{Gin}{width=\maxwidth,height=\maxheight,keepaspectratio}
% Set default figure placement to htbp
\makeatletter
\def\fps@figure{htbp}
\makeatother

\KOMAoption{captions}{tableheading}
\makeatletter
\@ifpackageloaded{bookmark}{}{\usepackage{bookmark}}
\makeatother
\makeatletter
\@ifpackageloaded{caption}{}{\usepackage{caption}}
\AtBeginDocument{%
\ifdefined\contentsname
  \renewcommand*\contentsname{Table of contents}
\else
  \newcommand\contentsname{Table of contents}
\fi
\ifdefined\listfigurename
  \renewcommand*\listfigurename{List of Figures}
\else
  \newcommand\listfigurename{List of Figures}
\fi
\ifdefined\listtablename
  \renewcommand*\listtablename{List of Tables}
\else
  \newcommand\listtablename{List of Tables}
\fi
\ifdefined\figurename
  \renewcommand*\figurename{Figure}
\else
  \newcommand\figurename{Figure}
\fi
\ifdefined\tablename
  \renewcommand*\tablename{Table}
\else
  \newcommand\tablename{Table}
\fi
}
\@ifpackageloaded{float}{}{\usepackage{float}}
\floatstyle{ruled}
\@ifundefined{c@chapter}{\newfloat{codelisting}{h}{lop}}{\newfloat{codelisting}{h}{lop}[chapter]}
\floatname{codelisting}{Listing}
\newcommand*\listoflistings{\listof{codelisting}{List of Listings}}
\makeatother
\makeatletter
\makeatother
\makeatletter
\@ifpackageloaded{caption}{}{\usepackage{caption}}
\@ifpackageloaded{subcaption}{}{\usepackage{subcaption}}
\makeatother
\ifLuaTeX
  \usepackage{selnolig}  % disable illegal ligatures
\fi
\usepackage{bookmark}

\IfFileExists{xurl.sty}{\usepackage{xurl}}{} % add URL line breaks if available
\urlstyle{same} % disable monospaced font for URLs
\hypersetup{
  pdftitle={Racismo, ideologia, propostas. E o artista?},
  pdfauthor={Leo Gilson Ribeiro},
  colorlinks=true,
  linkcolor={blue},
  filecolor={Maroon},
  citecolor={Blue},
  urlcolor={Blue},
  pdfcreator={LaTeX via pandoc}}

\title{Racismo, ideologia, propostas. E o artista?}
\author{Leo Gilson Ribeiro}
\date{}

\begin{document}
\maketitle
\begin{abstract}
Jornal da Tarde, 1990/03/16. Aguardando revisão.
\end{abstract}

\renewcommand*\contentsname{Table of contents}
{
\hypersetup{linkcolor=}
\setcounter{tocdepth}{2}
\tableofcontents
}
\bookmarksetup{startatroot}

\chapter*{\texorpdfstring{\textbf{?meta:cover.title}}{?meta:cover.title}}\label{section}
\addcontentsline{toc}{chapter}{\textbf{?meta:cover.title}}

\markboth{\textbf{?meta:cover.title}}{\textbf{?meta:cover.title}}

Leo Gilson Ribeiro foi certamente um dos intelectuais e críticos de
literatura mais engajados em denunciar o racismo, bem como especialmente
em divulgar a literatura negra - do Brasil, da América Latina, dos EUA e
da África - em nosso país, mas cuja memória, infelizmente, foi acometida
da mesma invisibilidade denunciada pelo personagem central de Ralph
Ellison em sua importante novela \emph{The Invisible Man}, relato este
utilizado sagazmente pelo próprio Leo Gilson Ribeiro no início de uma
conferência sobre a literatura negra em 1985 proferida no Centro
Cultural São Paulo.

Assim, tanto no que diz respeito à literatura brasileira, quanto em
relação à literatura caribenha, norte-americana ou africana raramente
encontramos hodiernamente alguma menção ao crítico que tanto fez para
colocar em evidência nos principais veículos de imprensa nacional nos
quais trabalhou a importância de alguns autores negros sobre os quais
ele havia escrito quase sempre em tom elogioso chamando a atenção do
grande público para os mesmos em décadas passadas nas quais quase
ninguém na grande imprensa fazia algo similar.

Desde 1959, isto é, logo após retornar de sua formação acadêmica na
Europa (1953-1958 nas universidades de Hamburgo e de Heidelberg), o
jovem professor (pois antes mesmo de concluir seu doutorado na
Universidade de Hamburgo sobre Teixeira de Pascoaes (\emph{Die} Saudade
\emph{als Form des Pantheismus veranchaulicht am Werke von Teixeira de
Pascoaes/ A saudade como forma do panteísmo ilustrada na obra de
Teixeira e Pascoaes}) Leo Gilson Ribeiro já atuava em Heidelberg como
\emph{Lektor} de Literatura Brasileira (o que fez de 1956 até 1958)
quando então teve de regressar à primeira universidade na qual havia
iniciado seus estudos na Alemanha - a Universidade de Hamburgo - para
realizar a defesa de sua tese de Doutorado. No Brasil, contudo, ele não
pôde ingressar na vida acadêmica nacional porque para isso ele teria de
ter enviado para Brasília o seu diploma original de Doutorado obtido em
fevereiro de 1958 junto à Universidade de Hamburgo com a possibilidade,
segundo relato pessoal do próprio autor, de perdê-lo, algo que ocorria
com frequência naquela época.

O retorno ao Brasil, dada à dificuldade enfrentada para o reconhecimento
de seu diploma de Doutorado, leva então o jovem professor a atuar no
campo do jornalismo cultural. Isso se deu inicialmente no Rio de
Janeiro, onde ele residia, por meio de sua contribuição a diversos
jornais (\emph{Diário de Notícias}, \emph{Correio da Manhã},
\emph{Jornal de Letras} e \emph{Jornal do Brasil}), periódicos
(\emph{Comentário} e \emph{Chuvisco}) e revistas (\emph{Manchete} e
\emph{Cruzeiro Internacional}), mas especialmente, cabe destacar aqui, a
sua atuação junto ao jornal \emph{Diário de Notícias}, veículo no qual
criou uma importante coluna cultural intitulada \emph{Caminhos da
Cultura}.

No \emph{Correio da Manhã} publica de julho a setembro de 1965 uma série
de cinco reportagens voltadas ao tema do racismo e da literatura negra
sob o título geral de ``Discriminação racial'' (``O problema crucial do
século XX'', ``O que significa ser negro'', ``\emph{Apartheid} - a
legalização da paranóia'', ``África do Sul -- um vasto campo de
\emph{Displaced Persons}'' e ``A África do Sul -- a conivência adia a
solução''). Em anotações pessoais datadas de 1966, encontramos um plano
exposto pelo autor de lançar um livro ainda naquele ano que seria
denominado ``Três desafios do século XX'' e que seria subdividido em
três partes, a primeira focada no tema da opressão cultural do artista
na União Soviética, a segunda na discriminação racial nos EUA e na
África do Sul e a última dedicada à explosão demográfica nos países
subdesenvolvidos. Nessas notas manuscritas, Leo Gilson Ribeiro menciona
que já havia escrito sobre vários tópicos relacionados ao racismo, tais
como: origens da discriminação racial, Gabineau e a deformação do
darwinismo, origens psicológicas do preconceito racial, a
vulnerabilidade do Sul e a Guerra Civil, a Ku Klux Klan, dentre vários
outros. De fato, encontramos entre seus papéis setenta páginas
datilografadas e inéditas que discutem esses e outros temas relativos ao
racismo dentro de uma pasta contendo na sua capa o título do livro que
pretendia publicar ``Três desafios do século XX''.

Já em São Paulo, para onde se muda em 1966, a fim de trabalhar, a
convite de Mino Carta, no recém criado \emph{Jornal da Tarde}, Leo
Gilson Ribeiro publica uma extensa e muito elogiada reportagem para a
revista \emph{Status} em novembro de 1976 intitulada ``Hitler está vivo:
na África do Sul''.

Durante 47 anos de produção em jornais e revistas, isto é, desde o seu
retorno ao Brasil até a sua morte ocorrida em 2007 Leo Gilson Ribeiro se
esforçou sempre para tornar mais conhecida a literatura negra escrita no
Brasil (Carolina Maria de Jesus, Lima Barreto e Paulo Colina dentre
outros), na América do Sul (Léon Damas e Aimé Césaire), nos EUA (Richard
Wright, James Baldwin, Charles Wright, Ralph Ellison e Toni Morrison) e
na África (Léopold Senghor, Castro Soromenho, Wole Soyinka, José
Luandino Vieira, Uanhenga Xitu e Chinua Achebe dentre outros).

A recolha desse material, injustamente esquecido, me parece um
importante documento para evidenciar como Leo Gilson Ribeiro procurou no
inteiro percurso de sua longa carreira difundir a literatura negra em
nosso país nos principais meios de comunicação (além dos já citados cabe
destacar os principais veículos nos quais trabalhou desde a sua ida a
São Paulo, a saber, o \emph{Jornal da Tarde,} o semanário \emph{Veja} e,
por fim, a revista \emph{Caros Amigos}) a que teve acesso como crítico
literário. Cabe mencionar igualmente que ele participou, sempre que
possível, dos eventos - acadêmicos ou não - organizados sobre a
literatura negra, entrevistou escritores negros do Brasil e do exterior,
e realizou perfis ou necrológios sobre autores negros, bem como produziu
diversos artigos sobre distintos escritores e escritoras negras e
especialmente, é claro, sobre seus livros.

A escolha de textos de autoria de Leo Gilson Ribeiro que compõem esse
volume sobre o racismo e a literatura negra evidentemente não pretende
exaurir toda a produção dele sobre o tema, mas procura oferecer ao
leitor interessado um roteiro de leituras vasto e diversificado pela
literatura negra - positiva ou negativamente avaliada por nosso crítico.
Uma literatura que Leo Gilson Ribeiro claramente previu, já em meados
dos anos 80 do século passado, que por meio de seus autores poderia
trazer ``a inovação indispensável e especificamente negra para a
Literatura Brasileira''.

Percebe-se da leitura atenta desses textos, evidentemente
circunstanciais como são os textos destinados à imprensa cotidiana, que
o foco de nosso crítico sempre foi a figura do artista - negro ou não -
que escreve sobre a situação do negro frente a uma sociedade racista e
preconceituosa na qual se encontra. Mais ainda, podemos perceber sempre
a importância extrema que tem para o nosso crítico a dimensão ética de
um autor. O escritor para ele deve, portanto, estar profundamente
ancorado em sua sociedade e refletir criticamente sobre ela não se
deixando levar por uma literatura de teses (como ele demonstra ter sido
o caso do primeiro romance de Aluísio de Azevedo) ou por um mero
conteúdo panfletário. Note-se bem que para o nosso crítico a literatura,
qualquer que seja a posição ideológica de uma autora ou autor, deve
sobretudo revelar uma excelência formal e imaginativa não cedendo em
hipótese alguma a uma verborragia pomposa, mas vazia ou a meras fórmulas
dogmáticas e panfletárias.

Outro aspecto que chama a atenção e que evidencia a formação acadêmica
do autor é seu pendor comparativista. Em diversos artigos, escritores e
escritoras de outras literaturas são mencionados para podermos pensar
melhor determinados aspectos da obra de um autor ou autora que está sob
análise. Deste modo, mas sem perder a singularidade de cada artista ou
sem deixar de acentuar a imensa dificuldade hermenêutica de mergulhar em
outros universos literários distantes do ocidental, Leo Gilson Ribeiro
procura sempre inserir uma escritora ou um escritor em um universo
literário mais amplo e com isso, obviamente, somos nós, leitores de seus
textos - notas, resenhas, perfis ou entrevistas - que ganhamos novas
intuições e pistas que poderão - caso alguém se dê ao trabalho de
segui-las e de se aprofundar nelas -- de nos levar a descobrir novos
horizontes literários, bem como novas áreas de pesquisa em literatura
comparada.

Um sinal interessante e ao mesmo tempo triste de nossa memória cultural
é que muitos dos livros comentados por Leo Gilson Ribeiro de autores
africanos, e que mais recentemente foram reeditados ou traduzidos pela
primeira vez, infelizmente, não fazem nenhuma menção ao nosso crítico em
suas respectivas introduções. Por outro lado, muitos autores por ele
citados, ainda continuam inéditos esperando alguma editora que os
disponibilize em boas traduções para o público ledor brasileiro.

\emph{Fernando Rey Puente}

\clearpage \thispagestyle{empty} \mbox{} \clearpage \mainmatter

\bookmarksetup{startatroot}

\chapter{Racismo, ideologia, propostas. E o
artista?}\label{racismo-ideologia-propostas.-e-o-artista}

Jornal da Tarde, 1990/03/16. Aguardando revisão.

\hfill\break

A complexidade francamente indescritível das relações inter-raciais nos
Estados Unidos, na África do Sul do nazista \emph{apartheid}, no Brasil
e agora na Europa Ocidental, sem esquecer que os povos eslavos também
são sobejamente conhecidos tanto por seu antissemitismo quanto por seu
ódio aos negros e orientais -, essa complexidade não foi liquidada pelas
conjecturas de um Sartre, nem de um Malcolm X, nem de Nabokov -- que
achava indecente haver o próprio conceito de racismo.

Agora, grupos radicais nos Estados Unidos reivindicam para os negros o
Sul do país, enquanto os radicais brancos querem fazer do Noroeste a
América branca, vizinha do Canadá. O excelente novelista negro Ralph
Ellison, em seu magistral livro de ácido humor, \emph{The Invisible
Man}, postulou que o negro é, em si, \emph{o homem invisível} nos
Estados Unidos, um ectoplasma transparente deliberadamente \emph{não
visto} pelos brancos que o odeiam. Alguns dos meus mais inteligentes
amigos (brancos) norte-americanos, simpáticos à causa negra e à extinção
do racismo, estão atônitos, perplexos e deprimidos: ``Creio que o
problema do racismo aqui no meu país é insolúvel'', escreve-me a mais
arguta delas.

E finalmente para a novelista prolífica negra norte-americana Joyce
Carol Oates o negro deve superar o tema obsessivo da discriminação
racial de que é vítima. Um escritor homossexual tem que escrever apenas
sobre o homossexualismo e não sobre toda a humanidade?, parece indagar
pertinentemente.

Esses múltiplos aspectos desembocam na pergunta formulada por uma
excelente poetisa norte-americana, de cor negra (estas eternas
explicações soam idiotas!) que indagou, sem nenhuma leviandade: ``Como
ressarcir uma raça do fardo incalculável que sofreu ao ser
escravizada?'' Deveria haver uma compensação, como a que a Alemanha
Ocidental fez aos judeus no Estado de Israel? Deveria haver
oportunidades excelentes para os negros estudarem, aprenderem
profissões, restaurar-lhes a dignidade aviltada?

Alguns legisladores, sabe-se lá se bem intencionados, propuseram -- não
sei se com ingenuidade ou cinismo -- que \emph{todos} os negros
norte-americanos fossem para a Libéria, um Estado artificial, na África,
comprado com dinheiro dos americanos brancos, para que lá os negros
voltassem ao solo pátrio africano e não se falava mais do assunto; ah,
não, claro que haveria um miniplano Marshall para a Libéria: ajuda em
dinheiro, em tecnologia, para que os liberianos, descendentes dos
escravos estadunidenses, instaurassem na África Negra um Estado próspero
e modelar. A ideia fracassou retumbantemente.

O pequeno, mas eloquente livro de Clóvis Moura, recém-publicado pela
Editora Ática, \emph{História do Negro Brasileiro}, traz mais um ponto
de vista forçosamente ideológico e, portanto, debatível, mas é preciso
reconhecer que o racismo e a ideologia que subjaz a ele \emph{são
inseparáveis}.

Também é preciso não esquecer que os russos, os poloneses, os tchecos e
os franceses, ingleses, holandeses, espanhóis, árabes e outros povos da
Europa e da Ásia são decididamente \emph{contra} o negro. O preconceito
não é um privilégio dos EUA, da África do Sul, do Brasil etc., \emph{de
forma alguma}.

Sobra então o que deve soar como uma blasfêmia aos ouvidos dos militares
e de grande parte da população brasileira: a diluição da noção de
Nação-Estado, desmembrando-se o Brasil, por exemplo, em territórios
africanos, áreas exclusivamente indígenas, outras brancas, outras de
mestiço? Afinal, a Itália e a Alemanha se uniram como Nações-Estado no
século passado: fracioná-las não seria impossível e por tudo que os
alemães orientais dizem, eles se recusam a ser ``digeridos'' pelo
capitalismo consumista da Alemanha Ocidental.

Um intelectual russo de grande renome, Afanasyev, argumenta \emph{a
favor} da liquidação da Nação-Estado soviética que é também um Império
de mais de 100 nacionalidades diferentes: será, crê, uma forma de
solucionar os entrechoques étnicos e impedir o colapso e o caos da
``Desunião Soviética''.

E nesse \emph{maelstrom} de ideias em conflito, o poeta, o artista, o
escritor de raça negra e dos dois sexos: que devem fazer?

\bookmarksetup{startatroot}

\chapter{Perfil da Literatura Negra}\label{perfil-da-literatura-negra}

Mostra Internacional de São Paulo, 1985/05/20-26. Aguardando revisão.

\hfill\break

\begin{quote}
``Eu sou invisível.

Não, não sou um fantasma como os que perseguiram Edgar Allan Poe. Nem
sou um daqueles ectoplasmas feitos nos filmes de Holywood. Sou um ser
humano feito de substância de carne e osso, fibras e líquidos -- e
talvez se pudesse até dizer de mim que tenho um cérebro. Sou invisível,
compreendem, apenas porque as pessoas se recusam a me ver\ldots{}

Nem a minha invisibilidade é uma questão de algum acidente bioquímico
que tenha acontecido com a minha epiderme. A invisibilidade a que me
refiro acontece devido a uma insólita disposição dos olhos daqueles com
quem entro em contato. Trata-se de uma construção de seus olhos
interiores, aqueles que as pessoas usam para ver, através de seus olhos
físicos, para ver a realidade\ldots{} Você se contorce de dor, ansioso
para se convencer de que você \emph{existe} no mundo real, de que você
faz parte dos sons da angústia circundante, e você passa a usar os
punhos, amaldiçoa sua condição e jura que conseguirá fazer os outros te
reconhecerem. Qual o quê! Raramente se consegue..''
\end{quote}

Quem está falando assim? Um judeu em plena Alemanha nazista? Um
homossexual? Um índio? Um velho? Uma mulher discriminada em todas as
áreas de seus direitos humanos? Um velho, mero ferro velho hoje
imprestável depois que foi usado até o fim? Um índio de terras, família
e cultura roubados impunemente e que reclama diante de autoridades como
que voluntariamente surdas?

Não: a voz do Homem Invisível -- ou da Mulher Invisível -- é a de um
negro ou negra. É a voz do extraordinário escritor norte-americano Ralph
Ellison em seu livro-chave \emph{The} \emph{Invisible Man}. Ele percorre
nesta narrativa fantástica todos os registros da experiência pela qual
passa \emph{quem não é visto} pelos outros. Haverá estratégias, modos de
comportar-me que mudem essa situação? Quem sabe se eu for obediente,
humilde, até mesmo subserviente, \emph{os outros} me verão e
responderão, mesmo que seja fracamente, a meus acenos para ser
reconhecido como um ser humano? Eu devo rebelar-me, usar a violência,
arrebentar todos os códigos? \emph{Eles e elas} me circundam, são
infinitamente mais fortes do que eu, eu quase diria que são todo
poderosos.

E se eu usar a inteligência -- será que serei visto? Ou devo isolar-me
num canto, exatamente como eles e elas querem, tornando-me uma figura
baça, apagada, quase inexistente, sem causar o transtorno da minha
presença-trambolho e me tornar invisível no meio de um gueto de outros
invisíveis como eu? Ou devo agir, desafiando a injustiça que me é feita
pelo preconceito, pela insensibilidade, pela inveja, pela crueldade,
pela deliberada intenção de me usarem e de fazerem de mim a polpa amorfa
de suas ordens e caprichos?

Não precisamos descer à banalidade histórica de mencionar que os negros
foram arrancados da África e transferidos à força para as Américas como
escravos, pois esse -- é óbvio -- é o nosso ponto de partida. Como,
porém, o marginal, o excluído antes de qualquer julgamento, o condenado
sem comparecer previamente diante de tribunal algum reage dentro de um
mundo que da escravidão o transformou num mero animal sem alma nem
identidade. O negro passou a ser, conforme o caso, uma minoria diante de
uma sociedade branca, como nos Estados Unidos, ou de uma sociedade que
mal emergiu do colonialismo imposto pelo europeu, como tantos países da
África Negra. Ou finalmente ele é ainda aquele em quem todos pisam, como
no regime racista do monstruoso \emph{apartheid} da África do Sul?

Recodificar pode parecer uma palavra pedante para tentarmos interpretar
como os poetas, artistas plásticos dançarinos, os compositores musicais,
os novelistas, poetas e dramaturgos negros \emph{viram} quem sempre se
recusou teimosamente a vê-los. Recodificar, com o significado de criar
um novo código é uma expressão difícil, hermética, por isso digamos mais
simplesmente: como o negro pôde sobreviver e achar o seu nicho nestas
sociedades: uma mais, outras menos intolerantes, uma mais hipócrita do
que outras em reconhecer seu preconceito racial contra ele?

O primeiro obstáculo, que reforça o desprezo aviltante que os racistas
demonstram pelos ``inferiores'' é, sem dúvida, o estereótipo, a
caricatura. O negro? Ora, é preguiçoso, cheio de superstições, pai ou
mãe de santo, lixeiro, prostituta disfarçada que desfila nua nas escolas
de samba, arrombador de casas, presidiário, no máximo jogador de futebol
ou corredor esportivo. Além, é lógico, de um garanhão insaciável, sempre
pronto para maratonas sexuais lendárias. Assim, cem milhões de
indivíduos arrancados à força e com engano do seu \emph{habitat} na
África Negra de seus ancestrais e trazido para as Américas como gado
humano simplesmente, \emph{ça va sans dire}, simplesmente NÃO TÊM
CULTURA. No máximo, são crianças supersticiosas e dóceis. Usam-se dois
critérios simultaneamente. Assim, quando se trata da Europa e partindo
da visão sacrossanta de que a Europa é o ÚNICO centro de civilização e
cultura do mundo, a literatura oral de Homero, na Grécia Antiga, é
louvada como tradição transmitida de geração em geração com seus versos
imortais da \emph{Ilíada} e da \emph{Odisseia}. Ignora-se, portanto,
criminosamente, dolorosamente, a riquíssima tradição oral da cultura e
da civilização autóctones da África Negra. Partindo de Benin, de Daomé,
do Quênia, do Senegal, da Nigéria e de dezenas de outras nações ao sul
do Saara que nos legaram testemunhos importantes como os códigos penais,
a genealogia das tribos, sobre as fábulas e apólogos de cada grupo e
sobretudo se joga no lixo o acervo riquíssimo dos rituais religiosos do
culto do sagrado que inflama o devoto coração negro. Sacudiam os ombros
os racistas dos colonizadores Impérios colonialistas, zombeteiros: é
inútil aprender línguas arrevesadas como o Wolof, o Bambara, o Peule, o
Bamileke, o Ewondo, o Kikongo, o Yoruba, o Hauoussa, o Kishauili ou o
Suahili. Aconteceu exatamente o que aconteceu no Zaire, ex Congo Belga,
onde os dominadores belgas só saíram apressados, quando da independência
de sua imensa e riquíssima colônia, e deixaram para trás apenas quatorze
africanos com diplomas universitários para tomar conta de um enorme país
emergente. Como os belgas, os demais colonizadores brancos -- com
raríssimas exceções -- ignoravam, arrogantemente, que toda aquela
tradição oral tão rica \emph{era também}, legitimamente, Literatura, era
História, era Sociologia; era Psicologia, eram os Mitos multisseculares
de povos inteiros! E como nas noites em que ``a negrada'' se reunia em
torno das fogueiras, nas clareiras da mata, exausta de cavar diamantes
para os patrõezinhos louros de Bruxelas ou Antuérpia na realidade
cantavam em coro zombando dos ``sinhozinhos'' e suas mentiras, com
aquela risonha ironia ferina do humor negro!

Depois, pouco a pouco, como uma neblina, desfez-se gradualmente a
invisibilidade. Grandes artistas europeus, grandes escritores e
filósofos brancos começaram a levantar o véu e a verificar que, como
diria Descartes, o negro pensa, logo existe! O quadro de Picasso
\emph{Les Demoiselles d'Avignon} testemunhava clara e conscientemente
influências das admiráveis máscaras negras, da escultura negra. Sartre
proclamava que o negro afrontado, subjugado, erguia do chão a pedra que
o branco arremessara contra ele com ódio e nojo que mal escondiam a sua
insegurança o seu medo e a sua inveja do ``Negro'', o negro passava a
usar as mesmas armas dos brancos. O negro começava a sua literatura
escrita, conforme os padrões ditados pelos brancos, por volta de 1900.
Mas não só a literatura escrita: nos Estados Unidos os \emph{spirituals}
(muito mais tarde o \emph{jazz}, o \emph{blues}) com seu conteúdo
bíblico, cantado nas igrejas protestantes ou no trabalho do campo, os
\emph{spirituals} se adaptavam perfeitamente ao sofrimento que no Velho
Testamento falava dos judeus, exilados de sua terra natal escravizados
no Egito, tangidos numa diáspora semelhante ao doloroso e involuntário
périplo do negro. Moisés estendia a visão paradisíaca da Terra Prometida
a todos os subjugados e expulsos de sua terra original. Os negros,
escravizados nos Estados Unidos, estavam no mesmo deserto espiritual, as
plantações de algodão do Sul rivalizando com o branco de suas carapinhas
e a religião de um homem que morreu na Cruz a lhes prometer a libertação
do cativeiro, quem sabe até a volta mítica à África de seus
antepassados?

Aqui também havia uma modificação do código que os brancos seguiam. Para
os negros a Terra Prometida, a Canaã, não era Israel, era a liberdade, a
justiça, a democracia, o reconhecimento de seus direitos humanos que
lhes dessem a possibilidade concreta, cotidiana, de serem considerados
adultos livres e pensantes, independentes e responsáveis. Assim como
teriam que gozar da plenitude de seus direitos humanos inalienáveis
porque vinham junto com sua imersão na água batismal. A Bíblia passava a
ser o livro da esperança e da promessa, não só judaica, não só cristã,
mas também negra. Não podemos, dolorosamente, deixar de reconhecer que
as Américas também neste ponto como em tantos outros, serviram de
ambiente para \emph{pogroms} e para a destruição dos negros, assim como
ocorreu, por exemplo, com o Quilombo de Palmares.

As Américas nasceram de dois holocaustos: o das tribos indígenas e o
holocausto: dos negros chicoteados de seu habitat natural rumo a países,
povos e línguas estrangeiras, como diz a esplêndida escritora inglesa,
Doris Lessing: a arrogância, a falta de humildade, de curiosidade por um
ser humano diferente, mas nunca inferior

A Liberdade tinha sido conquistada com a vitória das tropas de Lincoln?
A liberdade era boa para os brancos? Ela tinha que ser boa também para
aos negros. Ao grito de Marcus Gavey ``\emph{Come back to Africa!}''
(``Voltem para a África!'') quando os Estados Unidos fundaram a Libéria
justamente para se descartar daquela massa de trabalhadores gratuitos,
agora que eles tinham se tornado um estorvo pago e renitente, ele
lucidamente respondia que não, o lugar dos ex-escravos é aqui, na
América. O que nos pode servir de lição: W. E. Dubois precedeu os
movimentos de libertação de vários países africanos negros e suas obras
serviram de livro de cabeceira ou Bíblia para líderes africanos como
Kwame Nkruman e Jomo Kenyata entre outros. Em 1903 ele já tinha a
audácia de proclamar: será inútil a batalha dos negros pelos seus
direitos? Estará o negro condenado como os heróis punidos gregos,
Sísifo, Tântalo -- a ver sempre seus esforços baldados? Os racistas do
Sul se organizaram em sinistros grupos fanáticos e violentos: o Ku Klux
Klan, apavorantes encapuzados, brandindo tochas de fogo e procurando
``justiça e divertimento'': sair à caça de negros e católicos para
enforcar ou linchar e depois incendiar suas míseras casas. Sereno em sua
combatividade, Dubois fez o que para os brancos e para muitos negros que
queriam ``esquecer'' a África que consideravam simiesca e selvagem,
(conforme o chavão distorcedor da verdade repetido mil vezes): Dubois
pensou profundamente, conscientemente e proclamou, sem se abalar, e de
suas meditações surgiu o primeiro raio de luz que rompia aquela total
invisibilidade negra. Em primeiro lugar, dizia, o negro \emph{tem que
ser ele mesmo}. Vencer a degradação social que lhe é imposta. Vencer a
miséria, a ignorância, a inércia. Vencer a imitação servil e infecunda
dos modelos brancos. Não está aí, inteira, a semente da futura
\emph{Négritude} de Senghor e de Aimé Césaire, aquele movimento de
valorização total do mundo negro que genialmente Lima Barreto vira entre
nós, muitos anos antes do ilustre poeta-Presidente do Senegal e do poeta
das Antilhas com seu movimento da \emph{negrice}?

Chicoteados para fora de sua paisagem natural rumo a países
desconhecidos em fétidos navios negreiros, rumo a países e povos de
línguas e costumes estrangeiros o negro deparou com características dos
brancos dominantes que a magnífica escritora inglesa Doris Lessing foi
cáustica mas justa ao denunciar: a arrogância vazia, a falta de
humildade, de mera curiosidade por um ser humano diferente, mas nunca
inferior, levou o branco a destruir, sem cuidados impérios inteiros:
civilizações e culturas antiquíssimas da África, dos astecas, maias e
toltecas no México e Guatemala, dos Incas no Perú e Bolívia -- todos
caíram arrasados pela soberba branca -- mercantilista e protestante na
América do Norte, hipocritamente conivente por meio da Igreja Católica
que se diz seguidora do ``amai o próximo'' pregado por Cristo e que
fazia vista grossa com relação à escravidão e ao massacre dos negros e
índios no Brasil. Como a mesma Igreja, dita Católica, mais tarde
abençoaria o fascismo de Mussolini, os trens de Hitler que despachavam
soldados para a Segunda Guerra Mundial e calava seus púlpitos diante do
genocídio de judeus, eslavos, homossexuais, ciganos aos milhões nos
campos de concentração\ldots{}

Mas se uma parte crescentemente influente e importante da
intelectualidade branca se aliara, eloquente, à causa da abolição da
escravatura -- no Brasil o poeta baiano Castro Alves, o pensador da
aristocracia pernambucana, Joaquim Nabuco, nos Estados Unidos a
romancista Harriet Beecher Stowe e seu romance \emph{A cabana do pai
Tomás} -- persistia outro problema crucial. Como manter fielmente os
valores ancestrais africanos usando-se de uma língua estrangeira? O
português, no Brasil, em Angola, em Moçambique, em Cabo Verde e em
Guiné-Bissau; o espanhol, em Cuba, na Colômbia, no Peru, na Venezuela; o
inglês nos Estados Unidos, o francês nas Antilhas tinham outras
sonoridades, diferentes da estrutura e das sonoridades originais das
línguas e dos dialetos da África Negra. Foi uma escolha difícil,
decisiva mesmo, mas que brotou simultaneamente em todos os países: o
negro contornou essa dificuldade. Apoderou-se da língua dos dominadores,
da maioria branca. Não sei se cometo uma injustiça e se for o caso
penitencio-me dela desde aqui e agora, mas tudo leva a crer que foi nos
Estados Unidos que primeiro explodiu o orgulho de ser negro, a recusa de
macaquear os modos de vestir, de falar, de ser do homem branco. W. E.
Dubois, em 1890 -- quando a União nacional quase se esfacelara por
causa, principalmente, da abolição da escravatura nos EUA, nos Estados
Confederados do Sul agrário e escravagista -- Dubois atreveu-se a
proclamar o ``Movimento do Niágara''. O que queria esse negro assanhado,
que se formara em filosofia em Harvard e Berlim? Assediar a consciência
norte-americana até que os negros, afrontados em seus mínimos direitos,
lesados, sem direito a voto, pessimamente alojados, pessimamente
remunerados, sem acesso à instrução: cumpria torná-los cidadãos iguais
aos outros como rezava, formalmente, a Constituição.

Ao mesmo tempo que ele clamava por boas escolas (mas não de samba), pelo
voto que não fosse fraudado nas urnas controladas por brancos
inescrupulosos. Ele aconselhava o enriquecimento dos negros para poderem
competir em pé de igualdade com os ricos de pele clara, professava sem
nenhum fanatismo uma fé profunda e comovedoramente democrata: Liberdade
individual e de escolha de trabalho. Liberdade de associação. Liberdade
de pensamento e de opinião. Liberdade de culto e de educação. E
generoso, ultrapassando as falsas barreiras da cor da pele,
acrescentava, convincente:

``De tudo isso nós precisamos, não separados dos demais, mas sim todos
combatendo pelo mesmo ideal: o ideal da fraternidade humana. Sem
oposição nem desprezo pelas outras raças que constituem a República
norte-americana''.

Poeta, ao contrário de Dubois, que fundou a Associação para a Defesa das
Pessoas de Cor e lutou de 1910 a 1945 pela independência dos países da
África Negra subjugados pelos colonizadores ingleses, franceses,
holandeses, belgas, portugueses, alemães, o poeta Langston Hughes lança
a semente de um sentimento de exílio negro em meio à neve nos Estados
Unidos. E como o nosso poeta Jorge de Lima ao falar do '\emph{banzo}'',
da saudade triste que se apodera do negro no Brasil longe de seu meio
ambiente natural, na África longínqua; do negro que teme a civilização
materialista, melancólica, cinzenta, dos arranha-céus. Ele anseia pelo
sol, pelas palmeiras, pelos tambores nativos e espera algum dia não ter
vergonha de se sentar à mesa dos brancos: ``pois eu também sou a
América''.

Essa reação do negro transplantado para outros países se divide em
vertentes diferentes, naturalmente. De um lado, há os que anseiam por
voltar à África, à aldeia tribal, ao ambiente primevo dos avós e
bisavós: em sua busca de readquirir uma identidade perdida. De outro, a
aceitação consciente de que ``agora somos parte integrante e inalienável
do mosaico das Américas''. Esta aceitação de seu americanismo -- afinal
há mais ou menos 400 anos os negros aculturados no Novo Mundo são dos
mais antigos e básicos ``americanos'', do Sul, do Centro ou do Norte da
América -- não deixa de ter uma subdivisão amarga, cheia de ódio e
ironia, em que se emprega a mais ácida ironia contra o branco opressor.
Demonstra essa reação um dos poetas brasileiros mais talentosos dentre
os que Paulo Clina judiciosamente coligiu em sua pioneira coletânea de
poetas negros brasileiros contemporâneos, denominada \emph{Axé}, pela
qual lutei denodadamente e com razão na reunião da Associação Paulista
de Críticos de Arte que lhe atribuiu um prêmio justíssimo e até então
outorgado somente a autores brancos ou mestiços. O poeta Cuti
exemplifica essa atitude em seu poema intitulado ``Palavra'':

\begin{quote}
Palavra~

Que sai da violência, do sarcasmo~

Dos bêbados felizes~

Deitados na rede da ilusão dum Brasil branco~

Arrotando a vergonha-sobremesa do nosso olho~

Hesitante em olhar para trás~

Dos partidários de Rui Barbosa hoje~

Na queima da ``mancha negra da escravidão''~

Do esquecimento das feridas que escorrem pus~

Na carne de milhões de brasileiros~

Nós~

Que sai do cheiro podre de favela~

Onde a cor das pessoas~

É seta é meta direta da mentalidade escravagista~

Em branco processo disfarçado~

Palavra~

Arrancada da angústia que se torce no meio dos medos~

Do povo doente de não querer parentesco de escravo~

Palavra~

Fustigada : ``Racista!''~

Porque picha verdades cruas na cara da compostura~

Maquiada de sonhos europeizantes~

Palavra~

Cuspe de escravo na cara do amo~

Do amo-bibelô-vovô guardado~

Nas estrofes do poema encomendado pelo cofre~

Nos talheres da inconsciência~

No útero estéril da nobreza~

Palavra com olho d'água~

Com choro censurado~

Orvalho de tristezas sobre folhas de cafezal~

Em farpas de canavial~

No sórdido riso dourado da terra~

Palavra com palavra que se embola~

Mar de bolhas que fervilha~

Pela estrada dessa História de nós povo aqui~

Trazido~

Aqui sugado~

Aqui bagaço~

Cuspido~

Palavra-cobrança na porta emperrada da consciência~

Nacional~

Sinal do sentimento nosso~

Nessa língua estrangeira~

Por enquanto
\end{quote}

Nas Antilhas, Aimé Césaire e depois o senegalês Léopold Senghor
extrapolam a celebração do ser negro com a \emph{Négritude}, uma
tendência que o grande escritor brasileiro negro, Lima Barreto já
antecipara, como dissemos anteriormente, com seu conceito revolucionário
de celebração da ``negrice'' brasileira, que ele tentava articular em
verdadeiro movimento literário. O estopim desse Renascimento Negro parte
da revista criada por intelectuais da Martinica intitulada
\emph{Légitime Défense}. Os antilhanos de cultura francesa,
empanturrados do Parnasianismo quando Paris já tinha Rimbaud,
Baudelaire, Verlaine e os surrealistas se rebelam, com razão. A nota de
autenticidade africana é confirmada pela revista fundada em Paris
subsequentemente, em 1934: \emph{L'Étudiant Noir} (\emph{O Estudante
Preto}). Aí Aimé Césaire avança a tese de que nem o surrealismo nem o
marxismo podem servir de modelos para a \emph{Négritude}, pois são
movimentos que partem da Europa, dirigidos a europeus: ora, a África
Negra, demonstrava o etnólogo alemão Frobenius, de maneira
irrespondível, é civilizada até à medula dos ossos. Antes da chegada dos
brancos colonialistas à África, os negros indígenas nada tinham de
subdesenvolvidos no plano das artes, da literatura, das religiões, das
relações familiares, no plano jurídico, moral e político:

``A ideia do negro bárbaro não passa de uma invenção europeia'', afirma
Frobenius, dispondo de dados arqueológicos irrefutáveis, convincentes na
mão. Theodore Monot também denuncia a imbecilidade europeia de tomar a
experiência do homem branco como a única. No máximo, os europeus
poderiam ter um avanço \emph{tecnológico} -- o que hoje nos levou a
Hiroshima, aos mísseis SS-2L russos e aos mísseis Pershing e Cruise
norte-americanos.

A tecnologia que os índios astecas confundiam com a vinda dos deuses ou
com a magia ao vislumbrarem Hernán Cortés e seus soldados descerem de
navios montados a cavalo e com pedaços de madeira que cuspiam fogo e
matavam alvos ao longe. Um poeta e ensaísta dos Estados Unidos com quem
conversei hoje, durante esse frutífero congresso, o norte-americano Don
Lee, assumiu conscientemente e voluntariamente uma dupla personalidade:
é Don Lee por batismo na América anglo-saxônica e é também por eleição
própria Haki Madhubuti. Ele retoma a frase célebre de Aimé Césaire que
ainda considera válida hoje em dia:

``Nous autres nègres n'avons rien inventé~» (Nós, negros, nada
inventamos) para exortar a comunidade negra a apoderar-se dos
conhecimentos, vitais hoje em dia para qualquer raça, da era tecnológica
branca, na Rússia ou nos Estados Unidos ou do \emph{know-how} japonês
com sua nipônica coesão nacional e sua supremacia eletrônica alcançada
hoje em dia já em termos mundiais de competição com os EUA. Don Lee pede
a altos brados: menos televisão e mais estudo. Mais livros para os
negros, mais conhecimentos para os negros: este é o desafio para os
negros hoje em dia mais urgente, crê.''Não sou contra os brancos nem
tenho objetivos absurdos de querer tomar para os negros, como
compensação pela escravidão, os territórios dos Estados do Sul dos
Estados Unidos, a Geórgia, o Alabama, o Tennesse, o Texas etc., como
advogam, sem pensar, certos grupos fanáticos negros. O que eu quero é a
participação integral do negro na malha de informações e eficiência do
mundo moderno, porque mais do que nunca hoje em dia a informação, saber
é poder''

O negro encarou o código que lhe impunha o branco dominante e o
modificou estruturalmente, creio. O calor humano, a solidariedade, a
fraternidade, a criatividade artística, a reflexão serena em todas as
áreas da ação e do pensamento vieram dar vida às estátuas do Museu de
Cera em que se estão transformando as sociedades brancas racistas que se
idolatram a si mesmas, alijadas voluntariamente do espaço e do tempo.

A integração pacífica das raças, a miscigenação serão utopias a que
recorremos para ocultar a hipocrisia do preconceito generalizado? A
África do Sul, esse enclave hitlerista em pleno final do século XX,
``liberou'' recentemente o amor e a união matrimonial entre pessoas de
raças diferentes -- quanta magnanimidade! Isso embora marido e mulher de
cores diferentes ou amasiados terem que continuar a se sentar em bancos
de jardins públicos diferentes, a viajar em trens diferentes e tomar
água em bebedouros diferentes, além de outras restrições.

E não é sem propósito a menção específica à África do Sul. Eu lhes pedi
licença para iniciar esse apressado esboço homenageando as organizadoras
deste importante encontro inicial. Permitam que o encerre com uma nota
dupla: primeiro, uma nota de esperança, a fim de que a educação, o
respeito mútuo tragam uma contribuição recíproca cada vez maior a todas
as raças, visando ao objetivo maior da humanização de nosso convívio. A
segunda nota seria o reconhecimento, a homenagem sincera a todos os que
forjaram a libertação plural da África Negra, muitos deles autores
brancos como Alan Patton, Doris Lessing, Nadine Gordiner e os grandes
líderes atuantes hoje em dia: Dom Desmond Tutu, Nelson Mandela, sem
esquecer os corajosos predecessores, o Reverendo Martin Luther King, o
Mahatma Gandhi. Todos nos legam e legaram uma lição imorredoura no
coração e no espírito do ser humano, não importa a sua raça: a noção
superior do \emph{ahimsa}, o princípio-bússola da não-violência, o amor
e o respeito pelo nosso próximo. Sem esquecer aquela anônima Rose, uma
faxineira preta norte-americana que certa vez se recusou a ceder, num
ônibus do Sul dos Estados Unidos, o seu lugar, na parte detrás do
veículo em que viajava a um passageiro branco e com isso deflagrou todo
o processo de mudança em prol dos direitos civis dos negros naquele
país. Desde aí ecoaram até nós as estrofes do hino da esperança cantado
por multidões de todas as raças, sexos e religiões ao invadir
Washington, diante do Monumento à Memória de Lincoln:

\begin{quote}
``We shall overcome!''~

``Venceremos!''
\end{quote}

O esplêndido escritor irlandês George Bernard Shaw exprimiu esse mesmo
anseio ao anotar lucidamente''

\begin{quote}
``Você vê como estão as coisas e diz: Por quê? Mas eu sonho com coisas
que nunca existiram e então eu digo: Por que não?''
\end{quote}

É o sonho visionário da grande alma de Martin Luther King exclamando:
``\emph{I have a dream}!'' (``Eu tenho um sonho!''): o sonho da justiça,
da liberdade, do respeito mútuo, da pacífica e fecunda convivência de
todos os seres humanos no mundo: não é chegado o momento de começarmos a
realizar concretamente o que sonhamos?

\part{Literatura Brasileira}

\chapter{O negro na literatura
brasileira}\label{o-negro-na-literatura-brasileira}

Anais do Seminários de Literatura brasileira: ensaios, 1990. Aguardando
revisão.

\hfill\break

Talvez a primeira observação que seria pertinente fazer com referência
ao tema do negro na Literatura Brasileira seria naturalmente a
manifestação mais óbvia, qual seja a de que a Literatura Negra está
completamente à margem da marginalidade. Enquanto alguns grupos de
reinvindicações feministas conseguem publicar algumas coisas que falam
de seus problemas específicos e principalmente dos seus direitos civis
pisoteados por uma classe dominante, machista e que não reconhece nenhum
direito a uma minoria dentro do que eles chamam de ``uma democracia'',
isto é, a predominância absoluta de uma minoria, já o negro, por
exemplo, tem os seus cadernos -- \emph{Cadernos negros, Cadernos do
canto negro} - que são mais ou menos clandestinos e só circulam através
de um certo gueto brasileiro. Durante as pesquisas que fiz para este
seminário não consegui detectar, em nenhum autor brasileiro específico,
a não ser nos autores negros, evidentemente, uma consciência do que
fosse esse problema sociológico tão sério num país como o Brasil.

Em seguida gostaria de me referir ao fato de que a literatura, quando
trata do negro, quando ela não é, de forma clandestina, senha de
mensagem entre grupos negros e poetas e escritores negros, ela é feita
pela parte \emph{soi disant} branca do Brasil, de uma maneira bastante
paternalista. Tanto por parte da crítica como também por parte do
público, notamos que não existe um estudo específico sobre a produção
literária especificadamente negra no Brasil ou que focaliza o negro,
desde Gregório de Matos Guerra até hoje, no decurso de nossa literatura.
Não só todos os propósitos das reinvindicações negras como também as
suas contribuições importantes estão mais ou menos relegadas a um
terceiro plano. A noção de diferença de epiderme, de diferença de
pigmentação surge muito cedo no Brasil, já no Brasil de Gregório de
Matos Guerra. Podemos citar, entre seus versos satíricos, alguns versos
em que ele zomba da sociedade baiana em 1700, na sua própria época. Num
trecho de seu poema ele diz:

\begin{quote}
Muitos mulatos desavergonhados trazidos sob os pés~

Os homens nobres postos nas palhas,~

Estupendas usuras nos mercados;~

Todos os que não furtam, muito pobres~

E eis aqui a cidade da Bahia.
\end{quote}

Ele também se ri de todos os descendentes de índios, em outros versos
famosos, quando diz:

\begin{quote}
Aqui é fidalgo nos ossos, cremos nós,~

Pois nisso consistia o maior brasão~

Daqueles que comiam seus avós
\end{quote}

Não é, porém, apenas a tina vitriólica dos sarcasmos de parte da obra de
Gregório de Matos Guerra que precocemente instaura o racismo como
critério de julgamento \emph{a priori} no Brasil. À medida que o tempo
avança, um autor como Bernardo Guimarães, por exemplo, em meados de
1800, escreve a famosa, hoje famosa até na China, \emph{A Escrava
Isaura}. E baste um trecho de louvação da escrava Isaura, que passava
por branca, para sabermos como é que o autor a descrevia:

\begin{quote}
``A sua tez é como o marfim do teclado, alva, que não deslumbra,
embaçada por uma nuança delicada que não saberei dizer se é leve palidez
ou cor de rosa desmaiada. Na fronte calma e lisa como um mármore polido,
a luz do ocaso batia um róseo e suave reflexo. Di-la-eis, à luz do
ocaso, misteriosa lâmpada de alabastro guardando no seio diáfano o fogo
celeste da inspiração''.
\end{quote}

Em outro romance, chamado \emph{Rosaura, a enjeitada}, Bernardo
Guimarães duvida que exista um brasileiro apenas cujos ascendentes não
tenham puxado flecha ou tocado marimba.

Mas possivelmente haja um \emph{parti pris} meu aqui. A grande explosão
se daria com o Adolfo Caminha e seus romances extraordinários.
Interessa-nos aqui principalmente \emph{O Bom Crioulo} em que ele
enfrentou a Marinha brasileira e seus mitos e tabus, como a chibata, que
era usada contra violações mínimas nas castas inferiores aos oficiais,
e, mais ainda, ele ousou, para um escândalo inimaginável na época,
colocar, no mesmo romance, dois problemas absolutamente inéditos no
Brasil - o racismo e o homossexualismo. Dois grumetes, um branco e o
outro negro, têm uma relação passional. Quando o louro, inconstante,
belo, afeminado, tendendo ao bissexual é morto pelo negro que por ele
estava apaixonado, atinge-se um clima passional digno de Eugene O'Neill.
Expulso da Marinha o escritor cearense, de meados a fim do século
passado, é saudado hoje, até nos EUA como um genial precursor da
literatura em que o \emph{black power} e o \emph{gay power} se afirmam
como correntes poderosas que se opõem aos diversos preconceitos que
deformam a sociedade norte-americana em sua contemporânea rebelião
contra o cativeiro e a omissão a que são sujeitos esses dois segmentos
da população norte-americana.

Com grande propriedade, Adolfo Caminha exprime em seu livro, o que ele
chama de ``esse acervo de mentiras galantes e torpezas dissimuladas'',
esse ``cortiço de vespas que se denomina sociedade''. A corrente racista
continuava apenas os éditos da coroa portuguesa, que, já em 1726,
proibia aos mulatos e homens de cor o acesso a qualquer cargo municipal
em Minas Gerais até a quarta geração, medida também aplicada aos homens
brancos casados com mulheres de cor. Há notícias de que jesuítas baianos
comerciavam com escravos negros como se fossem moeda corrente na época,
o que dá uma ideia das mudanças que cada século traz, lentamente, à
posição do negro na nossa literatura e na nossa vivência diária.

Não é no período propriamente contemporâneo e sim no período moderno que
nós encontraremos personagens negros, aqui e ali, em autores tão
dispares quanto Coelho Neto, José Américo de Almeida, José Lins do Rego
e Jorge de Lima. O racismo do historiador Oliveira Viana imperava quase
que soberano, reforçado pela obra do francês Lapouge, \emph{Les
sélections sociales}. Oliveira Viana afirmava a superioridade intrínseca
da raça branca sobre a negra e todas as demais, ao passo que,
anteriormente, o Conde de Gobineau, no qual, se diz que Hitler e Himmler
teriam se inspirado para criar a sua teoria nazista da
\emph{Herrenrasse}, a raça superior, ariana e alemã, proclama aos quatro
ventos, no seu livro famoso \emph{Ensaio sobre a desigualdade das raças
humanas}, a predominância da raça loura, de olhos azuis, logo seguida
pela raça latina e, em último, a raça eslava, considerada o lixo da raça
branca. Quem já teve algum contato com fotografias de guerra já viu, por
exemplo, mulheres russas, grávidas, amarradas com correntes à neve, num
ritual macabro dirigido por Hitler, enquanto os tanques nazistas
passavam em cima dessas infelizes. Das raças de cor, nem falar! O Conde
Gobineau era um comensal frequente nos banquetes de Pedro II, achando
que o Brasil só progrediria quando trouxesse milhões de imigrantes
europeus para cá.

Com a proclamação da República, um autor prolixo e talvez desprovido de
outro talento que não o de se enamorar de suas próprias palavras, Coelho
Neto, escreve um estranho e confuso romance, \emph{O Rei Negro}. Nesse
romance, os pretos Macambira e Balbina, no meio rural do Estado do Rio,
evocam o cativeiro e o autor dá uma nota feminista às mulheres, que já
eram meros objetos de fecundação, sem maiores direitos. Coelho Neto
tenta repetir as deturpações da língua portuguesa pelos negros; a
alteração da pronúncia de palavras - ``raio de roda'', ``tá in cima'',
``déssi'', ``dexa'', ``tá'' e ``Deus é grandi''. Mas os negros são
mostrados aí como no realismo mágico da literatura hispano-americana,
como sabedores de rituais secretos, de superstições primitivas que os
levam à loucura, até que o negro Macambira, príncipe em sua terra de
origem, se embrenha pela floresta incompreendido pelos que se opunham à
escravidão e rechaçado pelos brancos.

Nós nos lembramos sempre quando lemos os poemas de Manuel Bandeira, que
ele tinha acolhido a preta Irene como digna de entrar no céu. A boa
preta Irene não precisa nem pedir licença para entrar no paraíso. Já
José Américo de Almeida, no conto ``A baiana'', fala de uma empregada
que salva uma criança frágil.

\begin{quote}
``Generosamente a patroa de Irene ficou tocada de gratidão. Chegou a
termos de pespegar um beijo no focinho úmido da Baiana. É verdade, assim
a batizaram, desde que chegara, não sei se por causa da cor ou porque
era ama de leite. Beijou só uma vez, mas beijou. Depois cuspiu muito e
esfregou sabão nos beiços, o que a fez cuspir ainda mais''.
\end{quote}

Com Jorge de Lima temos a primeira incursão moderna na poesia de um
autor francamente a favor dos negros. É verdade que Olavo Bilac, na sua
poesia a respeito do banzo, se referira à saudade do negro da sua África
original, mas se trata de um exercício parnasiano estético, sem maior
profundidade psicológica. Nós não precisamos nos referir ao famoso
``Essa negra fulô\emph{``}, de Jorge de Lima, em que o poeta subverte o
encanto da bela escrava e a coloca em primeiro plano pelo amor do
patrão, em detrimento da patroa branca e que leva uma vida parasita.
Jorge de Lima, católico militante, saúda o negro com sentimentalismo, é
verdade, mas com uma veemência inusitada na Literatura Brasileira. Ele
canta o negro como sendo o advento não só da libertação do cativeiro
como também como a primeira raça que traz para o Brasil uma doçura
desconhecida da raça branca, na Europa.

Finalmente chegamos, dentro dessa rapidíssima análise de algumas
personagens negras da Literatura Brasileira, às personagens de Jorge
Amado, que muita gente compara às mulatas de Sargentelli. Seriam as
personagens de Jorge Amado simplesmente mulheres de cor, usadas como
mero chamariz lascivo para o seu livro, como Gabriela e outras
personagens? Eu não creio. Na sua vontade de prestar homenagem à mulher
voluptuosa e de cor, o autor baiano retratou o protótipo do uso que
delas fazem os homens, os donos da sociedade machista brasileira, que se
limitam a querer apenas usar a mulher sexualmente. Gabriela possui, na
realidade, toda uma carga anárquica da preguiça, de ruptura de tabus
sexuais e, com a inocência de quem não sofre a limitação de outros
tabus, busca o prazer. Se ele exige casamento ou não, não lhe importa.
Se ele inclui o adultério e algum dinheiro para se manter, que diferença
faz?

Ultimamente, a \emph{Antologia dos poetas negros brasileiros
contemporâneos}, sobre a qual eu chamei a atenção da APCA (Associação
Paulista dos Críticos de Arte) de São Paulo, coloca alguns excelentes
poemas ao lado de outros execráveis, mas talvez seja possível
terminarmos com a citação de Paulo Lima:

\begin{quote}
Até quando as estrelas~

Encravadas nas gretas,~

No céu aberto~

Nas palmas das minhas mãos,~

Chorarão as ninhadas~

Deste destino de ratos?
\end{quote}

Ou, como diz Semog:

\begin{quote}
De repente, assim, assim~

Num passe de mágica,~

Como uma fonte atávica,~

Comera todas as palavras~

Ou será que nós já éramos mudos?
\end{quote}

Restam as duas grandes figuras negras da Literatura Brasileira: Lima
Barreto e Cruz e Souza. Com relação a Cruz e Souza, é engraçado
observarmos que certos escritores norte-americanos que muitas vezes não
compreendem bem o português, alguns dos nossos brasilianistas que não
deram certo, veem no Cruz e Souza os atabaques da África original. Eu,
até hoje, nem mesmo com um metrônomo, não consegui descobrir os
atabaques de Cruz e Souza na sua obra simbolista. Ao contrário, eu
creria que Cruz e Souza simbolista procurou em símbolos místicos e no
uso de maiúsculas, muito comum nos textos místicos, no seu rebuscamento
de palavras e na sua busca de mundos etéreos, com virgens louras, buscou
justamente um alívio para a sua vida feita de racismo, de
desconhecimento e de miséria.

Já Lima Barreto teria um destino bastante mais trágico. Não só a miséria
e o racismo o perseguiram, como também a loucura do pai, que esteve ao
seu cargo, e também, mais tarde, a loucura dele próprio e a frustação de
todos aqueles que, no Brasil, e em outros lugares, denotam sinais de
genialidade, desde Baudelaire até Lima Barreto. Com um pai louco e seu
problema de alcoolismo agravado, ele foi escorraçado da sociedade dita
``de moral e bons costumes'' do Rio de Janeiro e nada o conseguiu
salvar. Quem quiser ter uma impressão realmente comovente do final de
sua vida, basta ler a biografia que dele fez a Editora José Olympio, com
Francisco de Assis Barbosa. Lima Barreto não só se depara com o
preconceito de raça, como também, como todo artista, ele questiona as
ideias que lhe são dadas como aceitas e, quando titubeia ao receber
essas ideias sem questioná-las, morre cedo e incompreendido. Talvez,
como no sonho do grande líder norte-americano Martin Luther King, não
haja, amanhã, justificativa para as palavras do estudioso do negro,
Haroldo Costa, em epígrafe no seu livro, que nos sirvam de lembrança de
um passado ultrapassado:

\begin{quote}
``É preciso não carregar a pele como um fardo''
\end{quote}

Estas anotações, na realidade, sinto-as como prematuras. Amanhã, tenho
certeza, o negro fará parte decisiva da Literatura Brasileira. Trará à
criação nacional toda a brandura e sabedoria que falta aos rígidos
gêneros dos brancos desprovidos de fantasia, darão facetas novas da
literatura que não se enquadram no museu de formol da literatura
cartesiana europeia, natimorta hoje. Afinal é da África, é da América
Latina mestiça que nos vêm hoje as vozes mais importantes da metamorfose
que Mallarmé queria: ``da vida estuante na sua complexidade quase
indecifrável de um poema, de um romance, a vida vista pelo prisma de um
indivíduo e transformada naquele material de criação humana perene: o
livro''. Seja qual for a sua cor e a sua origem, o negro, quem sabe,
será amanhã a inovação indispensável e especificamente negra para a
Literatura Brasileira e deixará de ser, de certa forma, sinônima com
ela. Quem sabe o negro será amanhã a própria Literatura Brasileira, em
sua parte decisiva. Assim seja.

\chapter{Entrevista - León Damas e a Negritude
brasileira}\label{entrevista---leuxf3n-damas-e-a-negritude-brasileira}

Correio da Manhã, 1965/04/03. Aguardando revisão.

\hfill\break

O poeta León Damas, da Guiana Francesa, que constituíra com Césaire, da
Martinica e Senghor, hoje presidente do Senegal, o triângulo sobre o
qual se baseia o movimento da \emph{Négritude}, encontra-se pela segunda
vez no Brasil. Na sua sala de estudos de um apartamento em Copacabana
ele nos recebe com grande afabilidade, cercado de livros brasileiros e,
bem à mão, um dicionário francês-português. Veio pesquisar, com uma
bolsa de estudos, a contribuição do homem de cor à literatura brasileira
e faz questão de ressaltar, logo de início, a cooperação que vem tendo
do diretor da Biblioteca Nacional e o valor da obra de Antônio Olinto
sobre os \emph{Brasileiros na África}. A busca de uma identidade
autônoma por parte dos negros, que assume características de um
movimento de reivindicações civis nos Estados Unidos adquiriu entre os
intelectuais negros de formação francesa uma diretriz nitidamente
cultural. Entre nós, o poeta Damas reconheceu imediatamente que a
contribuição do negro não se limitava à literatura, extravasando-se na
música, no \emph{folklore}, na linguagem.

\begin{quote}
``Eu percebi imediatamente que no Brasil a integração do negro não é um
\emph{cliché}, pois ela foi preparada desde a abolição e talvez mesmo
antes, incutindo no negro sentimentos patrióticos e a certeza de
pertencer a um novo país, a uma nova nacionalidade, que sintetiza
pessoas de todas as cores e proveniências. Esta \emph{largesse} aliás o
Brasil herda já de Portugal, que sempre admitiu negros em sua sociedade,
antes mesmo da escravatura nas Américas.''
\end{quote}

\begin{quote}
``O Brasil já tinha a sua negritude antes mesmo que nós a
estruturássemos. Constituem os seus primeiros sinais os cantos de
trabalho, que correspondem aos \emph{spirituals} dos Estados Unidos,
depois as lendas europeias alteradas nas suas versões africanas,
preservando-se integralmente o legado da África Negra nos rituais
religiosos de candomblé da Bahia, no \emph{folklore} estudado por Câmara
Cascudo e Gilberto Freyre, estudando-se atualmente a contribuição do
negro à literatura de cordel do Nordeste, ao bumba-meu-boi e aos
desafios de trovadores''.
\end{quote}

\begin{quote}
``A \emph{négritude} explode no carnaval brasileiro, esse espetáculo
único de uma total fusão de raças. Em grande parte essa ausência de
violência e de constrangimento que distingue a integração do negro
brasileiro de todos os países do mundo se deve à ação lúcida de
intelectuais brancos brasileiros. De forma unânime e continua, através
dos séculos, ele vem defendendo o homem de cor, até hoje em dia
chegarmos a Manuel Bandeira, Carlos Drummond de Andrade e outros. A
única coisa que me surpreende é que as associações já existentes
(sobretudo em São Paulo), destinadas ao estudo da contribuição negra e
ao progresso desse segmento da população brasileira, não comemorem datas
importantes como a morte de José do Patrocínio, para citar somente uma,
que no dia 20 de janeiro passou quase inteiramente despercebida no
Brasil''.
\end{quote}

\begin{quote}
``O Brasil inspirou-me um poema, que já iniciei, denominado \emph{Les
Derniers Escales}. Ele refletirá, posso adiantar, a \emph{fisionomia
humana do Brasil}, que não é só uma frase de efeito, ninguém melhor do
que eu para comprovar como ela existe e como se confirma, na realidade,
a visão que deu deste extraordinário país Stefan Zweig na sua obra. Eu
provenho da Guiana Francesa, que tem tido uma contribuição literária que
supera de muito as suas modestas dimensões geográficas e a sua escassa
população; no entanto, e de Caiena um importante poeta jovem, Serge
Passiau, que em \emph{Le Mal du Pays} fala do sentimento de exílio
dentro do seu próprio país de que sofrem tantos intelectuais guianenses.
Esse exílio é fruto de uma alienação cultural, pois os universitários da
Guiana Francesa são forçados a estudar latim e grego e ter uma formação
claramente francesa, quando, na minha opinião, seria muito mais útil e
profícuo se estudassem no Brasil. Aqui, embora as universidades tenham
surgido tardiamente, atingiram em certos casos excelentes níveis,
sobretudo no setor da medicina tropical, dos estudos técnicos e
sociológicos, sem falar na afinidade cultural que existe entre os nossos
países do continente americano, de formação étnica parecida, em parte.''
\end{quote}

Para o poeta, que já foi deputado, há uma estreita vinculação entre
movimento poético e a ação política, citando como exemplos a poesia da
\emph{Résistence} francesa como fator na liberação da França dos
invasores nazistas, a \emph{Négritude} que influiu na elaboração da
Constituição da França de 1946, abrindo as portas do Parlamento aos
diferentes países africanos de língua francesa e ainda a ação de
Langston Hughes, de Richard Wright e James Baldwin nos Estados Unidos,
que complementam as ``marchas da liberdade'' do reverendo Martin Luther
King e unem, assim, \emph{le cœur et l'intelligence}.

\begin{quote}
``É ardilosa e absurda a afirmação dos marxistas que acusam a
\emph{Négritude} de ser um movimento afrancesado, que não reflete os
desejos e a personalidade do negro africano. A \emph{Négritude} não lhes
deve agradar porque é essencialmente democrática e não incita ódios de
classes nem inter-raciais e a prova disso está em que Sartre, que
certamente não pode ser acusado de fascista, saudou a \emph{Négritude}
um dos mais decisivos movimentos culturais do pós-guerra. Digo mais
ainda: a influência desse movimento se espraia pelos países africanos de
língua inglesa, como a Nigéria, cujos principais poetas admitem
publicamente sua inspiração. O que os detratores da \emph{Négritude} não
compreendem é que podemos ser diferentes sendo semelhantes, isto é:
através de nossa expressão poética em língua francesa trazemos uma
aportação inteiramente nova e inédita sea à poesia francesa, seja à
articulação de uma poesia negra, fiel aos seus anseios raciais
ancestrais. Os que nela veem uma forma de racismo negro desconhecem
também o seu conteúdo inteiramente pacífico e sobretudo se esquecem da
dolorosa história do homem de cor que certamente não será racista porque
conheceu o racismo e tendo conhecido o ódio não o abrigará. A
\emph{Négritude} antecipa a era do diálogo universal entre todos os
componentes da espécie humana, inclusive através das trocas culturais e
das confrontações artísticas entre os povos.~

``Quando digo que os negros são a raça que menos se poder acusar de ódio
não vejo uma contradição a essa afirmação na existência dos''muçulmanos
negros'' (\emph{black Muslims}) dos Estados Unidos, que pregam a
destruição violenta da raça branca. Nos EUA a situação do negro é
\emph{sui generis}. Se os Estados Unidos tivessem tido colônias
africanas como a França, a Inglaterra, Portugal, teriam resolvido de
forma diferente o seu problema. Mas lá o que acontece é que o negro
cria, dentro da própria metrópole, não na África distante, uma
concorrência econômica e sexual ao homem branco. Esta concorrência,
creio, é acentuada no Sul dos Estados Unidos (que defende uma situação
que foi alterada com uma Guerra de Secessão) e também entre os
imigrantes europeus. Segundo eu vejo a situação americana, são os
imigrantes de recente integração americana que fazem mais oposição ao
negro, ao passo que aqui no Brasil a lei que limitou severamente a
imigração, no período de Getúlio Vargas, veio acelerar a formação de uma
nacionalidade homogênea e compacta, eliminando o preconceito racial e
permitindo a formação de uma sociedade multi-racial harmônica.''
\end{quote}

Despedimo-nos do poeta artífice do movimento da \emph{Négritude}
enquanto ele se volta para os microfilmes tirados, na Biblioteca
Nacional, dos escritores brasileiros e de seus antepassados. Sua atenção
se detinha justamente na mãe de Castro Alves, na qual ele crê reconhecer
traços de uma mestiça, justificando assim a eloquência veemente do
``Navio Negreiro'', oriunda de uma revolta emocional e intensamente
pessoal. León Damas elabora uma obra sobre a contribuição do negro à
cultura brasileira, focalizando um aspecto ainda não explorado da
cultura que é do nosso País e dos seus ancestrais.

\chapter{Invenções e algemas de Castro
Alves}\label{invenuxe7uxf5es-e-algemas-de-castro-alves}

Veja, 1971/07/14. Aguardando revisão.

\hfill\break

Nem sempre o crítico e o leitor falam do mesmo Castro Alves. Existe a
imagem ``oficial'' do poeta romântico que se inflama na descrição épica
do ``Navio Negreiro''. E uma visão mais rigorosa, que peneira versos
esplêndidos da ganda volumosa de seu Amazonas poético. Os 24 anos de sua
vida intensa foram breves demais para disciplinar sua inspiração
desenfreada. Não lhe foi concedida a dilatação de prazo que pedira na
véspera de sua morte, há cem anos: ``Ai, o Quilombo dos Palmares! Seria
minha obra-prima\ldots{} Dai-me, meu Deus, mais dois anos de vida!''

Atualmente, perderam a validez a crítica excessivamente severa de Jamil
Almansur Haddad (``Castro Alves é o orgulho elevado ao auge'') quanto a
excessivamente entusiasta de Mário de Andrade (``Teve a maior glória, de
discernir, entregando-se a ela, a causa dos escravos''). Que visão do
poeta baiano se tem depois que a poesia brasileira recuperou sua poesia
social (com João Cabral de Melo Neto), sua poesia de inspiração
nativista, principalmente baseada no negro (com Jorge de Lima, e
apresentou um poeta de grandeza universal que expressa uma temática
brasileira complexa e profunda (Carlos Drummond de Andrade)?

Para a sensibilidade moderna, Castro Alves perdeu em renome retórico o
que ganhou em verdadeiro pioneirismo. Lidos cem anos depois, nota-se que
seus poemas intimistas, de aguda percepção da natureza tropical, ampliam
o itinerário esboçado por Casimiro de Abreu. Os sociais mantêm a força
de sua sinceridade e, sobretudo, Castro Alves precede o culto parnasiano
da imagem poética ousada. Mas não podava com devido rigor a linguagem.

A própria denúncia de uma situação de exploração humana aviltante -- a
escravidão - exige dele um estilo heroico, cheio de interjeições e
metáforas eruditas que um poeta consagrado pelo povo não consegue domar.
Cauteloso ao tratar com o maior desafio para a poesia romântica -- o
adjetivo -, Castro Alves cai no lugar-comum inexpressivo (``o rir calmo
da turba'', ``guerreiros ousados'', ``areias infindas''). A poesia de
exortação cívica e moral, próxima demais da oratória, arrebata o ouvido
de uma multidão empolgada pelo entusiasmo do momento. Mas na página
impressa a mediocridade de versos gratuitos susta o fluxo poético com
estrofes meditativas e de rimas fracas e insossas: ``Do Espanhol as
cantilenas/ Requebradas de langor/ Lembram as moças morenas/ As
andaluzas em flor''. A impressão que fica é a de um voo poético
arrastado, preso às algemas da retórica como o albatroz preso à terra.

Fruto de uma época que se empolgava com a palavra altissonante, numa
Bahia ainda acostumada à pirotécnica verbal de um padre Vieira, a
sonoridade esmaga momentos de autêntica identificação entre o poeta e
sua legítima indignação moral. Assim, a coragem e a vibração de versos
como ``E existe um povo que a bandeira empresta/ Para cobrir tanta
infâmia e cobardia!'', ou a musicalidade rítmica de ``Auriverde pendão
da minha terra,/ Que a brisa do Brasil beija e balança'', são anuladas
pelos versos seguintes, desprovidos de força e de seguimento lógico: ``E
deixa-a transformar-se nessa festa/ Em manto impuro de bacante fria!'',
ou ``Estandarte que a luz o sol encerra,/ E as promessas divinas da
esperança..''

Cultuado nas palavras do poeta senegalês Léopold Senghor como ``o poeta
dos continentes'' (A América e a África), Castro Alves, examinado depois
do estudo da cultura da África negra e das reivindicações da poesia da
\emph{Négritude}, revelaria, ao contrário, uma visão estereotipada do
negro como um caçador indolente, puro, bom e tanto quanto os selvagens
de José de Alencar, dotado de virtudes dos heróis clássicos da Grécia
antiga. Sua poesia amorosa -- inspirada por uma bela judia que morava
diante de sua casa em Salvador, por sua paixão pela atriz portuguesa
Eugênia Câmara, por Sinhá Lopes dos Anjos a quem dedicou seus versos num
baile em São Paulo, e pela jovem italiana Agnese Trinci -- também
decepciona pela artificialidade ``literária'' no sentido pejorativo do
termo. Ligadas a uma sensualidade sempre faminta, suas imagens
libidinosas eram ousadas para uma época que considerava pouco pudicos
versos que diziam ``Teu seio é vaga dourada/ Ao tíbio clarão da lua/ Que
ao murmúrio das volúpias/ Arqueja, palpita, nua''. Essa erotização da
paisagem chegaria ao cúmulo de comparar o rio São Francisco a um corcel
fogoso que, ao banhá-las, possui as terras de suas margens.

Mas, se a perspectiva do presente emagrece seu perfil qualitativo,
aumenta a importância de suas inovações. Castro Alves não hesita em
chocar os ouvidos afinados por uma gramática e uma paisagem rigidamente
importadas de Portugal. Utiliza a contração ``pra'' em vez de ``para'' e
integra uma geografia brasileira em versos que falam de ``picadas'' na
floresta, ``gerais'' e ``capinzais'' -- embora se limite a só ver
borboletas azuis forçado pela rima e cego à variedade de colorido de
nossa fauna. Nessa descrição intimista de estados da natureza --
alegria, melancolia, morte, esperança -- que refletem as emoções humanas
diante da paisagem, ele cria pequenas joias de perfeita adequação de
linguagem onomatopaica e frescor inventivo: ``Em músico estalo rangia o
coqueiro'', ou ``a surdina da tarde ao sol, que morre lento'', entre
dezenas de outros achados à procura de um garimpador paciente.

Ao formular sua poesia social -- em parte deslocada no tempo, pois alude
ao tráfico negreiro abolido quase vinte anos antes de seu nascimento --
ele se aproximaria da definição de poesia que Shelley formulou: ``O
poeta é o desconhecido legislador da humanidade''. Seus versos
aceleraram a proclamação da lei de 13 de Maio e apressaram indiretamente
a proclamação da República por sua arrebatadora comunicabilidade
popular. No continente americano, só Whitman teria, em sua época, um
impacto e uma grandeza democrática tão profundos.

\chapter{O mulato Raimundo faz cem
anos}\label{o-mulato-raimundo-faz-cem-anos}

Jornal da Tarde, 1981/5/23. Aguardando revisão.

\hfill\break

Felizmente para a nossa literatura, em geral é péssima a aclimatação no
Brasil dos ``ismos'' europeus para cá importados. O romantismo deu algum
sinal de vitalidade no lirismo chopiniano de um Casimiro de Abreu, na
veemência arrebatada e na sincera eloquência dos versos abolicionistas
de Castro Alves. Mas depois desse primeiro tímido transplante, houve uma
série de rejeições do corpo orgânico brasileiro desses enxertos vindos
de lá com grande atraso. O simbolismo, que teve como um de seus máximos
expoentes o catarinense Cruz e Souza, evaporou-se nos sonhos racistas e
empolados do vate negro que sonhava com loiras celestes e usava termos
alambicados para cantar sua Beatriz alvíssima a que não podia ter
acesso, preso ao inferno do preconceito contra qualquer epiderme escura.
O parnasianismo apresentou até elementos humorísticos inesperados quando
cultuados por nossos imitadores da frieza escultórica, marmoreamente
impassível de um gélido Lecomte Lisle ou de um estático Heredia:
excelente cultor da língua, Olavo Bilac deu ao impávido parnasianismo
uma vibração erótica tropical, não prevista no roteiro que nos foi
enviado da França.

Depois de ``ismos'' importados a granel pelos inventores da Semana (que
durou três noites) de Arte de Moderna de 1922 no Teatro Municipal de São
Paulo revelaram-se já devorados por teias de aranha -- cubismo,
expressionismo -- já quando desembalados dos caixotes \emph{made in
Europe} ou ao espalharem pelo ar o vírus fascista do canto ao futurismo
de um Marinetti.

Um destino semelhante estaria reservado ao naturalismo, mandado vir de
cambulhada com um positivismo mal digerido de Auguste Comte e os
romances hoje ilegíveis de Émile Zola. Felizmente, sempre o Brasil soube
desviar-se, mas suas obras perenes, dos tentáculos dos ``ismos'' e
quando não o fez os resultados foram -- quase sem exceções --
desastrosos. É o caso de \emph{O Mulato}, romance inseguro, dramalhão
mexicano vazado em vernáculo\ldots{} do maranhense Aluísio de Azevedo.
As efemérides obrigam a comentar esse romance apenas porque neste ano se
completam 100 anos da sua publicação, ou seja, em 1881. Melhor teria
sido esperar a comemoração do centenário de \emph{O Cortiço} (1890)
obviamente a obra mais madura e mais pujante deste polêmico e corajoso
autor do Norte do Brasil. Sem dúvida, Aluísio de Azevedo teve que
reconhecer -- fato que o acabrunhou para o resto da vida -- que o Brasil
já desde então não permitia a autor algum viver apenas, como se dizia,
``da sua pena''. Caricaturista mordaz, jornalista de afogadilho, autor
de folhetins mal cosidos e instantaneamente perecíveis como qualidade de
escritura ou feitura literária, Aluísio de Azevedo, irmão do divertido e
satírico comediógrafo Arthur de Azevedo, fez concurso para ingressar na
carreira diplomática e abandonou a literatura aos 37 anos de idade,
desgostoso com a lentidão de um país no qual um milhar de exemplares
levava anos para se esgotar. Foi uma decisão lamentável, peremptória,
que privou o Brasil de um dos seus mais vivazes talentos. Há um certo
exagero na aferição bonachona de José Lins do Rego ao atribuir a Aluísio
de Azevedo a entrada no palco da literatura brasileira do povo, da raia
miúda. Afinal, Taunay já captara de maneira palpavelmente vívida o povo
na sua prosa dinâmica e de grande beleza estilística e José de Alencar,
Euclides da Cunha, Lima Barreto e Adolfo Caminha distanciam-se poucos
anos antes ou depois do aparecimento de \emph{O Mulato}, colocando
vários setores do povo em cena. E Castro Alves, ao denunciar a
escravidão, estaria falando quiçá da aristocracia social do Brasil da
sua época?

Aluísio de Azevedo não se revela com todo o vigor do seu talento em
\emph{O Mulato}. \emph{O Mulato} é uma obra absolutamente imatura, de
estreia ``nas letras'' de um rapaz de apenas 20 anos, que comete vários
erros de português e espraia por páginas inteiras descrições da natureza
de São Luís do Maranhão, sua cidade natal, e diálogos hoje totalmente
inverossímeis, reminiscentes da radiofonização de \emph{Direito de
Nascer} de tão pateticamente ridículos.

O romance em torno do qual se solta um foguetório meramente ditado pela
cronologia existe, felizmente, numa edição que constitui uma absoluta
raridade entre nós, publicada pela Editora Ática, de São Paulo, é
vendida ao preço irrisório de cem cruzeiros, com abundância de notas
explicativas, uma boa capa e grande cuidado de preservação do texto
original (a segunda versão, isto é, a que dela fez o autor). Outra
vantagem que o leitor adquire quando prefere esta edição se refere à
objetividade didática das notas informativas e do Suplemento de
Trabalho, que facultam ao aluno captar bem a essência da obra e as
intenções de Aluísio de Azevedo. Que alívio para os que adquirem, por
exemplo, a monografia dedicada a Dalton Trevisan, publicada pela Editora
Abril, na qual os, digamos, comentadores, ao invés de se cingirem a
explicitar o texto para o aluno principiante, não: querem à força
transformar o magnífico contista paranaense num ``alienado'' e incutir
ao leitor a noção de que só a Libertadora das Gentes, a Revolução
Panacéia, é que trará autores ``engajados'' com a realidade brasileira,
provavelmente os aprovados por um futuro e onipotente Sindicato de
Escritores de modelo soviético\ldots{}

Mas se esta edição que respeita a inteligência do leitor, sem querer
catequizá-lo nem deturpá-lo, é digna de elogios, já o livro em si é um
amontoado de arrematadas tolices que não sensibilizam mais os leitores
que no Brasil já leram um J. J. Veiga, um João Antônio, um Alcântara
Machado, um Marcos Rey, um Rubem da Fonseca. Porque \emph{O Mulato} não
é um romance, no sentido estrito do termo: é um panfleto inchado de
palavrório artificialmente teatral, com episódios grotescos de tão
implausíveis e que, portanto, não existe -- a não ser em raros trechos
-- como obra literária de valor permanente.

Tinha plena razão o excelente romancista peruano, Mario Vargas-Llosa, ao
afirmar, em seu esplêndido ensaio sobre Flaubert, que o naturalismo,
como escola literária, se revelara um fracasso irrecuperável.
Infelizmente, \emph{O Mulato} está atado a essa escola iniciada por
Émile Zola na França, e a uma crença rígida do autor maranhense no
positivismo de Auguste Comte e sua implícita ``interpretação definitiva
e imutável, tanto quanto inquestionável'' da vida humana como o produto
apenas de leis científicas que o homem descobrira, colocando a sociedade
sob a lupa do Saber e explicitando TODOS OS FENÔMENOS culturais,
sociais, políticos etc. sob esse prisma único. Aluísio de Azevedo pagou
um tributo altíssimo, à custa do seu próprio talento: \emph{O Mulato} é
ilegível como ``romance'' hoje em dia. As ideias, as doutrinas raramente
são boas fontes de inspiração para romances inesquecíveis. De fato, a
fórmula panfletária produz quase sem exceção textos indigestos,
indeglutíveis, enfadonhos, risíveis ao extremo.

Que julgue o leitor por si próprio:

Nesta história rocambolesca em que Ana Rosa, branca é obrigada a não se
casar com Raimundo, um mulato ¾ branco, formado na Europa, culto,
refinado e dotado de todas as virtudes heroicas dos mitos da Grécia
Antiga, há dois temas fundamentais e, um secundário. Os temas forçados,
que obcecam o autor e maçam o leitor indescritivelmente são: o terrível
racismo que, segundo o escritor, paralisava a vida social no seu
Maranhão natal, mantendo as etnias em castas estanques: os brancos no
topo da pirâmide, os mulatos desprezados e os negros enxovalhados como
se fossem um lixo ou dejeto humano. Segundo tema: o anticlericalismo
ferrenho do autor, desgostoso com a lassidão e a hipocrisia de amplos
setores do clero maranhense -- logo no Maranhão, onde um dos
insuperáveis Mestres do idioma e do pensamento em português, o Padre
Vieira, por tantas vezes pronunciou sermões que são parte inalienável do
melhor que a língua portuguesa já formulou!\ldots{} O terceiro tema é a
descrição muito adocicada da paisagem tropical brasileira, com a
evocação dos seus traços populares e trovas, modinhas, cenas de as vezes
grande vivacidade plástica, traindo as origens do autor, que no
princípio queria seguir a carreira de pintor e só tarde veio a descobrir
a sua vocação literária. Uma vocação literária que salta à vista na sua
obra-prima, \emph{O Cortiço}, mas que aqui descamba em retratos
românticos -- nada afeitos à pretensa linha naturalista do autor -- do
ambiente topográfico de São Luís do Maranhão.

Com essas três pilastras o que Aluísio de Azevedo obtém?

Um teatro de fantoches falantes -- pois o que mais poderá ser este
\emph{O Mulato} de retas e nobilíssimas intenções e péssima execução?

Não há personagens críveis: há caricaturas de traços rápidos.

O cônego Diogo é, maniqueistamente, o arquétipo do Mal, da Mentira, da
Sordidez, da Cupidez, até o autor intelectual do crime que tira a vida
do incômodo mulato que dá título ao livro. Sempre, indefectivelmente, o
sacerdote é apresentado usando frases latinas, dessas que se tiram de
almanaques de citações latinas e algibeira, não sendo obviamente
compreendido pelos seus interlocutores, pasmados com tanta e tão
impenetrável e omnipresente erudição. Quando, por exemplo, o pai de Ana
Rosa, um português bonachão que não dá a mão da filha ao \emph{colored}
premido pelo preconceito que o circunda, elogia a beleza da filha, o
cônego incontinenti abre sua caixa de ditos latinos aptos para qualquer
momento e qualquer circunstância da vida:

\begin{quote}
``Manoel (o pai de Ana Rosa) bateu no ombro do cônego.~

\begin{itemize}
\item
  Meto-lhe inveja, heim, compadre?\ldots{} Olhe como o diacho da pequena
  está faceira, não é?~
\item
  \emph{Ne insultes miseris}!''
\end{itemize}
\end{quote}

O que, misericordiosamente, o editor explica em útil nota de pé de
página quer dizer: Não insultes os miseráveis!

Trata-se do futuro casamento de Ana Rosa? O pérfido cônego não tarde em
admoestar misteriosamente: ``\emph{Cui fidas vide!}'' (Vê em quem
confias!). O mulato que a tantos incomoda com a sua presença (exceto sua
amada-amante inacessível diante do altar, Ana Rosa), quando pede
explicações sobre uma sua propriedade ao barroquíssimo padre, este
responde com trejeitos: ``\emph{Horresco referens!}'' (Horrorizo-me ao
contá-lo). E assim por diante.

Além dessa estereotipação que é má literatura e não convence, Aluísio
Azevedo tem cenas e diálogos de um ridículo involuntário e impagável,
que superam as maiores expectativas dos que cultuam o \emph{kitsch}.
Quando Ana Rosa vê passar diante de sua janela o caixão com o cadáver do
seu amor, o mulato a quem se entregou fisicamente e que a engravidou,
reage da seguinte maneira:

\begin{quote}
``Era com efeito, ele.~

O povo olhou todo para cima e viu uma coisa horrível, Ana Rosa, convulsa
doida, firmando no patamar da janela as mãos, como duas garras,
entranhava as unhas na madeira do balcão, com os olhos a rolarem
sinistramente e com um riso medonho a escancarar-lhe a boca, as ventas
dilatadas, os membros hirtos.''
\end{quote}

As incongruências se sucedem quase que de página em página. É totalmente
inacreditável, por exemplo, que Ana Rosa fique sozinha em casa e consiga
praticamente violentar Raimundo, seu amado, que antes de sua pretensa
partida para o Rio de Janeiro, a vapor, viera despedir-se\ldots do pai
dela, depois te ter enviado à filha uma carta de amor e renúncia
estoica. É um folhetim da pior espécie com lances em que Ana Rosa se
arroja aos pés do amado e este, que na frase anterior a tratara de
``minha senhora'', agora exclama este primor de romance de capa e
espada:

\begin{quote}
``Antes assim (nota da redação: Ana Rosa julgara-o um impostor), juro-te
que o desejava! Mas supõe que eu seria capaz porventura de sacrificar-te
ao meu amor? Que eu seria capaz de condenar-te ao ódio de teu pai, ao
desprezo dos teus amigos e aos comentários ridículos desta província
estúpida? Não, deixe-me ir, Ana Rosa! É muito melhor que eu vá!\ldots E
tu, minha estrela querida, fica, fica tranquila ao lado de tua família;
segue o teu caminho honesto; és virtuosa, serás a casta mulher de um
branco que te mereça!\ldots Não penses mais em mim. Adeus.''
\end{quote}

O que se atribui de naturalismo a Aluísio de Azevedo se reduz a
descrições cruas de vômitos podres, de mendigos macilentos de fome e é
na parte coral, coletiva, do seu livro seguinte, \emph{O Cortiço}, que
deparamos realmente com um romancista plenamente apto a descrever,
comovedora e contagiantemente um ambiente social que equivale ao
inferno. Ao contrário dos naturalistas franceses, porém, ele é pudico.
Até a cena em que Ana Rosa força Raimundo a compartilhar do seu
``delírio carnal'' é toda cheia de reticências, sem nada que pudesse
fazer corar um leitor de 100 anos atrás, fora as beatas ignorantes de
que ele escarnece justamente em seu livro-libelo.

Aluísio de Azevedo mudou o final do seu romance-dramalhão. Na primeira
versão, era Ana Rosa que morria, nesta segunda e Raimundo que é abatido
a tiros, pelo caixeiro de seu tio, o Dias, que no final vem a casar-se
justamente com\ldots Ana Rosa. É óbvio que quem arquiteta, aos olhos da
sociedade e de Deus, o crime, armando o criminoso Dias, o assassino
anônimo do mulato Raimundo, é o sacerdote. Que distância astronômica
entre este padre untuoso e o Tartufo de Molière\ldots{} e quantos
anos-luz separarão esta Ana da Ana Karenina de Tolstoi? São dados
incomparáveis: Molière tinha uma sutileza e um domínio da psicologia
humana e do vocabulário a que o autor maranhense não tem acesso nem
remotamente e Tolstoi, felizmente, não dogmatizou sua candente crítica
social de \emph{Ana Karenina} pautando-a pelo rígido catecismo de
Auguste Comte e seu missal positivista.

O leitor que achar que este centenário é apenas uma coincidência de
datas tem todos os motivos para buscar refúgio e refrigério rápidos para
a maçada quase inútil de ter lido \emph{O Mulato} correndo a retirar da
estante \emph{O Cortiço}. Ali, sim, achará cenas excelentes, mas como a
própria literatura alemã, além da francesa, comprova de sobejo, o
naturalismo nasceu morto: quem hoje, na Alemanha, lê espontaneamente e
com algum proveito as peças naturalistas de Gerhart Hauptmann? Que
grande renascença tivera jamais, depois de publicados, os romances de
Zola como \emph{L'Assomoir} e \emph{Germinal}?

Deixemos de lado o restante da criação literária de Aluísio de Azevedo
(com exceção de \emph{O Coruja} e \emph{Casa de Pensão}), pois são
escritos apressados, em forma de folhetim, enviados celeremente para os
jornais e consumidos por um público nada exigente em matéria de
literatura. A não ser que se queira ler essa série de textos com veia
cômico-satírica: aí, sim, Aluísio de Azevedo consegue até superar o
talento cômico de seu célebre irmão comediógrafo Arthur de Azevedo.

Consul do Brasil no Japão, na Itália, na Argentina, Aluísio de Azevedo
separou-se da literatura definitivamente. É lastimável que ele não tenha
seguido o exemplo de Eça de Queiroz -- cuja influência ele assimilou tão
mal, ao basear \emph{O Mulato} em \emph{O Crime do Padre Amaro} e sua
vitriólica veia anticlerical -- e não tenha aproveitado os lazeres da
carreira diplomática do início deste século para escrever, já que tinha
o sustento assegurado. Certamente seu estilo se aperfeiçoaria e essa
obra tão gritantemente imperfeita, da sua juventude literariamente
imberbe, \emph{O Mulato}, possivelmente teria sido refundida ou quem
sabe até esquecida? Não se perderia muita coisa.

\emph{O Mulato} realmente só é importante pelo desassombro com que
descreve as múltiplas tiranias odiosas que persistem, atenuadas, até
hoje, no Brasil: a inferiorização da mulher, que não tem escolha a não
ser o casamento com alguém escolhido pelos pais, a mulher como um
elemento ornamental, de todo parasitário na sociedade brasileira; o
monstruoso preconceito racial que impedia a miscigenação no nível das
classes médias da então Província do Maranhão; a ignorância geral que
fazia do Brasil um Império de analfabetos, a inexistência de eficazes
ministérios da saúde, da educação; o desamparo do proletariado e dos
habitantes das zonas rurais; o obscurantismo de parte da Igreja\ldots{}
Aluísio de Azevedo é um defensor de teses fantasiadas de romance, pelo
menos neste duplamente infeliz \emph{O Mulato}. Mesmo quando cintilam
faíscas do seu real talento literário ao delinear a hipocrisia de alguns
personagens marginais, o escritor se sobressai, porém, pelo tom didático
do livro, que prega as coisas certas com um tom de sermão enfadonho. E
afinal a República com todas as suas mazelas, corrupções e mordomias,
não provou ser a panaceia que ele dela esperava, idealista impertérrito:
``Mas, como quer que o povo seja instruído num país cuja riqueza se
baseia na escravidão e com um sistema de governo que tira a sua vida
justamente da ignorância das massas?\ldots{} Por tal forma, nunca
sairemos deste círculo vicioso! Não haverá república enquanto o povo for
ignorante; ora, enquanto o governo for monárquico, conservará, por
conveniência própria, a ignorância do povo; logo\ldots{} nunca haverá
república!''

O saldo é óbvio:

Tardiamente aboliu-se a escravidão: proclamou-se a República: mas a
ignorância das massas, comprova-se hoje, não é privilégio da inépcia
monárquica. Este é o defeito dos romances-teoremas: passados os tempos,
as teses ruíram e nada resta do seu arcabouço. Pois só arroubos
bairristas maranhenses poderiam dar a um livro cheio de defeitos
notórios a classificação de ``obra-prima'': esse raciocínio só é válido
para os que não admitem nenhuma jocosidade ao descrever São Luís do
Maranhão como ``a Atenas brasileira''\ldots{} Seria, contudo, igualmente
infamante e impensável aceitar a crítica rasteira e sem base que um
setor da Igreja lhe moveu na época, através de seu órgão de imprensa,
\emph{Civilização}. Mais interessante do que o romance é a batalha
verbal que se trava entre um padre, por coincidência chamado Raimundo
(como \emph{O Mulato} que é o personagem central do livro) Alves da
Fonseca, a atacar com as invectivas mais baixas o autor em artigos
mentirosos e pusilânimes e, do outro lado, Aluísio de Azevedo, no jornal
\emph{Pensamento} a responder e a querer substituir a crença na religião
pela adesão ao pensar, hábito não aclimatado no Brasil, como se vê, até
os dias que correm.

É um centenário um tanto embaraçoso dos 100 anos da publicação de
\emph{O Mulato}, pois muito pouco há para se celebrar e literariamente
há mais para se calar. Zola dizia, ele mesmo fazendo o
diagnóstico-epitáfio de seus romances naturalistas, que ao escrever
queria proceder a ``uma autópsia cirúrgica no corpo da sociedade''. Como
\emph{O Mulato} infelizmente comprova, para se fazer uma autópsia é
preciso primeiro que exista um cadáver e este é de belíssimas feições
idealistas mas corroído por uma enfermidade insanável: a verborragia a
serviço da pregação de ideias, doença que a História da Literatura
demonstra ser fatal para o autor, o indefeso leitor e as estropiadas
ideias atingidas pelo cancro crescente de uma retórica oca e balofa.

\chapter{Resenha de LGR vetada e não publicada no JT sobre Adolfo
Caminha}\label{resenha-de-lgr-vetada-e-nuxe3o-publicada-no-jt-sobre-adolfo-caminha}

In `O Bom Crioulo', 1983. Aguardando revisão.

\hfill\break

\begin{quote}
``Esse acervo de mentiras galantes e torpezas dissimuladas, esse cortiço
de vespas que se denomina -- sociedade'' Adolfo Caminha
\end{quote}

A Editora Ática, que já se vinha distinguindo por revelar, entre nós,
coerente e corajosamente, os autores africanos, alguns dos melhores de
Angola (como José Luandino Vieira), recupera agora aquele que é talvez o
mais injustiçado de todos os romancistas brasileiros: Adolfo Caminha
(1867-1897). É, atualmente, quase impossível encontrar as obras do
escritor cearense que morreu com 29 anos de idade e que, no entanto,
pela sua rebeldia de espírito, pela sua visão lúcida e profunda do ser
humano, de sua luta apaixonada e teimosa contra os tabus de uma
sociedade hipócrita, deixou dois testemunhos incendiários de sua
precocidade. Zola, Flaubert e Eça de Queiroz já tinham abordado o
adultério, a Igreja, o meio provinciano, a exploração das classes
operárias, os bordeis e as fábricas massacrantes da moderna Revolução
Industrial do século passado. Mas haverá em toda a literatura ocidental,
na qual o Brasil se insere, o exemplo de um autor que tenha tão
destemidamente focalizado temas tão polêmicos quanto Adolfo Caminha?

É de se duvidar que sim. Adolfo Caminha esmiuçou desassombradamente o
feminismo em \emph{A Normalista} (publicado em 1893), simultaneamente
com os mitos da virgindade violada por um reles sedutor, o aborto
forçado pelas circunstâncias do meio e o desamparo da mãe solteira, além
de pinceladas de atração sexual de uma branca por machões negros,
idealizados em sua potência sexual como garanhões ao mesmo tempo
assustadores e cobiçados em sonhos -- ou pesadelos de sentimento de
culpa?

Em \emph{O Bom Crioulo} (de 1895!), o autor inconformista que nos é tão
oportunamente restituído pela Editora Ática, envereda, ao mesmo tempo,
por assuntos proibidos, em parte, até hoje, pelo menos na literatura
brasileira: o homossexualismo, o preconceito racial de uma
superestrutura branca contra os negros, o horror do castigos corporais
(chibatadas, prisão, obediência cega às ordens) e a injustiça social que
alija os pobres numa casta quase comparável à dos escravos ou dos
párias, os intocáveis da Índia onde predomina o Hinduísmo.

Adolfo Caminha, infelizmente, nunca teve críticos à sua altura. Os de
sua época, Veríssimo e Romero, no mínimo o ignoraram ou o mencionaram
com o desprezo de uma nota apressada. Em seu livro \emph{O Realismo}
(Editora Cultrix), João Pacheco resume, felizmente para os leitores, sua
apreciação a uma série de palavras ocas:

\begin{quote}
``\ldots{} Se não chegou a desenvolver plenamente os seus dotes --
faleceu aos trinta anos (corrijamos: aos 29) -- tinha todo o estofo de
um romancista, pois sabia manejar cenas e personagens com naturalidade.
Enquanto \emph{O Bom Crioulo} tem \emph{fabulação mais segura} (grifo de
espanto meu) e nele \emph{se recortam mais firmemente os caracteres}
(idem!), \emph{A Normalista} põe em ação um mundo mais amplo, a
desdobrar-se num cenário mais vasto -- por isso menos realizado em
partes, porém mais rico em perspectivas''. É a arte do escrever sem
dizer absolutamente nada\ldots{}
\end{quote}

Também em duas páginas e meia M. Cavalcanti Proença em seus
\emph{Estudos Literários} (Editora José Olympio) ``termina'' com o
autor, frisando que ``em arte não há assuntos proibidos'' e atribui a
Adolfo Caminha o que certamente o autor republicano e abolicionista
recusaria com um sorriso sutil de comiseração: ``Um submundo
erroneamente estruturado determina a convergência de vícios (sic!) e os
homens, afinal, são produto do meio, vítimas do meio muitas vezes!'' É
uma perspectiva crítica que não se sustenta nem mesmo quando se lê ou
relê \emph{O Bom Crioulo} sob a ótica estreita de um pretenso
cientificismo pretensamente naturalista.

Caminha tem, ao contrário, um estilo que começa incerto, influenciado
por um romantismo a extrair da terra um lirismo estuante, vitalizante:
em \emph{A Normalista} seu canto, influenciado por José de Alencar,
celebra a pureza e o vigor da natureza sertaneja. Depois, passa a
magistral em \emph{O Bom Crioulo}. Absolutamente nada no livro confirma
que o autor estivesse, como moralista falso, hipoteticamente desvendando
taras, aberrações, perversões. Vários integrantes da tripulação do navio
-- incluindo oficiais - bordo do qual serve o Bom Crioulo participam,
dissimuladamente, de relações que vem, no Ocidente, do tempo das
esquadras e batalhões de Esparta, na Grécia Antiga. Nem o autor se
propõe, em momento algum, a \emph{julgar} tais relações: a tragédia de
\emph{O Bom Crioulo} está muito mais enraizada na repulsa vivíssima que
Adolfo Caminha sente pelos castigos corporais brutais impostos
caprichosa e arbitrariamente pelos oficiais da Marinha de então e no
choque de temperamentos de Amaro e Aleixo. Enquanto Aleixo, branco,
catarinense, é descrito, no início, como de formas e atitudes que
convencionalmente se chamam de ``femininas'', Amaro, negro ou mulato
escuro, é a sua antítese em tudo: másculo mas sensível, íntegro e
corajoso, passional e, no final, alucinado pela sua paixão.
Evidentemente que o livro todo alude, frequentemente, ao preconceito que
segrega os negros à escala mais baixa ou imóvel dos estereótipos:
``inferior'', ``selvagem'', ``incapaz de controlar seus instintos mais
baixos'', ``indolente'', ``emotivo'', ``irracional'' etc.

\emph{O Bom Crioulo} surpreende não só pela quase inacreditável
precocidade do romance, um romance tão complexo, escrito por um jovem de
menos de 30 anos, no Rio de Janeiro, em 1895. Essa audácia talvez faça
dele, até prova em contrário, o primeiro romancista das Américas e da
Europa eminentemente \emph{moderno} e até mesmo \emph{contemporâneo}:
não é só hoje que Gore Vidal focalizou o homossexualismo em seu ruidoso
\emph{The City and the Pillar}? Não é só hoje que assomaram com livros
candentes e bandeiras do E.R.A. (\emph{Equal Rights Amendment} ou Emenda
(constitucional) por Direitos Iguais na sociedade norte-americana para
mulheres e homens), feministas como Kate Millet, Betty Friedan, as três
Marias portuguesas, Germaine Grier, Rose Marie Muraro? E finalmente não
é de hoje apenas a eclosão do movimento negro, iniciado por Martin
Luther King e retomado pela \emph{Négritude} dos poetas das Antilhas e
da África Negra (Aimé Césaire, Senghor), simultaneamente com os estudos
africanos nas universidades norte-americanas e o \emph{black is
beautiful} lançado pelo \emph{Black Power} nos Estados Unidos?

O autor cearense distingue-se porém pela sua abrangência. Jamais --
cremos -- lhe passou pela cabeça escrever um romance de tese: agora vou
\emph{provar} que na Marinha brasileira há discriminação racial; agora
vou \emph{provar} que nela, em todos os escalões, a prática do
homossexualismo está camuflada mas inequivocamente difundida. Erraria
também redondamente quem quisesse, apressadamente, equiparar esse bom
crioulo a um Otelo em ambiente da Corte Imperial, no Rio de Janeiro
pré-republicano e pré-abolicionista. Estranhamente, Caminha não faz
sermão, não prega isto ou aquilo. Quase sempre com total isenção,
limita-se a descrever situações, a captar perfis psicológicos desenhado
não com rótulos esterelizantes mas, como os de Balzac, exemplos vivos da
riqueza da fauna humana, com suas virtudes e abjeções. Mesmo quando usa
a nomenclatura convencional da época quanto ao negro ou quanto ao
``vício nefando'', nota-se que ele põe em dúvida tais classificações
\emph{a priori} e como que se indaga, remetendo a pergunta ao leitor:
Tal premissa está certa eticamente? Corresponde aos fatos?, deve ser
modificada ou aceita? Sem se prender à mentirosa moral de seu tempo,
Adolfo Caminha é muito mais autenticamente ``moralista'' no sentido
exemplar do termo: ele questiona a ``moral'' passageira vigente por
considerá-la um monte de vilezas, e comodismos, de preconceitos, de
injustiças, de covardias. Terá mudado a ``moral''?

Causa impressão extraordinária, ao contrário, a pudicícia sincera com
que meramente \emph{insinua} cenas de outro modo escabrosas como a cena
em que o personagem principal se atormenta com sua inesperada
reviravolta de sentimentos em relação ao rapazinho catarinense. Lembra o
Riobaldo durante quase todo o longo romance de Guimarães Rosa,
\emph{Grande Sertão Veredas}, que quase enlouquecerá por sua obsessão
pelo misterioso Diadorim, sem companheiro de lutas e jagunço, como ele:

\begin{quote}
``E agora, como é que não tinha forças para resistir aos impulsos do
sangue? Como é que se compreendia o amor, o desejo da possa animal entre
duas pessoas do mesmo sexo, entre dois homens? Tudo isto fazia-lhe
confusão no espírito, baralhando ideias, repugnando os sentidos,
revivendo escrúpulos. -- É certo que ele não seria o primeiro a dar o
exemplo, caso o pequeno se resolvesse a consentir\ldots{} -- Mas --
instinto ou falta de hábito -- alguma cousa dentro de si revoltava-se
contra semelhante imoralidade que outros de categoria superior
praticavam quase todas as noites ali mesmo sobre o convés..''
\end{quote}

Adolfo Caminha trata com sutileza e bom gosto qualquer cena que em
outras mãos seria luxuriosa, doentia, chocante. Assim, quando Aleixo
concorda com o primeiro encontro com Amaro, o bom crioulo; é uma cena
que hoje em dia talvez uma ginasiana pudesse ler sem enrubescer, uma
descrição tão sóbria, pudica mesmo:

\begin{quote}
``Depois de um silêncio cauteloso e rápido, Bom-Crioulo, conchegando-se
ao grumete, disse-lhe qualquer cousa no ouvido. Aleixo conservou-se
imóvel, sem respirar, encolhido, as pálpebras cerrando-se
instintivamente de sono, ouvindo, com o ouvido pegado ao convés, o
marulhar das ondas na proa, não tece ânimo de murmurar uma palavra. Viu
passarem, como em sonho, as mil e uma promessas de Bom-Crioulo: o
quartinho da Rua da Misericórdia no Rio de Janeiro, os teatros, os
passeios\ldots{} ; lembrou-se do castigo que o negro sofrera por sua
causa; mas não disse nada. Uma sensação de ventura infinita
espalhava-se-lhe em todo o corpo. Começava a sentir no próprio sangue
impulsos nunca experimentados, uma como vontade ingênita de ceder aos
caprichos do negro, de abandonar-se-lhe para o que ele quisesse -- uma
vaga distensão dos nervos, um prurido de passividade\ldots{}

\begin{itemize}
\tightlist
\item
  Ande logo! Murmurou apressadamente, voltando-se.
\end{itemize}

E consomou-se o delito contra a natureza.''
\end{quote}

Não há nenhum trecho mais \emph{forte} do que este em todo o livro. Há,
isso sim, quadros espantosos da miséria proletária daquele Rio de
Janeiro em que Oswaldo Cruz ainda não debelara a febre-amarela, enquanto
o Imperador Pedro II subia Petrópolis, para fugir do verão carioca:
``Era justamente em dezembro, mês de epidemia e de insuportável calor''.

Dir-se-ia que aqueles homens, operários e marinheiros, não tinham
aparelho respiratório, não tinham pulmões, ou estavam saturados de
miasmas.

Trabalhavam cantando e martelavam assoviando, com uma indiferença
heroica, sem pensar no grande perigo que os ameaçava.

Pela noite, desde o escurecer, o odor pestilento aumentava e então não
havia remédio: a marinhagem toda se precipitava para fora, como um
formigueiro alvoroçado, tapando o nariz: - Foge! foge! Olha a febre
amarela!''

Republicano atrevido, que aproveitou uma festa colegial em homenagem a
Victor Hugo par elogiar a República e a Abolição diante do Imperador
aturdido, Adolfo Caminha não traça apenas um retrato social terrível das
hierarquias sociais do Brasil de fins do século passado. Obtém
impressionantes retratos de indivíduos, como a finória D. Carolina,
rameira, como se dizia naquele tempo, aposentada, mas sequiosa da beleza
e da juventude de Aleixo. A fotografia que tira dela é inesquecível e
reminiscente de leituras de Eça de Queiroz:

\begin{quote}
``Há dias metera-se-lhe na cabeça uma extravagância: conquistar Aleixo,
o bonitinho, tomá-lo para si, tê-lo como amantezinho do seu coração
avelhentado e gasto, amigar-se com ele secretamente, dando-lhe tudo
quanto fosse preciso: roupa, calçados, almoço e jantar nos dias de folga
-- dando-lhe tudo enfim.

Era uma esquisitice como outra qualquer: estava cansada de aturar
marmanjos. Queria agora experimentar um meninote, um criançola, sem
barba, que lhe fizesse todas as vontades. Nenhum melhor que Aleixo, cuja
beleza impressionara-a desde a primeira vez que se tinham visto. Aleixo
estava mesmo a calhar: bonito, forte, virgem talvez\ldots{}

Arranjava-se perfeitamente, sem que Bom-Crioulo soubesse. Mas como falar
ao grumete, como propor-lhe negócio? Ele talvez ficasse ofendido, e
podia haver escândalo\ldots{}

\ldots{} Viu-se ao espelho e notou que realmente ``ainda prestava
serviço'': - Qual velha! Nem um pé-de-galinha sequer, nem uma ruga --
pois isso era ser velha? Certo que não. Lá quanto à idade, ninguém
queria saber. A questão era de cara e corpo\ldots{} Ora, adeus!\ldots{}

Começou a fazer-se muito meiga para o rapazinho, guardando-lhe doces,
guloseimas, passando a ferro, ela própria, seus lenços, gabando-os na
presença de estranho, fingindo-se distraída, quando queria mostrar-lhe a
exuberância de suas carnes -- perna, braço ou seios\ldots{} Uma ocasião
Aleixo vira-a em camisa curta, deitada, com as pernas de fora; porque os
aposentos da portuguesa davam para o corredor e, nesse dia, ela
esquecera de fechar a porta. O grumete voltou o rosto depressa, todo
cheio de respeito, com se aquilo fosse uma profanação; mas, depois, ao
lembrar-se do caso, tinha sempre uns arrepios voluptuosos, não podia
evitar certa quebreira, certo desfalecimento acompanhado de ereção
nervosa.

Nunca mais lhe saíra da lembrança aquela cena de alcova: uma mulher
deitada com as pernas à mostra, muito gordas e penugentas -- num
desalinho irresistível, braços nus, cabelo solto -, devia de ser
esplêndido a gente dormir nos braços de uma mulher: A portuguesa até que
não era mazinha\ldots{}

Aleixo, porém, estava longe de supor que D. Carolina, aquela D.
Carolina, que o tratava como filho, bondosa e meiga, pretendesse fazê-lo
seu amante,''
\end{quote}

Se a obra de Adolfo Caminha já era coerente e audaciosamente anterior de
quase um século ao espírito de nosso tempo, também sua vida se regeu
sempre por uma insubordinação inata: apaixonando-se, na provinciana
Fortaleza de então, por uma moça noiva, ``raptou-a'' ao noivo e com ela
teve dois filhos. (Segundo outros autores, sua esposa, Isabel Jataí de
Paula Barros, já era casada quando ela e o escritor se encontraram).
Afastou-se (segundo outros ``foi afastado'') da Marinha e tornou-se
amanuense da Tesouraria da Fazenda. Ruem por terra, portanto, quaisquer
críticas que se lhe queiram fazer, acusando-o de ``combater em causa
própria'' ao enfocar o homossexualismo e a discriminação racial: branco
e heterossexual, Adolfo Caminha rebelou-se também contra a Sociedade
Literária que ajudara a fundar no Ceará, aquela curiosa ``Padaria
Espiritual'' que deveria trazer novo alento à modorrenta capital, com
suas reuniões denominadas ``fornos'' e seu jornal batizado de \emph{O
Pão}.

Morreu, muito jovem, de uma doença característica do movimento
romântico: a tísica, como era chamada a tuberculose. Sem dúvida, trechos
de seus romances -- principalmente \emph{A Normalista} -- têm trechos
inteiros de um romantismo nativista que Alencar assinaria sem hesitar.
Nem a sua impassível atitude de ``não julgar'' tinha nada do
exibicionismo pseudo-clínico e farsesco de Júlio Ribeiro e sua hoje
natimorta \emph{A Carne}. Adolfo Caminha -- sobre quem é urgente que se
faça um alentado estudo crítico, impossível no espaço de um breve artigo
de jornal -, porém, não pôde coitado!, escapar aos críticos que lhe são
póstumos.

Em futuras edições dessa pequena obra-prima, a Editora Ática deveria
proceder, por assepsia, à eliminação do prefácio (de Saira Youssef
Campedelli, da Faculdade Ibero-Americana) e sobretudo de suas
hilariantes notas de pé de página. Ela rotula, por exemplo, a expressão
italiana \emph{dolce far niente} de italianismo. Mais adiante, talvez
levada pelo pudor, consigna o termo ``calipígio'' como equivalente de a
``eu tem belas formas'', quando evidentemente esse adjetivo significa
``que tem formosas nádegas'', do grego \emph{kalos}, belo e \emph{puge},
nádegas. Quem sabe, se a prefaciadora tivesse redigido tais notas depois
de ter visto o excesso de nádegas e outras particularidades da anatomia
humana esmiuçadas como numa aula de vivissecção pelos \emph{cameramen}
que insistiram apenas nessas imagens, obsessivamente, durante a filmagem
calipigiamente fanática do nosso Carnaval na televisão, quem sabe ela
corrigiria sua etimologia? Da mesma forma erra nossa professora
ibero-americana ao traduzir \emph{clownerie} como ``termo francês usado
para significar acrobacia de palhaço''. \emph{Clownerie} quer dizer
facécia, gracejo, piada de palhaço, palhaçada\ldots{} \emph{O
Bom-Crioulo}, apesar da incompetência dos trechos pseudo-didáticos desta
prefaciadora, conserva, no seu texto original aqui reproduzido
integralmente pela Editora, todo o viço dos clássicos, tão contundente e
polêmico quando foi publicado, 88 anos atrás.

As questões complexíssimas que \emph{O Bom-Crioulo} suscita fazem deste
livro uma ruptura, na literatura brasileira, entre o antigo e o moderno.
Caminha não faz preleções, não ``moraliza'' no sentido estreito que as
digníssimas e zelosas Senhoras da Lapa dão a esse termo. Quer dizer: não
permanece indiferente ao tradicionalismo das ideias feitas e nunca
questionadas. O autor cearense está fundamentalmente empenhado com a
Ética, cético à ``moral'' imposta segundo os ditames variáveis de uma
época ou de um período histórico e social. À saia baião não se sucederam
o biquini, o \emph{topless} e a pílula anticoncepcional? Ao processo e
encarceramento de Oscar Wilde não sobrevieram a adoção de leis liberais
e a conscientização do \emph{Gay Movement} em vários países? A Lei
Afonso Arinos não penaliza, pelo menos juridicamente, a discriminação
racial?

É singular, por último, como mero fecho destas anotações sucintas sobre
uma de nossas obras-primas, destacar: o autor \emph{não se exime} de
formular uma pergunta muito mais importante do que as indagações sobre a
libido, a sexualidade permitida ou não e até mesmo capaz de ultrapassar
a injustiça que circunda a crueldade e a estupidez do racismo. Qual deve
ser o procedimento ético, humano, de um ser humano para com o outro?

\chapter{A revolta de um precursor}\label{a-revolta-de-um-precursor}

Veja, n.41, 1969/06. Aguardando revisão.

\hfill\break

No dia 13 de maio de 1888, o mulatinho Afonso Henriques de Lima Barreto
comemora sete anos de idade na Rua do Riachuelo, no Rio de Janeiro, em
meio a foguetes e danças dos negros libertos, que lhe dão a primeira
noção confusa do preconceito racial e da escravidão que se abolia
naquela data. Em 1922, o romancista social de \emph{O Triste Fim de
Policarpo Quaresma}, o jornalista cáustico de \emph{Bagatelas}, morria
-- no mesmo ano da Semana de Arte Moderna em São Paulo, cujas reformas
mais importantes ele antecipara de decênios. Esta semana, a Editora
Brasiliense comemora os sessenta anos do seu primeiro romance lançado
pela primeira vez em Portugal, \emph{Recordações do Escrivão Isaías
Caminha} (numa época em que Machado de Assis e Euclides da Cunha não
tinham editores brasileiros para suas obras).

Os dezessete volumes de Lima Barreto que a Brasiliense agora reedita
permitem definir a modernidade quase profética do autor que via no
romancista ``alguma coisa do descobridor''. Contemporâneo de Machado de
Assis, por quem sentia antipatia, Lima Barreto precederia a tendência
abrasileirante do movimento modernista: ``Eu tenho notado nas rodas que
hei frequentado\ldots{} uma nefasta influência dos portugueses.
Ajeita-se o modo de escrever deles, copiam-se-lhes os cacoetes, a
estrutura da frase, não há dentre ele um que conscienciosamente procure
escrever como seu meio pede e requer..''

Rebelando-se contra uma literatura meramente sonora, cheia de retórica e
sutilezas verbais inúteis, como a de Coelho Neto, seu contemporâneo.
Lima Barreto é um seguidor do filósofo e crítico de literatura francês
Taine, que vê na literatura o retrato social de um ambiente, de uma
época e de uma raça. A raça seria uma de suas preocupações maiores. ``É
triste não ser branco'', anota Lima Barreto em seu diário, estruturando
desde seu primeiro romance a defesa do negro espoliado que Richard
Wright e James Baldwin fariam explodir mais tarde na literatura moderna
americana: ``Não sei que estranha tenacidade leva (a raça negra) a viver
e por que essa tenacidade é tanto mais forte quanto mais humilde e
miserável''. Como a adesão ao que ele chama de ``negrismo'' seguiria de
perto as tendências atuais da \emph{Négritude} de Léopold Senghor e Aimé
Césaire: ``Pretendo fazer um romance em que se descrevam a vida e o
trabalho dos negros numa fazenda. Será uma espécie de \emph{Germinal}
negro, com mais psicologia especial\ldots{} Essas ideias que me
perseguem de pintar e fazer a vida escrava com os processos modernos do
romance, e o grande amor que me inspira -- puderá! -- a gente
negra\ldots{} Dirão que é o negrismo, que é um bom indianismo\ldots{}
mas e a glória e o imenso serviço à raça a que pertenço?'' Ao lado de
repostas às teorias racistas do Conde de Gobineau (que serviram de base
para o nazismo e para o regime de \emph{apartheid} da África do Sul, ele
alinha as fases da sua batalha solitária contra a estreiteza do meio, o
preconceito e o álcool, que o levaria ao hospício duas vezes. Ao
contrário de Machado de Assis, que via no funcionalismo público uma
forma de ascensão social, Lima Barreto, ao deixar o Liceu Nichteroy (com
os estudos custeados por seu padrinho, o Visconde de Ouro Preto),
fracassa na tentativa de ingressar na Escola Politécnica para ser
engenheiro e consegue um emprego de amanuense obscuro a Secretaria da
Guerra, onde seu talento é hostilizado. ``As repartições'', exclama Lima
Barreto, como um Kafka carioca e mulato, ``são como a vida em geral:
amam os medíocres.''

Fugindo ao desfile das elegantes na Rua do Ouvirdor e a inauguração do
cinema, Lima Barreto estuda filosofia sozinho, em enciclopédias
francesas, anota palavras para formar um vocabulário e é, acima de tudo,
uma personalidade cheia de contrastes. Apaixona-se pelo positivismo --
mas, ao mesmo tempo, paga promessas à sua Madrinha, Nossa Senhora da
Glória; exalta o marxismo e a Revolução Russa -- mas tem o ceticismo de
um Voltaire, que não acredita em reformas de fora sem uma revolução
interior; é anarquista, mas candidata-se quatro vezes à Academia
Brasileira de Letras, sem sucesso. Sensível, polêmico, Lima Barreto
incorporaria o povo brasileiro -- nas suas camadas mais humildes -- à
literatura romanesca do Brasil, precedendo toda a literatura social
nordestina de Graciliano Ramos e José Lins do Rêgo, assim como
anteciparia Guimarães Rosa na pesquisa do folclore brasileiro para
utilização como matéria-prima literária. Além de trágico e
inconformista, Lima Barreto seria um autor satírico mais cáustico e mais
substancial na críticas dos costumes de seu tempo que qualquer outro
autor brasileiro, com suas história dos Bruzundangas, República que
simboliza o país das trapalhadas e da mixórdia, o Reino de Jambon, o
Brasil-Presunto ``que até aqui tem sido muito roído: roem-no os de fora,
roem os de dentro, mas o diabo dessa perna de porco resiste à voracidade
externa e interna de uma maneira perfeitamente milagrosa''.

\chapter{Lima Barreto - altamente inovador. E quase
desconhecido}\label{lima-barreto---altamente-inovador.-e-quase-desconhecido}

Jornal da Tarde, 1981/5/9. Aguardando revisão.

\hfill\break

``Nasci sem dinheiro, mulato e livre''

Lima Barreto

Apesar da excelente biografia de Lima Barreto que lhe dedicou o
finíssimo crítico paulista Francisco Assis Barbosa (\emph{A Vida de Lima
Barreto}, Livraria José Olympio/MEC Editora, coleção Documentos
Brasileiros), o destino trágico e a literatura profundamente inovadora
do escritor carioca continuam quase desconhecidos no Brasil.

Ofuscado por Machado de Assis, morto precocemente aos 41 anos de idade,
no ano da Semana de Arte Moderna de São Paulo, 1922, Afondo Henriques de
Lima Barreto desafia desde a sua geração a todos que queiram ter uma
noção melhor do avanço da literatura brasileira, sem os ``ismos''
importados tardiamente para cá pelos revolucionários pândegos de 22 e
sem a idolatria cega votada à Machado de Assis. Mulato como o autor de
\emph{Quincas Borba}, Lima Barreto não cultivou o pessimismo do seu
contemporâneo nem se afastou da luta pelas liberdades civis e conquistas
do operariado moderno. Sua vida, porém, nada tinha que o levasse a
acreditar nos seres humanos, na sua bondade, na sua coragem. Talvez as
artes no Brasil, desde Aleijadinho, não encerrem uma existência tão
monstruosamente trágica a par de uma criação filosófica e estética
superior e perpétua.

Lima Barreto foi vítima de inúmeros torpes inimigos simultâneos: o
hipócrita, dissimulado preconceito racial que disfarça nossas castas
étnicas sob o manto mentiroso de uma autêntica ``democracia racial''; a
pobreza aviltante da ``carreira'' de funcionalismo público a que esteve
sempre precariamente atrelado; a mediocridade insuperável da inércia
intelectual e cultural brasileira que tolhe, até em nossos dias,
qualquer iniciativa tendente a alterar o \emph{status quo} de
incultíssima sonolência.

Como Monteiro Lobato, com quem manteve uma correspondência vivaz, Lima
Barreto não se amoldou à pasmaceira \emph{de rigueur} no campo do
pensamento, da erudição, da pesquisa intelectual, das realizações
artísticas, da audácia criativa cultural. Alcoólatra prematuro dos
trinta anos de idade em diante, com o pai louco a seu cargo, um posto
medíocre de amanuense como profissão, irrealizado no plano sentimental,
tímido com as mulheres e saindo enojado de bordeis, Lima Barreto
esboçou, porém, uma das mais geniais incursões no aprendizado solitário
de um estilo literário entre nós. \emph{O Triste Fim de Policarpo
Quaresma} -- possivelmente sua obra-prima - , \emph{Recordações do
Escrivão Isaías Caminha} e \emph{Vida e Morte de M. J. Gonzaga de Sá}
trazem a marca do seu inconformismo múltiplo: contra o isolamento que o
condenava o preconceito de cor, contra a miséria do deserto intelectual
brasileira, contra os acadêmicos, ditadores, uma Igreja ladina e
pusilânime, mestra em acrobacias de uma flexibilidade inacreditável da
qual abusa sempre que passa de um ``partido'' político a outro, em
defesa sempre de seus interesses: se eles penderem para a salvaguarda
dos ``humildes e ofendidos'', lépida, ela passará a ser seu ``anjo
protetor'', com a mesma agilidade com que antes se aliava
peremptoriamente aos poderosos de cada um dos momentos da História.

Com tantos inimigos ao mesmo tempo e sem contemporizar com a boçalidade
das suas manifestações -- da imprensa estúpida e venal aos aduladores
palacianos do momento -, não estranha que Lima Barreto tenha sucumbido,
só e incompreendido até o fim. No entanto, nunca a literatura brasileira
uniria espíritos tão díspares e singulares num só romancista: o autor de
\emph{Policarpo Quaresma} tem de Gogol a noção arraigadamente trágica da
existência; de Sterne a ironia alegre; de Dickens o traço vigoroso que
desenha uma caricatura humana sucinta e exemplar. A essas afinidades ele
acrescentou uma brejeirice e uma doçura brasileiras que estabelecem o
mais inesperado contraste com a sua vida e seus dissabores quase nunca
mitigados por momentos de calma. Que calma poderia haver para quem
tivera com única herança o apego à cultura e como encargo pesado o
cuidado do pai, alienado mental mantido em casa? O único consolo era o
da Arte, principalmente da Literatura que lhe permitia ombrear-se com os
brancos ricos e ultrapassar o marasmo brasílico da cretinice endêmica
neste país, e sem sintomas de melhora até hoje. Gentilíssimo de trato,
ensimesmado, avesso a intimidades e obscenidades, Lima Barreto deixa um
retrato verídico e sulfúrico dos balangandãs pseudocultos do brasileiro
médio, que do alto de seus anéis de ``doutores'' com rubis falsos ria
daquele passageiro mulato, mal vestido, solitário, que viajava no trem
da Central do Brasil, rumo à repartição bocejante ou de volta à casa,
filial do hospício:

\begin{quote}
``A presunção, o pedantismo, a arrogância e o desdém com que olhavam as
minhas roupas desfiadas e verdoengas sacudiam-me os nervos e davam-me
ânimos à revolta.
\end{quote}

O brasileiro é vaidoso e guloso de títulos ocos e honrarias chochas. O
seu ideal é ter distinções de anéis, de veneras, de condecorações, andar
cheio de dourados, com o peito \emph{chamarré d'or}, seja da Guarda
Civil ou da atual segunda linha. Observem. Quanto mais modesta for a
categoria do empregado -- no subúrbio pelo menos - mais enfatuado ele se
mostra. Um velho contínuo tem-se na conta de grande e imensa coisa, só
pelo fato de ser funcionário do Estado, para carregar papeis de um lado
para outro; e um simples terceiro oficial, que a isso chegou, por
trapaças de transferências e artigos capciosos nas reformas, partindo de
`servente adido à escrita', limpa que nem um diretor notável, quando
compra, se o faz, a passagem no guichê da estação. Empurra brutalmente
os outros, olha com desdém os mal vestidos, bate nervosamente com os
níqueis..''

Quase sem amigos, arredio, aquele moço sério, educado, passava o tempo
disponível, depois do trabalho maçante de amanuense de uma Secretaria
governamental, a frequentar a Biblioteca Nacional, naquele Rio de
Janeiro que ainda era uma cidade provinciana, antes das reformas
urbanas, da ação saneadora e heroica de Oswaldo Cruz. E para si próprio
ele redige um ``Curso de filosofia feito por Afonso Henriques de Lima
Barreto, segundo artigos da \emph{Grande Encyclpédie Française du XXème
Siècle}, outros dicionários e livros fáceis de se obter''. A par da
filosofia, que lhe permitiria, fora dos limites da raça, da classe
social, da nacionalidade, interpretar a vida e suas ilusões, descalabros
e raras alegrias, ele se realizava inteiramente apenas na literatura:

``Mais do que qualquer outra atividade espiritual da nossa espécie, a
Arte, especialmente a Literatura, a que me dediquei e com quem me casei,
mais do que ela nenhum outro meio de comunicação entre os homens em
virtude mesmo do seu poder de contágio, teve, tem e terá um grande
destino em nossa triste Humanidade\ldots{} Quer dizer: que o homem, por
intermédio da Arte, não fica adstrito aos preceitos e preconceitos de
seu tempo, de seu nascimento, de sua pátria, de sua raça: ele vai além
disso, mais longe que pode, para alcançar a vida total do Universo e
incorporar a sua vida na do Mundo.''

A literatura -- antes dos modernistas de 22 -, ele afirmava, não era a
gramatiquice lusitana emperrada e artificial, mas a oralidade inculta e
saborosa do falar brasileiro, mais do que uma forma vazia e altissonante
à la Coelho Neto, ``mestre'' das lantejoulas de literatice da época em
seus romances natimortos. A literatura era a ``a exteriorização de um
certo e determinado pensamento de interesse humano, que fale do problema
angustioso que nos cerca, e aluda às questões de nossa conduta na
vida''.

Nossa conduta na vida: essa raiz ética o prende a uma consciência
incapaz de aceitar, conciliações com tudo que é chucro, chulo, vil,
menor e o degreda para uma solidão, uma incompreensão nunca sanada, a
par do fardo melancólico do pai delirante a bradar que tranquem as
portas pois a polícia cercou a casa e vem buscá-lo, nas noites de
insônia e alucinações tétricas de pavor. Lima Barreto não tinha
``pistolões'', não recorria a conterrâneos bairristas que pudessem
ajudá-lo, não aceitava a petulância fátua dos acadêmicos nem teve o
bálsamo de um amor feminino que pudesse servir-lhe de apoio na vida
adversa. Os seus inimigos, ferozes e poderosos, vingaram-se da maneira
mais solerte possível: ignorando-o, impedindo-o de ganhar uma vida
melhor, de ter o renome que seu engenho extraordinário lhe permitiria.
Um de seus alvos preferidos foi a Igreja mimética, camaleão a adaptar-se
às cores predominantes nas mutáveis cortes do poder:

\begin{quote}
``A tática seguida pelo Vaticano consiste em sustentar a classe poderosa
no momento, com unhas e dentes, desculpar os seus erros e crimes, para
poder viver; e quando ela, a classe poderosa, é derrubada e abatida,
alia-se à poderosa que lhe sucede\ldots{}

\ldots{} Não creio, portanto, que a Igreja possa resolver a questão
social que os nossos dias põem para ser solucionada urgentemente.

Se os socialistas, anarquistas, sindicalistas, positivistas etc, não a
podem resolver, estou muito disposto a crer que o catolicismo não a
resolverá também, tanto mais que nunca foram tão íntimas as relações do
clero com o capital, e é contra este que se dirige toda a guerra dos
revolucionários.''
\end{quote}

Os revolucionários, em sentido estrito do termo, seriam os que traziam
reformas, principalmente para o operariado: redução da jornada diária de
trabalho, aumento salarial: nunca Lima Barreto aderiu totalmente à
ideologia alguma, a nenhum credo, a nenhum dogma político. Há tentativas
várias de querer incorporá-lo ao positivismo, que abraçou por pouco
tempo, ao anarquismo, ao comunismo, mas ele mesmo, de forma categórica,
confessa sempre sua independência individual: a favor da liberdade, a
favor dos Jecas Tatus entregues à miséria, à maleita e ao analfabetismo
que Monteiro Lobato denunciara de forma tão veemente e irrepetível em
\emph{Urupês}, mas sem rótulos nem diretrizes partidárias.

Deixa documentado nitidamente:

\begin{quote}
``Não obedeço a teorias de higiene mental, social, moral, estética de
espécie alguma. O que tenho são implicâncias parvas: e é só isso.
Implico com três ou quatro sujeitos das letras, com a Câmara, com os
diplomatas, com Botafogo e Petrópolis: e não é em nome de teoria alguma,
porque não sou republicano, não sou socialista, não sou anarquista, não
sou nada; tenho implicâncias. É uma razão muito fraca e subalterna; mas
como é a única, não fica bem à minha honestidade de escriba escondê-la''
\end{quote}

O próprio escritor, porém, que bradava não amar nem a pátria; nem a
família, nem a Humanidade, na realidade buscava a compreensão do próximo
e seus romances são dos mais impregnados de ternura humana comovente já
escritos no Brasil. Não erraram alguns críticos da época em,
percucientemente, apontarem o personagem de Policarpo Quaresma como uma
espécie de dom Quixote brasileiro em miniatura. Em que outras páginas da
literatura brasileira haverá tanta graça leve, tanta caricatura mordaz
mas não sádica, tanta doçura humana a definir personagens que Sterne e
Dickens aprovariam com entusiasmo? Quaresma, o brasileiro extremado, que
escreve em tupi-guarani uma carta a um ministério e que dedica seu tempo
e escasso dinheiro a ler sobre os rios do Brasil, as riquezas do Brasil,
a flora e a fauna do Brasil, terminaria vítima da ditadura, do hospício,
do esmagamento execrável pelos poderosos do momento e pela indiferença
dos circunstantes. Nem mesmo uma figura feminina -- à mulher Lima
Barreto reserva sempre um quinhão maior de bondade altruísta e ativa --
consegue interceder por ele, os amigos evitando envolver-se com quem não
fosse do partido governista\ldots{} Mas o final, da impotência contra a
prepotência do arbítrio, não consegue transmitir uma anulação total da
atmosfera paradoxalmente risonha que Lima Barreto incute a seus
personagens, vítimas de obsessões hilariantes.

Aquele marginal, Lima Barreto, não era um misantropo: amava a Humanidade
a seu modo rude, áspero, e não há justificativa para sua própria
descrição como a de um homem ``de coração árido''. Fugazmente boêmio nas
rodas literárias do Rio do seu tempo, cidade então de poucas e parvas
distrações, cioso de sua descendência lusitana e africana, Lima Barreto
escrevia para comunicar-se, não para traduzir o Belo e o Perfeito em
estilo literário. Frequentemente se jactava de sua incorreção gramatical
e sua ``implicância'' com Machado de Assis é célebre. Ao mestre criador
de \emph{Memórias Póstumas de Brás Cubas}, ele reservava apenas frases
de sarcasmo: ``Machado escrevia com medo do Castilho e escondendo o que
sentia, para não se rebaixar''. Seu magnífico biógrafo, Francisco Assis
Barbosa, cita depoimentos tanto de Austregésilo de Ataíde quanto de
Sérgio Buarque de Holanda, segundo os quais o mero nome de Machado de
Assis enfurecia Lima Barreto a ponto de imprecar para quem quisesse
ouvir: ``Machado é um falso em tudo. Não tem naturalidade. Inventa tipos
sem nenhuma vida''. Considerava Aluísio de Azevedo superior a Machado de
Assis como romancista. Pior ainda: Machado de Assis seria um omisso,
meramente aludindo a subentendidos que de tão abstratos se tornavam
ocos: ``Machado era um homem de sala, amoroso das coisas delicadas, sem
uma grande, larga e ativa visão da Humanidade e da Arte. Ele gostava das
coisas decentes e bem postas, da conversa da menina prendada, da
garridice das moças.''

Nesse julgamento impiedoso e sectário, o autor de \emph{Policarpo
Quaresma} embutia talvez sua revolta contra um mulato que não aludia a
essa condição, ao contrário dele, Lima Barreto, que já então denominava
(``impertinentemente'', como queriam muitos de seus inimigos brancos) de
``Vila Quilombo'' sua casa e queria, muito antes da \emph{Négritude} de
Senghor e Aimé Césaire, criar uma literatura sobre os negros, um
``negrismo'', como chegou a denominá-la, que tirasse essa componente
decisiva da população brasileira do esquecimento a que estava entregue
pelos escritores brancos ou mulatos que se tinham por brancos ou pretos
``de alma branca'', como queria o preconceito da época.

Era compreensível que o desespero o levasse muitas vezes -- antes do
alcoolismo e das entradas no hospício, à semelhança do pai
incuravelmente louco -- a pensar no suicídio como uma forma de escapar
da masmorra em que o tinham enterrado vivo. É a seu personagem em grande
parte autobiográfico, Isaías Caminha, que ele confia seus pensamentos
íntimos:

``Eu tinha uma imensa lassidão e uma grande fraqueza de energia mental.
Quis descansar, debrucei-me na muralha do cais e olhei o mar. Estava
calmo; a limpidez do céu e a luz macia da manhã faziam-no aveludado. Os
últimos sinais da tempestade da véspera tinham desaparecido. Havia
satisfação e felicidade no ar, uma grande meiguice, em tudo respirava; e
isso pareceu-me hostil. Continuei a olhar o mar fixamente, de costas
para os bondes que passavam. Aos poucos ele hipnotizou-me, atraiu-me,
parecia que me convidava a ir viver nele, a dissolver-me nas suas águas
infinitas, sem vontade nem pensamento; a ir nas suas ondas experimentar
os climas da terra, a gozar todas as paisagens, fora do domínio dos
homens, completamente livre, completamente a coberto de suas regras e
dos seus caprichos\ldots{} Tive ímpetos de descer a escada, de entrar
corajosamente pelas águas adentro, seguro de que já ia passar a uma
outra vida melhor, afagado e beijado constantemente por aquele monstro
que era triste como eu.''

Da escrivaninha à mesa dos cafés literários da época e daí à cama de um
bordel, do desvelo ao pai incurável à rebelião retesada anos a fio sob o
tacão da discriminação, Lima Barreto morreria sem ter terminado sua
obra, aos 41 anos de idade, o organismo corroído pelo álcool, as mãos
agarradas aflitas a um tomo da revista francesa, \emph{Revue des Deux
Mondes}, 48 horas antes do triste fim de seu próprio pai. Talvez na
literatura brasileira não haja documentos mais patéticos do que os
transcritos por ocasião de seu internamento nas clínicas do Instituto de
Psiquiatria da Universidade do Brasil, refúgio que alternava com a
sarjeta em seus momentos de dor incontida: ``Comemorativos pessoais e de
moléstia: Cópia da guia policial: -''Nada informa dos antecedentes de
hereditariedade. Acusa outros no rapto de manuscritos. Acusa insônias,
com alucinações visuais e auditivas. Estado geral bom. Boa memória. Já
teve sarampo e catapora, blenorragia, que ainda sofre, e cancros
venéreos. Confessa-se alcoolista imoderado, não fazendo questão de
qualidade. Está bem orientado no tempo e no meio. Memória íntegra:
conhece e cita com bastante desembaraço fatos da História antiga, média,
moderna e contemporânea, respondendo as perguntas que lhe são feitas,
prontamente. Tem noções de álgebra, geometria, geografia. Nega
alucinações auditivas, confirmando alucinações visuais. Associação de
ideias e de imagens perfeitas; assim como perfeitas são a atenção e a
percepção. Cita seus autores prediletos que são: Bossuet, Chateaubriand,
``católico elegante'' (sic), Balzac, Taine, Daudet; diz que conhece um
pouco de francês e inglês. Com relação a esses escritores faz
comentários mais ou menos acertados; em suma, é um indivíduo que tem
algum conhecimento, e inteligente para o meio em que vive. Interrogado
sobre o motivo de sua internação, refere que indo à casa de um seu tio
em Guaratiba, prepararam-lhe uma assombração, com aparecimentos de
fantasmas, que aliás lhe causam muito pavor. Nessa ocasião, chegou o
tenente Serra Pulquério, que, embora seu amigo de ``pândegas'',
invectivou-o por saber que preparava panfletos contra seus trabalhos na
vila proletária Marechal Hermes. Tendo ele negado, foi conduzido à
polícia, tendo antes cometido desatinos em casa quebrando vidraças,
virando cadeiras e mesas. A sua condução para a polícia só se fez
mediante o convite do comissário, que lhe deu aposentos na delegacia até
que o transferiram para a nossa clínica. Protesta contra o seu
``sequestro'', pois vai de encontro à lei, uma vez que nada fez que o
justifique. Nota de certo tempo para cá animosidade contra si, entre os
seus companheiros de trabalho, assim como entre os próprios oficiais do
Ministério da Guerra de onde é funcionário. Julga que o tenente Serra
Pulquério teme a sua fama ``ferina e virulenta'', pois, apesar de não
ser grande escritor nem ótimo pensador, adota as ideias anarquistas e
quando escreve deixa transparecer debaixo de linguagem enérgica e
virulenta os seus ideais. Apresenta-se relativamente calmo,
exaltando-se, contudo, quando narra os motivos que justificaram a sua
internação. Tem duas obras publicadas: \emph{Triste Fim de Policarpo
Quaresma} e \emph{Memórias} (sic) \emph{do Escrivão Isaías Caminha}.
Marcha da moléstia e tratamento: Purgativo-Ópio. Saída: Transferido em
27 de agosto de 1914.

2ª Entrada

Nome: Afonso H. de Lima Barreto

Cor: parda -- Idade: 38 anos -- Nacionalidade: brasileira. Estado civil:
solteiro -- Profissão: Jornalista. Entrada : em 25 de dezembro de 1919.
Diagnóstico: Alcoolismo.''

Vendo a existência humana com um estoicismo valente, incapaz de ater-se
a uma fé mística, não crendo que fosse possível ao ser humano devassar o
Mistério que circunda o nascimento e a morte, Lima Barreto, no entanto,
não deixa como mensagem final o desalento. Confirma, é verdade, o estado
calamitoso em que se encontram todos os seus conterrâneos e
contemporâneos. Mas, como sucede frequentemente na prosa lírica
arrebatada desse Mestre que não pôde perfazer a sua Perfeição, há lugar
para uma remota esperança no futuro. O passado foi canibalesco,
hediondo, apavorante e sem remédio, mas quem sabe uma Humanidade futura
aprenderá o amor, a comoção, a solidariedade, a grandeza generosa da
alma? É o que subentende claramente o trecho final de seu belo,
comovente, miniaturesco \emph{O Triste Fim de Policarpo Quaresma} com
seu ritmo de \emph{adagio} melodioso e solene:

\begin{quote}
``Saiu e andou. Olhou o céu, os ares, as árvores de Santa Teresa, e se
lembrou que, por estas terras, já tinham errado tribos selvagens, das
quais um dos chefes se orgulhava de ter no sangue o sangue de dez mil
inimigos. Fora há quatro séculos. Olhou de novo o céu, os ares, as
árvores de Santa Teresa, as casas, as igrejas: viu os bondes passarem;
uma locomotiva apitou; um carro, puxado por uma linda parelha,
atravessou-lhe na frente, quando já a entrar no campo\ldots{} Tinha
havido grandes e inúmeras modificações. Que fora aquele parque? Talvez
um charco. Tinha havido grande modificações nos aspectos, na fisionomia
da terra, talvez no clima\ldots{} Esperemos mais, pensou ela; e seguiu
serenamente ao encontro de Ricardo Coração dos Outros.''
\end{quote}

\chapter{Lima Barreto - doce, feroz, iluminado. E
esquecido}\label{lima-barreto---doce-feroz-iluminado.-e-esquecido}

Jornal da Tarde, 1984/04/14. Aguardando revisão.

\hfill\break

\begin{quote}
``Além de mulato, talentoso!''
\end{quote}

Não era o cúmulo do desaforo?!

Abanando-se com leques para refrescar sua fúria, não editando seus
livros, os bem-pensantes donos da opinião deste país têm conseguido a
contento sufocar quae inteiramente a figura e a obra de Lima Barreto até
hoje.

Para suprimi-las, não faltam pretextos. Em primeiro lugar, o racismo
cruel e ignorante que indaga atônito: ``Mas, afinal, o que queria aquele
negro metido a escritor?'' E respondem em coro: ``O pai dele morrera num
hospício, incurável. Mais tarde ele também. Logo, geneticamente, quem
sai aos seus..'' Dessa visão nazista da Raça Superior,
\emph{Herrenrasse}, ao moralismo mais hipócrita, é um pulo fácil: ``Ele
próprio não era um bêbado contumaz, recolhido pelas autoridades de
Limpeza Urbana?, na rua do Ouvidor das duas primeiras décadas desde
século, no Rio de Janeiro?''. Subversivo, elemento perigoso para outros,
pregava o Anarquismo, radiografava uma sociedade baseada na empulhação,
no roubo, na agiotagem, no esmagamento de imensa maioria por uma minoria
de ricaços, a plutocracia inteiramente colonizada pelo capital e pelas
ideias estrangeiras.

Nem como material de propaganda para os devotos que se arrastam de
joelhos até a múmia embalsamada de Lenin no Palácio do Kremlin, em
Moscou, as farpas agudas de Lima Barreto servem. Sua gargalhada e seu
espanto ele reservava para qualquer ``ismos'' elevados como hóstia
diante do altar da Santíssima Trindade Marx-Engels-Lenin.

Vários trechos das dinâmicas discussões políticas de seu livro de
estreia, \emph{Recordações do Escrivão Isaías Caminha} (Editora Ática)
publicado integralmente em Portugal, em 1909 (tendo o autor, paupérrimo,
aberto mão de seus direitos autorais para que seu romance saísse em
Lisboa), reforçam a visão precocemente anárquica de Lima Barreto.
Entrelaçam-se com trechos significativos de seus diários e artigos
contidos em \emph{Bagatelas} (Lima Barreto, obras completas, Editora
Brasiliense, 1961). Não terá sido o autor carioca o primeiro, na nossa
literatura, a citar Kropotkin, o teórico do Anarquismo russo? Na obra de
ficção, na realidade uma autobiografia tenuemente disfarçada, as
tendências anarquistas ficam claras:

\begin{quote}
``Não há na Natureza nada que se pareça com a nossa sociedade governada
pelo Estado\ldots{} Observe o senhor que todas as sociedades animais se
governam por leis para as quais elas não colaboraram, são como
preexistentes a elas, independentes da sua vontade; e só nós inventamos
esse absurdo de fazer leis para nós mesmos -- leis que, em última
análise, não são mais que a expressão da vontade, dos caprichos, dos
interesses de uma minoria insignificante\ldots{} No nosso corpo há uma
multidão de organismos, todos eles interdependentes, mas vivem
autonomamente sem serem propriamente governados por nenhum, e o
equilíbrio se faz por isso mesmo\ldots{} O sistema solar\ldots{} Na
Natureza, todo equilíbrio se obtém pela ação livre de cada uma das
forças particulares.''
\end{quote}

Um eco evidente de seu contato com Kropotkin que leu traduzido em
francês:

\begin{quote}
``As partes componentes de um ser vivo ajudam-se umas às outras. Assim,
em todas as relações dos entes animados, a luta pela existência tende a
tornar-se a luta pela coexistência''
\end{quote}

A noção de ``luta de classes'' lhe soa tão ineficaz quanto as fórmulas
pretensamente mágicas do Positivismo e, por extensão, do totalitarismo
nazifascista ou comuno-soviético.

\begin{quote}
``Eu ouvi-o sem coragem de contestar, embora não compartilhasse as suas
crenças. Não era a primeira vez que ia ao Apostolado, mas quando vi o
vice-diretor sair rapidamente por detrás de um retábulo, na absida da
capela, ao som de um tímpano rouco, arrebatando a batina, com aquele
laço verde no braço, dava-me vontade de rir às gargalhadas. Demais,
ficava assombrado com a firmeza com que ele anunciava a felicidade
contida no Positivismo e a simplicidade dos meios necessários para a sua
vitória: bastava tal medida, bastava essa outra -- e todo aquele rígido
sistema de regras, abrangendo todas as manifestações da vida coletiva e
individual, passaria a governar, a modificar costumes, hábitos e
tradições. Explicava o catecismo. Abria o livro, lia um trecho e
procurava o caminho para alusões a questões atuais, repetindo fórmulas
para se obter um bom governo que tendesse a preparar a era normal -- o
advento final da Religião da Humanidade..''
\end{quote}

Franco-atirador murado em sua fortaleza elevada, Lima Barreto já
precocemente previra a manipulação do futebol como uma fonte de
distração rendosa, a fim de desviar a atenção do povo de seus reais
problemas. Denunciara os prefeitos que derrubavam morros e florestas,
alterando o que hoje se chama de equilíbrio ecológico de um ambiente.
Antecedera-se a Léopold Senghor, recém-eleito para a cômica Academia
Francesa dos ``imortais'', na criação de uma literatura voltada para o
negro, a sua cultura, não uma \emph{négritude} comparável à de Aimé
Césaire mas um ``negrismo'' que ressaltasse o elemento negro como a
argamassa de união nacional, a construtora de toa da estrutura econômica
e, em parte determinante, também da cultura do Brasil.

\begin{quote}
``Estas \emph{'Recordações'I''} não têm, porém, outro propósito senão o
de fazer a biópsia desse organismo vivo mas em grande parte apodrecido
que é a imprensa, na qual colaborou, no \emph{Correio da Manhã}, do Rio
de Janeiro. Sem dúvida, não alude quase às exceções: aos jornalistas
competentes, inteligentes, honestos, que não vivem de adulação e
ignorância arrogante. Por isso as redações que descreve não são
mitigadas por figuras humanas, parecem um museu de monstros auto-ungidos
em deuses pelo engodo que impingem à massa de leitores amorfos,
moldáveis, estúpidos:~

``Nada há tão parecido com o pirata antigo e o jornalista moderno; a
mesma fraqueza de meios, servida por uma coragem de salteador;
conhecimentos elementares do instrumento de que lançam mão e um olhar
seguro, uma adivinhação, um faro para achar a presa e uma
insensibilidade, uma ausência de senso moral a toda prova\ldots{} E
assim dominam tudo, aterram, fazem que todas as manifestações de nossa
vida coletiva dependam do seu assentimento e da sua aprovação\ldots{}
Todos nós temos que nos submeter a eles, adulá-los, chamá-los gênios,
embora intimamente os sintamos ignorantes, parvos, imorais e bestas. Só
se é geômetra com o seu \emph{placet} (beneplácito), só se é calista com
sua confirmação e se o sol nasce é porque afirmam tal cousa\ldots{} E
como eles aproveitam esse poder que lhes dá a fatal estupidez das
multidões! Fazem de imbecis gênios, de gênios imbecis; trabalham pra a
seleção da mediocridade..''
\end{quote}

E sucedem-se, ferozes, as cenas e retratos da redação: um diretor
enfurece-se e se preocupa com o que dele dirão os grandes gramáticos do
idioma, a ele que estava entregue a salvaguarda da língua pátria,
conspurcada por erros de português? E responde à pergunta se está certo
dizer ``um copo d'água'' ou ``um copo com água'':

\begin{quote}
``- Conforme: se se tratar de um copo cheio, é um copo d'água; se não
estiver cheio, um copo com água.''
\end{quote}

Outro se enraivece quando se usa o adjetivo ``eminente'' para outra
pessoa que não for José Bonifácio e não atina com quem seja Ruskin, o
admirável esteta inglês estudioso das catedrais gótica francesas,
traduzido por seu grande admirador, Marcel Proust, possivelmente o
escritor de prosa francesa mais importante deste século. Ainda outro
``paquiderme plumitivo'' exerce uma tirania de árbitro do bom gosto no
jornal: ``O seu estágio diplomático em Quito dava-lhe também um
infalível julgamento inenarrável nas maneiras de tratar duquesas e
princesas.''

Como há os que se arvoram em ``sábios'' mandam proceder à análise
antropométrica de um casal assassinado, aos quais tinham tirado as
cabeças. Citando doutamente autoridades em medicina legal, chegam à
conclusão de que o morto era\ldots{} um mulato. É inútil descobrir-se
que o declarado ``mulato'' não passava de um italiano com carteira de
identidade e ficha dactiloscópica: ``Um dia antes dessa elucidação, o
doutor Franco de Andrade (autor dessas esdrúxulas mensurações
antropológicas) era nomeado diretor do Serviço Médico-Legal da Polícia
da cidade do Rio de Janeiro''.

Haveria inúmeros outros exemplos da imbecilidade que sufoca a esmagadora
maioria da imprensa até os nossos dias, de mãos dadas com a mais cínica
e impune distorção das informações dirigidas ao leitor. Nos EUA já se
esboça um movimento de grandes proporções gravíssimas para aqueles
jornalistas que se julgam consagrados, auto ungidos senhores da verdade,
mestres em manipular, conforme seus caprichos, inclinações ideológicas
ou preguiça recheada de ignorância e burrice agressivamente arrogante,
as notícias em seu poder.

Romance inicial de um dos poucos gênios da Literatura Brasileira que o
descaso brasileiro pela cultura e pela informação relegou quase ao
anonimato, Lima Barreto surpreende pela doçura que se esconde por trás
de tanta e tão justificada amargura. Para quem, de forma devastadora,
concluíra que o preconceito levava os brancos racistas (a imensa
maioria) a enxergá-lo \emph{menos} do que se enxerga uma árvore, as
meditações estoicas no seu desespero diante do ódio \emph{a priori} que
lhe dedicavam continuam dolorosamente atuais no Brasil, que, desde a sua
morte (ou assassinato por omissão dos demais), não mudou:

\begin{quote}
``\ldots{} Mas, não é a ambição literária que me move a procurar esse
dom misterioso para animar e fazer viver estas pálidas
\emph{Recordações}. Com elas, queria modificar a opinião de meus
concidadãos, obrigá-los a pensar de outro modo, a não se encherem de
hostilidade e má vontade quando encontrarem na vida um rapaz como eu e
com desejos que tinha há dez anos passados. Tento mostrar que são
legítimos e, senão merecedores de apoio, pelo menos dignos de
indiferença''.
\end{quote}

\chapter{Policarpo Quaresma, uma obra-prima envolta em doce
amargura}\label{policarpo-quaresma-uma-obra-prima-envolta-em-doce-amargura}

Jornal da Tarde, 1983/4/9. Aguardando revisão.

\hfill\break

A compreensão do espírito que animou o romancista carioca Lima Barreto a
compor o personagem título de suas páginas espantosas, \emph{Triste Fim
de Policarpo Quaresma}, pode ficar em parte impedida pela tradução
imperfeita da frase de Renan que lhe serve de epígrafe. No original
francês de seu Mar Aurèle, está claramente exposta a noção de ruptura
entre o idealista e a realidade que o circunda:

\begin{quote}
``Le grand inconvénient de la vie réelle et ce qui la rend insupportable
à l'homme supérieur, c'est que, si l'on y transpose les principes de
l'idéal, les qualités deviennent des défauts, si bien que fort souvent
l'homme accompli y réussit moins bien que celui qui a pour mobiles
l'égoïsme ou la routine vulgaire~»
\end{quote}

Como qualquer aluno médio da Aliança Francesa sabe, a tradução em
português que vem logo abaixo está incorreta:

\begin{quote}
``O grande inconveniente da vida real e o que a torna insuportável ao
homem superior é que, se para ela transportamos os princípios do ideal,
as qualidades se tornam defeitos, se bem que frequentemente o homem
íntegro aí se sai menos bem que aquele que tem por causas o egoísmo e a
rotina''
\end{quote}

\emph{Si bien que} significa a tal ponto que, de tal modo que, correção
que, convenhamos, modifica todo o sentido da frase.

Tivesse tal erro primário sido cometido por um principiante no estudo do
francês e não teria sido tão grave. Mas numa edição cuidada, que se quer
erudita, com frondosas citações de Luckács e outros críticos, fica
lesada a inteligência do leitor que não puder discernir na infeliz
tradução uma das linhas-mestras que delineiam todo o romance de Lima
Barreto: \emph{Triste Fim de Policarpo Quaresma}, Editora Ática.

É óbvio o parentesco do major, patriota e funcionário Policarpo Quaresma
com Dom Quixote ou com o príncipe Mishkin de \emph{O Idiota} de
Dostoievsky ou com qualquer personagem que queira sobrepor seu idealismo
à rudeza destruidora e diluidora da realidade que o circunda. Não há
maior lugar-comum do que o afirmar que os ``inconformados'' e ``animados
por um ideal elevado'' são os mártires da história política (Ghandi,
Tiradentes, Martin Luther King), cultural (Baudelaire, Bach, Mozart),
social (Wilhelm Reich, Rosa Luxemburgo) ou qualquer outra esfera do
penoso progresso humano. De certa forma, \emph{Triste Fim de Policarpo
Quaresma} é o típico \emph{Erziehungsroman} em sua forma trágica: é um
romance em que o herói ou anti-herói aprende, por sua própria e dolorosa
experiência, a reconhecer os limites da sua esfera ideal de ação e os
obstáculos, intransponíveis, que os interesses dominantes erigem contra
seu inconformismo ou ingenuidade.

Os fatos patéticos que marcaram a vida de Lima Barreto são notórios:
mulato, pobre, escorraçado pelo preconceito racial das ``rodas''
literárias do Rio de Janeiro da sua época (1881-1922), pela futilidade
dos prosadores e poetas brasileiros que, na sua esmagadora maioria,
estavam tão distantes do Brasil quanto possível, tentando aclimatar-se a
uma ``selva'' oposta em tudo ao refinamento de Paris e suas escolas de
parnasianismo, simbolismo e outras flores exóticas mal importadas por
nossas alfândegas intelectuais. Lima Barreto passou do alcoolismo à
loucura, depois de lutar asperamente para manter o pai que enlouquecera,
dedicando-se ao jornalismo, a empregos públicos, e vindo a morrer depois
de propor profeticamente o lançamento da ``negrice'', uma corrente
literária que desse ao negro o relevo que tem na sociedade brasileira e,
comovedoramente, depois de ter focalizado os males que corroem o Brasil
com um misto inusitado, original, de esperança, de amargura, de
pessimismo e de ironia.

Inimigo acérrimo de Machado de Assis, Lima Barreto isolou-se de seu meio
ambiente: jamais quis pertencer a nenhuma academia literária, abominava
a descaracterização cultural do Brasil, já predominante em seu tempo.
Embriagava-se de lieratura francesa exatamente para não copiá-la
servilmente, mas, ao contrário, enraizar a literatura brasileira fora da
estufa onde se cultivavam tendências europeias e retratar as diferentes
camadas de nossa população em romances, cujo realismo é tocado sempre de
um comovente elo de calor humano e de poesia. Excelente, a biografia que
fez dele Francisco de Assis Barbosa (Editora José Olympio, \emph{A Vida
de Lima Barreto}), aviva e aclara os traços fundamentais desse grande
precursor dos conceitos principais da Semana de Arte Moderna de 1922.

Não que fosse possível reduzi-lo a mero ``precursor'', pois com Lima
Barreto as correntes que prendiam o romance brasileiro às matrizes do
Velho Mundo já se rompem, concreta e comprovadamente. Policarpo Quaresma
é, em grande parte, o próprio autor, com pinceladas caricaturais que,
porém, não lhe retiram a aura de doçura e perspicácia intelectual que
distinguem o romancista carioca. Farto de verificar, diariamente, quanto
o Brasil, estrangeirado até a medula, se avilta, se apequena e se nega a
si próprio, esse humilde funcionário público idolatra um conceito que os
internacionalismos hoje em modo tornaram risível e ao qual o senador
Teotônio Vilela se refere desassombradamente: a Pátria. Quaresma envia
ao Congresso um ofício, pedindo que se adote o tupi-guarani como nossa
língua oficial. Quaresma saúda quem vai visitá-lo com os prantos típicos
de uma tribo tupinambá em vez do distante aperto de mão, o importado
\emph{handsshake} britânico. Considerado louco, passa no hospício uma
temporada que, se não arrefece o seu patriotismo, o amargura
profundamente. Seu contato com a roça é igualmente desolador; a roça, o
interior são lugares abandonados pelo governo, entregues às saúvas que
devoram as colheitas, às doenças dos Jecas-Tatus que uma espécie de
maleita mental impede de defender seus direitos, quando não o quintal
das arengas da baixa politicagem. Quaresma, por não apoiar um partido
político ou mesmo se definir em termos políticos, é injustamente
multado, de forma dolosa e impune. Escreve ao Marechal Floriano Peixoto
presidente da República e ingenuamente se enreda em facções militares e
termina condenado ao pelotão de fuzilamento. Um romance picaresco? Uma
história divertida como é engraçado o Tartarin de Tarascon, de Daudet?
Uma confissão apavorante de impotência diante do acomodamento e do
servilismo das oligarquias brasileiras que o esmagam ao mesmo tempo ao
mesmo tempo que obliteram a sua memória?

\emph{Triste Fim de Policarpo Quaresma} roça, frequentemente, a estatura
de uma obra-prima: ``E desse modo ele ia levando a vida, sem ser
compreendido, e a outra metade na repartição, também sem ser
compreendido. No dia em que o chamaram de Ubirajara, Quaresma ficou
reservado, taciturno, mudo, e só veio a falar porque, quando lavavam as
mãos num aposento próximo à secretaria e se preparavam para sair,
alguém, suspirando, disse: `Ah! Meu Deus! Quando poderei ir à Europa!' O
major não se conteve: levantou o olhar, consertou o \emph{pince-nez} e
falou fraternal e persuasivo: 'Ingrato! Tens uma terra tão bela, tão
rica, e queres visitar a dos outros! Eu, se algum dia puder, hei de
percorrer a minha de princípio ao fim!''

Num país em que a memória nacional é carunchada em igrejas, bibliotecas,
velhos casarões ainda de pé, ele se espanta: ``Como é que o povo não
guardava as tradições de 30 anos passados?'' Os que fazem da adulação
ignóbil seu modo de ``subir'' na vida povoam estas páginas ao lado de
Quaresma: ``Empregado do Tesouro, já no meio da carreira, moço de menos
de 30 anos, ameaçava ter um grande futuro. Não havia ninguém mais
bajulador e submisso do que ele. Nenhum pudor, nenhuma vergonha! Enchia
os chefes e os superiores de todo incenso que podia. Quando saía,
remancheava, lavava três ou quatro vezes as mãos, até poder apanhar o
diretor na porta. Acompanhava-o, conversava om ele sobre o serviço, dava
pareceres e opiniões, criticava este ou aquele colega e deixava-o no
bonde, se o homem ia para casa\ldots{} Na bajulação e nas manobras para
subir, tinha verdadeiramente gênio''. A falsa erudição que até hoje o
brasileiro médio pensa ``tirar de letra'' é uma escada para os altos
cargos e para a admiração da patuléia ignara, entontecida por aquela oca
retórica plagiada de tomos vetustos e compreendidos pela metade. Os
movidos pelo egoísmo ou pela rotina corriqueira sobressaem-se, como na
reflexão ética de Renan, mal traduzida como epígrafe deste livro: os
loucos, os miseráveis, os idealistas é que estão à margem do
``triunfo'', do ``sucesso'', nessa pluralidade de ``infernos sociais''
que a vida apresenta:

\begin{quote}
``Casas que mal dariam para uma pequena família são divididas,
subdivididas, e os minúsculos aposentos assim obtidos alugados à
população miserável da cidade. Aí nesses caixotins humanos, é que se
encontra a fauna menos observada da nossa vida, sobre a qual a miséria
paira com um rigor londrino. Não se podem imaginar profissões mais
tristes inopinadas da gente que habita tais caixinhas..''~

Mas mais ``democrática'' e universal é a loucura. ``Quem uma vez esteve
diante deste enigma indecifrável da nossa própria natureza, fica
amedrontado, sentido que o gérmen daquilo está depositado em nós e que
por qualquer coisa ele nos invade, nos toma, nos esmaga e nos sepulta,
numa desesperadora compreensão inversa e absurda de nós mesmos e dos
outros e do mundo\ldots{} Não é só a morte que nivela: a loucura, o
crime, a moléstia passam também a sua vassoura pelas distinções que
inventamos\ldots{} Saiu o major mais triste ainda do que vivera toda a
vida. De todas as cousas tristes de ver, no mundo, a mais triste é a
loucura; é a mais depressora e pungente''.
\end{quote}

Como Adolfo Caminha, Lima Barreto antecipou-se excessivamente a seu
tempo: de temperamento anárquico, revoltado com sua condição, mas
saudoso da monarquia, horrorizado com a República imposta pelos
militares positivistas, ele ironiza o ``progresso'' material que não
modifica minimamente a ética do comportamento humano. É um trecho que
adquire imediata atualidade se se substituir o termo ``positivismo''
pelo de ``materialismo científico'':

\begin{quote}
``Eram os adeptos desse nefasto e hipócrita positivismo, um pedantismo
tirânico, limitado e estreito, que justificava todas as violências,
todos os assassínios, todas as ferocidades em nome da manutenção da
ordem, condição necessária, lá diz ele, ao progresso e também ao advento
do regime normal, a religião da Humanidade, a adoração do grão-fetiche,
com fanhosas músicas de cornetins e versos detestáveis, o paraíso enfim,
com inscrições em escritura fonética e eleitos calçados com sapatos de
solas de borracha!\ldots{} Os positivistas discutiam e citavam teoremas
e mecânica para justificar as suas ideias de governo, em tudo semelhante
aos cantos e emirados orientais''.
\end{quote}

A carta em que Quaresma se dirige à irmã e reexamina a sua vida é o
momento culminante deste romance melancólico, trágico, pungente:

\begin{quote}
``Esta vida é absurda e ilógica; eu já tenho medo de viver, Adelaide.
Tenho medo, porque não sabemos para onde vamos, o que faremos amanhã,
deque maneira havemos de nos contradizer de sol a sol\ldots~

O melhor é não agir, Adelaide; e, desde que o meu dever me livre destes
encargos, irei viver na quietude, na quietude mais absoluta possível,
para que do fundo de mim mesmo ou do mistério das cousas não provoque a
minha ação o aparecimento de energias estranhas à minha vontade, que
mais me façam sofrer e tirem o doce sabor de viver\ldots~

Além do que, penso que todo este meu sacrifício tem sido inútil. Tudo o
que nele pus de pensamento não foi atingido, e o sangue que derramei e o
sofrimento que vou sofrer toda a vida foram empregados, foram gastos,
foram estragados, foram vilipendiados e desmoralizados em prol de uma
tolice política qualquer\ldots~

Ninguém compreende o que quero, ninguém deseja penetrar e sentir; passo
por doido, tolo e maníaco e a vida vai-se fazendo inexoravelmente com a
sua brutalidade e fealdade.''
\end{quote}

Não só a loucura, a miséria, a guerra esgotam esta saga urbana: sensível
à condição da mulher, Lima Barreto já denuncia a ``casa de bonecas''
ibseniana em que a mulher brasileira é obrigada a viver, sem participar
das ações, reflexões e decisões nacionais: seu único ``destino'' e
casar-se e, uma vez casada, ``conhecer o seu lugar''. Nem mesmo a Olga
que, superando os preconceitos que lhe querem tolher os movimentos,
tenta, inutilmente, salvar aquele ``pândego'' e só depara com homens
pusilânimes e medíocres consegue superar essa crosta de imobilidade.

Há quem queira ver nas frases finais do livro uma esperança: ``Tinha
havido grandes modificações nos aspectos, na fisionomia da terra, talvez
no clima\ldots{} Esperamos mais, pensou ela; e seguiu serenamente ao
encontro de Ricardo Coração dos Outros''.

Será justa essa aferição da conclusão de Lima Barreto? É impossível
determinar, a não ser por um \emph{parti pris} ideológico, que sim ou
que não. De qualquer maneira, o que o leitor constata é uma mecânica da
imobilidade do Brasil: de 1911, quando foi publicado esse livro, até
hoje, o que mudou senão a aparência? Que modificações houve senão de
alcance estatístico? De forma confusa, no entanto, sem rótulos precisos
de ``esperança'' ou ``desânimo'', Lima Barreto incute em quem o lê a
noção dual de que o prisma das coisas tristes é sem dúvida mais amplo e
mais abrangente do que o da alegria, mas viver, como um inexplicável
paradoxo, imbui as derrotas de uma atmosfera que só o enigma da doçura
define e delimita.

\chapter{A Mãe Coragem negra de Canindé. Sobre os diários de Carolina
Maria de
Jesus}\label{a-muxe3e-coragem-negra-de-caninduxe9.-sobre-os-diuxe1rios-de-carolina-maria-de-jesus}

Christ und Welt, n.28, ano XV, 1962/07/13. Aguardando revisão.

\hfill\break

Através do mapa de São Paulo corre uma linha sombria que representa
simbolicamente a estrutura social e a rígida hierarquia da metrópole
industrial da América Latina de quatro milhões de habitantes. Dos
terrenos ondulados dos quarteirões aristocráticos -- Jardim Europa,
Morumbi -- com suas mansões milionárias essa linha desce para o centro
comercial e para os quarteirões proletários, Brás e Vila Maria, até que
ela desemboca na escuridão putrefata da favela Canindé, localizada entre
as águas negras do rio Tietê e a reluzente autoestrada que leva ao Rio
de Janeiro.

Não há nenhum elo entre esses mundos separados hermeticamente um do
outro. No apartamento de luxo do Conde Matarazzo, o mecenas brasileiro
ao estilo de Rockfeller, estão dependurados nas paredes quadros de Miró,
Cézanne e Rouault que foram pleiteados ciumentamente por alguns museus e
leiloeiros da América do Norte e da Europa, quando eles estiveram à
venda. Nos quarteirões proletários, na floresta de fábricas do Conde
Matarazzo e do \emph{playboy} Baby Pignatari, para mencionar apenas
esses dois, trabalham assiduamente italianos, japoneses, poloneses e
alemães que sonham o sonho capitalista de enriquecer rapidamente,
enquanto em Canindé os vencidos pela vida -- a maioria negros e
retirantes do Nordeste ressequido -- vegetam em barracões de metal e
madeira assediados pela fome e promiscuidade, alcoolismo e desesperança.
Um dia, contudo, os finos fios que conectam fragmentariamente entre si
as células sadias e doentes desse organismo gigante, provocaram, por um
instante, uma reação em cadeia que embaralhou todos esses valores
hierárquicos.

Audálio Dantas, um jovem repórter, que devia estar fazendo uma
reportagem rotineira na favela sobre os arruaceiros que roubavam com
violência as crianças do bairro na sua praça de brinquedos, ouviu
próximo a ele os gritos indignados de uma negra alta e elegante.
``Inacreditável, esta embalagem!'', vociferou ela majestosamente. ``Elas
vêm todas em meu livro para que não sejam esquecidas''.

O faro do repórter deixou-o entrever uma história que poderia emergir do
pântano de tantas esperanças humanas. Ele ganhou a confiança da
pretendida autora e no primitivo barraco de madeira construído por ela
leu o diário dela. Escrito penosamente à mão; notas gramaticalmente
inábeis desvelam a terrível descrição de um submundo, de um mundo de
subhomens, o mundo da favela -- um mundo como o Ocidente, desde o livro
\emph{Recordação da Casa dos Mortos} de Dostoiévski, não mais havia
vivenciado. ``Canindé é a filial do Inferno na Terra'' proclama Carolina
Maria de Jesus. Como outrora das poesias de Stadler e Heym, emerge as
visões horripilantes de uma grande cidade, em meio à qual ``o fedor de
carne e peixes putrefatos\ldots{} crianças esfarrapadas berrando sobre
pobres brinquedos\ldots{} enquanto ao longe a cidade ecoa no estrondo da
autoestrada''.

Depois de três anos de uma persistente batalha o repórter conseguiu
encontrar uma editora para a sua ``descoberta proveniente da selva
humana'': uma editora conservadora que até aquele momento tinha
publicado apenas livros escolares e que então queria arriscar uma ``nova
linha (editorial)''. A elite brasileira -- intelectuais, jornais e redes
de televisão, professores e estudantes, ministros e deputados -- recebeu
com uma reação relâmpago a descarga elétrica deste desmascaramento de um
mundo que nas capitais significa um problema social de dimensões
catastróficas; somente no Rio há duzentas favelas; apesar disso se trata
de um mundo escondido, fechado e omitido que na vida cotidiana é banido
de nossa consciência.

``Assim vivem as pessoas na favela?'' perguntaram muitos Cândidos que de
tal miséria não tinham ``noção alguma''. E o desmascaramento desta
realidade desconhecida -- Sartre disse no Rio de modo lapidar:
``Copacabana é apenas a janela, a realidade por trás dela é a favela''
-- segue rapidamente uma tragicomédia sul-americana que emparelha em uma
confusão barroca o grotesco com o comovente.

Em uma semana venderam-se de uma só vez dez mil exemplares - um recorde
no Brasil -, e logo esse diário desalojou Graham Greene, Bertrand Russel
e o mais popular escritor nativo, Jorge Amado, da lista dos mais
vendidos para os lugares abaixo. O importante jornal conservador \emph{O
Estado de São Paulo} fala de um dos melhores livros brasileiros deste
século, o liberal \emph{Diário de Notícias} o denomina ``uma bofetada
estrondosa no rosto da administração brasileira''. Ao mesmo tempo os
comunistas agarram avidamente o material de propaganda excelente e
inesperado contra a ``decadência do sistema capitalista no Brasil''.

Durante toda a semana a autora negra diariamente teve de dar entrevistas
na televisão, falar sobre problemas sociais nos encontros e aparecer
como estrela em inúmeros coquetéis em livrarias de luxo, onde ela
amigável e sorridentemente escrevia dedicatórias pessoais em seus livros
para senadores e até mesmo para o Ministro do Trabalho. Na escolha da
\emph{Miss São Paulo}, ela recebeu, como convidada de honra, a coroa da
rainha da beleza. Na visita de uma elegante casa noturna ela teve de
subir ao pódio a fim de deixar-se aplaudir. Os alunos da prestigiada
Faculdade de Direito da USP em São Paulo nomearam essa mulher, que
frequentou por apenas dois anos a escola primária em uma aldeia afastada
do interior, membro de honra da Faculdade -- um título que deveria ser
concedido a Sartre; Carolina de Jesus, contudo, foi preferida pelos
futuros advogados, porque ela seria ``incomensuravelmente mais valiosa
na luta pela liberdade'' do que o filósofo do existencialismo.

O estrangeiro se interessou por este -- visto historicamente -- talvez
atrasado protesto de um proletariado oprimido, que trouxe junto à
literatura o exótico colorido local dos trópicos e do meio negro. Do
Japão e da França, Alemanha e América do Norte chegaram com velocidade
de entrega postal ofertas em dólares: todas dirigidas à mulher, cujo
manuscrito havia sido recusado por treze editoras, dentre as quais a
norte-americana \emph{Reader's Digest}, com a lacônica observação ``sem
interesse para nós''. Revistas ilustradas com milhares de exemplares,
\emph{Life}, \emph{Paris-Match} e outras solicitaram entrevistas
exclusivas. A outrora coletora de sucata e papel velho viajou então com
seus três filhos através de todo o Brasil; o sonho de sua vida, o de
possuir uma casa, realizou-se com os seus crescentes direitos autorais.
No Rio, como atração da Feira do Livro, ela hospedou-se no mesmo hotel,
que havia sido atacado alguns anos atrás com pedradas, porque este hotel
havia recusado abrigar a cantora negra norte-americana Marian Anderson.

A segunda parte da comédia teve lugar no napolitanamente despreocupado e
ensolarado Rio, que está em forte contraste com a cidade industrial de
São Paulo milanesamente firme e obstinada. Consciente de seu propósito,
a mulher, que literalmente apanhava sua comida cotidiana da lixeira dos
ricos, caminha em direção a sua ``nova vida''. ``Eu gostaria de imitar
os ricos a fim de que eu possa inseri-los em meu próximo livro'',
afirmou a obcecada escritora pouco antes do jantar em sua homenagem
junto ao governador do Estado. Ela começou a ``imitar'' em seu segundo
livro o seu novo meio circundante, no momento em que a sua conta
bancária aumentou até aproximadamente dez mil dólares.

Como atua o mundo da burguesia culta e especialmente da \emph{High
society} brasileira na enérgica, simples, mas perspicaz mulher, que no
sentido mais verdadeiro da palavra alcançou um lugar ao sol? Maria
Carolina de Jesus reagiu sobretudo com a mesma sinceridade desarmada com
a qual ela lidava com pessoas influentes e companheiros de pobreza:
``Tirem as mãos do dinheiro do povo, bons homens. Deem-no antes aos
pobres. Vocês já são tão gordos, como vejo, deixem os favelados
engordarem um pouquinho''. No restaurante mais chique do Rio, \emph{Au
Bom Gourmet}, reúne-se toda a multidão ao redor dela: ricos industriais,
mulheres que compram seus vestidos de dois mil dólares na Dior ou na
Givenchy, frívolos críticos sociais, partidários dos príncipes Orleans e
Bragança, todos a queriam tocar, admirar, falar com ela, estar ao redor
dela, de modo semelhante a como, no século XVI, se comportavam os nobres
da corte francesa ao avistaram os primeiros índios brasileiros. Carolina
Maria de Jesus ceticamente balança a cabeça ao deixar o local. ``Nenhuma
dessas pessoas refinadas daria um único centavo aos negros esfarrapados
que mendigam aqui nas ruas. Elas falaram comigo sobre caridade e
injustiça social simplesmente porque elas gostam de se deixarem
fotografar junto a uma personalidade importante quando isso sairá nos
jornais. Isso é apenas palavrório vazio. Quando elas finalmente se
tornarão humanas? O dinheiro estrangulou nelas a compaixão pela dor
alheia?''

Significativo é também como ela caracterizou os círculos intelectuais,
que em Paris ou no Rio aprendem insidiosamente a caluniar pelas costas:
``A maldade de nossos intelectuais é como a nuvem: infinitamente
refinada, mas penetra em alguém através dos ossos''. Frequentemente ela
reflete sobre as causas sociais e culturais da favela, frequentemente
ela sente um profundo sentimento de solidariedade com os ``irmãos que eu
deixei na favela''. Isso, muito embora os vizinhos dela, furiosos acerca
dos retratos desprovidos de beleza que ela fez deles, à sua partida
atiravam pedras nela. De vez em quando ocorrem a ela observações
filosóficas, cuja essência poderia ter ocorrido a Marco Aurélio ou a
Bertolt Brecht, conforme elas possuam uma tonalidade elegíaca ou
revolucionária: ``Preconceito de raça\ldots? O ser humano é tão efêmero
sobre a Terra\ldots{} ele devia durante a sua viagem mundana viver em
paz e não odiar nenhum dos seus próximos..'' Ou: ``Os pobres são
tratados tão mal, que eles obrigatoriamente devem renunciar a seus
bens..'' E: ``Se o sol pertencesse à Terra, ele seria destinado ao
privilégio de poucos seres humanos..''

Sente-se ela de fato mais feliz agora ``na outra margem, a margem rica
da vida?'' Há pouco uma peça foi encenada inspirada em seu livro; um
assistente de De Sica acena com uma oferta para a filmagem de sua saga
de miséria. Agora seus filhos pela primeira vez podem comer bem e com
regularidade, podem ir para a escola calçados, podem dormir em colchões.
O novo esplendor descoberto custa um banho quente. Como um ser humano
religiosamente determinado ela sente profunda gratidão a Deus; dado que
agora ela ``vive no Paraíso''; ela se alegra com o fato de que ela
poderia fazer o bem a milhares de pessoas: ``Como o sol que é um só e
apesar disso envia o seu calor a todos igualmente.''

Logo, contudo, ela teve de reconhecer sobriamente que sua nova vida era
um simples Purgatório, e que por trás das máscaras dos ``protetores'',
``salvadores'', ``conselheiros'' e ``amigos'' ela descobre os
interesseiros, os parasitas, os inventores necessitados de capital. Se
ela já ficava horrorizada no bairro miserável com a depravação dos seres
humanos -- agora a impressiona a depravação ética de uma sociedade
irresponsável que finalmente, como ela percebeu, é corresponsável pela
pobreza de milhões de pessoas.

``Às vezes eu reflito sobre isso: na favela há seres humanos toscos que
são ignorantes. Aqui há rivalidades, cobiça. Nenhuma sinceridade\ldots{}
Eu me sinto como se eu vivesse em um mundo de joias e ornamentos
falsos.'' Ao redor do nome deles, cujo ``bem é valioso'', disputam
políticos, publicitários, imprensa, associações de negros; com uma
imagem ela ilustra a sua nova posição: ``Eu tenho a impressão que eu sou
um cadáver e que os abutres aguardam ao meu redor. Os carniceiros
humanos têm fome de dinheiro.''

E assim se encerra como em uma balada o círculo de sua vida: os mundos
extremos se tocam por um instante: Carolina, a mulher andrajosa, janta
com um membro da família Matarazzo. Nisso deveria residir o valor de seu
livro? No contato entre as camadas distintas que compõem a sociedade
brasileira? Ou, sobretudo - como especialmente alguns estrangeiros a
partir de uma falsa perspectiva afirmam -, talvez o seu livro sirva para
a dissolução institucional da favela Canindé? Desconhecendo os traços
demagógicos dessas medidas atrasadas e insuficientes, eles esquecem as
incontáveis favelas da América Latina nas quais milhões ainda vegetam e
nas quais continuamente se devoram como um câncer ulceroso.

Igualmente falsa é a afirmação otimista de que o diário dela seria uma
espécie de ``Cabana do Pai Tomás'' brasileira. Qualquer um que conheça
as condições brasileiras e que leia esse importante testemunho, diria
antes que ele tem o valor social de um ardente protesto contra a
administração e a oligarquia brasileiras. Mais ainda: Provavelmente ele
é a chave para o entendimento da explosiva realidade latino-americana --
que facilmente pode conduzir para soluções ao estilo de Castro.

Carolina Maria de Jesus é uma autêntica representante do povo
brasileiro: o amor dela pela justiça e pelos pobres, a religiosidade
dela e o estoicismo de sua vida o testemunham -- mas também o profundo
ceticismo dela em relação à intenção dos políticos brasileiros. Todavia,
ela se diferencia da massa por meio de sua capacidade de discernimento,
que permite que ela tome partido de modo coerente em relação a esta ou
aquela personalidade, graças a sua recusa da superstição e graças ao seu
amor pela instrução.

Talvez nós encontremos o sentido profundo de seu documento à miséria
humana, modesto e quase analfabeto, em uma surpreendente analogia com a
obra de Kafka - naturalmente em outro nível do dizer literário. Não
apenas na célebre citação de Kafka: ``Eu escrevi assim, porque eu vi a
vida assim!'' se fundamenta essa inesperada afinidade. Também na
semelhança da temática por eles tratada se baseia a aproximação entre
ambos habitantes solitários de guetos sociais e culturais: é a descrição
do medo e da angústia. Sem dúvida, a cronista da favela se ocupa de um
medo físico por sua sobrevivência biológica, enquanto o criador do
inalcançável ``castelo'' sentia angústia metafísica. Nas suas confissões
Carolina escreve: ``Há pessoas que desesperam da vida e somente pensam
na morte como solução. Eu me defendo sempre contra isso na medida em que
eu escrevo o meu diário.'' Como um eco distante soam as palavras de
Kafka: ``No ato de escrever há uma consolação especial, enigmática,
talvez perigosa, talvez salvadora\ldots{} Talvez a literatura leve à
oração..''

O livro da brasileira testemunha similarmente uma crença absoluta na
transcendência da palavra, na sua força de mudar o mundo circundante,
como ela muito concretamente deveria vivenciar. Como um \emph{negro
spiritual} esse livro amargo contém ao lado de tanta tristeza uma faísca
de consolação quando Carolina afirma que: ``O ser humano não nasce
despido - veste-o a esperança.''

\chapter{O negro nos livros, poemas e
teses}\label{o-negro-nos-livros-poemas-e-teses}

Jornal da Tarde, 1982/12/25. Aguardando revisão.

\hfill\break

\begin{quote}
``África! Áfirca da reconquista das liberdades/ África do Negro,/ Não há
ninguém na África'' (Bernard Dadie, poeta da Costa do Marfim)
\end{quote}

Livros com enfoques diversos tornam este ano que termina talvez aquele
que, na última década, mais refletiu sobre o negro e mais revelou sobre
a Negritude. O luxuoso, informativo e rico primeiro tomo de uma
\emph{História Geral da África}, nas suas quase 800 páginas e redigido
por cientistas das mais diversas nacionalidades, remonta às origens das
diversas culturas e civilizações africanas (publicação da Editora Ática
com a cooperação da Unesco).

\emph{A Abolição} de Emília Viotti da Costa (Editora Global, 101
páginas) é um estudo interessante do processo abolicionista do Brasil,
embora mutilado por sua visão \emph{a priori} e dogmática de que a
abolição da escravatura representou no Brasil a já surradíssima e
obsoleta teoria marxista de uma ``luta de classes''\ldots{}

\emph{África, o Povo} de Carlos Contini (Editora Achiamé, 114 páginas)
pareceu-me uma tediosa enumeração das etnias africanas, precedidas do
documento míope e mecanicista da UNESCO que acredita ``cientificamente''
na desigualdade econômica e social como \emph{única} origem do racismo,
teoria que, convenhamos, também vem indiretamente do legado empoeirado
de Marx \& Co., hoje fundamentalmente corrigido em 180 graus pela
psicologia, pela antropologia cultural e outras, estas, sim, ciências
humanas que não pretendem ser a demonstração forçada de um teorema em si
redutivo e pobre.

Mais importante, porque tem uma visão mais plural e mais abrangente das
relações entre as raças no Brasil, é um livro-chave: \emph{Fala,
Crioulo}, de Haroldo Costa (261 páginas, Editora Record). São
depoimentos colhidos pelo jornalista carioca Haroldo Costa junto às
personalidades mais diversas do segmento negro que compõem,
marcadamente, a etnia brasileira. Já lucidamente, no seu prefácio, o
escritor Jorge Amado destacara o elemento afetivo como impulso vital
para a miscigenação, abandonando todos os ``ismos'' de um cientificismo
pedante e ideologicamente desfigurador.

Assim, um professor de Direito Civil e Romano, de 65 anos, José Pompílio
da Hora, alude a uma das alavancas que poderia, realmente, contribuir
para a emancipação e a elevação do negro, depois do primeiro passo que
foi a Abolição, em 1888. É lógico que a educação, nos países das
Américas, é fruto de classes dominantes, de origem europeia, branca. É
lógico também que o eurocentrismo sempre erigiu \emph{a sua} cultura e
suas premissas como \emph{o único critério} pelo qual se pode avaliar a
inteligência e o avanço de um povo. Portanto, a educação, no Brasil,
teria como tarefa primordial \emph{liberar-se da obsessão tecnológica}
(grifo meu) que classifica povos e civilizações inteiras segundo apenas
a óptica de seu possuem ou não siderúrgicas, estaleiros navais, armas
mortíferas, enfim, se ultrapassaram a primeira revolução industrial,
iniciada na Inglaterra há cerca de 200 anos. Mais ainda: os livros
escolares deveriam ser revistos profundamente para não acolher mais a
História distorcida que nos é ensinada e da qual o negro é o grande
ausente. Embora reconhecendo que por enquanto a educação é feita
\emph{por brancos para brancos}, ou os negros e todos os outros grupos
(japoneses, chineses, coreanos etc.) se submetem a esse processo
uniformizante ou colocam ao lado das premissas brancas ocidentais alguns
de seus valores próprios, que se chocam com o utilitarismo materialista
da sociedade de consumo que nos é imposta a todos (brancos, asiáticos,
negros e mestiços) pelas agências de publicidade, manipuladoras da
psique de milhões de passivos espectadores.

Além destas considerações que faço em torno da meta da educação do
negro, proposta pelo professor Pompílio da Hora, o deputado federal
Adalberto Camargo, 58 anos, vê na ascensão política um dos instrumentos
de conscientização e defesa dos grupos brasileiros negros. Ele argumenta
que assim como a vinda de um presidente da Itália ou de um soberano
japonês ao Brasil mobiliza multidões de brasileiros descendentes de
italianos ou de japoneses, por que multidões igualmente numerosas de
brasileiros descendentes de africanos não acorrem ao aeroporto para
saudar o presidente do Senegal, o magnífico poeta da \emph{négritude},
Léopold Senghor?

Dom José Maria Pires, arcebispo de João Pessoa, 62 anos, enfatiza o
total descaso com que a Igreja sempre encarou a questão da escravatura
no Brasil: ``Fazendo causa comum com os dominadores, a Igreja nunca
esteve ao lado dos negros em suas lutas de libertação''. Enquanto,
comentário meu, hipocritamente a Conferência Nacional dos Bispos do
Brasil (CNBB) se coloca, depois de quase dois mil anos de existência da
Igreja (diríamos um tanto tardiamente?) ``ao lado'' dos que hoje chama
de ``pobres e oprimidos'' e lança até um Conselho Indígena Missionário,
por que nunca houve uma Pastoral do Negro? Além da omissão, setores
importantes da Igreja como a Companhia de Jesus achavam que ``não é
escandaloso pagar as nossas dívidas em escravos, pois eles são a moeda
corrente do país (página 236), como as próprias Constituições Primeiras
do Arcebispado da Bahia incluíam, como impedimento para o
sacerdócio:''Se tem parte de nação hebreia, ou de qualquer outra
\emph{infecta} (grifo meu) ou de negro ou mulato'' (página 237).

Enquanto a advogada e orientadora educacional, 56 anos, Gracierre
Ferreira da Costa pergunta: por que as luxuosas Escolas de Samba
cariocas não abrigam, ao lado das fantasias de carnaval para quatro dias
de deslumbramento, verdadeiras \emph{escolas} que durante o ano letivo
inteiro alfabetizassem os negros e lhes ensinassem algum ofício útil na
vida? E faz uma referência dolorosa ao desamparo, que considera
duplamente terrível, ``da mulher negra, sem instrução, sem marido,
sacrificada, enganada por brancos e negros, essas mães solteiras que não
têm amanhã nem para si nem para os seus filhos''.

Pelé, Édson Arantes do Nascimento, que aos 17 anos de idade já era a
suprema glória do futebol brasileiro na Suécia, hoje com 40 anos de
idade, repetiu sempre a adesão a partidos públicos e tem sido
sistematicamente agredido e manipulado por uma parte da imprensa rígida
que queria a sua adesão cega a ideologias do Partido Único. Pelé é o que
melhor harmonizou toda a temática negro e branco. Ele a engloba de forma
duplamente fecunda: tem noção de, com o esporte, ter feito alguma coisa
pela sua raça (modéstia de quem recebeu em Paris, em 1981, o prêmio
inédito de \emph{O Esportista do Século}). Por outro lado, ele não tem
nenhum resquício de racismo às avessas, vingativo, e, sim, a grandeza e
nobreza de visão de dizer: ``Eu não tenho problema dentro de mim, dentro
do meu coração, contra o branco apenas por ser branco. Aprendi sempre a
valorizar o homem através do seu pensamento e das suas ações''.

A \emph{Antologia Contemporânea da Poesia Negra Brasileira} (Editora
Global, 103 páginas), subintitulada Axé e organizada por Paulo Colina, é
um contraponto às palavras de Guerreiro Ramos, que servem de epígrafe o
livro anterior, de Haroldo Costa (\emph{Fala,} \emph{Crioulo}):

\begin{quote}
``É preciso não carregar a pele como um fardo''.
\end{quote}

Conselho fácil de ser dado teoricamente, mas e na prática diárias?

Na \emph{práxis} da sua poesia, grande número dos poetas aqui reunidos
tem como inspiração constante o choque plural de ser negro num país que,
felizmente, está distante da África do Sul com seu anti-humano e
criminoso \emph{apartheid} e até mesmo da selvageria da discriminação
instituída, por exemplo, nos Estados do Sul dos Estados Unidos, mas, ao
mesmo tempo, saber que a alforria não significou uma ascensão social. De
ignorar as suas origens, perdidas na diáspora africana. De sonhar com
quilombos hoje inexistentes. De cantar impelido por uma melancolia
talvez incurável na sua desesperança existencial.

Pelo menos nesta amostragem, que às vezes inclui poemas esplêndidos, os
poetas dissentem dos depoimentos em prosa, que estoicamente aceitam a
luta contra a discriminação e apontam saídas para a emancipação
democrática dos brasileiros de cor negra. Seriam os poetas os profetas
de uma vingança futura ou retardatários de evocações doloridas, mas hoje
atropeladas pelas reivindições diárias de toda uma comunidade -- branca,
negra, asiática, mestiça, índia -- heroicamente em luta pelo seu quinhão
justo na sociedade brasileira?

Ninguém poderia emitir \emph{um juízo de valor}, sociológico ou
político, abusivo, se aplicado à criação artística. O fato é que poetas
mineiros como Adão Ventura, gaúchos como Oliveira Silveira ou paulistas
como Abelardo Rodrigues e Cuti representam, concretamente, uma poesia
nova, inédita no Brasil. Vistos sob um prisma internacional, esses
criadores de admiráveis obras-primas da poesia estão cronologicamente
(ou fora do tempo?) afinados com a poesia reivindicatória de um
antilhano, Aimé Césaire, de um senegalês, David Diop; dos
norte-americanos Countee Cullen ou Sterling Brown. Opondo-se ao fraterno
perdão de Senghor, eles mascam a sua fúria na esperança de um amanhã
sangrento e mais tarde igualitário ou abandonam-se introspectivamente a
uma visão da vida que equivale à impotência, à resignação, à lembrança
do passado dos quilombos ou da paisagem africana que lhes foi
arrebatada. Invariavelmente são versos pungentes, fortes, concisos,
admiráveis:

\begin{quote}
``Tratocracia''~

Ele Semog (poeta carioca)~

Quebraram-lhe todos os dentes~

E suas costelas~

Furtaram-lhe a alma~

E a dignidade também~

Mas lhe deixaram a loteria~

Pois sabiam que a miséria~

Não se toma de ninguém.~

Ou, de Oliveira Silveira, do Rio Grande do Sul:~

``Casas de Negros''~

``\ldots{} laranjeiras, currais e algumas casas de negros''~

Saint-Hilaire~

casas de negros~

queixa e resmungo~

casas de negros~

cantigas do Congo~

casas de negros~

feijoada e charque~

casas de negros~

santo e orixá~

casas de negros~

reza e batuque~

casas de negros~

palavras em choque~

Nas casas de negros~

coisas escravas~

passando livres~

para a casa-grande.
\end{quote}

\emph{Com o poema} ``Charqueada'' a noção irônica, estoica, épica, só
sofrimento atinge um alto nível emotivo com um destacamento, um
distanciamento de si mesmo similar ao de atores que sigam esse
alheamento proposital ensinado por Bertold Brecht:

\begin{quote}
\begin{itemize}
\tightlist
\item
  Os negros estão despidos~
\end{itemize}

senhora pelotense~

trabalhando no sol.~

\begin{itemize}
\tightlist
\item
  Os negros estão desnudos~
\end{itemize}

senhora pelotense~

trabalhando no sal.~

Eles vieram de longe~

de campos tão distantes~

repontados pela estrad~

com seus mugidos fundos~

brancos homens de preto a tocá-los~

e um ponteiro a chamar: Venha, venha!~

Eles vieram~

poleangos assim~

e foram embretados~

e passaram por todas as facas~

pelo sal~

pelo sol~

senhora pelotense~

e chegaram a pretos velhos~

com as marcas na pele~

Na carne~

na alma~

senhora pelotense~

charqueados.
\end{quote}

Essa originalidade coesa em poucos e simples versos sem atavios
grandiloquentes e inúteis transforma-se em tristeza niilista em Éle
Semog:

\begin{quote}
PANO DE BOCA~

De repente, assim, assim,~

Num passe de mágica~

Com uma fome atávica~

Comeram todas as palavras!~

Ou:~

IMPASSE~

às vezes a vida~

me passa de reluz~

como um imenso~

e inevitável funeral.~

ÀS MINHAS CUSTAS~

Tudo que sei do número treze~

É que é o grupo do galo~

E que é o dia de azar.~

Tudo que sei de liberdade~

é continuar escapando~

Da penitenciária~

Pois não existem quilombos~

Para me guardar
\end{quote}

Abelardo Rodrigues ousa mesclar a uma imagem que normalmente despertaria
piedade um tom grotesco de títeres que caem pelo chão, torpes,
conduzidos por mãos inábeis, misturando o Carnaval e o ritmo de
submissão que é uma forma de autonegação para um macabro ritual sádico e
masoquista da subjugação e do engodo, ao lado do escárnio e da perda da
dignidade humana: ``Agora choraremos/ em compassos e apitos/ nossos
passos/ de caranguejos./ Pesadamente como o tanque/ miraremos o canhão
para nosso espelho/ até sermos felizes/ como viúva saciada''.

É lógico que os demais poetas dessa antologia variam na qualidade da
forma e conteúdo de seus versos. Inegavelmente, há participantes,
outros, dessa coletânea que nada têm de poetas, mas, sim, de oradores
empolados de uma retórica babosa, ou pessoas, que confundem poesia com a
letra inconsequente de um sambão tradicional. É inevitável que, ao lado
de grande e legítimos talentos, que crescerão com o tempo, haja vocações
fracassadas e que deveriam abandonar o cultivo da poesia por um cultivo
mais rendoso, das rosas que abundam em seus pseudopoemas, às bactérias
que infirmam a banalidade dos que têm ou bisonhamente julgam que têm o
que dizer. É o caso palpável, por exemplo, do, digamos, poeta paraibano
Arnaldo Xavier. Usando de artimanhas gráficas que já tinham cabelos
brancos quando E. E. Cummings, o poeta menor norte-americano que se
recusa a usar maiúsculas e confundia \emph{layout} gráfico com
experiência poética. De que adianta, realmente, colocar entre parênteses
invertidos as palavras )Agonia(, )Grito(, )Medo(, )Tristeza(, se o final
é um cretino:

\begin{quote}
Ei-la aqui.~

Ei-la aqui!~

Ei-la aqui!?~

Ei-la AQUI.~

\begin{itemize}
\tightlist
\item
  Não há mais como camuflar a Dor.
\end{itemize}
\end{quote}

Em sua, como diremos, poesia intitulada ``Até o Mais Herético dos
hereges reza quando ama'', o candidato de Musas esclerosadas começa
declarando:

\begin{quote}
Quando falo~

:)Amor(~

Soa falso como uma Árvore~

Se esboço um gesto de Luz~

)redijo(~

Escuridão~

Em todos os movimentos~

Na Certezabsurdônika: DE QUE,~

ESTAMOS MORRENDO A CADA MOMENTO
\end{quote}

Com tanta imbecilidade disposta como o diagrama da própria estultice, o
rapaz/homem/senhor depois de admitir com maiúsculas que ``não vomito, Eu
sou o próprio vômito'', reconhecimento cínico impecável pela certeza da
diagnose ele se põe a inventar uma ``Niicanção'' (que será), ``Estrelas
Quadrúpedes'', ``Sóis Bípedes'' para, linhas penosas, adiante, relatar
que ``Dor foi expulso/ por cuspir no rosto do Goleiro Amor/ e troca de
pontapés com Paixão'', resultado de uma partida nefasta \emph{contra} a
inteligência, a renovação poética autêntica, a sensibilidade e a cultura
do leitor e, quem sabe?, soberbo ``deleithe'' (à moda do autor) por sua
superioridade intrínseca? Afinal, em outro, que termo se pode usar?
Criação de sua larva ele ``inova'' com termos como ``O Pássaro
Homem/Só/Queria Voar no gerúndio'' e por medo de contágio de tanta perda
atônita de neurônios linha após linha o leitor, com um suspiro de
alívio, passa a outro bardo.

Esta antologia não teria desculpas que pedir por incluir alguns nomes
que ombreiam com os maiores criadores de \emph{kitsch}, pois no Brasil,
afinal, o \emph{kitsch} serve até para eleger à Câmara dos Deputados, em
Brasília, o ``poeta'' J. G. de Araújo Jorge, fora outras manifestações
Certezabsurdônika que galgaram, através do voto, o Planalto ou até mesmo
a Assembleia diante do Parque Ibirapuera, nas últimas eleições.

Como a literatura é por essência democrática, não caberia ao crítico
mostrar-se didático, complacente nem intolerante para com esses poetas
desiguais no talento e na obtenção de suas metas poéticas. No entanto, o
que se constata, visivelmente, é que uma parte decisiva da população
brasileira, a que tem uma origem étnica africana, dia a dia desfaz os
chavões que, como esparadrapos amorfos, são colocados no negro como
futebolista, carnavalista, garanhão erótico, temperamento infantil,
quando não malandro, assaltante e quimbandista. Essa antologia comprova
que, se há, certamente, uma temática negra no vasto repertório da poesia
brasileira -- e em alguns casos, parece-me, da melhor qualidade - , ela,
inexoravelmente, se funde com a população de outras etnias, na luta por
um avanço efetivo da Massa de todo o povo brasileiro, na qual está
fundida, inevitavelmente, seu segmento negro. Não esmorecer diante do
muro dos preconceitos, da prepotência, da injustiça e da ignorância,
insuflar à fisionomia do Brasil o traço que mais caracteriza a raça
negra: a dignidade. Atualizar os livros que omitem sua participação
decisiva na construção deste país, atingir níveis altos de educação e
manter os valores ancestrais de sua cultura. Haverá ideal que mais se
coadune com os brasileiros de origem africana?

\chapter{Poesia. Uma obra reunindo poetas negros de várias
épocas}\label{poesia.-uma-obra-reunindo-poetas-negros-de-vuxe1rias-uxe9pocas}

Jornal da Tarde, 1986. Aguardando revisão.

\hfill\break

\emph{A Razão da Chama} (Edições GRD), apesar de cuidadosamente
organizada pela competência do poeta Oswaldo de Camargo, deixa uma
impressão estranha de já lida, já percorrida. São vários poetas negros
brasileiros, de Caldas Domingos Barbosa a Luís Gonzaga Pinto da Gama, de
Cruz e Souza e Lino Pinto Guedes a Solano Trindade, até chegar aos novos
poetas contemporâneos: Cuti, Paulo Colina, Abelardo Rodrigues, Éle Semog
e outros mais. A impressão que se tem é a de uma releitura, ampliada, da
coletânea \emph{Axé}: a temática é estreita e se reduz à constelação de
que a epiderme escura é discriminada monstruosamente no Brasil. Dessa
verificação inicial se passa à gama inteira de sentimentos que ela
causa: revolta pela discriminação, ódio e vingança, orgulho e
autenticidade étnica e reconciliação e perdão pela invenção e exercício
do racismo pelos brancos e pelos que se consideram brancos no Brasil
etc.

É pouco.

Evidentemente que a humilhação causada por um preconceito, a revolta
contra a injustiça e o \emph{parti pris} de atitudes baseadas nos mais
vis instintos do ser humano são elementos que bloqueiam a expressão de
outros temas. Mas, se nos detivermos, como criadores artísticos, apenas
nessa barreira, como poderemos focalizar os mil aspectos da vida de que
participam todos os seres humanos de qualquer cor? A celebração da
natureza, a discussão de ideologias políticas mas com isenção de
posturas dogmáticas, o erotismo e o lirismo, a dificuldade das
inter-relações humanas, a discussão de assuntos ecológicos, de perigos
inerentes à instalação de usinas nucleares, os temas sociais, a
meditação filosófica sobre o efêmero da vida humana, a angústia
implícita nesta condição, os combates contra regimes políticos e
econômicos iníquos -- há literalmente um sem-número de motivos a serem
tratados pelos nossos poetas negros, fora desse esboço aqui traçado.

James Baldwin, Lima Barreto, Aimé Césaire, Senghor, Soyinka e tantos
mais romperam a prisão da discriminação e criticaram as suas respectivas
sociedades, às vezes de forma áspera e eficaz e essa ampliação temática
me parece ser o próximo passo do poeta negro no Brasil, demonstrada já
fartamente no genuíno talento de que são dotados. E fica ainda no ar a
pergunta: Mário de Andrade e Cassiano Ricardo, ambos mulatos, não são
incluídos na antologia justamente por não terem falado apenas desse fato
em suas poesias? A minha é apenas uma opinião e como tal pode estar
profundamente errada, mas julgo que o artista negro -- e não só no
Brasil -- em uma tarefa infinitamente mais ampla: a de reumanizar o
campo das expressões estéticas como comprava a sua atuação vital,
decisiva, no campo da música e da dança: a experimentação que faz um
Ishmael Reed com a linguagem, nos Estados Unidos, não é sumamente
interessante e renovadora, recordando a de Raymond Queneau, na
literatura francesa?

Por último, sempre estive convencido de que é através dos textos
escolares e dos meios de comunicação de massa que se pode combater pelo
menos com alguma eficiência a discriminação. Quando se fizer justiça
histórica ao negro e à sua contribuição a todos os campos da atividade
humana, não será mais difícil a inoculação de racismos torpes por parte
de adultos ignorantes e acometidos de lepra moral?

Nesta antologia, o tom galhofeiro da sátira corrosiva se inicia, com
graça e vigor, com a epopeia cômica de Luiz Gonzaga Pinto da Gama: ``Lá
Vai Verso''. À maneira de Camões e outros vates renascentistas, ele
anuncia:

\begin{quote}
``Quero a glória abater de antigos vates,~

Do tempo dos herois armipotentes;~

Os Homeros, Camões -- aurifugentes~

Decantando os Barões da minha Pátria!''
\end{quote}

A lista que se segue se assemelha muito à idiotice da campanha pública
destes dias que antecedem as eleições de 15 de novembro, com candidatos
disputando o título de Cretino Máximo, sem deixar de lembrar os
escândalos de fraudes de ex-ministros da Justiça, de deputados que votam
dolosamente por companheiros ausentes da Câmara em Brasília, a corrupção
vertiginosa -- e impune -- que corrói o Brasil como cupins ou saúvas
famintas:

\begin{quote}
``Com sabença profusa irei cantando~

Altos feitos da gente luminosa,~

Que trapaça movendo portentosa~

A mente assombra, e pasma à natureza!~

Espertos eleitores de encomenda,~

Deputados, Ministros, Senadores,~

Galfarros Diplomatas-chupadores,~

De quem reza a cartilha da esperteza''.
\end{quote}

A repulsa pelos que traficam no infame comércio negreiro, a comicidade
dos pedantes e postições ``doutores'' de sabença empolada e divertidas
variações em torno do termo de intenções pejorativas -- bode -- para
designar os que não são brancos temperam os versos de ``Quem sou eu?''
já dentro de um clima quase surrealista e estonteante.

Cruz e Souza exige de seus admiradores uma sintonia com seu vocabulário
luxuriante (revel, mucilaginoso, lutulentos), até mesmo em seu famoso
poema de revolta contra os que não podiam associar as palavras artista e
negro, ``Emperedado'', mas uma leitura mais atenta descobrirá um
surpreendente clima baudelairiano de contrastar o substantivo e o
adjetivo, criando uma sensação original no leitor, como no trecho em que
ele se refere às crianças negras abandonadas como ``tenebrosas flores''
em ``Crianças Negras'' e sua veemência de uma eloquência por vezes
retórica, discursiva e hiperbólica. Já Lino Guedes, um poeta menor, se
distingue pelo seu tom emocional, por certo generoso: ``Assim esqueço o
castigo/ que recebi de sua mão'', estendendo a mão fraternal de paz para
quem foi seu carrasco; mas impressiona que ele consinta em aderir a
ideias recebidas sem questionamento, que o levam a aconselhar o negro a
ser ``um homem direito'', de cujo ``proceder'' apenas surgirá uma
modificação de seu \emph{status}. Que exemplos, que modelos tinha o
negro -- e ainda tem -- em profusão para ``ser direito''? Imitando as
trapaças e a iniquidade dos brancos? Fechando-se no gueto da
superioridade racial, mito que alimenta, desde tempos imemoriais, um
povo como o japonês, por exemplo? E qual é o ``proceder'' que deve ser
seguido? O estoico? O servil? O cordato com a injustiça?

Carlos Assumpção introduz um estilo certamente mais requintado e
eloquente quando se alça à altura dos versos:

\begin{quote}
``O alicerce da nação~

Tem a pedra dos meus braços~

Tem a cal das minhas lágrimas~

Por isso a nação é triste~

É muito grande mas triste''
\end{quote}

Uma dor que se espraia pelos poemas de Oswaldo de Camargo -- ``Estou no
meio de vós/ como a peste no ombro da desgraça, / como o laço da
garganta do cativo/ ou a tristeza que amansa vossa mão..'' e atinge, com
o gaúcho Oliveira Silveira, um \emph{pathos} comovedor:

\begin{quote}
``Treze de maio traição~

iberdade sem asas~

e fome sem pão~

Treze de maio -- já dia 14~

a resposta gritante:~

pedir~

servir~

calar''
\end{quote}

Paulo Colina, um poeta a meu ver já se afirma em plena maturidade do seu
canto, diz, porém, as palavras que ecoam as considerações iniciais
feitas acima, com uma concisão e propriedade filosófica certeiras:

\begin{quote}
``ser marginal todavia~

bastaria ao poema apenas~

a cor da minha pele?''
\end{quote}

Com exceção de Abelardo Rodrigues e de Éle Semog, os demais poetas não
apresentam, nesta antologia (nem o incisivo Cuti), nada de
extraordinário e é com a expectativa de futuras coletâneas, melhores,
que fechamos esta agora, magra e insatisfatória no seu todo.

\chapter{Nossa poesia negra, tentando falar
alemão}\label{nossa-poesia-negra-tentando-falar-alemuxe3o}

Jornal da Tarde, 1990/03/17. Aguardando revisão.

\hfill\break

Na Alemanha e na Suíça, um volume pequeno com o título bilíngue
\emph{Schwarze Poesie, Poesia Negra} divulga, nos países de língua
alemã, a poesia brasileira feita por poetas negros. E, francamente, aí
reside seu único mérito, pois a introdução (de Moema Parente Augel) está
eivada de todos os chavões possíveis e, ai de nós, com as mais
prosaicas, menos poéticas traduções para o alemão de Johannes Augel.

Quem procurar este pequeno livro terá de importá-lo da Edition Diá (com
endereços na Suíça e na Alemanha). Vale a pena, reitero, quase que
unicamente pelas páginas da esquerda, em que são otimamente selecionados
alguns dos supremos poetas afro-brasileiros contemporâneos.

Em artigo da revista trimestral norte-americana \emph{The Black Scholar}
já fiz referência, na seção \emph{Biblioteca} que redijo às
sextas-feiras neste jornal, à importância e relevância da abrangente
coletânea de versos em suahili, em português, em inglês, do Haiti, da
área de todo o Caribe, da África - enfim em todos os locais, quase, da
Diáspora negra imposta brutalmente pela escravidão.

No original e não nas capengas traduções, reitero, estão os versos
refinados e sutilmente inteligentes, para começar, de Cuti. Com requinte
verbal e sem ser pedante jamais, Cuti pede:

\begin{quote}
``Leva

a lava leve do meu vulcão~

pra casa~

e coloca na boca do teu~

se dentro do peito~

afogado estiver de mágoa~

O fogo de outrora~

do centro da terra~

virá sem demora~

Porque não há~

por completo~

vulcão extinto no peito'' (``Oferenda'')
\end{quote}

Eloquente, vibrante, ele alterna o desafio a ferro e fogo e a doçura
sonhada de um mundo justo em ``Esperança'':

\begin{quote}
``Há uma esperança decisiva na ponta do fuzil:~

a morte ou a vida enriquecida~

aquecida de amor e comida.~

Há uma esperança levantada nos punhos fechados:~

a morte ou a vida cheia de vida~

plena de igualdade e verdade.~

``Há uma esperança na faca da sombra:~

a morte ou a vida dos meninos~

meninas homens mulheres e os sinos.~

Há uma esperança de tocaia na fúria:~

a vida crivada de sonhos~

de balas de mel na boca do mundo''
\end{quote}

Se a poesia vigorosa de Cuti enlaça-se com a prosa imorredoura de Martin
Luther King em sua mensagem esplêndida \emph{I Have a Dream} (``Eu tenho
um sonho'') e com a ameaça de vingança de \emph{Da Próxima Vez, Fogo!},
de James Baldwin, nem por isso ele deixa de falar dos ``modelos''
brancos impostos à raça negra depois da libertação da escravatura do
preconceito nesta nossa inexistente ``democracia racial'' mas
``hipocrisia racial''. É verdade que não temos o nazismo do
\emph{apartheid} monstruoso da África do Sul nem o ódio racial que
talvez a maioria dos cidadãos brancos assume perante seus conterrâneos
negros, notadamente nos Estados do sul dos Estados Unidos.

Evidentemente, neste artigo delimitado fortemente pelo pouco espaço, no
entanto a violência dos versos de Oliveira Silveira não pode ser
esquecida:

\begin{quote}
``Um charque esta alma retalhada~

um charque esta alma ressentida~

um charque esta alma aqui~

um charque~

charque sal~

charque sol~

charque sul~

esta carne rasgando-se sem lâmina~

este sangue ancestral ferindo ardendo~

esta alma negra sal e sol nos lanhos~

um charque~

charque sal~

charque sol~

charque sul~

você sabe uma faca abrindo fendas~

na carne um raio um terremoto um mar~

de sangue pelo meio uma alma repartida~

um charque~

charque sal~

charque sol~

charque sul''
\end{quote}

A veemência já intolerante de mesuras e mentiras explode igualmente nos
versos incendiários, drásticos de Adão Ventura como

\begin{quote}
ALGUMAS INSTRUÇÕES DE COMO LEVAR UM NEGRO AO TRONCO~

``Levar um negro ao tronco~

e cuspir-lhe na cara.~

levar um negro ao tronco~

e fazê-lo comer bosta.~

levar um negro ao tronco~

e sarrafiar-lhe a mulher.~

levar um negro ao tronco~

e arrebentar-lhe os culhões.~

levar um negro ao tronco~

e currá-lo no lixo.''
\end{quote}

Houvesse mais vagar, não cometeríamos a injustiça de não focalizar
outros poetas importantes, de linha mais urbana como Oswaldo de Camargo;
Éle Semog; a angústia existencial de Paulo Colina, provavelmente mais
liberto dos temas de escravidão e abolição da escravidão; o sarcasmo
cortante de Abelardo Rodrigues. Nesta antologia sumamente feliz na
escolha dos versos e poetas, não poderia deixar de ser mencionada, ainda
que por último, a voz impressionante, decisiva de Lourdes Teodoro, que
vê a perspectiva urbana e a contrasta com o passado de Quilombos e
Palmares de forma indelével:

\begin{quote}
BALADA DEL QUE NUNCA FUÉ A PALMARES~

``Somos pivetes,~

balconistas,~

assaltantes,~

e quantos mais~

que de Palmares nem~

ares~

que de Palmares~

só os ais~

helicópteros,~

Eerrepês,~

patrulhas,~

volks-w,~

sobre favelas, baixadas,~

vilas e areais,~

metralhadoras,~

trinta e oitos~

pistolas e pontapés,~

socos e beliscões.~

Salve 20 de Novembro~

eu, de Palmares~

nem os ares,~

eu de Palmares,~

só os ais.''
\end{quote}

\chapter{O Evangelho da Solidão de Eduardo de
Oliveira}\label{o-evangelho-da-soliduxe3o-de-eduardo-de-oliveira}

O Estado de São Paulo, 1970/7/23. Aguardando revisão.

\hfill\break

Na parte final desta sua coletânea de poemas, Eduardo de Oliveira
transcreve a opinião hiperbólica de Tristão de Ataíde, que o considera
``o novo Cruz e Souza'' brasileiro, a par de outra que o define como
``um triste sonhando coisas lindas\ldots{} (seu livro) é um porto de
miragens, gemendo a insatisfação milenária dos poetas.'' A mesma
indecisão caracteriza sua utilização dos versos:

A parte numericamente maior de seus poemas prende-se à forma parnasiana
e à expressão de sentimentos melancólicos (``de tristeza em tristeza me
transporta/ esse pesar de que não me liberto/ e cuja dor meu peito não
suporta'') ou de sofrimento virtuoso como \emph{laissez passer} para o
céu (``É preciso sofrer. Sem sofrimento/ a humanidade não se purifica/ A
dor se esvai um dia e o bem que fica/ nos há de dar conformidade e
alento.'') ou de confissão amorosa igualmente soturna (``Penso em você,
quando a tardinha desce/ triste e chorando, como estou agora./ Penso em
você, quando desperta a aurora/ que vem da noite que desaparece).

Mas a parte nitidamente melhor de sua poesia encontra-se, sem dúvida,
nos acentos de revolta pessoal em que o poeta de cor evoca a África de
sua origem com ``Tumbeiros do Além'' que se inicia:

\begin{quote}
``Eu sou um pedaço d'África~

jogado no chão do mundo.~

Tumbeiros malditos~

Tumbeiros do Nilo~

Tumbeiros-Saara~

Tumbeiros do Caos~

Tumbeiros-Tumbeiros~

Tumbeiros do Além.''
\end{quote}

Essa sinceridade emotiva é porém desvirtuada por influência de Gonçalves
Dias na métrica:

\begin{quote}
``Nas plagas distantes~

a que me atiraram~

tristezas chegaram~

cravando-se em mim.~

nas terras do norte o negro é fantasma~

terrível miasma~

de angústias sem fim.''
\end{quote}

Essa adesão a métricas cerceadoras do ímpeto expressivo está aliada a um
tom em certos pontos condoreiro, que recorda as ``Vozes d'África'' de
Castro Alves, com sua retórica declamatória hoje caída em desuso. São
esses enganos, frutos talvez de leituras voltadas para o passado, de
valores já consagrados mas cristalizados em sua época específica, que
prendem seu voo poético.

Estas observações não querem dizer que o poeta deva se limitar,
forçosamente, aos temas raciais. Significam somente que é na temática
brotada da \emph{négritude} de Aimé Césaire e de Léopold Senghor que
Eduardo de Oliveira encontra sua maior força. Uma força incerta, que
descamba para o lugar-comum de efeito:

\begin{quote}
``Se o negro levanta~

seu porte de ébano~

é eletrocutado~

é decapitado~

a bem do país~

a bem da nação~

que um dia com sangue~

ajudou a construir.''
\end{quote}

Mas uma força que adquiri um ritmo e uma expressão próprias, quando o
poeta não interfere \emph{intelectualmente} na sua confecção, mas, como
queria Rimbaud, deixa que seu canto flua instintivamente:

\begin{quote}
``Bocas negras~

negras vozes~

que têm fome de justiça e de música.~

Almas negras~

negras preces~

que se prolongam num mistério de sombras de infinito.~

Cantos negros,~

negros hinos~

\begin{itemize}
\tightlist
\item
  todos feitos de banzo e de atabaques.~
\end{itemize}

Luzes negras,~

negras luas~

caídas numa bola de paz e de dor que vem das Áfricas.~

Olhos negros~

negros prismas~

projetando futuros e mocambos.~

Belas negras,~

negras prenhes~

de castas fecundações que os sóis não trazem.~

Corpos negros,~

negras frontes~

que dão manhãs escuras de alegrias.~

Sonhos negros,~

belos sonhos~

com soluços de noites e pedaços do meu povo.''
\end{quote}

O contacto com esta raiz expressiva e com os exemplos vigorosos da
poesia negra -- de Langston Hughes aos poetas contemporâneos do Congo --
poderá trazer uma diretriz certa às suas hesitações poéticas, tornando-o
uma expressão inédita da nossa poesia virgem -- a da negritude
brasileira original.

\chapter{Prefácio ao livro de Paulo
Colina}\label{prefuxe1cio-ao-livro-de-paulo-colina}

In COLINA, Paulo. A noite não pede licença, Roswitha Kempf Editores,
1987. Aguardando revisão.

\hfill\break

Dramaturgo, tradutor, animador de encontros culturais na União
Brasileira de Escritores, em São Paulo, o poeta Paulo Colina tem uma
personalidade artística nítida, forte, a destacá-lo do conjunto de
importantes poetas negros contemporâneos no Brasil de hoje.

A poesia de Paulo Colina já ultrapassou, há muito, o tom de queixume
derivado do angustiante preconceito racial: a sua poesia não é um muro
de lamentações sobre o passado da ignomínia -- a chibata, a senzala, o
navio de escravos, o estigma, a orfandade da Abolição de 1888.
Contemplativa, cheia de meditações filosóficas em forma de metáfora
poética, a poesia de Paulo Colina me parece, sobretudo, a, de uma
sensibilidade plural, moderna, que se depara com o contexto urbano,
repetição da crueldade do Brasil rural transformada na pobre geometria
de nosso \emph{sky line} urbano, a buscar o lucro e o logro nos céus.

Habitante da sua época e da sua cidade escolhida, Paulo Colina tem a
inventividade como passo seguro para não cair na banalidade tumular do
lugar-comum, do sentimentaloide:

\begin{quote}
Infinda gravidez de ausência~

no ventre da cidade
\end{quote}

ele decifra na desolação cinzenta da cidade-acampamento à beira do
dividendo, da Bolsa de Valores em alta ou em baixa, do mercado de
empregos, essa forma ``moderna'' de pelourinho em mim, e não se examinam
os dentes do escravo, lhe impõem porém o acorrentamento do seu tempo,
pés e mãos atados a escritórios onde a fraude, dia a dia, goteja e faz
fortunas.

\begin{quote}
O som arrastado dum carro de bois

nos ombros largos da noite
\end{quote}

ecoa como volta obsessiva do passado retido em gravuras de Rugendas e
Debret, até que o corte lancinante de um avião que possa ``fabrica'' as
manchetes dos jornais:

\begin{quote}
o jato leva e traz o dia seguinte

rompendo a barreira do nosso sonho
\end{quote}

o poeta, como desterrado de toda as repúblicas, desde a Cidade concebida
por Platão, impregna estes versos (que têm a rara dádiva de uma
abstração que não se prende ao piegas, ao kitsch do palavrório oco que
no Brasil muitas vezes passa por ``poesia''\ldots) de uma originalidade
expressiva característica.

Seria possível destacar linhas soltas que inauguram uma novidade de ver,
sentir e exprimir:

\begin{quote}
o limo do tempo apenas conserva

o fogo campeia a memória
\end{quote}

ou imagens sugestivas, de feitura aparentemente simples, espontâneas,
como:

\begin{quote}
Abrir as mãos

e soprar a pena

sentimento do mundo
\end{quote}

Lúcido, o poeta se vê, recolhendo o que resta de seus anseios,
esperanças desfeitas, limitações impostas externamente:

\begin{quote}
Sou todo cacos de vidro
\end{quote}

E nós todos, buscando no bar, no álcool, no esquecimento, estarmos
refletidos estoicamente nos ``espelhos do Nada''. Esse naufrágio
coletivo da ``Nave dos Tolos'' medieval que somos nós, os seres humanos,
captados em nossa insignificância, não escapa ao poeta invadido pela
melancolia, em meio ao modismo de letreiros luminosos em inglês na
noite. Seu cosmopolitismo, que o leva a conhecer, em inglês, poetas e
prosadores decisivos do nosso tempo como a traduzir, com ajuda de outrem
poesias de um outro artista desgarrado da vida, japonês, pobre,
possivelmente ignorante do orgulho étnico de seu povo que se considera
uma \emph{Herrenrasse} (uma raça de dominadores) pois descende, só ela,
de deuses, aquele infeliz Takuboku Ishikawa -- tudo isso não faz perder
a seiva paulistana, contemporânea, desse poeta que livro a livro, poema
a poema, amadurece. Os grilhões de um passado histórico que lhe foi
roubado permanecem: Paulo Colina não é um ``alienado'' como entoariam em
coro as vozes que seguem mais o ``materialismo histérico'' do que
``histórico''. Consciente de que a Princesa, quem sabe, esqueceu-se de
assinar a carteira de trabalho que completaria a Abolição, ele
sutilmente ironiza o carnaval como epopeia da raça, da mesma maneira que
não veria no futebol, creio, a glorificação do artilheiro negro.

Mais elevada e menos efêmera é esta poesia que começa da constatação da
quarta-feira e cinzas cotidiana, longe das passarelas, do ópio da cor,
das luzes, do som e dos aplausos da multidão. Quero crer que Paulo
Colina forja um canto muito mais abrangente e que desafia um período de
tempo tão escasso e tão artificialmente celebrado. As suas conquistas de
estilo, a concisão e o impacto de seus versos nos asseguram que ele
depura, cada vez mais, o seu ritmo contemporaneamente sincopado,
traspassado de \emph{blues} -- uma voz autêntica da ebulição do Brasil
deste final da década de 80 e que todos os prognósticos tranquilamente
indicam que será uma das vozes decisivas da poesia brasileira a fincar a
sua inspiração de talento, inteligência e vigor neste solo nem sempre
mãe gentil. Mas não é o destino dos poetas autênticos da modernidade, de
Baudelaire a Fernando Pessoa, tecerem seu canto como quem se sobrepõe às
correntes adversas de um tempo sombrio, o mais ferozmente armado de
códigos e látegos para ``disciplinar'' o poeta?

\chapter{Paulo Colina - o poeta das
cinzas}\label{paulo-colina---o-poeta-das-cinzas}

Revista Goodyear, n.47, 1988. Aguardando revisão.

\hfill\break

``Que Abolição temos que comemorar? A Princesa Isabel, talvez por
ingenuidade, cedeu à pressão dos grandes latifundiários. Daí por diante,
enquanto milhões de europeus chegavam para substituir o trabalho escravo
no negro, o Brasil `branqueava' sua população. O negro da senzala foi
atirado à favela, à marginalidade, à fome, ao biscate, à ignorância.''

Incisivo, o poeta paulista Paulo Colina, 38 anos, é um dos artistas
brasileiros que mais se destacou na célebre \emph{Antologia dos Poetas
Negros Contemporâneos} organizada por Oswaldo de Camargo, o incansável
animador e memória dos movimentos culturais negros do Brasil. A poesia
de Paulo Colina (\emph{A Noite não Pede Licença}, \emph{Entre Dentes},
\emph{Plano de Voo} e \emph{Fogo Cruzado}) não é um muro de lamentações
que abandona o presente e vai se colocar no passado da chibata, do navio
negreiro, do pelourinho. Melancólica meditação sobre a perversão do
mundo e do nosso pobre \emph{sky line} urbano, a cultivar o lucro e o
logro à custa do próximo, ela é cosmopolita e não banal ou
sentimentalóide:

\begin{quote}
``infinda gravidez de ausência~

no ventre da cidade''
\end{quote}

Decifra na desolação das massas penduradas do dividendo, atadas a
empregos reles onde a fraude incha e faz fortunas, um tom de
\emph{blues} resignado a meditar sobre o infortúnio da condição humana e
capta:

\begin{quote}
``o som arrastado dum carro de bois~

nos ombros largos da noite''
\end{quote}

Não é um artista preso ao passado da senzala, a modernidade rasga seus
versos como nota dominante:

\begin{quote}
``E o jato que leva e traz o dia seguinte~

rompendo a barreira do nosso sonho''
\end{quote}

e soa uma originalidade austera, concisa na decifração da realidade:

\begin{quote}
``o limo do tempo apenas~

conserva;~

o fogo campeia a memória''
\end{quote}

até uma dubiedade de dor e voo (``pena'') a unir a tristeza e a saudade
da liberdade:

\begin{quote}
``Abrir as mãos~

e soprar a pena~

sentimento do mundo.''
\end{quote}

Lúcido o poeta resume em si as nuances de uma sensibilidade de seus
anseios e esperanças:

\begin{quote}
``Sou todo cacos de vidro.''
\end{quote}

O centenário que este ano se comemora não é propriamente o da Abolição,
pois o negro não teria como festejar a abolição do preconceito nem dar
vivas à sua habitual orfandade neste país de um mentiroso clima de
``democracia racial''. A discriminação é sutil, ardilosa, diária,
enrosca-se em pretextos fúteis, absurdos mas está sempre presente. A
recusa polida de uma moça em querer dançar ``com uma pessoa de cor'', os
empregos que milagrosamente ``já foram preenchidos'' assim que um negro
se candidata a eles, o passageiro de metrô ou de ônibus que pede licença
e se levanta quando vem sentar-se ao seu lado uma pessoa ``diferente''.
Quando ainda era rapaz em Pirituba, zona noroeste de São Paulo, Paulo
Oliveira -- depois Colina, em homenagem à cidade onde nasceu -- jogava
partidas de futebol, ``peladas, num time misto de brancos, negros,
mulatos e até um nissei''.

Hoje sorri: ``Foi aos poucos que comecei a compreender certos silêncios,
certos afastamentos mudos e me dei conta de que a pobre e solidária
Pirituba da minha infância não era o''mundo, vasto mundo'' de que fala
Carlos Drummond de Andrade. Mas o que tinha de vergonhosa a minha cor de
pele? Fazia diferença se meu cabelo e minhas feições eram diferentes
daqueles meninos filhos de italianos, de húngaros, de portugueses, que
todo dia brincavam comigo de bola de gude, de mocinho e bandido, de
guerra?''

Começou para ele um difícil aprendizado: o do ABC do racismo. Um
preconceito velado, disfarçado. ``Hipócrita, usemos logo a palavra
certa'', Colina acrescenta. ``Eu nunca tinha notado que era de repente
um marginal contra a minha vontade e contra o meu conhecimento. Da noite
para o dia eu passara a ser um negro jovem `atrevido', metido a falar
certo, rodeado de livros. Eu só podia, para muitas pessoas, ser
malandro, quem sabe ladrão ou preguiçoso ou, com a fama que os negros
têm, um garanhão perigoso, em busca de moças brancas. Minha mãe, a vida
inteira cozinheira de patrões brancos, que moravam em mansões dos
bairros nobres como Morumbi ou os Jardins, me dizia sempre que diante de
Deus todos têm a mesma cor. E rezava muito. Meu pai, motorista
particular, não era homem de muitas palavras. Mesmo assim, atribuía o
preconceito à `ignorância das pessoas'. Mas nem minha mãe nem meu pai
sabiam explicar aquela atitude contra mim: o que eu fizera de mal? Eu me
achava `normal' como os outros mas me ensinaram que não, eu não era não.
Nós tínhamos vindo de Colina, uma cidadezinha do interior do Estado,
para São Paulo, metrópole, atraídos pelo sonho de uma vida melhor, de
empregos bem pagos. Nossa vida tinha altos e baixos. Mais baixos, aliás.
Sempre que possível, mandávamos buscar avós, tios, primos, a parentada
toda. Mas a situação estava piorando, o dinheiro cada vez mais curto. Eu
ficava deslumbrado com os anúncios luminosos do centro da cidade, com as
livrarias de vitrinas atulhadas de livros coloridos. A realidade era que
eu tinha que poupar ao máximo meu único par de sapatos; havia sempre
tantas contas pra pagar! Com tudo aumentando de preço na feira e no
armazém, lápis, papel e livro viraram luxo para mim.''

Mesmo depois que Paulo Colina começou a trabalhar, primeiro como
\emph{office-boy}, depois na firma de produção de alimentos onde sua mãe
se empregara como cozinheira e lhe arranjara uma colocação, as coisas
não melhoraram. Havia um filho de portugueses que o detestava e fazia
tudo para tornar sua vida no escritório um inferno. ``Ele não gostava de
pretos'', era a explicação que não explicava nada. ``Colegas da minha
raça procuravam me animar, me aconselhando a `não ligar para isso' ou
dizendo que `a gente se acostuma, depois nem liga mais'. Quando chegava
o domingo, havia trégua naquela guerra sem pé nem cabeça, eu pegava um
livro de Lima Barreto que comprara num sebo -- sempre fui rato de
livraria de segunda mão, fuçava tudo que podia -- e saía para olhar o
tio Tietê. Ia ler ou reler aquele negro que para mim fora um
deslumbramento. Antes do Lima Barreto eu não imaginava que existiam
escritores negros a não ser o Cruz e Souza. E logo tratei de aprender
inglês, quem sabia inglês conseguia empregos bem remunerados e podia ler
tantos autores também..''

Poetas, romancistas, ensaístas começaram, em edições de bolso, a
enfileirar-se em sua estante no quarto. ``Muitas vezes eu preferia
passar o almoço com um copo de leite e um sanduíche que levava de casa
pra poder comprar livros de Leroy Jones, Richard Wright e hoje Toni
Morrison, essa analista profunda e sutil do mundo negro em \emph{Song of
Solomon}, por exemplo. Então, eu raciocinava triunfante, os negros não
eram só bestas de carga, a se curvar e dizer servilmente `sim, sinhô'
para os brancos, nem eram apenas craques de futebol ou sambistas.
Consciente da minha falta de preparo, dos meus erros e tropeções,
comecei a garatujar uns versos. Hoje reconheço que eram horríveis! Mas
eu queria escrever, expressar o que sentia. Além de poeta eu queria ser
dramaturgo, criar peças para o Teatro do Negro, escrever romances sobre
o que o Lima Barreto chamava de `negrice', muito, muito tempo antes de
se falar da `\emph{Négritude}' com os poetas africanos e antilhanos.''

Atualmente, ele acha, o negro continua explorado, depois da fase em que
o samba, o jongo, eram coisas de morro, dava polícia em cima, samba era
o folclore da favela pendurada sobre a paisagem do Rio de Janeiro,
barracão de zinco ``cantado'' como ``beleza comovedora'', implorando
clemência à cidade a seus pés. Lorotas! Paulo comenta: ``Agora, o
Carnaval virou indústria. Indústria que rende para os bicheiros, pelo
menos em termos de prestígio, mas rende para o Estado também, trazendo
divisas estrangeiras para o tesouro nacional. São desfiles milionários,
com cinco mil pessoas, muito luxo do tipo Hollywood. A negrada fica na
bateria, que é coisa de negro mesmo, ou então empurra os carros
alegóricos no muque e dança e dança quando pode: negro não é forte,
negro não é só músculos e sexo? O Carnaval serviu também para dar
emprego a milhares de artesãos que o ano inteiro confeccionam
fantasias''.

Pára para repetir a frase de sentido duplo: ``Confeccionar
fantasias\ldots{} é, essa é a função do negro carnavalesco hoje. Forjar
a fantasia de que a vida é mansa. Vamos ser Maria Antonieta e o Rei da
França durante três, quatro dias de apoteose? Vamos comemorar a
Abolição, essa ficção que não houve? Da `liberdade' de não saber ler, de
não ter formação profissional, nem de um plano financeiro monarquista ou
republicano para apoiar o negro e formá-lo para o mercado de trabalho
ninguém cogitou. Resultado: o negro `sobrou' na nova sociedade `livre'
como `sobrou' antes. Foi ser capinador na roça, abrir valas para esgotos
na cidade, viver de `bicos' ou, como ladrão, arrancar ela força aquilo
que era seu por direito e lhe negavam. Sempre um marginal, sempre o
último a ser contratado e o primeiro a ser despedido. É o jogo das
classes dominantes, ou tem sido até agora, não é? E por que nenhuma
escola de samba milionária pensa em fundar uma \emph{escola} de verdade
onde se dê instrução para os negros? Já dizia Noel Rosa que''samba não
se aprende no colégio'', mas toda a situação do negro no Brasil mudaria
por meio da educação, essa alavanca que iria destruir até as favelas.''

Tornado insuportável dia-a-dia na firma onde a mãe se curvava horas e
horas sobre panelas e caldeirões, Paulo Colina entrou numa grande
companhia importadora e exportadora japonesa e ainda hoje, em outra
empresa, trabalha neste ramo. Era a única pessoa de cor entre nisseis,
em sua maioria, e um ou outro brasileiro de outra origem étnica.
Traduziu com um colega Masuo Yamaki, os versos do gênero \emph{tanka} --
mais popular que os refinados \emph{hai-kais} de Bashô -- do poeta
Takobuko Ishiawa. ``Eu senti logo uma afinidade surpreendente com ele.
Era um poeta contemporâneo, que se defronta com a modernidade, um poeta
urbano, angustiado, sozinho, que bebia muito, frequentava prostíbulos e
morreu jovem. Eu sentia muitas das tristezas e incompreensões que ele
sofreu. Traduzir seus versos era quase uma tarefa fácil para mim, embora
muito dolorosa.''

Apolítico, o poeta brasileiro desmente a frase de Khuane Nkrumah, ``a
não-violência é anacrônica'' e acredita, ao contrário, que a violência
só aumenta os problemas. Seguidor de Martin Luther King, o grande líder
religioso e social dos Estados Unidos, que inspirado na doutrina da
não-violência, de Ghandi, libertou as massas negras oprimidas, ele
exemplifica a espiral de violência com a década de 60 no Brasil,
``quando as guerrilhas causaram mais mortes inúteis em porões de
torturas e no campo. Está provado que a violência, definitivamente, não
é a solução''. O radicalismo de Malcom X ou dos Panteras Negras não é a
resposta, assim como lamenta quem queira emigrar para um país da África
Negra: ``O brasileiro negro que for para a Nigéria ou para o Quênia, por
exemplo, vai se deparar com choques tribais, com hostilidade, com uma
falta quase total de possibilidades de se realizar profissionalmente, em
países onde a luta pelo pão ainda precede a luta pelo lápis e o papel, a
reconstrução dos países devastados pelo colonialismo ainda é muito
recente e árdua. A nossa luta é aqui. Somos brasileiros há 400 anos,
aqui é que demos nosso suor, nosso sangue, por que escapar? É uma
ilusão.''

Sem se deixar rotular de pessimista, ele vê o ser humano ainda em um
estado muito primário, cheio de brigas, de preconceitos. ``O homem, em
geral, ainda é um bicho nocivo. Nocivo para seus semelhantes e para a
natureza, que destrói. Pra ser sincero não vejo, rigorosamente, um único
país civilizado na face da terra onde os homens vivam fraternalmente. É
uma busca frenética do poder, ainda que para obtê-lo seja preciso pisar
na jugular do outro. Claro, quanto ao racismo, o Brasil não é a África
do Sul, onde reina o pavor, mas estamos muito longe, mesmo hoje em dia,
de uma verdadeira democracia. Nossa democracia, e não só do ponto de
vista racial, não passa de uma balela, um engodo. Francamente, eu
preferiria que o racismo brasileiro não fosse disfarçado. Que fosse como
nos Estados Unidos onde o preconceito é às claras, mas onde a lei está a
favor do negro, há recursos legais para combater a discriminação. Os
negros norte-americanos já têm uma classe média com alto poder
aquisitivo, acesso à educação universitária e cargos eletivos de
prefeito, governador ou, na economia de mercado, a postos de executivos
de firmas importantes. E aqui? A Lei Afonso Arinas? É mais uma prova de
que existe o racismo, senão a lei não precisava existir. E ela é inócua:
quem a viola é denunciado, vai à delegacia mais próxima, paga uma multa
insignificante e fica tudo por isso mesmo\ldots{} Agora, na nova
Constituição, parece que a discriminação é definida como crime
inafiançável. Só que no capítulo das chamadas minorias os evangelistas
se colocaram contra a proteção a proteção dos homossexuais. Por que eles
estão excluídos? Será que não são filhos de Deus lá na Bíblia deles?''

As vocações de Paulo Colina e James Baldwin coincidem em sua relutância
em limitar seus escritos a ensaios anti-racistas como \emph{Da Próxima
Vez, Fogo!} ou a fazer uma poesia panfletária, obcecada apenas por um
tema: ``A visão que se tem de dentro do gueto negro, eu digo sempre, é
uma visão menor da realidade, que é múltipla e muda rapidissimamente.
Não posso bater só nessa tecla: sou negro. Gosto de ser negro. Amo
minhas raízes africanas. Não sou um disco quebrado. Quem for escrever,
negro ou branco, tem que ter uma única coisa indispensável: talento. No
`movimento negro' temos de tudo. Há negros bajuladores de Maluf, do
Jânio e, antigamente, do Adhemar de Barros. Dedicam livros babosos aos
poderosos e querem usar sua subliteratura rastejante para subir
socialmente. Basta de o negro eternamente `morder o granito' das
estátuas que povoam as praças das cidades brasileiras.'' Seu é o sonho
de Martin Luther King transportado para o Brasil: o de um dia haver uma
efetiva democracia brasileira, quando a cor da pele do indivíduo não
pesar mais na avaliação do seu valor, do seu caráter, da sua dignidade:
``Será que estou sonhando alto demais?''

Recusa-se a banalizar palavras como solidariedade ou povo, despejadas de
qualquer palanque político e toda a propaganda governamental. Afinal,
sua poesia e sua ação social estão longe dos desfiles do ``reinado de
Momo''. Estão fincadas firmemente na realidade da Quarta-Feira de
Cinzas, ``de olhos bem abertos e sentidos alerta''.

\part{Literatura Norte-Americana}

\chapter{Literatura negra nos Estados Unidos - James
Baldwin}\label{literatura-negra-nos-estados-unidos---james-baldwin}

Correio da Manhã, 1965/3/27. Aguardando revisão.

\hfill\break

O cartão entregue a cada um dos milhões de refugiados da Europa Central
acampados em barracas à espera do visto de imigração é simbólico da
nossa era atômica -- a era das \emph{displaced persons}. A expressão
inglesa é cruel na sua especificação de que uma pessoa está ``fora do
seu lugar'', no desajuste, no desenraizamento que são a forma século XX
da angústia mais desesperada que o \emph{mal du siècle} dos românticos e
o \emph{Weltschmerz} dos melancólicos poetas alemães. Kafka foi o arauto
dessa multidão em busca de uma Canaã mítica -- perdida, inacessível ou
inexistente? -, dos judeus que \emph{displaced} na Alemanha de Hitler,
terminaram cremados nos infernos de Dachau e de Ausschwitz. Mas são
também desambientados os jovens irados ingleses e os \emph{beatnicks}
que procuram no orgasmo e no Zen budismo, na maconha e no \emph{jazz}
uma expressão para o seu inconformismo e finalmente são \emph{displaced}
em seu sexo os travestis mimetizados com um ideal feminino.

Nos Estados Unidos, um décimo da sua população simboliza, de forma
candente, esse \emph{status} moderno, fruto amargo do desenraizamento
violento do seu \emph{habitat} africano: os negros. Libertados da
escravidão ao preço de uma Guerra Civil e emancipados pelo sinistro
resgate de seu mais luminoso defensor -- Lincoln -, os negros americanos
assumiriam uma posição \emph{sui generis} entre os negros de todo o
mundo.

Não lutavam, como seus irmãos africanos, pela libertação política, nem
afirmavam, através da \emph{Négritude}, a sua fisionomia cultural
autônoma. Constituindo educacionalmente a ``elite'' dos negros em
qualquer país, compartilhando em grande parte a prosperidade econômica
americana nada tinham em comum com os \emph{Simbas} canibais do Congo
belga ou com os bantus, de nível pré-histórico, vítimas da fanática
\emph{apartheid} sul-africana. Os negros dos Estados Unidos encontram-se
na situação paradoxal de serem os que mais assimilaram da civilização
ocidental, nela se integrando inteiramente, e ao mesmo tempo os
elementos menos acatados, em geral, por essa parte da sociedade
ocidental que é o povo norte-americano, sobretudo nos Estados do Sul. Os
negros que forjaram o destino da América viam-se agora despojados, como
``cidadãos de segunda classe'', dos direitos elementares aceitos pelas
comunidades em que moravam, de poderem decidir dos rumos políticos e
econômicos da nação que era sua também e, em última instância, das leis
que regeriam as suas próprias vidas. Dentro da gama de saídas para esse
impasse e para esse absurdo, surgiram movimentos cívicos que vão desde
os que exercem uma ação persuasiva na defesa dos direitos do negro, até
os que, como os \emph{black muslins} (muçulmanos negros) preconizam a
destruição violenta da raça branca. Dentre todos, destaca-se o reverendo
Martin Luther King, Prêmio Nobel da Paz do ano passado, pela sua
grandeza moral, pela sua lucidez, pela sua fé inquebrantável na doutrina
da não-violência inspirada em Gandhi. O extraordinário líder protestante
opõe à doutrina do ódio e do terror as suas marchas da liberdade, a
pressão sobre as cúpulas governamentais, a prece aliada à ação.

James Baldwin encarna nessa autêntica ``guerra fria'' nacional o papel
do intelectual autêntico, do novelista de grande sucesso, do ensaísta
brilhante. Muito superior à Richard Wright, ele supera também a
influência dos artistas de \emph{jazz} e até dos atores negros porque,
ao contrário destes, não apela para a sensibilidade artística dos
brancos apenas mas também para o seu raciocínio, para a sua
introspecção. Talvez mais importante como ensaísta e pensador do que
como ficcionista, suas argutas considerações sobre a situação do negro
nos Estados Unidos publicadas na prestigiosa revista \emph{The New
Yorker} e logo depois enfeixadas em livro permanecerem 29 semanas
consecutivas à frente da lista de \emph{best-sellers} do seu país,
\emph{The Fire Next Time}, título desse abalador testemunho, foi
aclamado na Inglaterra, na França, na Itália, na Alemanha, enquanto as
mais importantes universidades norte-americanas convidavam seu autor a
pronunciar conferências perante seus milhares de alunos e os editores
europeus duelavam com cifras e telegramas par obter os direitos de
tradução de suas obras.

\emph{The Fire Next Time} (\emph{Da Próxima Vez, Fogo}!) não granjeou ao
escritor de 39 anos apenas fama mundial -- uma capa da revista
\emph{Time}, um número especial da publicação liberal inglesa
\emph{Encounter} -- revelou a fundo uma situação humana contundente para
o homem branco pela acusação veemente que encerrava, o \emph{De
Profundis} de um pária que falava com a linguagem compreensível aos
opressores. Amaríssimo na sua diagnose e na sua rebelião, este volume
violento não preconiza, porém, uma solução violenta. Seu apelo final e
sóbrio, altivo, digno, em prol da cooperação das duas raças na tarefa
que lhes é comum de banir a barbárie da intolerância, da ignorância
cifrada no preconceito, da desumanidade expressa pela discriminação. É
para impedir um \emph{Palmares} de proporções imprevisíveis entre os
fanáticos dos ``muçulmanos negros'' e os sequazes da Ku Klux Klan
sulista que ele adverte contra os Hitlers em potencial, os Lee Oswalds
sempre prontos a assumir um soturno papel nas catástrofes da História.
Ainda há esperança no seu brado final, que encerra seu anátema contra a
humilhação imposta pelos brancos: ``Tudo agora, devemos supor, está em
nossas mãos, não temos o direito de supor outra coisa. Se nós -- quero
dizer, nós, os brancos e os negros relativamente conscientes, que temos
de insistir junto à consciência dos outros -- se nós não esmorecermos
agora no cumprimento do nosso dever, poderemos, embora sejamos poucos,
por termo ao pesadelo racial e concretizar o destino do nosso País e
mudar a História da humanidade. Se não ousarmos tudo agora, o
cumprimento daquela profecia, recriada da Bíblia numa canção composta
por escravos, recairá sobre nós:''Deus deu a Noé o sinal do arco-íris --
Basta de água, da próxima vez: fogo!''

Embora repelindo as teorias violentas e apelando para a cooperação entre
as raças, Baldwin reconhece não só a sua solidão como intelectual negro
dentro da massa negra como também a inadequação, a insuficiência das
palavras diante dos fatos, numa conclusão semelhante à de Sartre quanto
à impotência do escritor, como escritor, diante da injustiça social. De
fato, a História registra as modificações trazidas às vidas dos povos
pelos Césares, pelos Napoleões e pelos Hitlers; mas a modificação
aportada pelos Shakespeares e pelos Prousts não se traduz nunca em
vitórias militares, em conquistas territoriais, em fatos que podem ser
medidos estatisticamente. A mais inflamada peça de Brecht, mais
inspirado poema de Lorca ou o mais inteligente ensaio de Sartre jamais
conduziu um público à tomada de Bastilha alguma. Ao reconhecer que não
tem ``poderio político nem econômico'', Baldwin intrinsecamente
especifica quais são as alavancas suscetíveis de causar qualquer
modificação concreta. Como \emph{A Cabana do Pai Tomás}, como os
discursos de Joaquim Nabuco e José do Patrocínio, os seus livros podem
``acelerar'' a conscientização de um problema, formulá-lo, expô-lo,
denunciá-lo, nunca solucioná-lo. (O trágico fim da nossa Inconfidência
política urdida por poetas comprova a disparidade de meios para a
alteração de uma situação de fato.)

Lucidamente, porém, ele lança mão da sua forma de participação, sem
jamais transformar as suas observações penetrantes em matéria
panfletária, primariamente didática e ineficaz. Volta do exílio
voluntário em Paris, depois de nove anos, para enfrentar as ``condições
normais'' de vida do negro americano no seu país de origem.

Essas ``condições normais'' foram vividamente descritas por um homem
branco, John Howard Griffin, que após intensa radiação com iodo e
raspagem do cabelo conseguiu adquirir uma aparência de negro,
dirigindo-se ao Sul dos Estados Unidos. Nos Estados ultra-racistas do
chamado \emph{Deep South}, ou sejam o Mississipi, o Alabama e a
Louisiana, ele experimentou durante semanas o estigma candente. Além das
humilhações constantes e diabólicas das salas de espera só para brancos,
dos bebedouros e bancos de jardim só para brancos, dos restaurantes em
que lhe era recusada comida, dos postos de gasolina onde lhe era negada
a venda de gasolina devido à pigmentação de sua pele -- além de todas
essas humilhações sórdidas e mesquinhas que marcam a fogo a
sensibilidade dos negros, havia sempre presente o medo de ser linchado,
a expectativa acuada das palavras ofensivas, dos olhares de desprezo e
dos risos de chacota coletiva.

James Baldwin ultrapassa essa incursão que terminou quando cessou o
efeito das radiações, pois para ele é válida não a experiência, mas a
vivência determinada biologicamente, por fatores genéticos imutáveis. O
seu libelo não deve atingir só os brancos racistas norte-americanos, mas
a totalidade da raça branca, herdeira involuntária do delito da
escravidão, partícipe, contra a sua vontade dessa injustiça, já que
integra com os americanos brancos a coletividade de raça branca:

``A brutalidade com que os negros são tratados na América simplesmente
não pode ser exagerada. A princípio, o negro \emph{não acredita} que os
brancos o tratem dessa forma\ldots{} Ainda não existe uma linguagem
sequer para descrever o sofrimento pessoal do negro neste país..''

Agora, James Baldwin forjou admiravelmente esta linguagem, que utiliza
como uma catapulta de ideias e reflexões. A nossa época, que já conta
com o \emph{Diário de Anna Frank} e o de Maria Carolina de Jesus, além
do \emph{Journal du Voleur} de Jean Genet, tem agora completo o
triângulo dos párias de vários regimes e várias épocas: o judeu, o negro
e o homossexual. São misteriosas e muitas vezes ilógicas as trajetórias
das palavras de um reformador que se exprime através da literatura. Não
se exclui assim a possibilidade de que, apelando para a razão do
americano branco consciente das suas responsabilidades, o autor nascido
no Harlem atinja o ponto central do qual unicamente poderá partir a
modificação que ele propõe e todos ambicionamos. Exemplarmente, a língua
inglesa fala de \emph{a change of heart} quando uma pessoa muda de
opinião ou de ideia. É nesta ``transformação de corações'' e de cérebros
cegados pelo racismo que reside a nossa esperança.

\chapter{James Baldwin e o negro nos
EUA}\label{james-baldwin-e-o-negro-nos-eua}

Correio da Manhã (Caminhos da cultura), 1965. Aguardando revisão.

\hfill\break

Nessa esplêndida alegoria moderna que é \emph{Les Nègres}, Genet faz o
personagem Archibald exclamar, dirigindo-se a seus irmãos de cor: ``Eu
vos ordeno serem negros até a profundeza de vossas veias e de nelas
arrastar sangue negro. Que a África nele circule. Que os negros se
anegrem. Que eles se obstinem até à loucura naquilo que os condena a
ser, em seu ébano, sem seu cheiro, em seus olhos amarelados, em seus
gostos de canibais\ldots{} Que se (os brancos) mudarem a nosso respeito,
não seja por indulgência, mas por terror\ldots{} Inventai, não o amor,
mas o ódio\ldots{} O trágico estará na cor negra! Será ela que vós
amareis, reunireis, merecereis. É ela que é preciso conquistar!''

A busca de uma identidade para o negro contemporâneo tem assumido as
formas que cada país, cada civilização, cada escola de valores culturais
apresenta no local onde o negro vive. É claro que, nas regiões de vida
tribal e de prática do canibalismo como partes do Congo e da chamada
África Negra, a reivindicação de uma fisionomia própria que caracteriza
os movimentos negros nos Estados Unidos, na África do Sul, no Senegal e
na Martinica. A par da \emph{négritude} de Léopold Senghor e dos poetas
africanos de expressão francesas reunidos pela publicação \emph{Présence
Africaine} em Paris, existe toda uma gama de nuances nos Estados Unidos
dessa tentativa de forjar uma personalidade do negro distinta da
personalidade do branco, com valores de ritmo, de emoção, de
espontaneidade diametralmente opostos a muitos dos cânones da cultura e
da arte do Ocidente. A conclamação guerreira do personagem de Genet ecoa
desafiadoramente na doutrina violenta e fanática de Elijha Muhhamda, o
chefe da seita negra dos chamados \emph{black muslins}. Contando com
mais de 300.000 adeptos espalhados por todos os Estados Unidos, esse
grupo semirreligioso te 30 ``templos'' em vários Estados e como doutrina
implacável a de exterminar todos os brancos, para triunfo final dos
negros sobre os ``demônios'' de pele clara. Allah, em sua concepção, é
um sanguinário Deus negro, de vingança e inclemência. Repudiam o
Cristianismo -- religião inventada pelos brancos -- e não querem ter nem
mesmo os sobrenomes em comum com seus antigos senhores brancos. Passam a
usar só um nome seguido de um X. Ao exigirem para os negros
norte-americanos uma pátria separada, a ser retirada do território
estadunidense, eles de certa maneira confirmam o \emph{apartheid}
sul-africano dentro de um racismo puramente negro.

Por outro lado, o movimento negro reivindicatório de melhorias sociais
urgentes dentro da sociedade americana abrange líderes de extraordinária
envergadura moral, cultural e intelectual. Um deles, Lester Granger da
\emph{Urban League}, presidiu em Petrópolis, em 1962, a reunião
internacional de assistentes sociais ali congregada. Outro, Martin
Luther King, foi agraciado no ano passado com o Prêmio Nobel da Paz pela
sua defesa dos direitos da minoria negra por meios pacíficos delineados
por Gandhi na sua doutrina do \emph{ahimsa} (não-violência).

Uma única figura, porém, se destaca como arguto e lúcido intelectual
negro, escritor, ensaísta e novelista de renome nos países de língua
inglesa. Trata-se de um autor que tem dedicado uma série brilhante e
palestras, em universidades americanas, ao tema das relações
inter-raciais em seu país e que em seu último e brilhante livro,
\emph{The Fire Next Time}, analisa agudamente os problemas que surgem
com o empecilho para a ascensão sócio-cultural do negro norte-americano.

James Baldwin é quase totalmente desconhecido no Brasil, exceto pelas
minorias que leem inglês. No entanto, cremos que sua ação como crítico
da situação das pessoas de cor supere de muito duas atividades de
novelista e dramaturgo. O jovem intelectual no seu ácido libelo --
publicado em livro depois de reunido em capítulos na famosa revista
\emph{The New Yorker} -- condena igualmente as soluções violentas como a
que houve recentemente em Harlem, bairro negro de Nova York, imensa
favela de cimento armado que se ergue em plena metrópole cosmopolita. No
entanto, Baldwin argumenta lucidamente ao afirmar que teme não haver
outra alternativa, fora a violência, se a maioria branca norte-americana
não tomar medidas urgentes tendentes a dar ao negro seu compatriota um
lugar ao sol, subtraindo-o à posição de inferioridade a que tem sido
relegado há séculos.

A análise que Baldwin faz das relações raciais nos Estados Unidos é não
só de brutal franqueza como também de amarga, amaríssima, revolta.
Referindo-se à decisão do Supremo Tribunal \emph{yankee} que em 1954
declarou ilegal a segregação nas escolas, ele atribui esta resolução
histórica aos interesses da política externa norte-americana, ``que
cortejava a África recém-emersa do colonialismo e rica em minerais e em
petróleo''. As críticas vitriólicas que faz aos \emph{mores} e aos
valores materialistas americanos -- é sabido que Baldwin viveu longos
anos em Paris -- causaram impacto e parecem aumentar a corrente de
autores americanos rebelados contra ``a maneira de viver americana''.
Essa corrente, recordemos, provém de John dos Passos, Gertrude Stein e
Hemingway até Henry Miller, em nossos dias, que definiu a civilização
americana como ``um pesadelo dotado de ar condicionado''.

A veemência de James Baldwin não é menor absolutamente com relação ao
ressentimento negro:

\begin{quote}
``O negro norte-americano tem a grande vantagem de nunca ter acreditado
naquela coleção de mitos a que estão presos os americanos brancos: o de
que seus antepassados eram todos heróis amantes da liberdade, o de que
nasceram no país mais extraordinário que o mundo já viu, o de que os
americanos são invencíveis na guerra e sábios na paz, o de que os
americanos sempre agiram honradamente com os mexicanos, os índios e
outros vizinhos inferiores, o de que os americanos são os homens mais
direitos e viris do mundo e de que as mulheres americanas são puras''.

``Como pode alguém respeitar, muito menos adotar, os valores de um povo
que não vive, em nenhuma forma imaginável, de maneira que diz ou da
maneira que afirma deveria viver? Não posso aceitar a declaração de que
o labor intenso do negro americano durante quatrocentos anos deve
resultar meramente na sua integração no nível atual da civilização
americana. Estou longe de deixar-me persuadir de que valeu a pena
terem-me liberado do curandeiro africano se agora -- a fim de apoiar
minhas contradições morais e a aridez espiritual da minha vida -- eu
tenho que me tornar dependente do psiquiatra americano. É uma troca que
recuso absolutamente''.
\end{quote}

Seu impressionante documento pessoal e humano termina com um apelo aos
negros e aos brancos ``relativamente conscientes''. É tarefa desses
elementos lúcidos da sociedade americana despertar a consciência dos
demais e cumprirem juntos o dever inadiável de por a termo ao ``pesadelo
racial'', coo o denomina Baldwin, e mudar o curso dos acontecimentos nos
Estados Unidos e no mundo. Se não ousarmos tudo para atingir esse
objetivo de inspiração bíblica contida na canção do escravo que cantava:
``Deus deu a Noé o sinal do arco-íris: Chega de água, da próxima vez:
fogo!''

\chapter{O negro Baldwin luta pelos
seus}\label{o-negro-baldwin-luta-pelos-seus}

Jornal da Tarde, 1966/10/29. Aguardando revisão.

\hfill\break

O aguilhão, cravado desde o nascimento na pele luzidia e negra, não pára
de doer nunca quando você é negro em certas regiões do mundo. A
princípio atônito, quando ainda criança, James Baldwin não compreende
porque seu pai, pastor protestante, o proíbe de brincar com crianças
brancas na calçada, de entrar no parque onde estão as palavras
incompreensíveis ``só para brancos'' e porque ele abraçou a religião que
fala do amor ao próximo, com seus hinos religiosos de força e poesia.

Confusamente, o pai e a mãe explicam que brancos afastados do amor
pregado por Nosso Senhor Jesus Cristo é que puseram a tabuleta no
parque, onde há o escorrega e o balanço e onde ele gostaria de pular e
brincar om os outros meninos da sua idade. Mas as explicações não
explicam, falam sempre de paciência, de um Reino Futuro de Justiça e de
Bondade e o jovem James está interessado no presente, no agora.

Na escola, continuam as humilhações incompreensíveis. E seus primos do
Sul lhe dizem que na cidade é melhor, ``lá em casa'', na terra de onde
sua família veio, os pretos são perseguidos por ferozes cachorros
policiais, por policiais armados de gás lacrimogêneo e de mangueiras
d'água de enorme violência. Mas lá longe pouco importa, ele dá de
ombros: estou vivendo aqui mesmo. E não é que se possa dizer que estou
morando como eu quero! De fato, o Harlem é um círculo do inferno, o
inferno de que falam os sermões do pai na Igreja Batista aos domingos: o
fogo é o calor infernal, os suplícios são os mesmos: as misérias, os
ratos, o amontoado de famílias juntas, a falta de perspectiva para os
jovens que se tornam criminosos ou aceitam tornar-se ``cidadãos de
segunda classe''. O que quer dizer: sem os privilégios do país mais
privilegiado em riqueza e em progresso na terra. Mas não em humanidade.
E James Baldwin parte. Escreve. Vive intensamente: o amor, a fome, o
desespero, raramente a esperança e a fé. Parte para a Europa, dando-se
por vencido. O preconceito, o ódio, a indiferença que o circundam são
mais fortes do que a sua insistência em lutar, em afirmar-se contra a
hostilidade generalizada. Na Europa, passa 9 anos, a maioria dos quais
em Paris. Como milhares de outros americanos, às margens do Sena,
isolados da maneira de viver americana que um compatriota irreverente,
Henry Miller, chamara de ``um pesadelo dotado de ar condicionado''.

Na França, Baldwin escreve:

\begin{quote}
``Nos Estados Unidos, a cor da minha pele se erguera sempre entre mim e
a minha personalidade; na Europa, aquela barreira caíra\ldots{}
Revelou-se então que a indagação:''quem sou eu?' não fôra solucionada
porque eu me afastara das pressões sociais que me ameaçavam\ldots{} A
pergunta sobre quem eu era tornou-se finalmente uma questão pessoal cuja
resposta teria que ser encontrada em mim mesmo''.

Mas os quadros de Rembrandt, as cantatas de Bach, as peças de
Shakespeare e a Catedral de Chartres permanecem tão distantes dele,
quanto na América nativa, o Empire State, o mais alto edifício do mundo,
como que a encarnação maciça de uma indiferença monumental para com a
sua sorte:

``Nada disto tudo representava criações minhas, não continha a história
da minha raça; eu podia procurar em todas essas manifestações, em vão,
eternamente, um pálido reflexo de mim mesmo sem nunca encontrá-lo. Eu
era sempre um intruso, um pária''.
\end{quote}

Na Suíça as crianças, os adultos de vilas serenas espelhando a majestade
de suas montanhas nos lagos azuis aproximam-se dele como de um animal
manso e perdido perto dos Alpes. Tocam-lhe a pele como se acaricia um
filhote de lobo e passam as mãos pela sua carapinha, riem, escondem-se
com medo.

Baldwin assiste à conferência sobre a \emph{Négritude}, quem sabe viria
da África original, ventre de que brotou a sua raça, a reposta que o
investiria de uma identidade finalmente perdida entre aqueles estranhos
ídolos que não eram os seus? Aimé Cesaire, Senghor, o poeta que se
tornou presidente de uma República Africana, depois da libertação do
colonialismo, Léon Damas da Guiné Francesa, mais os representantes de
novas e orgulhosas nações negras o Ghana, a Nigéria, estão todos
reunidos em Paris. Discutem, debatem, desentendem-se dias a fio sobre o
papel do negro na civilização moderna, ventilam problemas políticos
incandescentes, marxistas, liberais e reacionários acusando-se com furor
mutuamente. Não, não será daquela discórdia que lhe virá a paz tão
ambicionada.

James Baldwin parte novamente para o Norte, para a Suécia. Entrevista
Ingmar Bergman, o mago dos filmes de extraordinária densidade lírica, de
pregnância de pensamento e requinte de imagem e palavra. O grande
diretor lhe fala de amor, de erotismo, de morte e de misticismo, como
que refletindo em sua voz pausada e nos seus olhos azulados os temas que
inspiram seus poemas cinematográficos: \emph{Morangos Silvestres},
\emph{A Fonte da Donzela}, \emph{O Silêncio}.

Pela última vez, James Baldwin parte. Desta vez de volta. O seu lugar é
a América. É na própria arena em que os gladiadores travam o combate
desigual com a ferocidade humana que ele deve estar, que ele
forçosamente, pela força da sua cor tem que estar. Em 1955 sua primeira
coleção de ensaios sobre a discriminação racial irrompe como a explosão
de uma bomba-relógio: a mesma que já explodira cem anos antes na Guerra
de Secessão sangrenta e que quase desmembrara o país em duas Repúblicas
antagônicas.

\emph{Notes of a Native Son} revela perante uma América estarrecida o
``filho bastardo da civilização ocidental'', o negro, \emph{displaced
person} de todo um Continente, arrancadas suas raízes, pela violência
dos negreiros de outros séculos, da sua própria terra natal. Depois do
degredo físico, a masmorra espiritual agora naquela terra que ele
próprio ajudara a construir com seu sangue e seus músculos -- e como
comprovavam milhões de mulatos -- com o ventre de sua mãe violada na
senzala. Poderia \emph{não ser} amargo, monstruoso, incendiário, um
livro que como um vulcão faz estremecer a superfície aparentemente
plácida de toda uma sociedade.

No entanto, a inteligência, o brilho, o ressentimento justo perante a
iníqua crueldade, nenhum desses elementos fundamentais dos ensaios de
Baldwin lhe permite escrever um libelo unilateral, vingativo e cego pelo
ódio. Ele vislumbra a solidariedade -- arduamente conquistável -- entre
as duas raças se, dentro do espírito cristão, agirem com boa vontade ou,
dentro do espírito prático, reconhecerem que o bom senso imporá a
integração fatalmente. Até mesmo \emph{The Fire Next Time} que contém
por certo algumas das mais candentes e irrespondíveis acusações à
civilização criada pelo homem branco, nem mesmo esse longo e sombrio
lamento termina com uma nota de desalento, mas sim com um apelo a
brancos e negros para que deponham as armas do fanatismo e da ignorância
em prol de uma América, como vaticinam suas moedas cunhadas com a efígie
e Lincoln, feita uma através da diversidade de seu cadinho de raças, de
religiões, de nacionalidades e culturas. A América futura.

Contudo, é feita de várias facetas, de várias camadas, a tragédia de
James Baldwin. \emph{Giovanni's Room} revela francamente sua segunda
marginalização, ao documentar sua paixão erótica pelo mesmo sexo.
\emph{Nobody Knows my Name} volta a insistir na impossibilidade de um
intelectual negro ser absorvido pelos movimentos religiosos que pregam a
destruição da raça branca, a sua extirpação da parte da América --
vastas regiões do Sul -- que integrarão uma República Negra da América
do Norte:

\begin{quote}
``A única coisa que todos os americanos têm em comum é a de que a sua
identidade como povo é única e está sendo forjada exclusivamente aqui,
neste Continente''
\end{quote}

E será neste Continente que juntos perecerão ou criarão uma nova
América.

A última faceta da angústia de Baldwin desvenda-se então, nas palavras
dos críticos e no espanto do grande novelista inglês E. M. Forster:
``Este livro foi escrito por um negro? \emph{Wonderful}!
\emph{Wonderful}!'' James Baldwin sacrificou a sua vocação de novelista,
de ficcionista, em prol dos seus libelos veementes a favor da paz
racial. O ensaísta arguto, brilhante, sagaz, matou o romancista com a
pujança de um talento dramático e pungente.

São apressados porém os seus críticos, embora honestos no reconhecimento
unânime dos seus dons verbais inigualáveis entre os autores de cor de
língua inglesa.

Porque, em seu último livro, Baldwin \emph{assumiu} plenamente a sua
identidade americana, com tudo de esplêndido e de terrível, de promissor
e de responsável que isso acarreta:

\begin{quote}
``In short, I had become na American''. Em poucas palavras: eu me tornei
um americano. Eu ingressara, tropeçara, inevitavelmente, na confusão sem
fim que é tão individual e tão pública da República Americana''.
\end{quote}

É legítimo esperar do seu destemor que se adentrou na raiz de todo o seu
crucial problema a floração, agora, da sua imaginação. Agora que sua
mente lúcida produziu já, em abundância, os seus frutos de agridoce
sabor, que seu coração nos falou de tão perto, será a vez de sua
fantasia criadora trazer a uma humanidade faminta da sua voz as novas
perspectivas e a nova esperança que o negro traz para uma Civilização
exausta e necessitada da sua redenção.

\chapter{James Baldwin (necrológio)}\label{james-baldwin-necroluxf3gio}

Jornal da Tarde, 1987/12/2. Aguardando revisão.

\hfill\break

James Arthur Baldwin, o escritor e ensaísta norte-americano que morreu
de câncer no estômago ontem na cidadezinha de Saint-Paul de Vence, no
Sul da França, aos 63 anos de idade, foi cronologicamente o último
artista negro destruído pelo racismo. O peso do preconceito duplo --
contra a cor da sua pele e contra seu homossexualismo -- desvirtuou seu
grande talento literário, transformando-o predominantemente em um
porta-voz involuntário e candente da igualdade racial. A vida pendular
que escolhera morando na França desde 1945, e nos Estados Unidos, onde
nascera no gueto do Harlem, contrasta com sua deliberada intenção de
fixar-se nos Estados Unidos, como membro e como membro de uma minoria,
portanto, duplamente escorraçado pela sociedade majoritária
numericamente branca e heterossexual, no início de sua carreira. Numa
das passagens autobiográficas mais comoventes de seus livros ele se
interroga sobre a majestade da civilização europeia, em seus momentos
supremos: que podia significar para ele a imponência das catedrais
góticas, a genialidade de Shakespeare ou de Dante, um quadro de
Rembrandt ou de Vermeer? Nada daquilo tinha alguma coisa a ver com a sua
formação, com o seu sofrimento cotidiano, com o racismo da Klu Klux Klan
e seus homens encapuzados no Sul dos Estados Unidos, a castrar e
enforcar negros pelo simples fato de serem negros. Ele assume, pelo
menos temporariamente, a sua condição, uma vez aprendida a lição da
Europa:

\begin{quote}
``Em resumo: eu me tornara americano. Eu ingressara, tropeçara,
inevitavelmente, na confusão sem fim que é tão individual e tão pública
da República Americana''.
\end{quote}

A volta, porém, fora decepcionante. Ninguém se importava que ele fosse
um \emph{escritor}. Ele tinha que se tornar era o arauto da
discriminação sofrida pelos negros. Alguns grupos mais radicais o
acusavam de ``ter sido cevado pela ração branca'', referindo-se às
bolsas de estudos que ele recebera de empresas como a Ford Foundation.
Que outra empresa negra poderia lhe dar o mesmo apoio? James Baldwin
continua tão deslocado quanto Kafka, pertencente a uma minoria judaica
que fala alemão em Praga: círculos concêntricos do exílio do artista.
Ele, característica de um sem número de homossexuais, identificava-se
com os ensinamentos da mãe, que pregara sempre, apesar de todas as
advertências cotidianas, abster-se do ódio, reconhecer no outro um
irmão, fosse ele branco, amarelo, negro, índio ou mestiço. Malcolm X, em
parte seguindo os ensinamentos apocalípticos de Fritz Fannon, propunha a
destruição implacável dos ``demônios de olhos azuis'', os brancos. Os
grupos das \emph{Black Panthers} (Panteras Negras) iam mais longe: a
enorme fatia do Sul dos Estados Unidos pertenceria, ``de direito'', a
uma futura República Negra Norte-Americana, da qual seriam expulsos
todos os brancos.

James Baldwin afastou-se de tais protestos que pregavam a violência:
embora visse iminente, não queria participar dela. Afinal, entre seus
empregos para sobreviver como garçom, operário, lavador de pratos e
outros, ele conseguira aliar-se ao movimento pelos direitos civis dos
negros, na década de '60 de Martin Luther King, o esplêndido profeta de
um novo sonho americano: o da igualdade racial entre todos os
componentes das diversas etnias que constituem os Estados Unidos.

E na África, onde estavam suas raízes -- estaria lá a reposta que ele
buscava tão ansiosamente? Participar de uma Conferência Sobre Problemas
Africanos aterrorizou-o. Muitas das nações africanas recém-libertadas do
jugo colonialista tinham se transformado em ditaduras sangrentas de um
tirano corrupto. A violência tribal dilacerava a Nigéria numa guerra
civil ferozmente fratricida. E a África do Sul era uma vasta prisão para
sua população negra, segregada nas praias, nos trens, nos bebedouros,
nos guichês dos bancos, nos restaurantes, nos bairros. O que esperar dos
árabes? Indiferença. Historicamente tinham sido eles os primeiros a
reavivar a prática da escravidão dos negros, avidamente seguidos pelos
portugueses, ingleses, espanhóis, todos sequiosos do lucro que a
diáspora africana lhes rendia.

Havia a \emph{Négritude}, com o poeta-presidente do Senegal, Léopold
Senghor, o antilhano Aimé Césaire. E havia os sábios brancos como Leo
Frobenius que revelavam as antigas civilizações do Benin, de Daomé, da
Nigéria, à época em que a Europa ainda era um miserável amontoado de
choças. Baldwin se ajustava perfeitamente à visão raramente profundo de
Sartre, que distinguia o negro, como Gilberto Freire discernira no
Brasil, a predominância dos valores \emph{afetivos}. Quem sabe seria
essa a redenção do mundo futuro, devastado pela brutalidade do homem
branco?

Essa desesperada e fraterna emotividade de Baldwin, ele próprio
reconheceu antes de morrer, fracassou. Talvez as futuras gerações
terminem a tarefa interrompida de civilizar o ser humano em seu sentido
profundo do termo. No entanto, Baldwin não se restringe a essa
camisa-de-força, o binômio negro e homossexual. Em seus livros ele se
refere constantemente às minorias esquecidas, os índios, os drogados, os
veteranos da guerra do Vietnã. Sua preocupação com o sofrimento e a
injustiça já o tinha levado precocemente, aos doze anos de idade, a
focalizar o massacre dos republicanos espanhóis pelo terror franquista e
o holocausto dos judeus sob a monstruosidade nazista em seus escritos
adolescentes, mas já cheios de fogo e paixão.

Durante algum tempo ele equaciona como equivalentes, o racismo e o
poder. Mas logo percebe que mesmo os que são pisados pelo poder carregam
a mesma dose de racismo e preconceitos que as camadas dominantes. Mesmo
com tantas amargas decepções, eletriza-se ao recordar o passado de seus
ancestrais escravos nos Estados Unidos:

\begin{quote}
``Esse passado de cordas, de fogo, de tortura, de castração, de
infanticídio, de estupro, de morte e humilhação, de medo noite e dia, de
párias, de ódio e crimes''.
\end{quote}

As frases um tanto retumbantes de que os negros não se estarreceram com
as atrocidades dos campos de concentração alemães em Dachau, Auschwitz e
outros, pois achavam os brancos capazes de qualquer sordidez -- como se
os negros fossem incapazes de sentir compaixão pelas vítimas do nazismo
-, cedem, pouco a pouco, a um discernimento mais sutil. Seu contato com
a França, que não tinha tido escravos em seu território, esclarece a
existência de outros racismos: a França tinha suas derrotas na Indochina
e na Argélia, com os ``sub-humanos'' vietnamitas e árabes. Não era,
literalmente, os ``negros'' que a maioria dos franceses desprezava
ostensivamente e que hoje constituem a base para o ódio racial contra os
imigrantes da África do Norte de um político fascista como Le Pen na
França dos dias que correm?

Panoramas que antes ele via de forma demasiado genérica adquirem nichos,
nuances que ele não pensara existir anteriormente. Por exemplo: André
Gide, o corajoso novelista francês que admite publicamente em seus
livros escandalosos para a época seu homossexualismo, já representara os
marginais da Europa bem-pensante, como Oscar Wilde, imolado pela
hipocrisia da era vitoriana na Inglaterra, Baudelaire condenado pelo
Ministério da Justiça por seus poemas ousados e magníficos de \emph{As
Flores do Mal}. Havia várias formas de ser negro na Europa
também\ldots{}

Encontraria refúgio na religião? Esta pergunta lhe despertava uma
resposta fulminante de amargura e desencanto: ``Como se os
autoproclamados cristãos abandonaram o cristianismo e se a Igreja é o
pior lugar para se aprender o cristianismo?''

A mutilação do talento criativo de James Baldwin fica talvez
cristalinamente documentada em seu romance \emph{Giovanni's Room}, uma
tentativa, que redundou em fiasco, de unir os temas do racismo e do
homossexualismo. Talvez ele seja mais lembrado por seus ensaios
publicados na revista \emph{New Yorker} e depois reunidos em livro:
\emph{Da Próxima Vez, Fogo!} Não, é óbvio, que esse talento tenha sido
canalizado para a frase de efeito, para o panfletarismo: nunca. É que a
ferida da discriminação racial arde mais intensamente quando ele reflete
sobre a ``obscenidade'' do preconceito de cor e se vê acossado como um
animal, um tarado, sempre um estranho e um suspeito, o primeiro a ser
algemado pela polícia apenas por ser negro, exatamente como no Brasil de
hoje e de ontem.

É forçoso reconhecer que James Baldwin, afinal, ficou numa posição
intermediária entre Richard Wright, que a princípio o protegeu e ajudou,
até se desentenderem ``por uma questão de choques de ponto de vista de
gerações diferentes'', e Ralph Ellison, o trágico e farsesco autor de
\emph{O Homem Invisível}: o negro que todos os brancos fingem não
existir, não ver, não reconhecer como \emph{pessoa}.

A sua solidão foi a de um eterno marginal que, ao contrário de Jean
Genet, a querer destroçar à sociedade que o oprime, quer uma
\emph{integração} de todos os seres humanos, sejam quais forem suas
origens raciais e suas preferências sexuais. Como ele próprio confessou,
francamente, a um entrevistador, não faz muito tempo:

\begin{quote}
``A celebridade é uma nova solidão''.
\end{quote}

A celebridade, ele parece dizer, é a miséria presente que será festejada
pela geração vindoura, como os quadros de Van Gogh disputados hoje a
milhões de dólares, depois que seu atormentado e magnífico autor morreu,
anônimo, paupérrimo, um suicida a mais em um remoto asilo de loucos.

Ou, exprimindo seu pensamento de outra maneira: as leis não \emph{mudam}
o coração dos homens. Será que a biogenética conseguirá dotar o homem
moderno de um coração sem ódio?

\chapter{Resenha do livro O Mensageiro de Charles Wright (Editora Nova
Crítica,
1969)}\label{resenha-do-livro-o-mensageiro-de-charles-wright-editora-nova-cruxedtica-1969}

Jornal da Tarde, 1969. Aguardando revisão.

\hfill\break

Quando surgem nos jornais manchetes sobre conflitos raciais ou marchas
de protesto dos negros nos Estados Unidos, lutando pela defesa de seus
direitos covis, faltam sempre exemplos claros daquilo que a população de
cor reivindica.

\emph{O Mensageiro} é um romance lírico, cômico, trágico na denúncia da
falta de perspectivas de um jovem negro sensível, sufocado pela favela
de cimento armado em que mora em Nova York, com um emprego que não o
satisfaz -- e o que é pior: sem as perspectivas de melhoras no futuro
que estão reservadas aos colegas brancos de sua idade;

Charles Wright não tem a veemência de James Baldwin ao ameaçar em seus
excelentes ensaios \emph{Da Próxima Vez, Fogo!} Ele não escreve um
libelo colérico e político. Seu romance transcende os limites da revolta
contra o racismo em si para formular uma pergunta jovem, inquietante,
sobre o sentido da própria vida individual, cercada pelos preconceitos e
pela agonia de uma geração fossilizada que não quer enterrar seus mitos
materialistas, intolerantes e hipócritas.

Charles Wright é um dos melhores escritores negros ou de qualquer cor
atuais. Ele ambienta seu romance poético, intenso, sofrido, numa grande
metrópole como Nova York mas documenta também o passado rural dos negros
vindos do Sul. Tudo com mão de mestre: uma obra-prima de sofisticação,
de estilo e de solidão humana.

\chapter{A influência decisiva de Richard
Wright}\label{a-influuxeancia-decisiva-de-richard-wright}

Jornal da Tarde, 1992. Aguardando revisão.

\hfill\break

Richard Wright (1908-1960) consegue desagradar quase que de maneira
unânime. Extraordinário escritor negro do Sul dos Estados Unidos,
momentaneamente cativado pelo Partido Comunista e logo expulso dessa
associação que lhe pareceu limitá-lo como uma camisa-de-força, seu
talento passional e franco até as últimas consequências o levou a
abandonar o país em que nasceu e cresceu para abrigar-se junto aos
existencialistas franceses, Sartre e Simone de Beauvoir, em Paris.

Este resumo da sua meteórica carreira sintetiza também os ferozes
inimigos que foi fazendo à medida que seus livros atingiam leitores e
metas, um público e uma ambição literária cada vez mais inesperados. Ter
sido selecionado e censurado em parte pelo Clube Mensal do Livro que
mutilou livros fundamentais como \emph{Native Son} e \emph{Black Boy},
mas os editores brancos eram de opinião que sabiam mais que ninguém o
que seus associados brancos estavam dispostos suportar, partindo de um
talento negro, vulcânico como Dostoievsky, em sua análise dos crimes e
castigos reservados aos discriminados na sociedade norte-americana. Sua
franqueza e mordacidade com relação aos escritores negros nos Estados
Unidos alienaram os que hesitavam em aceitar juízos categóricos como,
por exemplo, o de que os autores negros de seu país até então tinham se
limitado a escrever literatura sem valor, buscando a esmola do
reconhecimento da maioria branca: ``Na maioria dos casos, esses
embaixadores artísticos foram recebidos como se fossem cachorrinhos
poodles capazes de mil acrobacias mirabolantes e engraçadinhas''. E que
desleixado era Richard Wright!, reclamavam os que celeremente se
dispunham a ``corrigir'' sua pontuação, sua ``verbosidade'', sem
compreender que ele usava técnicas de Joyce ao escrever sua literatura
incandescente.

Agora, dois volumes que, juntos, ultrapassam mais de 1.500 páginas,
englobam desde \emph{Lawd Today!}, \emph{Uncle Tom's Children} até
\emph{Black Boy} e \emph{Outsider}, numa publicação monumental da
\emph{The Library of America} (New York; edited by Arnold Rampersad). O
lúcido crítico Alfred Kazin no \emph{The New York Times Book Review}
agita as páginas tantas vezes sonolentas desse suplemento literário
colocando a revolução que esse escritor sulista trouxe para a literatura
norte-americana, ou seja, um romancista que provém justamente daquela
que é a mais rica região literária dos Estados Unidos, com gigantes como
Faulkner e Mark Twain, entre outros. Kazin demonstra até que ponto foi
fatal para Wright o exílio europeu. Por mais paradoxal que pareça, uma
vez afastado do centro de sua inspiração incendiária o racismo que
denunciou com um vigor e um sarcasmo talvez sem igual, Richard Wright
cairia de qualidade artística, como sucedeu tragicamente com James
Baldwin. Richard Wright ateou fogo a toda uma geração, inflamado pela
crença de H. L. Mencken, o crítico que atribuía um poder cataclísmico
``às palavras (usadas) como armas''.

Seus livros não trouxeram a revolução que ele esperava? O racismo
continua vivíssimo nos Estados Unidos? Richard Wright não fez diferença
alguma, então? A distância do tempo decorrido comprova que, até o limite
imposto à literatura que não for meramente panfletária, seus livros
possivelmente estão por detrás de Martin Luther King e da revolta
estudantil de Little Rock. Ao contrário de Alfred Kazin, numa
perspectiva mais abrangente, ser a ignição de todos os movimentos
reivindicatórios nos Estados Unidos e logo na \emph{Négritude} do
antilhano Aimé Césaire ao senegalês Senghor. Sem dúvida, há falhas em
determinados trechos grandiloquentes do estilo de Wright, mas hoje é
mais fácil compreender como James Baldwin errou ao julgá-lo o autor de
um livro ``tão simplório'' quanto \emph{A Cabana do Pai Tomás}. Nem
\emph{Native Son} é ingênuo nem \emph{Uncle Tom's Cabin} pode ser
considerado ``simplório''. Se \emph{Invisible Man} de Ralph Ellison, é
mais perfeito, \emph{Native Son} é uma primeira semente fecunda, lançada
na década de 30, numa literatura que a partir daí não é, realmente, a
mesma de antes.

\chapter{Entrevista - Toni Morrison e os negros da
América}\label{entrevista---toni-morrison-e-os-negros-da-amuxe9rica}

Jornal da Tarde, 1990/11/08. Aguardando revisão.

\hfill\break

Na realidade, o amor que eu sentia pela extraordinária escritora de cor,
a norte-americana Toni Morrison, já vinha entre cada linha e cada fala
de seus personagens em livros líricos, nada ortodoxos, vibrantes como
\emph{Tar Baby}, \emph{Song of Solomon} e \emph{Beloved}. A incendiária
e controvertida entrevista que dera em duas páginas à revista
\emph{Time} em 1989 dividira os EUA entre os que a defendiam e os que a
achavam exagerada. 300 anos de escravidão a incitavam a escrever sobre
seu povo, menos do que com a pesquisa, mais com a imaginação poética e
uma sutilíssima sensibilidade. Há trechos da revolucionária prosa de
Toni Morrison que têm o mesmo impacto das canções plangentes de Billy
Holliday, em sua voz toda tristeza e abandono.

Pergunto-lhe se o sonho idealista de Martin Luther King, o supremo líder
negro, de ver uma América de brancos e negros irmanados e livres de
preconceitos ruiu como um prédio ou uma ideia condenados. ``Seu ideal
sofreu muito desde que foi formulado até hoje'', ela explica, paciente.
Para ela, desde então surgiram áreas de indiferença por parte dos
brancos e de desespero dos negros. Mas continuam a existir os que estão
horrorizados com a situação atual e outros que dariam a vida para
modificá-la, acrescenta. Várias práticas legais, vários hábitos sociais
daqueles 30 anos até hoje diluíram-se, tornaram-se símbolos esvaziados
de qualquer conteúdo. Secaram muitos ramos da assistência social aos
negros mais pobres, com o corte das verbas do \emph{Welfare}; outras
fontes de ajuda financeira desapareceram também nas duas últimas
décadas. De modo que ``os supostos beneficiários jamais produziram uma
segunda geração'', pela evaporação das verbas. A situação que daí se
deriva é igual a de uma ponte semiconstruída sobre um rio caudaloso.
Todos dirão vendo-a: '' Mas não se pode passar por ela de uma a outra
margem, ela ficou fragmentada!'' Sim, ela concorda, essa parada se deve
em grande parte aos problemas econômicos agudos da sociedade
norte-americana: o meio milhão de brancos sem teto, espalhados,
maltrapilhos, pelas ruas de Nova York e outras metrópoles foram os
recipientes da ajuda governamental, não há dúvida.

E Miss Morrison dá números monstruosos para apoiá-la: depois da
administração Carter, aquele tradicional 1,5 \% da população total que
controlava 26\% da riqueza do país, quando terminou a presidência de
Ronald Reagan, de 8 anos, passou a deter 6\% a mais do que já tinha em
mãos. Daí a falta de verbas para cursos especiais para os negros, de
moradias, o que se transforma numa acusação pessoal contra os negros por
serem pobres. Os que discordam dela acham que as medidas tomadas de cima
eventualmente beneficiarão os negros desfavorecidos na escala social.
Ela, ao contrário, acha que é preciso \emph{aumentar} o dinheiro para
combater o analfabetismo, a ``indústria'' das drogas, as doenças que
dizimam os guetos negros e os impedem de se desenvolver plenamente.
``Como enfatizou Buckminster Fuller, a dívida para acabar com esses
problemas é infinitamente menor do que a dívida para se criar e se
manter uma sociedade onde predominam a violência, a falta de opções, a
insegurança geral''. Mesmo as camadas de cor que ascenderam em termos de
dinheiro, de poder aquisitivo, nada contribuem para melhorar esses
males, identificam-se com os valores da sociedade branca majoritária,
perdendo sua própria identidade, identificando-se como outra classe
social.

Ela não vê a diáspora africana escravizada como um holocausto comparável
ao dos judeus e outros grupos nos campos de concentração alemães durante
o período nazista. Sim, foi um genocídio, é claro, com um total de 60 a
200 milhões de negros que morreram nos navios negreiros, nas plantações
ou não resistiram à vida crudelíssima que lhes era imposta por seus
``donos''. Mas nota diferenças: no caso da escravidão, os
``proprietários'' estavam comprando capital produtivo, de graça
praticamente, e sem ter que arcar com os filhos dele, ou a esposa, a não
ser no caso de garanhões mantidos apenas para reprodução de mais
escravos.

O direito do comprador era ilimitado: podia-se violar e até matar quem
bem se quisesse: não havia ninguém que se opusesse às práticas mais
hediondas.

Quando lhe conto que uma sensível amiga norte-americana me disse,
desalentada, que cada vez que vai aos EUA adquire a certeza já quase
inabalável de que o racismo nos Estados Unidos é insolúvel. ``A sua
amiga tem razão em grande parte'' -- Toni Morrison responde --
``Aumentou o número de linchamentos causados apenas por motivos raciais
e a Ku Klux Klan que tem contactos mundiais, realizou há pouco uma
enorme conferência internacional em Londres''. Sem nos esquecermos de
que a sede mundial do Nazismo é em Chicago e ela nunca esteve inerte.
Sim, e a Klan já tem até um Senador no Congresso, um legislador, mas já
tivemos até Presidentes da Nação que pertenceram à Ku Klux Klan, como
Truman, que depois abandonou a organização, mas há outros, defensores às
ocultas da ``intrínseca superioridade branca'', apesar dos esforços dos
Republicanos de se distanciarem oficialmente deles\ldots{}

Para eles, ``todos os imigrantes que entraram nos EUA procedentes da
Europa se uniram na discriminação contra \emph{o Outro}, o diferente, o
negro. Estabeleceu-se uma hierarquia e ninguém -- nem os católicos, nem
os irlandeses, nem os judeus etc. -- queriam ficar por baixo, portanto
os negros, obviamente, tinham que ser o fim da linha. E, por força, o
negro tornou-se''o inimigo'', como o índio, já que ambos destoavam do
único padrão nacional que era a brancura da pele.

Os índios só obtiveram a cidadania europeia em 1912! Apesar de tudo, sou
otimista com relação ao racismo. No mundo de hoje não só se revelou a
falácia do racismo como sobretudo o racismo \emph{não funciona mais!} O
racismo, estou certa, é uma tarefa para educadores, as pessoas têm que
\emph{desaprender} o que a tradição do preconceito lhes transmitiu de
ideias pré-concebidas a respeito dos negros.

Para a magnífica escritora norte-americana, escritores brancos sensíveis
como Mark Twain, Faulkner, Styron e outros já refletiram profundamente
sobre os conceitos de liberdade, de ética, de humanidade com relação ao
negro. O Huck de \emph{Hucklerberry Finn}, por exemplo, considera o
escravo fugitivo que protege um ser humano e não apenas um
\emph{nigger}.

E que método ela preconiza para acabar com o racismo ou diminuir
consideravelmente as limitações que ele impõe? ``Em primeiro lugar quero
reconquistar todo um território que nos foi roubado''. E o primeiro
passo seria mudar os livros escolares, que omitem o negro de suas
páginas? Seria um dos passos imediatos, opina. Mas, contradigo, os
alemães eram um povo culto e foi entre eles que brotou o nazismo, não?
Eles não eram cultos, ela retruca, tinham frequentado escolas, o que é
diferente. Temos que pôr abaixo toda a literatura de pensamento e
preconceito que se deriva de uma falsa educação, acentua. A partir dos
8, 9 anos, quando eles ainda não se encrustaram na mente da criança.

Pergunto sobre a naturalidade com que as comunidades negras, em seus
livros, falam de fantasmas, convivem e dialogam com eles: os fantasmas
até influem na vida dos vivos. Mas é claro, ela responde: é um mundo
mágico, e os fantasmas realmente existem, minha mãe a vida inteira
conversou com eles, eu sei que estamos rebaixando um pouco o
conhecimento factual, racional, tradicional, mas o mundo do além
\emph{completa} o nosso, é um fragmento do todo. E como são as relações
entre as preocupações com o racismo, a raça, e a literatura como alvo
estético? ``eu deliberadamente eliminei as personagens brancas de meus
livros, só para ver como seria o mundo vivido apenas pelos negros. Como
músicos que, quando se juntam, tocam um para o outro, discutem peças
musicais e interpretações, sem interferência de estranhos, entende? Eu
não queria a intervenção de outros critérios que não fossem os dos
negros em meus livros, queria um ambiente sem a \emph{reação} negra aos
juízos que porventura se façam deles.

É uma tarefa difícil, porque me proponho a escrever ficção, se possível
boa, mas sem panfletarismo, sem confundir literatura com proselitismo ou
com um editorial veemente -- ``O que é difícil de se conseguir, há que
se ter muito cuidado.''

Outra característica extraordinária de Toni Morrison, como escritora,
que ela compartilha com Ishmael Reed, é seu propósito de capturar todo
um mundo oral, um mundo de sons emitidos por negros, um folklore negro.
Mas ela gostaria de um dia escrever um livro em que não tivesse que
identificar protagonistas por sua cor de epiderme. Duas vezes já,
acredita, conseguiu isso em \emph{Beloved} (\emph{Amada}), quando há
cenas, não importa se curtas, em que o leitor não pode distinguir se se
trata de brancos ou pretos. Quando ela confirma minhas suspeitas de que
os brancos são vistos, genericamente, como estúpidos e claramente
insensíveis em seus romances, ela se apressa em corrigir: os brancos que
se ``protegem'' do mundo usando o fato de serem brancos como escudo e
como justificativa para qualquer coisa que façam, esses são os brancos
que lhe parecem inferiores. ``Porque, realmente, ver tudo sob um prisma
racista significa aceitar um mundo mais estreito, mais pobre, menos
variado. Os seres \emph{humanos}, pessoas que são \emph{gente} e não
apenas a cor de sua pele, não são racistas. De todos os livros que já li
-- e são muitos, muitos mesmo -- \emph{nunca} deparei com uma cena em
que uma personagem branca toca numa personagem negra a não ser por meio
do estupro ou da violência, \emph{nunca}, acredita? É um sentimento
geral dos europeus, discriminar outras raças, você tem razão: mesmo na
Rússia e em outros países eslavos, o horror ao negro é patente: mesmo
que os negros estejam lá só para estudar em faculdades, não estão
disputando empregos com os brancos, no entanto, o preconceito europeu
parece estar sempre presente e onipresente.

Ah, sim, os árabes parecem que partilham esse sentimento: não
decididamente, o preconceito racial não foi inventado em nossos países!
``Há um livro extraordinário de Orlando Patterson chamado \emph{Slavery
and Social Death}, em que ele vai às fontes da escravidão em todas as
regiões e épocas e chega à conclusão de que não há ninguém que não
descenda de escravos''. É verdade, lembro, a antropologia contemporânea
fala de uma mãe universal, negra, africana, mãe dos mais louros
escandinavos e até de Hitler\ldots{} E mesmo a majestosa Constituição
dos Estados Unidos ignorou o \emph{status} dos negros, dos índios: eles
teriam alma? Seriam seres inteiramente humanos? ``É, o Iluminismo e o
racismo surgiram juntos, eis outra contradição da condição humana, Toni
Morrison observa, prosseguindo: temos, \emph{é indispensável} que
mudemos isso antes da possível guerra no Iraque, sei lá, nos levar de
roldão a todos pelos ares''. E concorda comigo quando aos livros
escolares e às leis anti-racistas acrescento o amor: ``Realmente, o amor
humano, embora não possamos legislar o coração. São os três elementos
necessários além de eliminarmos o trauma de \emph{todos} os
preconceitos, não só o racismo. A literatura, a arte, têm poder, por
isso estiveram sempre vigiadas por regimes totalitários, por
puritanismos, pela censura. Mas mesmo que dure gerações e gerações, é
imperativo mudarmos o mundo''.

\chapter{Nota sobre o livro O Olho mais Azul de Toni
Morrison}\label{nota-sobre-o-livro-o-olho-mais-azul-de-toni-morrison}

Caros Amigos, n.71, 2003/02. Aguardando revisão.

\hfill\break

A consagrada novelista negra norte-americana Toni Morrison, a primeira a
ganhar o Prêmio Nobel de Literatura, agora em \emph{O Olho mais Azul}
(Companhia das Letras), para muitos leitores brancos norte-americanos e
não racistas se ressente, realmente, de uma óptica voltada
exclusivamente para a mulher negra. Vilipendiada, sem direitos diante do
negro machista, ela logo adquiriu a noção (seria a humilhante certeza?)
de que ``por serem feias'', as mulheres negras correspondem ao
estereótipo de que ``os negros são inferiores'' intelectualmente,
produto de uma genética hierarquizada, criada e mantida pela raça
branca, anterior ao \emph{black is beautiful}. Seviciada sexualmente
pelo próprio pai, esmagada em sua infância pela massacrante publicidade
dos filmes de Hollywood que só propagam e exploram os cachinhos loiros
de Shirley Temple, a garota de ouro (e que trouxe muito ouro para os
estúdios de Hollywood), ela sucumbe a seu sonho irrealizável: ser uma
criança negra, mas de olhos intensamente azuis.

Este livro, que levou 25 anos para ser publicado e foi recusado por
todas as editoras, consigna a frase pungente de Toni Morrison: ``Ouvindo
línguas `civilizadas' aviltar seres humanos, vendo exorcismos culturais
aviltar a literatura, vendo a mim mesma preservada no âmbar de metáforas
desqualificativas, posso dizer que meu projeto de narrativa é tão
difícil hoje quanto o foi trinta anos atrás''.

\part{Literatura Africana}

\chapter{A África em lendas e contos. Lembrando nossa
História}\label{a-uxe1frica-em-lendas-e-contos.-lembrando-nossa-histuxf3ria}

Jornal da Tarde, 1982/1/30. Aguardando revisão.

\hfill\break

Determinante na formação da cultura popular brasileira, os negros para
aqui trazidos da chamada África Negra -- a região situada ao sul do
deserto do Saara e que exclui toda a parte Norte, arabizado, do
Continente, bem como a parte sul, hoje feudo do odioso regime racista do
\emph{apartheid} sul-africano -- raramente chegam até nós manifestações
culturais do Congo, da Nigéria, do Quênia, da Tanzânia. Embora a sua
parte antropológica seja tênue, apresentando a cultura negra com certo
paternalismo e desejando orientá-la segundo critérios europeus de valor,
como por exemplo ao tentar ressaltar a ``antiguidade'' no tempo das
civilizações do Benin e outras, no entanto, \emph{Contos de Lendas da
África} (Editora Melhoramentos) de autoria de Margret Carey nos traz um
apanhado genérico interessante sobre um dos aspectos -- o da tradição de
relatos orais de contos, que remonta a períodos autóctones, quando a
África Negra ainda não estava codificada nem pela colonização dos
missionários cristãos nem pela conversão de vultosos segmentos de suas
populações à religião muçulmana, imposta pelo Islã conquistador.

Com o advento caótico da tecnologia ocidental e do mundo comunista, as
sociedades predominantemente agropastorais dessa parte da África
sofreram uma perda de sua substância autêntica, tendendo a acelerar e
deturpar o ritmo de vida africano, como diz a autora, embora de modo
superficial, ``tentando assimilar em poucos anos aquilo que a
civilização europeia levou de cinco a oito séculos para desenvolver''. O
que já mostra uma angulação viciada: a de impor os critérios
eurocêntricos como objetivos que a África ``atrasada'' deva atingir.
Muito mais relevante para o leitor teria sido um rápido resumo da
história ignominiosa da partilha do território e dos habitantes da
África Negra, na Conferência de Berlim, quando as potências dominantes
da Europa anterior a 1945 retalhavam tribos e fronteiras, impondo
arbitrariamente embriões de Estados, colônias inglesas, belgas,
francesas etc. que deveriam subsistir como futuras nações modernas,
infensas às tradicionais rivalidades tribais. A importância da
demarcação dos territórios forma a manchete de parte do nosso século,
com secessionismo dos ibos contra os iorubas, por exemplo, na luta
fratricida da Nigéria, na província de Biafra.

Aimé Césaire, o grande poeta antilhano negro, que concebeu a poesia da
\emph{négritude}, logo abraçada pelo senegalês, mais tarde presidente de
seu país, o poeta culto, europeizado, Léopold Senghor (uma corruptela do
português Senhor, tornado sobrenome naquele país de língua oficial
francesa), constitui uma tentativa de retomada da consciência autóctone
da África Negra. Mas é seu um dos versos mais amargos e pungentes da
poesia contemporânea, quando ele, entre um lamento e um muxoxo de
desprezo, descreve os negros como ``aqueles que nada inventaram'': nem a
penicilina nem o avião, nem a televisão nem a bomba atômica, nem o
computador nem o trem ou o navio a vapor. Estudos mais profundos, de
religiões comparadas, levadas a cabo principalmente na Universidade de
Cambridge, na Inglaterra, revelam a afinidade, até hoje insuspeitada mas
hoje claramente documentada, entre as religiões negras da África e a
mitologia da Grécia clássica. Grande parte dessa cosmogonia de deuses,
em suas relações com os humanos, conservou-se no Brasil, com seu acervo
de cerimônias religiosas e seu panteão umbandista baiano. Nada disto o
leitor encontrará, infelizmente, nesta obra feita mais com o intuito de
``justificar'' os africanos do que propriamente apresentá-los sem
comparações ridículas com a Europa. Para os africanos, como para nós
brasileiros, pouco importante é a origem cronológica de um dado
cultural: as lendas populares que contam -- em termos europeus -- apenas
500 anos não nos interessam por sua relativa antiguidade mas pelo que
revelam da psicologia africana -- cujo traço mais interessante é o da
originalidade com que, ao contrário de meras ``estórias'', homens,
deuses e animais convivem o dia-a-dia, de forma muito semelhante às
fábulas de origem índia ou africana transplantadas para o Brasil. Se
aqui, devido às tradições indígenas, o jabuti e o cágado são mestres de
esperteza e finórios malandros, no que restou da África pré-colonialista
a tartaruga e a lebre são símbolos da argúcia dos mais fracos que vencem
a força bruta do homem e de outros animais de maior porte. Infelizmente,
com o advento do domínio árabe a vastas partes do território negro, ao
sul do Saara, houve uma distorção mutiladora: o contador oral de
histórias profissionalizou-se ao se dedicar a entreter as multidões, em
sua maioria analfabeta. No entanto, a linha geral da história contada
africana sofreu menos: ``As histórias não são usadas como veículo para
expressar o desejo de auto-realização, a injustiça é aceita, o herói nem
sempre triunfa e os crimes podem passar sem castigo. Esta última
característica aplica-se principalmente às histórias populares que têm
por herói um malandro. Um deles, Kwaku Ananse, de Gana, homem e aranha
ao mesmo tempo, é sobejamente conhecido''. Em todo o continente
repete-se as aventuras que têm por protagonista a tartaruga e a lebre
(cognome do Coelho Brer). ``Seria privar o leitor, porém, do encanto que
estes relatos só aparentemente simples mas, na realidade, profundamente
filosóficos, oferecem, não transcrever alguns de seus melhores
momentos''.

(História dos hotentotes que, junto com os bosquímanos, constituem os
mais antigos sobreviventes da raça africana.)

\begin{quote}
``Conta-se que certa vez a Lua enviou um inseto aos homens,
recomendando-lhe: Vá até eles e diga-lhes: Assim como eu morro e depois
renasço, assim vocês morrerão e depois ressuscitarão!''

O inseto partiu com a mensagem, mas, na viagem, foi apanhado pela lebre,
que lhe perguntou: ``Quem é você?''

O inseto respondeu: ``A Lua me enviou aos homens como mensageiro para
dizer-lhes que, assim como ela morre e depois renasce, assim eles
morrerão e não renascerão''.

Depois a lebre voltou à presença da Lua e disse-lhe o que havia dito aos
homens. A Lua ficou furiosa, respondendo: ``Como você ousa dizer aos
homens uma coisa que eu não disse?''

E assim falando apanhou um porrete e golpeou o nariz da lebre. Desde
esse dia o nariz da lebre ficou fendido, mas os homens continuam a
acreditar naquilo que ela lhes disse''.
\end{quote}

A fina ironia da crendice estulta dos homens, o elemento mágico como
parte ``natural'' da vida formam um elo unitário entre todas as regiões
culturais da África Negra, mas a civilização, como nós preconcebidamente
a rotulamos de tal, eclode com mais amplidão e mais fundo embasamento no
tempo na região da África Ocidental, que desce pelo litoral, desde o
Senegal, passando pela Guiné, Serra da Leoa, Libéria, Costa do Marfim,
Ghana, Daomé, até atingir a Nigéria e Camarões. Área de intensa presença
europeia -- principalmente holandeses e portugueses, cujas fortificações
dos séculos XV e XVI ainda existem -- o comércio de grãos, de marfim, de
outro estendeu-se, por iniciativa árabe logo seguida pelos colonizadores
rapaces europeus, ao nefando comércio escravagista. Exilados à força, e
com a conivência de chefes tribais africanos que vendiam os prisioneiros
de tribos vencidas como botim de guerra aos mercadores brancos, esses
africanos de compleição robusta, altos, saudáveis, despertaram a cobiça
dos europeus estabelecidos nas Caraíbas em ``importá-los'' como força
escrava. Ora, os cultos ancestrais dessas regiões ricas não tardaram a
se aclimatar no solo das Américas, como \emph{vudu} do Haiti: toda uma
religião animista, que inclui o culto dos ancestrais, a visão mágica da
vida, o comércio com deuses influenciáveis pelas oferendas humanas,
refletiu, no entanto, algo de mais profundo e mais autenticamente
inerradicável, as sociedades secretas dessa parte da África litorânea.
Os rituais são a tônica dessas sociedades, como a Sociedade do Leopardo,
que chegam ao assassínio, embora a maioria das sociedades se assemelhe
às maçonarias, a sindicatos de mútua proteção, ou clubes de benefício
recíproco em que a riqueza de poucos é compartilhada pela comunidade dos
pobres numerosos. Outras seitas, como os bundos ou porôs, iniciam os
jovens de ambos os sexos na vida adulta, zelam pela manutenção da lei e
da ordem, regulam a pesca e a colheita. Uma das riquezas artísticas
dessas culturas é constituída pelas máscaras, das mais variadas
concepções e finalidades, no entanto as máscaras -- que por via indireta
fecundaram as artes plásticas da Europa Ocidental no quando \emph{Les
Demoiselles d'Avignon} de Picasso entre outros numerosos exemplos -- são
apenas a parte mais conhecida de uma estatuária abundante. As cabeças de
terracota encontradas em Nok, no norte da Nigéria, datam, segundo os
testes de radiocarbono, de cerca do ano 250 antes de Cristo, enquanto a
tradição de uma variedade de escolas de utilização artística do bronze
``foi trazida de Ife para Benin por volta do ano 1400 depois de Cristo,
a pedido de Obá Oguola de Benin, em 1897, fruto em parte da estrutura de
organização tribal, com os Estados-nações já avançados como Achanti,
Daomé e os reinos dos iorubas Benin e Nupe. Da Serra Leoa provém a
fábula poética e simbólica intitulada \emph{Por que a Cobra muda de
Pele} :

\begin{quote}
``No princípio a morte não existia. A morte vivia com Deus, e Deus não
queria que a morte entrasse no mundo. Mas a morte tanto pediu, que Deus
acabou concordando em deixá-la partir. Ao mesmo tempo fez Deus uma
promessa ao homem: apesar de a morte ter recebido permissão para entrar
no mundo, o homem não morreria. Além disso, Deus prometeu enviar a homem
peles novas, que ele e sua família poderiam vestir quando seus corpos
envelhecessem.

Pôs Deus as peles novas num cesto e pediu ao cachorro para levá-las ao
homem e sua família. No caminho, o cachorro começou a sentir fome.
Felizmente, encontrou outros animais que estavam dando uma festa. Muito
satisfeito com sua boa sorte, pôde assim matar a fome. Depois de haver
comido fartamente, dirigiu-se a uma sombra e deitou-se para descansar.
Então a esperta cobra aproximou-se dele e perguntou o que é que havia no
cesto. O cachorro lhe disse o que havia no cesto e por que o estava
levando para o homem. Minutos depois o cachorro caiu no sono. Então, a
cobra, que ficara por perto a espreitá-lo, apanhou o cesto de peles
novas e fugiu silenciosamente para o bosque.

Ao despertar, vendo que a cobra lhe roubara o cesto de peles, o cachorro
correu até o homem e contou-lhe o que acontecera. O homem dirigiu-se a
Deus e contou-lhe o ocorrido, exigindo que Ele obrigasse a cobra a
devolver-lhe as peles. Deus, porém, respondeu que não tomaria as peles
da cobra, e por isso o homem deveria morrer quando ficasse velho. Desde
então o homem passou a ter um ódio mortal à cobra, e sempre que a vê
procura matá-la. A cobra, por seu turno, sempre evitou o homem e sempre
viveu sozinha. E, como ainda possui o cesto de peles fornecido por Deus,
pode trocar a pele velha pela nova''.
\end{quote}

Já a África Oriental, segundo as escavações de Leakey; berço do homem
africano aborígene, revela um cadinho de raças que no Quênia de hoje se
solidifica numa sociedade multi-racial, na qual entram não só elementos
negros, como também brancos e orientais, um pouco à feição do Brasil,
não fosse pela miscigenação extremamente tênue ainda existente naquele
país do Continente Negro. Em grande parte ainda nômades, segmentos
importantes de suas populações são constituídas de pastores das tribos
nilo-hamitas como os massais, os suk e os karamodjos. Para algumas
dessas tribos, o gado é sagrado, existindo mesmo palavras especiais para
designá-lo. Os nandis chegam a tomar cuidado para não misturar carne com
leite numa mesma refeição. Costuma-se considerar os massais como o grupo
mais aristocrático: são elegantes, orgulhosos e conservam muita coisa do
seu gênero de vida original. Os homens dessas tribos usam geralmente
muito pouca roupa, enquanto as mulheres trajam longas vestes de couro,
usando como adereços espirais de latão colocadas nas pernas, nos braços
e no pescoço.

Os homens são circuncidados e iniciados em grupos etários e,
eventualmente, tornam-se guerreiros. O guerreiro massai deve matar um
leão com sua lança, utilizando apenas uma das mãos, para mostrar sua
coragem. Por volta dos trinta anos eles se aposentam, casam-se e
tornam-se chefes. O gado constitui a principal riqueza, requerida para a
compra de uma esposa. Entre algumas tribos, o sangue, tirado de uma veia
do pescoço da vaca, é o alimento favorito dos homens.

Quanto à religião, todas as tribos acreditam num ser supremo, sendo
também importante o culto dos ancestrais, que atuam como mediadores. O
curandeiro exerce um papel de relevo: pratica adivinhação, faz chover,
proporciona fertilidade, aconselha no plantio e decide se os augúrios
são favoráveis à guerra. A África Oriental, fortemente cristianizada,
arabizada e hoje apresentando um mosaico de opções políticas que vão de
um marxismo \emph{sui generis} de Moçambique ao despotismo recém-deposto
de Idi Amin Dada em Uganda até o ``socialismo'' de feições cristãs
protestantes da Tanzânia e Estados extremamente carentes como o Burundi,
a Ruanda e o Malawi. Para o Ocidente, reveste-se de dramaticidade,
poesia e alegoria a lenda tradicional da tribo baganda, de Uganda, que
relata com uma pré-ciência multi-secular, a impossibilidade que os
judeus já tinham expresso por meio do ``Golem'' e a tecnologia moderna
por meio do robô humanizado de repetir o ser humano por meios mecânicos,
como nos ensina de maneira saborosa e filosófica o relato de \emph{Como
Walukaga, o Ferreiro, Respondeu ao Rei}: ``Há muito, muito tempo, havia
um ferreiro chamado Walakaga, artesão habilíssimo, o mais capaz de todo
o país. Chefe dos ferreiros reais, fazia ele todo tipo de trabalho --
enxadas para os lavradores e mulheres, espadas para os homens e
guerreiros, foices e machados para cortar as florestas; e, além de tudo,
era capaz de fazer lindas figuras de ferro para o rei.

\begin{quote}
Um dia o rei mandou um mensageiro chamar Walukaga, pois tinha um
trabalho muito especial a encomendar-lhe. Walukaga obedeceu de bom grado
e, após vestir sua melhor roupa, foi ao palácio do rei, sendo recebido
no pátio interno, onde o rei estava sentado para a audiência. Walukaga
aproximou-se do rei e fez-lhe uma mesura, tocando o chão com a cabeça. O
rei disse então: '' -- Walukaga, você é o chefe dos meus ferreiros, o
mais hábil de todos. Nenhum deles consegue fazer figuras de ferro tão
bonitas como as suas. Mas agora tenho um grande trabalho para você, pois
ninguém mais é capaz de executá-lo''.

Dito isto, bateu palmas e alguns criados apareceram com uma grande
quantidade e ferro, pronto para ser forjado. O rei continuou:

'' -- Walukaga, quero que você pegue este ferro e, com o seu martelo,
forje um homem para mim. Não quero uma estátua pequena, nem uma estátua
de ancestral. Desejo um homem de verdade, que possa falar e andar, com
sangue as veias, conhecimentos na cabeça e sentimentos no coração''.

Walukaga escutou o rei com assombro e desespero, mas tornou a
inclinar-se e levou o ferro para casa sem esboçar o menor protesto.
Sabia muito bem quão absolutos eram o poder e a vontade do rei. Se não
conseguisse fazer o que o rei mandava, ele e toda a sua família seriam
obrigados a tomar veneno do pote e morrer. A partir desse momento, não
teve paz. Por mais que se esforçasse, não sabia nem por onde começar.
Visitou todos os seus colegas de profissão, e todos os seus amigos,
contou-lhes o seu problema, implorando-lhes para ajudá-lo. Infelizmente,
ninguém estava à altura de dar-lhe algum conselho.

Houve, naturalmente, sugestões impraticáveis. Ele poderia tentar fazer
uma estátua de ferro oca e colocar alguém lá dentro para fazê-la falar e
andar. Mas isso não seria honesto e o rei poderia desconfiar. Ou então,
poderia sair do país e ir para outro lugar que ficasse a muitos dias de
viagem, fora do alcance do rei, onde ninguém tivesse ouvido falar nele,
e ali começar a vida nova, pois a um bom ferreiro nunca falta trabalho.
Mas isso significaria deixar seus amigos e parentes a mercê da cólera
real.

Um dia Walukaga estava voltando para casa, após ter visitado uns amigos
em busca de conselho. No caminho, encontrou um velho conhecido que
ficara louco e agora vivia no mato, sozinho. Walukaga não sabia que ele
tinha enlouquecido, mas quando o cumprimentou o louco o reconheceu e
respondeu-lhe de maneira bastante racional. Sentaram-se, falaram de
cousas e lousas e finalmente o louco perguntou a Walukaga o que o estava
atormentando. A isto Walukaga, suspirando profundamente, respondeu que
estava a braços com um problema desesperador de cuja solução dependia
sua própria vida. O louco mostrou-se interessado e pediu-lhe para contar
sua história. Walukaga, refletindo que o amigo parecia bastante lúcido e
que, afinal, contar sua história não faria nenhum mal, explicou-lhe a
ordem que recebera do rei e sua dificuldade para cumpri-la. O louco
ouviu-o em silêncio e quando Walukaga, meio sério, lhe perguntou o que
fazer, explodiu em gostosa gargalhada.

Disse então ao amigo: ``- Se o rei lhe está pedindo algo impossível,
você não deve fazer por menos. Vá até o rei e diga-lhe que, se ele
realmente deseja que você faça esse homem prodigioso a partir de ferro
frio, capaz de andar e conversar, com sangue nas veias, conhecimentos na
cabeça e sentimentos no coração, é essencial que você disponha, além do
ferro, de um carvão especial para o fogo e de uma água especial para
apagar o fogo e evitar que ele arda em excesso. Diga-lhe então para ele
mandar todas as pessoas do reino raspar a cabeça e queimar o cabelo até
completar mil cargas de carvão, e para elas chorarem até que a água de
sues olhos encha cem potes''.

Walukaga ficou muito grato pelo conselho, que era de longe o melhor que
recebera, e não perdeu tempo: foi à casa do rei e pediu uma audiência.

Quando foi admitido à presença do rei, Walukaga inclinou-se
respeitosamente e disse: ``- Senhor, se deseja realmente que eu construa
esse homem fabuloso da maneira como me descreveu, preciso de combustível
especial e de água também especial para apagar o fogo''. O rei estava
tão ansioso pelo homem de ferro, que prontamente concordou em dar a
Walukaga tudo o que lhe fosse necessário. Portanto, Walukaga continuou:
``- Senhor, mande todas as pessoas do reino raspar a cabeça e queimar os
cabelos, até perfazer mil cargas de carvão, para eu poder aquecer o
ferro. Depois, mande-as juntar cem vasos de lágrimas, para com elas eu
apagar o fogo e evitar que ele arda em excesso. O carvão comum de
madeira e a água comum dos poços não se prestam para forjar um homem de
ferro''.

O rei enviou mensageiros a todas as partes do reino, exigindo que todo
os seus súditos raspassem a cabeça para o carvão e derramassem lágrimas
para a água.

Ninguém deixou de atender a essa ordem, pois todos temiam o poder do
rei. Mas, depois que todos haviam dado o melhor de si, e todas as
cabeças estavam raspadas e todos os olhos secos, o resultado foi de
apenas uma carga de carvão e menos de dois vasos de lágrimas.

Os chefes do reino foram até o rei e o informaram desse fato. O rei
ponderou por alguns momentos e então mandou chamar Walukaga. Adivinhando
o que estava por vir, Walukaga dirigiu-se o palácio do rei, tremendo. No
entanto, quando olhou para cima após a reverência, o rei disse-lhe: ``-
Walukaga, não precisa mais fazer o homem de ferro para mim. Não posso
dar-lhe o carvão e a água que me pediu.

Walukaga tornou a inclinar-se até o chão e agradeceu ao rei. Então,
olhando para cima disse: ``- Senhor, foi por saber que não seria capaz
de obter cabelo suficiente para o carvão nem lágrimas suficientes para a
água que eu os pedi; o senhor pediu-me para fazer o impossível, ao
mandar-me forjar um homem de verdade, capaz de falar e andar, com sangue
nas veias, conhecimento na cabeça e sentimentos no coração.

Ouvindo isso, todos os cortesões se puseram a rir e disseram:
``-Walukaga diz a verdade''.
\end{quote}

O livro, farto em ilustrações, relata mais algumas lendas tradicionais,
belas e interessantes, justificando sua leitura até o capítulo final,
dedicado aos povos Suailis: neste trecho cessa realmente o interesse por
que a cultura africana se desfigura, mera copiadora dos preceitos
árabes, impostos pela conversão religiosa forçada aos Islamismos e
meramente repetem, sem nenhuma originalidade, preceitos e preconceitos
do Corão, sufocando a autenticidade autóctone e tornando supérflua essa
parte do livro, como seria inútil um relato bíblico ``africanizado'' por
mentes colonizadas por premissas e metas inteiramente estranhas à
mentalidade aborígene da África Negra.

Mas no Brasil de hoje em que, nunca é demais repisar no assunto, como
que uma muralha de silêncio se ergue entre a informação existente no
mundo e que não chega, por meio de traduções, até nós, em qualquer
campo, da ecologia à ideologia e à memória histórica, esse livrinho abre
uma fresta minúscula sobre a herança africana e nos deixa entrever
quanto de fecundante, para o Brasil que os trouxe, além-oceano, e deve
grande parte de sua especificidade como nação, reconhecidamente, ao
elemento africano que aqui se aclimatou e com outros segmentos étnicos
criou uma cultura que, dia a dia, delineia-se como inconfundivelmente
brasileira, fruto de miscigenação como talvez o exemplo único em todo o
mundo contemporâneo.

\chapter{A Negritude - Transfiguração Poética do Rosto
Africano}\label{a-negritude---transfigurauxe7uxe3o-pouxe9tica-do-rosto-africano}

Correio da Manhã, 1964/12/20. Aguardando revisão.

\hfill\break

Em sua obra-prima sobre os anos que viveu na África, a escritora
dinamarquesa Karen Blixen (que usa o pseudônimo de Isak Dinesen)
referiu-se à diversidade dos povos negros e à sua integração plena na
paisagem indígena:

\begin{quote}
``Os indígenas eram a África em corpo e sangue\ldots{} Eram expressões
diversas de uma ideia única, variações sobre o mesmo tema. Não uma
amálgama sintética de átomos heterogêneos, mas uma amálgama
heterogênea\ldots{} em que os negros estão sempre em eterna harmonia com
os elementos da sua terra natal..''
\end{quote}

Dispersos pela História forjada pelo homem branco, em vários
continentes, guardam porém uma origem comum, um passado de ritos, de
ritmos, de danças e mitos ancestrais, herança que perdura confusamente
em sua memória e em seu sangue. Durante séculos, a humilhação do
cativeiro, a destruição violenta de culturas, de hierarquias de nobreza
e de seitas religiosas os segrega, na ``casa grande'' das Américas. Em
contato com uma civilização estruturada, sua erradicação forçada de seu
\emph{habitat} original explode em complexo de inferioridade, em
impossibilidade de integração num novo mundo ao qual lhe era negado o
acesso por meio da pressão econômica, da relegação a um estágio
primitivo em que o analfabetismo constituía o primeiro obstáculo à
ascensão social e cultural. Durante séculos, das plantações de cana de
açúcar se erguem cantos melancólicos, de resignação, de fé no Deus
crucificado dos brancos e cantos nostálgicos que exprimem o desespero
silente desses seres murados entre o trabalho, a promiscuidade e a
tristeza -- são os \emph{blues} e os \emph{spirituals}. O \emph{jazz} é
a primeira manifestação híbrida de uma alegria reconquistada, de uma fé
no futuro, de uma exaltação dos sentidos: forma africana com
instrumentos da Europa.

Mas se a abolição da escravatura nos Estados Unidos custara uma guerra e
a morte de Lincoln, constituíra ao mesmo tempo a primeira conquista do
negro que teria consequências artísticas além de sociais -- da sua raiz
brota a flor do ritmo sincopado de Chicago, Nova Orleans, Manhattan. A
libertação do jugo colonial na África criaria as condições
indispensáveis à nova expressão da alma e da consciência negras, mas em
circunstâncias diferentes. Em contato com a cultura eminentemente
literária da França, fascinados ele também como os americanos da
``geração perdida'' pela Cidade em que germinaram os pensamentos de
liberdade, de igualdade e fraternidade, pela aceitação da arte negra por
Picasso e pelos pintores cubistas, os intelectuais negros -- expoentes
de sua raça -- tomam consciência de sua ``negreza'', nas palavras de
Heidegger: identificam-se ``como negros no mundo''. É a
\emph{Négritude}.

Léoplod Senghor a defende claramente:

\begin{quote}
``Quando nós, poetas negros, cunhamos o conceito de \emph{négritude},
criamos com isso o recipiente para o qual confluíram todas as correntes
da África\ldots{} O homem africano tem que compenetrar-se do seu
passado, das suas origens e aceitá-los conscientemente, isto é a
\emph{négritude} -- uma inter-relação total de todos os valores da
cultura neo-africana e ao mesmo tempo uma defesa da nossa dignidade'':

``E eu me erguerei, ó África, para te anunciar com o olhar imóvel como o
escultor de máscaras''. É a própria máscara que passa a simbolizar o
misticismo e a magia negras, a natureza de uma beleza violenta, o amor e
a dança, o sofrimento e a liberação:

``Tu, semblante de máscara, imaterial e sem olhos, voltado para o
passado,

Tu és perfeita, cabeça de bronze. A pátina do tempo não é maculada pela
\emph{maquillage}, pelo \emph{rouge}, pelas rugas, nem por vestígios de
lágrimas e beijos.

Oh, semblante que Deus criou antes da memória dos tempos,

Semblante da madrugada radiosa, não assumas o encanto,

De um pescoço macio, para excitar minha carne.

Eu te adoro, oh Beleza de olhos mortos e monótonos!''
\end{quote}

Mas não é só na África que a \emph{négritude} encontra sua expressão
poética: Léon Damas, da Guiana Francesa, celebra em seus versos a mesma
paixão por um mundo perdido e canta a inocência da África violada pelos
capitães de negreiros:

\begin{quote}
``Dá-me de volta minhas bonecas pretas;

Quero brincar com elas

Os brinquedos sem peias de meus instintos,

Quero ficar à sombra das minhas leis, reencontrar minha coragem, minha
ousadia, sentir-me eu mesmo de novo, sentir o eu que eu era ontem,

Ontem,

Sem complicações,

Ontem,

Antes da hora de arrancarem minhas raízes''
\end{quote}

Para o movimento da \emph{négritude}, que se alçava justamente quando as
nações africanas emergiam para a independência e para o século XX, não
existe o \emph{ressentiment} cego e fanático contra o colonizador de
ontem. Movimento ``evangélico'' na sua essência, como o definiu Sartre,
constitui politicamente uma expressão de um humanismo pan-africano, uma
reconciliação com o homem branco, senhor de ontem, irmão de agora no
soerguimento material das populações africanas com a sua técnica -- ao
invés do racismo, é uma solidariedade humana superior às diferenças da
pele:

\begin{quote}
``Mas protege-me, meu coração, de todo ódio,

Não faças de mim um servo do ódio,

Pois só odeio o próprio ódio,

Integra-me, funde-me numa única raça,

Pois conheces meu amor que abrange todo o mundo''
\end{quote}

Pela miscigenação, pelo contato direto de raças que convivem através dos
séculos, a \emph{négritude} pode também significar simbiose, harmonia,
encontro pacífico, como a reminiscência de uma remota origem
parcialmente portuguesa de Senghor, cujo nome é uma corruptela de
``Senhor'', nome trazido pelos portugueses a Joal, aldeia natal do
poeta, num oásis do sul de Dacar:

\begin{quote}
``Escuto dentro de mim o canto de voz sombria da saudade.

Será a voz antiga, a gota de sangue português,

Que ressurge do fundo das idades?

Meu sangue português perdeu-se no mar de minha \emph{Négritude}''
\end{quote}

Sobretudo, porém, essa consciência africana é fruto da reflexão do negro
sobre a as suas características raciais: para o êxtase dessa descoberta
de uma identidade própria não há lugar para recriminações coléricas
contra o usurpador de ontem nem contra a discriminação que perdura ainda
hoje, na África do Sul, nos Estados sulistas dos Estados Unidos. À
\emph{Négritude} falta, por escolha, a revolta amarga de seus
contemporâneos norte-americanos cifrada na criação de um Richard Wright,
de um Langston Hughes como também não possui o lúcido poder de análise
de uma inteligência ativa como a que permeia os ensaios de um James
Baldwin. Pois a \emph{négritude} antes de reivindicar novos territórios,
explode em celebração quase mística do seu próprio rosto que contempla
agora no espelho, pela primeira vez, assumindo consciência e
responsabilidade pela sua condição de negro:

\begin{quote}
``Mulher nua, mulher negra,

Vestida com tua cor que é vida, com tua forma que é beleza!

Cresci à tua sombra, a doçura de tuas mãos vedava meus olhos.

E eis que no cerne do verão e do meio-dia eu te descubro.

Terra prometida, do alto de um desfiladeiro calcinado

E tua beleza me fulmina em pleno coração,

Como o relâmpago de uma águia.

Mulher nua, mulher obscura,

Fruto maduro de carne firme, êxtase sombrio de vinho negro,

Boca que faz lírica minha boca,

Savana de horizontes puros, savana que vibra

Às carícias ardentes do vento do Leste

Tantã esculpido, tantã tenso que ribomba sob os dedos do Vencedor:

Tua grave voz de contralto é o canto espiritual do Amado.

Mulher nua, mulher obscura,

Óleo que nenhum sopro enruga, óleo calmo no flanco do atleta,

Nos flancos dos príncipes do Olali,

Gazela de vínculos celestes, as pérolas são estrelas

Na noite da tua pele;

Delícias espirituais, os reflexos de ouro rubro e tua pele que se
ondeia.

À sombra de tua cabeleira, os sóis próximos dos teus olhos iluminam
minha angústia.

Mulher nua, mulher obscura,

Eu canto tua beleza que passa, forma que fixo no Eterno

Antes que o Destino ciumento te reduza a cinzas

Para nutrir as raízes da vida.''
\end{quote}

\chapter{Angola escreve. Uma arte mágica e sofrida vinda de onde não se
sabe}\label{angola-escreve.-uma-arte-muxe1gica-e-sofrida-vinda-de-onde-nuxe3o-se-sabe}

Jornal da Tarde, 1981/07/04. Aguardando revisão.

\hfill\break

África Negra. A mera enunciação das duas palavras desperta
interpretações e fantasias conflitantes. Para alguns, o adjetivo será
considerado insultuoso, típico do racista que mantém o preconceito
inflexível contra tudo o que for ``coisa de negro'', portanto, nesta
acepção \textbf{a priori} os termos significam tudo que for desprezível,
pejorativo, ``inferior. Há depois as nuances de''piedade'' dos
``coitadinhos'', matizes que vão do missionário em catequese sincera,
querendo salvar aquelas almas pagãs para a única Fé, a de Jesus Cristo,
como a devoção dos bem-intencionados que querem transmitir às populações
negras o alfabeto, a roda, a escrita, a indústria, a medicina em vez dos
feiticeiros da tribo, a favela em vez das choças tribais.

Mas a África Negra evoca também escravidão, repulsa ética pela violência
que árabes, portugueses, ingleses, franceses, holandeses, belgas e
outros praticaram, arrancando à força de seu continente de origem
famílias inteiras, desmembradas e levadas do Brasil às Antilhas e à
Virgínia integrando o cativo trazido em navios negreiros até a Nação que
reconhecia em sua Constituição separatista da colonizadora Inglaterra
``\textbf{como fato evidente por si mesmo que todos os homens nascem
iguais perante Deus}''. Nos Estados Unidos da América entendia-se que os
homens têm até o direito constitucional de ``procurar sua felicidade''.
E que felicidade poderia haver nos campos de trabalho forçado no cultivo
do algodão, nas fazendas do Sul, a não ser a alegria dominical do coro
negro nas igrejas protestantes do Senhor branco? Aí se esqueciam as
dores, cantavam-se os salmos que falavam do cativeiro dos judeus na
Bíblia. Eram lamentos que no estrangeiro se adaptavam perfeitamente aos
negros cativos em terras da América. A melancolia resignada, temperada
pela Fé dos \emph{spirituals}, aquela melodia plangente adaptada aos
versos severos do Velho Testamento é a parte sacra da arte do desterro
negro. Os \emph{blues} serão a sua saudade e tristeza profanas, em
contraste com a Fé rural, os olhos voltados para vida plena depois da
morte, no Céu sem epidermes de Jesus e seus Apóstolos, regido por um
Deus justo, amoroso e sem etnias de almas. Os \emph{blues} falando de
amor frustrado, de miséria, de marginalização social urbana dos
discriminados entregues à bebida, à pobreza, ao trabalho mal pago, à
promiscuidade -- eram ``\textbf{coisas de negro}''.

Depois, com a figura excelsa do reverendo Martim Luther King, que se
inspirou na nobreza ética e religiosa de Gandhi, na Índia, de não apelar
para a violência a fim de obter a defesa de direito dos oprimidos,
surgem as marchas em favor do voto negro, do acesso dos negros às
Universidades, aos \emph{drugstores}, aos motéis, aos hospitais, aos
bebedouros e bancos de jardim ``só para brancos''. A revolução do
\emph{black pride} -- o orgulho de ser negro -- explode nas cidades,
pinta murais de negros ilustres nas paredes de Chicago, impõe o estudo
de línguas como o suahili, do Quênia, muda o nome do boxeador negro
Cassius Clay para Mohamad Ali, convertido à fé islâmica dos \emph{black
muslins}. Esquecido, é verdade, em seu fervor muçulmano impulsivo de que
os árabes foram, historicamente, dos primeiros ``inventores'' da
escravidão de negros, um ``comércio'' rendoso nas ``partidas'', para as
Américas, de ``cargas'' humanas.

Não importa: James Owens, atleta negro, vencerá as Olimpíadas em Berlim,
no ano de 1936, quando a ``superioridade intrínseca e imbatível da raça
branca'' tinha já sido proclamada absurdamente pela doutrina nazista.
Hitler, no estádio repleto, se recusara a apertar a mão daquele ``macaco
corredor'', e daí? A vitória do negro nos esportes só iria se acentuar
nas décadas seguintes. No futebol, Pelé, no boxe Joe Louis; na música, o
\emph{jazz}; no canto erudito, Marian Anderson; na música popular do
Brasil, da Jamaica, dos Estados Unidos, a liderança e a criatividade
negra sempre determinantes. De repente, é bonita a carapinha, o
denegrido ``cabelo ruim'' desprezado pelos brancos ou pelos mulatos
claros passa a ser moda e autenticidade. Nada de alisamentos a ferro:
como campânulas, os cabelos aureolam rostos que não se mostram mais
submissos e que criam uma cultura negra dentro do \emph{Establishment}
dominante branco.

Numa euforia até hoje insuperável, descobre-se a celebração literária:
grandes poetas como Aimé Césaire e Léopold Senghor (ex-presidente do
Senegal) elevam a \emph{Négritude} a seus píncaros: exalta-se a beleza
da mulher africana, como Picasso se influenciava pelas máscaras da
África Negra; reconhecem-se os valores negros como incomparavelmente
superiores, eticamente, à crueldade do egoísmo e à injustiça do branco:
em vez da luta individual consumista e sem nexo, a ternura, a doçura, a
solidariedade comunitária, o desapego às conquistas meramente materiais:
o emprego bem pago, mas que escraviza o ser humano à linha de montagem
repetitiva e alienante, ao automóvel, ao álcool, ao ter muitas posses e
ser apenas um dente de uma engrenagem que tritura todos seus
componentes.

Os argumentos dos racistas se tornam, pelo avesso, motivo de orgulho: os
pretos nunca inventaram nada, nem a roda, nem a crise econômica, nem o
clorofórmio, nem a penicilina, nem a pólvora, nem a escola, nem as
letras, nem o trem, o vapor, a fábrica, a poluição, a perda da
identidade de cada um na massificação anônima das violentas metrópoles
modernas. Paralelamente, uma a uma, dezenas de ex-colônias na África
Negra se emancipam, depois de séculos de dominação europeia, n década de
60. Se, de meados de 1800 até meados deste século a África foi uma torta
repartida ao sabor dos comensais do banquete africano nas conferências
europeias da alegre partilha do Continente Negro, agora no pós-guerra,
depois de desmoronado o nazifascismo, com Hitler, Mussolini e logo
Franco e Salazar, as nações africanas surgem, miseráveis mas com um
potencial plural inimaginável: petróleo, minério, terras férteis, mentes
abertas e alertas, reerguendo tradições seculares pisadas pela
arrogância do homem branco missionário do Partido Único da Fé
Verdadeira. A minúscula e sórdida administração do Congo, composta por
funcionários belgas, por exemplo, ao debandar de sua imensa colônia
africana deixa\ldots{} por mais inacreditável que pareça\ldots{} deixa
apenas 14, exatamente dez mais quatro, formados por universidades: quem
assume o poder no novo país, o Zaire, é obviamente, o caos.

Mesmo Portugal, que com mão férrea estrangulara o Brasil até 1822, é
forçado a abrir mão de suas remotas ``províncias ultramarinas'': os
níveis médios da hierarquia militar portuguesa reconhecem e demonstram a
inutilidade, o desperdício, a injustiça estéril de querer se manter uma
ilusão alimentada séculos a fio. A independência política, porém, não
passa de uma etapa formal, propensa a retóricas fáceis e retumbantes nos
discursos nem populistas de tão demagógicos: populeiros na sua exaltação
de uma soberania \emph{pro forma}, que só seria real quando tivesse o
sustentáculo dos adjetivos substanciais: econômica, cultural,
ideológica, efetiva. Pobre Angola! Mal passado o longo pesadelo do
colonialismo do fascismo de direita, é vítima do fascismo de esquerda,
das tropas de Fidel Castro a recolonizar, como prepostos da Mamãe
Rússia, as terras ricas, as mentes fecundáveis, a economia, como dantes,
espoliável. Novamente um poeta sobe, na África Negra, ao poder:
Agostinho Neto. Mas sobe sobre armas, apoiado na facção que venceu a
dilaceração fratricida, o MPLA, até hoje em escaramuças mais ou menos
sérias com o grupo do Unitas. Aliás, cada vitória, ou derrota depende do
\emph{parti pris} de cada agência noticiosa ou de cada jornalista e sua
interpretação abusivamente subjetiva dos fatos que nos chegam
deturpados, longínquos, praticamente inverificáveis\ldots{} A criação
artística, essa brotava mesmo entre as grades da prisão como já se
tornara comum na literatura desde Dostoievsky preso pelo Czar, até Ho
Chi Minh e Solzhenitsin e Graciliano Ramos, se tomarmos em seu sentido
mais lato a criação literária, depoimento e volição. Agostinho Neto,
antes de morrer, assinala em seus versos candentes, vigorosos, seu
desprezo pela civilização branca, ocidental, \emph{grosso modo}, que
jogou o negro de Angola no Rio de Janeiro, na Bahia, no Harlem, nos
mesmos ``alagados'', nas mesmas favelas, nos mesmos \emph{slums}
fétidos, e infra-humanos. A exploração do operário preto a cavar nas
minas dos brancos na África do Sul do \emph{apartheid}, a abrir os
esgotos das castas claras dominantes é uma elegia densa, trágica que ele
traça candentemente em seus poemas. São os momentos em que o caderno de
versos repousa ao lado das armas, com a ajuda maciça dos Volódias
(apelido carinhoso que se dá em russo para os homens de nome Wladimir),
sua ajuda em seus apetrechos bélicos vários: Kalinishkovs, tanques,
aviões Illyushin e uma chusma imensa de ``irmãos cubanos'' e
``conselheiros'' soviéticos que vivem segregados em guetos longe dos
habitantes aborígenes de Luanda. Como são numerosos os ``conselheiros''!
Angola precisa, urgente, de milhares de ``adidos da Embaixada'', depois,
sim, de agrônomos, professores, engenheiros, peritos em balística e
estratégia militar, enfermeiros, operadores de rádio, operários
graduados da indústria siderúrgica, mineira ou de extração de petróleo
ou diamantes, pois não?

Já o Brasil, que sempre fizera ouvidos moucos à libertação das colônias
ultramarinas portuguesas, para não desfazer o elo sentimental do
triângulo de rebuçados de Lisboa a se desenhar sobre o Atlântico: lá em
cima a Terrinha-Metrópole insaciavelmente ávida de bens materiais --
terras, pedritas preciosas, impostos, escravozitos, minérios, ai,
qualquer coisa que pudesse ser abocanhada sem escrúpulos e com farsas de
cristianíssima vergonha. Este era o vértice principal do triângulo
cabalístico-mítico que desaguava, em suas partes inferiores
(geograficamente, bem entendido) nas terras do Brasil e da África, onde
se tentava falar, com certa graça e até com açúcar (de cana) o
``\textbf{pretoguês}'', ora, pois!

Hoje, a avidez pelos mercados africanos da defunta África Lusitana (ai,
o rio Luando, não leva mais riquezas par o rio Mondego!) mudou os
vértices do triângulo: sôfrego, o Brasil foi o primeiro país a
reconhecer o governo marxista-leninista da Angola do MPLA: também, tudo
pesado, quem ajudará remotamente a Unitas a reconquistar Angola amanhã?
E é hoje que o Brasil precisa do petróleo abundante de Angola e de
colocar no mercado africano seus manufaturados, suas exportações de
alimentos, tudo envolto no manto retórico-hipócrita sentimental das
``afinidades étnicas''. É pena que não tenhamos elefantes, mas em
compensação os africanos não têm índios, nem falam tupi-guarani; fora
disso, o lema é sagrado, perdão, sagrada e a Revolução. Como, qual
Revolução? A da libertação de Angola, é lógico. E vivam tanto Angola
livre como os mercados livres de Angola e suas jazidas minerais! Não
insistamos nas leves incongruências de um País que defende a civilização
democrática e cristã abraçar afoito um irmão que nega Deus pela Bíblia
leiga escrita por Marx e Engels (que muitos nativos acham serem a mesma
e uma pessoa, nova encarnação daquele estranho filho de Deus, Jesus
Cristo, que morreu em nome da redenção de \textbf{todos} os homens).
Isso são detalhes.

Nessa tragi-comédia político-econômica, sobre a literatura. Tirando meia
dúzia de profundos conhecedores brasileiros do que já havia de
literatura importante no mundo de expressão portuguesa, como se diz
pedantemente na eufemística Avenida da Liberdade de Lisboa, desde os
tempos de Salazar \& Cia. Ltda., a Editora Ática, paulista, está
trazendo para cá raridades nunca sonhadas pelo leitor brasileiro. Como
sempre acontece, porém, quando intervém o colonialismo (português ou
russo, belga ou francês, não importa) junto com a separação das tribos
milenarmente assentadas em determinadas regiões devido ao recorte ditado
pela cobiça europeia na Conferência oitocentesca de Berlim, também os
livros e autores nos chegam desamparados, órfãos de qualquer
ancestralidade. Teria a literatura angolana brotado do nada? \emph{Ex
nullo magnificat}? Porque os vários autores que nos chegam, quase vinte,
nos são apresentados, tanto em prefácios como em notas da Editora, como
criadores de uma literatura contemporânea, vivaz. De acordo, mas
surgidos não se sabe de que origens. É infantil querer dar como única
matriz literária a fogueira do patriotismo da emancipação política: mais
de 400 anos decorreram também em Angola antes da independência de
Portugal. Em Angola, não há os nossos equivalentes a Gregório Matos
Guerra? Um piedoso sacerdote português, êmulo do Padre Vieira? Um poeta
que sonhasse, em versos, com a independência africana como entre nós o
mineiro Claudio Manoel da Costa? Se não houve acesso por parte dos
escritores angolanos a um arcadismo português, por que não a um
romantismo, um realismo, um simbolismo, embora todos ainda de importação
europeia?

Se os diversos autores que a Editora Ática nos apresenta não têm
antecessores mencionados, logo se destacam tendências na prosa de
Angola, uma prosa, é preciso reiterar, recentíssima, como nos é
apresentada, pois são todos autores das últimas duas décadas. Sem nenhum
maniqueísmo, mas como constatação apenas, são duas as correntes
principais de Angola.

Há os autores políticos, que escrevem profundamente engajados com a
ideologia da Libertação, com os traumas e ferimentos da Revolução, sobre
a longa violência da dominação branca e discriminatória de forma sutil,
vale dizer: hipócrita.

E há os escritores -- na maioria pretos, mas não exclusivamente -- que
se dedicam recapturar o lirismo nativo, o mundo de magia, de ironia
cheia de graça, de pilhéria, de ternura e hierarquia etária típica das
sociedades comunitárias africanas, que não votam o velho ao exílio
social, em que os nossos idosos apodrecem nos ``asilos'' ou a
perambular, esmolando, pelas ruas.

Em ambas as vertentes, há uma acentuada mescla de idiomas -- diálogos
inteiros são reproduzidos em quimbundo, um dos idiomas aborígenes
angolanos, ao lado da narrativa, que transcorre predominantemente em um
português de gramática incorreta quase sempre. Um português que para um
leitor brasileiro soa castiçamente lisboeta no léxico, com termos que
caíram em desuso no Brasil, se é que aqui tiveram guarida durante algum
tempo: \emph{miúdo} em vez de menino, \emph{bichas} em vez de filas,
\emph{rebuçados} em lugar de balas de confeitaria, além de muitos outros
vocábulos. As falas de transcrição fonética do quimbundo (vez ou outra
se menciona um dialeto ou outra língua nativa) se, por um lado, dão todo
o interesse que o desconhecido e a magia da estrutura de sons trazem a
um texto novo, por outro lado interrompem o fluxo narrativo, pois o
leitor tem de recorrer, forçosamente, às notas de pé de página ou, mais
laborioso ainda, aos glossários do final do livro, onde os diálogos e as
palavras estão classificados alternadamente por ordem alfabética ou pelo
número de página em que aparecem: maçante depois de alguns minutos de
prazer interrompido pela incompreensão.

Devido à fartura de títulos e nomes, a escolha sabidamente arbitrária de
uma análise genérica baseada em gêneros -- romance, conto, poesia --
visa a evitar o caos d mera enunciação de obras e autores. Assim, neste
primeiro segmento, o da literatura em prosa, destacam-se, nitidamente,
de um lote de seis escritores, os dois primeiros: Uanhenga Xitu (que
também usa o pseudônimo aportuguesado de Agostinho Mendes de Carvalho) e
José Luandino Vieira, criadores de livros encantadores ou aterradores,
conforme o tom de graça colorida de fantasia e lirismo, de Uanhenga
Xitu, ou o relato político de tortura e da repressão da litografia
veemente de José Luandino Vieira. Desbotados os restantes: Pepetela
(nome de guerra de Artur Pestana), Manuel Pedro Pecavira. Um caso à
parte -- e de extraordinário impacto dentro de uma visão ortodoxamente
marxista, mas literariamente válida e, por vezes, fascinante de
inteligência e sátira -- é o de Mauel Rui, contista absolutamente
excepcional.

Os livros que mais inegavelmente se destacam, tão objetivamente quanto
for possível uma avaliação que quer ultrapassar o subjetivismo de uma
avaliação crítica, honesta, isenta de \emph{parti pris}, são, sem
dúvida: \emph{A vida verdadeira de Domingos Xavier}, de José Luandino
Vieira, e os dois tomos de Uanhenga Xitu: \emph{Maka na senzala} (com o
subtítulo de \emph{Mafuta}) e \emph{Manana} -- sobretudo este,
deliciosamente reminiscente de um Lima Barreto \emph{naïf} , ambas
edições cuidadíssimas portuguesas, das Edições 70, com distribuição no
Brasil entregue à Livraria Martins Fontes, de São Paulo.

José Luandino Vieira vem precedido de resenhas tão laudatórias quanto
arrevessadas de críticos e professores portugueses -- fado amargo a que
não escapam os livros maravilhosamente singelos de Uanhenga Xitu\ldots{}
Mas José Luandino Vieira tem uma lição espantosa a recordar aos
brasileiros de hoje, fora de qualquer perspectiva ideológica ou de
participação ativa na Revolução de libertação de Angola e sem a menção,
literariamente secundária, dos empregos que teve ou de sua idade (46
anos par aos curiosos de cronologias precoces ou tardias). José Luandino
Vieira vê a tragédia do colonialismo do racismo, da opressão diária,
mesquinha e múltipla, com os olhos voltados para o fraco, o pisoteado, o
que em Angola quer dizer, em 90\% dos casos, o negro, \emph{A Vida
Verdadeira de Domingos Xavier}, no entanto, restabelece aquela noção que
o dramaturgo Vianinha (Oduvaldo Viana Filho) já, antes de morrer, aos 38
anos de idade, considerava basilarmente normativa para a literatura que
retratasse o real sem deturpações nem sem ``rebaixar-se até o povo'' por
meio de maniqueísmos idiotizantes ou de uma pobreza de conceitos e de
vocabulário que, na realidade, consistiam, na realidade, numa afronta ao
povo. \emph{A Vida Verdadeira de Domingos Xavier} é uma narrativa que
deslumbra o leitor pela sua veracidade: há seres humanos, com matizes de
qualidade e defeitos, não há heróis sem mácula nem tiranos monolíticos.

É claro que José Luandino Vieira nunca esconde os pormenores
horripilantes do preconceito racial, mesmo entre os portugueses, que de
todos os povos europeus é, talvez, o único a se destacar por uma quase
(quase, acentue-se) total cegueira étnica. Como as senhoras portuguesas
que reclamam da entrada de um passageiro preto em farrapos numa
condução, pois isso feria suas narinas ociosas do lar.

\begin{quote}
``O motorista até já tinha espreitado, resmungando qualquer coisa. O
operário, pedreiro ou caiador, trazia o fato coberto de nódoas de cal e
os seus pés se escondiam nuns velhos quedes. Assim como estava, o
cobrador achava que ele não podia viajar. Duas senhoras brancas
concordaram, acrescentando que, qualquer dia, nenhuma pessoa decente
podia andar nos maximbonbos (ônibus) por causa do cheiro dos negros
(sic)''
\end{quote}

O autor não pinta bonecos de cera do Museu de Mme. Tussaud: ao
contrário, os cipaios (tropas africanas a serviço dos antigos
colonizadores portugueses) e os pretos empregados nas empresas
comerciais ou estatais dos dominadores são indivíduos: uns servis, sem
dignidade diante de tudo o que o branco afirma ou inventa, outros se
rebelam surdamente; outros se envaidecem com os cargos de brilharecos
com os quais o regime colonialista lhes adoça a alienação e a escravidão
disfarçada; outros ainda ignoram os brancos, fiéis à sua tradição
aborígene, mas sem delatar os que tramam a derrubada do regime fascista
português daquela época anterior à libertação formal de Angola. Domingos
Xavier adquire, sem exageros, a grandeza histórica de um Jean Moulin, o
inacreditável membro de uma \emph{Résistance} francesa que morreu vítima
da Gestapo, depois que os nazistas lhe arrancaram os testículos, as
unhas, os dentes, e os olhos mas não lhe arrancaram os nomes dos
companheiros da luta contra o ocupante alemão. São inesquecíveis as
páginas secas, carregadíssimas de horror e altivez, que descrevem, com
uma secura que não desagradaria a Graciliano Ramos, as torturas e a
resistência sobre-humana que um humilde tratorista negro lhes opõe com
seu silêncio destemido e teimoso: ``Não digo!'' tornando-se o trissílabo
do seu vigor incoercível. O leitor merece ler este livro terrível, que
fala de um problema que se coloca no Brasil, no Uruguai, na Argentina,
em Cuba, na União Soviética, no Cambodge, onde quer que haja torturados
e mártires voluntária ou involuntariamente silentes.

Para um país como o Brasil, que se apresentou praticamente de gavetas
abertas, iniciada a abertura política, este livro é um soberbo exemplo
de literatura política válida, sem reduções ridículas a chavões e sem
ceder ao pieguismo ou ao doutrinarismo político: um Hemingway radicado
em Angola e em certos pontos mais certeiro do que o autor de \emph{Por
quem os Sinos Dobram}'', é esse José Luandino Vieira, cujas obras
restantes serão aguardadas com ansiedade justificada pelo público no
Brasil que se interessa por literatura político-social de tão alta e
rara categoria. Sem ser um Goytisolo nem um Jorge Semprum, José Luandino
Vieira, incomparavelmente menos culto o que estes, mais direto e mais
voluntariamente testemunha simples de seu tempo, é talvez o único --
haverá outros -- que no campo da língua portuguesa nos recorda que a
arte engajada tem o seu primeiro e inevitável engajamento com a arte
como depoimento, não como sectarismo panfletário e artisticamente
famélico.

Uanhenga Xitu é muito mais difícil para quem não conhece o quimbundo, de
que se serve com uma fartura talvez exagerada em suas evocações de uma
África brincalhona, galhofeira, , impregnada de feiticeiros, de sortes
tiradas quando surgem pássaros portadores de decifrações do futuro, de
uma população em transição entre os doutores do hospital (europeu,
ocidental) e os feiticeiros curandeiros da milenarmente ancestral
tradição tribal. É, porém, retrato contagiantemente delicioso da mulher
africana que Senghor já cantara em seus esplêndidos versos da
\emph{négritude}, que Uanhenga Xitu desenha com mil ardis, mil trapaças
ao lado do homem casado que esconde da linda moça de um bairro distante
do seu, em Angola, seu casamento para pedir-lhe a mão e fazer toda uma
patusca encenação de namora, noivado e casamento, até o desenlace
trágico e a destruição do precário triângulo amoroso. As figuras
femininas destacam-se na composição de Uanhenga Xitu: é a sogra
obediente ao machismo prevalente, aos valores impostos pelos brancos à
revelia dos valores africanos; é a esposa traída, sofrida e compreensiva
no seu desemparo e sua suspeição do marido Malazartes e Don Juan
estroina e simpaticíssimo \emph{made in Africa}.

Por último, Manuel Pedro Pacavira com \emph{Nzinga Mbandi} (também
Edições 70), parece-nos dispensável, pois adere com demasiado ardor à
linha -- piorada -- de \emph{Raízes}, o fantasioso \emph{best-seller}
norte-americano levado à televisão. Revela verdades tão monstruosas
quanto inegáveis a respeito da escravidão dos negros pelos europeus e
árabes. Mas sua \emph{Nzinga Mbandi} não é nenhuma Xica da Silva,
nenhuma versão feminina do Zumbi dos Palmares; é uma composição oca,
falsa, com alguns trechos convincentes porque meramente documentais do
desaparecimento de pessoas, tribos inteiras acorrentadas rumo aos navios
negreiros eu os despejavam no litoral brasileiro. Contudo, não há
impacto que o autor consiga transmitir à comoção do leitor, por
raquitismo de talento? Por desleixo? Nas cenas de aprisionamento, de
queimada de lavouras, de choque das intenções espirituais dos jesuítas
com a cupidez dos colonos e da Coria de Portugal a se apoderar de
terras, gentes e riquezas dos ``gentios'', Manuel Pedro Pacavira escreve
com vigor, mas infelizmente com arremedos de erudição histórica,
enfileirando descrições tediosas e enumerações inúteis. Ele será, com
otimismo, um autor do qual se possa esperar -- passe o chavão -- um
aperfeiçoamento do seu frágil talento.

Desta primeira leva de angolanos, permanecem o desenho e o colorido. O
colorido ingênuo, adorável de Uanhenga Xitu. O desenho trágico, goyesco,
de José Luandino Vieira, em qualquer cena colhida a esmo entre suas
poucas e magnificas páginas:

``O corpo do tractorista caíra em cima dos presos já adormecidos àquela
hora da noite. Era princípio da madrugada, com o silêncio vencendo todos
os ruídos, e o ranger da grande porta acordou nos presos, que ficaram de
olhos abertos, querendo adivinhar no escuro quem tinha sido trazido para
ali. Ouviram o cipaio dar voltas à grande fechadura, correr a tranca,
afastar-se em seguida, conversando em voz baixa com alguém.

Uma lua grande brilhava no céu sem nuvens, cheio de estrelas, e a luz
branca entrava a jorros pela janela banhando os corpos estendidos no
chão. Eram muitos, sem cama, dormindo aos montes no cimento, ou
embrulhados em trapos e velhos cobertores. Criados trazidos nos patrões,
homens desempregados apanhados sem cartão assinado, bêbados agarrados na
porta de tabernas sempre abertas, pequenos larápios, desordeiros, unidos
no mesmo destino, de porrada e trabalho na estrada. A entrada de mais um
preso era sempre acolhida com indiferença, alguém resmungava qualquer
coisa, mas ninguém se preocupava. Só de manhã se ia ver quem era o
infeliz, antes da hora de sair nos trabalhos forçados.

Mas logo que os passos dos cipaios deixaram de se ouvir e o silêncio
caiu novamente, os presos, acordados pelo cair do corpo em cima deles,
se levantaram admirados daquele patrício que nem se mexia!

Um homem virou Domingos Xavier de costas e, na pálida luz do luar, a
cara inchada do tractorista apareceu entre os farrapos da camisa suja de
sangue. Um vento de frio correu no meio dos homens. Era terrível aquela
cara, quase sem feições, sangrenta, mas um sorriso teimoso nos lábios. O
mais miúdo se abaixou e, tirando um lenço, começou a limpar com todo o
cuidado o sangue na cara de Domingos Xavier. O homem alto e forte
deitou-lhe, depois, cm muito jeito no chão, enquanto um velho, ainda
cheirando a vinho, começava a choramingar. Alguém que tinha um cobertor
abriu-lhe em cima do tractorista e cobriu com ele o corpo magro e
torturado. O miúdo baixo e forte continuou a limpar a cara sangrenta com
cuspo que punha nas pontas do lenço. Uma expressão de muita
tranquilidade se sentou na cara endurecida e inchada do tractorista. A
respiração era muito fraca, mal mexia a camisa, e o miúdo, cada vez que
ele respirava, limpava o pequeno fio de sangue que estava a sair no
canto da boca. Domingos Xavier, olhos fechados, nem se mexia, não gemia
sequer. Só sentia a vida esvaziar naquele corpo martirizado''

Até o coro que se ergue na prisão, paralelo ao sangue que se esvai:

\begin{quote}
``Uexile kamba diami

Una uolobita

Uafu

Mukonda kajimbuidiê'':

Era meu amigo

Aquele que vai a passar

Morreu

Porque não quis falar''
\end{quote}

\chapter{Angola Escreve. Alguns grandes autores do romance ao
conto}\label{angola-escreve.-alguns-grandes-autores-do-romance-ao-conto}

Jornal da Tarde, 1981/08/04. Aguardando revisão.

\hfill\break

Se já no plano puramente temático o romance constitui um gênero no qual
sobressai, alta, a figura de José Luandino Vieira, o conto angolano
também nos reserva surpresas compensadoras. O próprio José Luandino
Vieira atribui essa designação de ``estórias'' a várias coletâneas suas:
\emph{Luuanda}, \emph{No Antigamente, na Vida}, \emph{A Cidade e a
Infância} e \emph{Macanduba}. Nesses contos longos, porém, a temática
político-social cede à evocação de uma vida com episódios bucólicos,
nostálgica e poeticamente evocados pelo autor. Há cenas dignas de
comédia, com os mais velhos assumindo uma postura de incredulidade e
depois de indignação diante dos acontecimentos que os livros transmitem
nas escolas aos filhos:

\begin{quote}
``- Não adianta! Aldabrõe! Cambada de aldabrões! A terra anda? Queimem
os livros, queimem os livros!~

Berrava. Os monandengues à volta assomavam de mão na boca. Se virava,
então, professoral:~

\begin{itemize}
\tightlist
\item
  Meus filhos, não aceitem! Não acreditem! São uns aldabrões. Se a terra
  anda, eu dou saltos, a minha casa vem me dar encontro..''
\end{itemize}
\end{quote}

Com um raro talento narrativo, José Luandino Vieira faz desfilar uma
Angola lânguida como a Bahia, ao lado de cenas de entrechoque racial
latente e dificilmente oculto. O escritor parece ter uma afinidade à
primeira vista insuspeitada com Guimarães Rosa, o Guimarães Rosa que
unia neologismos a termos locais e arcaicos, principalmente em
\emph{Grande Sertão: Veredas}. A semelhança é evidente após a leitura
deste texto, trecho de ``Lá em Tetembuatubia'', incluído em \emph{No
Antigamente, na Vida}'':

\begin{quote}
``Apontou, branca pomba voando para o todo azulídeo ar, alarando a mão
dele. E vimos as sete partidas do céu, girarem olhos nossos, misturar
belezas.

Que era tudo o macio fogo sem chama em pós azulados do vento, cada vez
polvilho de luz filtrada depois da meia tarde e as todas passaradas
avoejantes, regresso no lar -- os brancos jindeles silenciosos nas rotas
corriqueiras do sul; os compactos guanguastros, nuvens de mentira,
reviengas súbitas exactas, de virar cores de bando; e uma que é outra --
a viuvinha-catembo com seu sozinho rabo enorme, flecha de escuridão, os
todos os mais que até pico-rei, mania de única, se passaram duas a duas.
Tudo em luminoso fundo de nuvens velhas, asas em campo de areai ardente
-- que, de lá, do ocaso dele, só berridava altas sombras das esferas da
noite mais camuela de belezas''
\end{quote}

Até daí a duas páginas, a interrogação de cunho filosófico que recorda
as meditações do solilóquio de Riobaldo:

\begin{quote}
'' -- Ah -- a gente que somos é de maior desconfusão, aula dele, Turito,
``Que cada qual é dono de muitas almas -- em simple bufo de nova vida
usada, usada fica nos capinhos dos caminhos de antigamente\ldots{} --
Nave hexacolor, só arco-íris de Deus é superior dela\ldots{} Só ouvia o
frufruir do aroxigêneo, nos motores-alhetas, impacientes. Que era de
multiplicados, nada de construção apressada, tudo motores ronronantes,
sotavento e barlavento'',
\end{quote}

Anteriormente:

\begin{quote}
'' -- Eles são nossos altarêgos. Zeca amigo! E se sorria, límpido.
Porque a verdade do Turito era essa, futura: pessoa que vai vir um dia
ocupar lugar de nosso tosco corpo materialesco, falam. Nós? Porcos
lenços, só de guardar lugar no cego cinema do mundo. Ele que quem via
via verdadeiras criaturas luminosas, a gente só as iluminadas figuras de
mentira''
\end{quote}

A adesão do autor a Angola levou-o a mudar de nome: José Vieira Mateus
da Graça transformou-se m José Luandino Vieira. Encarregado pelos
ocupantes portugueses de servir como primeiro-cabo ``para tomar conta de
livros (na Biblioteca do Quartel-General de Luanda). O Exército colonial
não tinha realmente a vocação da leitura e eu passava os meus dias do
seguinte modo: de manhã, com qualquer desculpa, ia para a praia (depois
de içar a bandeira portuguesa, que era o trabalho do cabo da
Biblioteca); sétimo ano do Liceu que nessa altura, estudava''por fora''
com o Antônio Cardoso e o Hélder Neto. Estudávamos todas as disciplinas
do 7º ano de Letras: Todas: Latim, Grego e Alemão e Inglês e Francês e
não sei mais o quê, parece que OPAN -- Organização Política e
Administrativa da Nação''\ldots{}

A política cultural ou apenas educacional do regime colonial português
com relação a Angola era ambígua. Se por um lado havia a já em seu
próprio título arcaica ``Casa dos Estudantes do Império'', onde se
procurava peneirar os ``valores novos'' de uma assim chamada
``aristocracia africana'' (?), por outro a Pide, polícia secreta
salazarista, impunha uma censura férrea e lançava mão de delatores,
informantes que denunciavam a suposta tendenciosidade de poemas,
artigos, ensaios, contos, conferências, romances, leituras importadas de
Neruda, de editoriais hispano-americanos, de textos de marxismo. Só com
a sonolenta prosa de Manoel Ferreira, relatando um pouco da biografia de
José Luandino Vieira, no prefácio alambicado e longo que escreveu para
\emph{A Cidade e a Infância} é que se vem a saber, inopinadamente, que
José Luandino Vieira, com seu tomo \emph{Luuanda} recebera, em 1965, o
Grande Prêmio de Novelística, conferido pelo júri da Sociedade
Portuguesa de Escritores. Para que se possa aquilatar não só o acerto da
comissão julgadora como igualmente a sua coragem, basta acrescentar que
dos cinco componentes da banca examinadora, quatro foram presos e o
quinto, por não atribuir o Prêmio a \emph{Luuanda}, foi absolvido. O
regime português colonialista declarou imediatamente extinta tal
Sociedade que ousara outorgar tal honraria a um escritor angolano
``internado nessa época no Campo de Concentração do Tarrafal de Cabo
Verde, acusado de terrorista, cumprindo a pena de 14 anos''. Os
escritores Alexandre Pinheiro Torres, Augusto Abelaira; Fernanda Botelho
e Manuel da Fonseca foram conduzidos à delegacia policial, o quinto
votante, o excelente ensaísta João Gaspar Simões, que discordara do voto
da maioria, foi posto em liberdade.

José Luandino Vieira traz para os leitores brasileiros uma dificuldade
de difícil superação: seus textos são predominantemente bilíngues,
misturando o português e o quimbundo, um dos 7 idiomas africanos falados
correntemente em Angola. Sem um glossário, o autor fica
semi-compreendido apenas pelo leitor brasileiro. Faltam notas
explicativas sobre centenas de palavras e frases sem tradução como
\emph{jinguba}, \emph{maboque}, \emph{gajajas}, \emph{quinqueras},
\emph{mufetes}, até frases inteiras, para nós enigmas indecifráveis
como:

\begin{quote}
'' -- ``eie, ngana kimitudi kia nganga ia'ngu, eme muene ngi-di-kolo: Um
ngongo ioso ki muene munzangala ngasoko nê mu kuiiba o muxima; hanji nê
mukuetu Kandidi dia Sabalu dia Nvula letu, nê hanji Xana dia Inana ia
jingondo, eme ki ngasoko nâ..''???
\end{quote}

Se tradução toda a dramaticidade, graça ou lirismo da narrativa se perde
para quem não dominar os dois idiomas. Além de se praticar uma sintaxe
da língua portuguesa diferente da que usamos no Brasil e em português de
Portugal, o que às vezes dificulta para nós a leitura, principalmente
pela regência dos verbos, sem as preposições que pedem ou permutadas por
outras preposições (exemplos: empregados entregues NOS patrões, em vez
de empregados entregues PELOS patrões como no trecho citado de
\emph{Domingos Xavier}.

Mas José Luandino Vieira no texto mais antigo, \emph{A Cidade e a
Infância}, se mostrava ainda muito tateante e amadorístico no seu
maniqueísmo político, racial, só nos demais livros é que ele se vai
aprimorando e ousando experiências de estilo que deixam entrever um
autor de possibilidades luminosas ainda não totalmente tocadas com vigor
literário plural e extraordinário.

O que a seleta de contistas angolanos revela mais ainda, porém, é uma
figura inclassificável, em certos pontos reminiscente da irreverência de
uma \emph{Serafim Ponte Grande} de Oswald de Andrade e de outros autores
que, na África, ironizaram ferinamente o conformismo dos colonizados em
assimilar toda uma estrutura de valores postiça, europeia, já
pré-digerida e nunca questionada, mas na realidade inadaptável novos
meios -- africanos, asiáticos, ou latino-americanos, como nos versos
satíricos ácidos do poeta malgache Flavien Rainavo. Manuel Rui é, como
contista, a mais fulgurante presença angolana que essa coleção nos
apresenta. Dele é a filosofia fecundamente lúcida de um Alioune Diop que
já na década de 20 definia o projeto da \emph{Présence Africaine} com
uma amplitude e uma eficácia certeira e ainda plenamente atual.

Ao apelar a todos os intelectuais africanos ara que usem os recursos que
a Europa coloca à sua disposição, ele se bate pela livre expressão de
cada ser humano em sua singularidade individual e irrepetível. Só
através de sua unicidade cada pessoa contribuirá para a formação de uma
opinião pública e para humanizar uma civilização que apenas se mostrou
tecnologicamente superior às demais, ao emergir da Europa branca e que
como um Ptolomeu teimoso insistia que a Europa continuava a ser o centro
único do mundo, apesar de todas as demonstrações em contrário. Ele
retoma os versos amargos de Aimé Césaire, que assumia integral e
altivamente a versão racista de que ``o negro não contribuiu em nada
para a civilização'' e não procura desmenti-la: rebate-a pelo avesso.
Não mostra as civilizações requintadas do Gabão, do Benin, não insiste
na estética deslumbrante das danças, das roupagens, da escultura, das
máscaras, das pinturas, da música tribal, das rapsódias orais, do
espírito de solidariedade comunitária, de toda a complexa estrutura
espiritual do mundo negro africano. Alioune Diop prefere falar do
presente voltado para o futuro, afirmando:

\begin{quote}
``O negro que brilha por sua ausência na elaboração da cidade moderna
poderá, gradualmente, marcar a sua presença contribuindo para a
recriação de um humanismo talhado realmente segundo as dimensões do ser
humano.

Pois é certo que não poderíamos legitimamente esperar que surgisse um
universalismo autêntico se, na sua formação, só interviessem
subjetividades europeias. O mundo de amanhã será construído por todos os
homens (de todas as raças)

Nós, da África, devemos abordar as questões que aparecem no plano
mundial e meditar sobre elas como os demais, a fim de nos encontrarmos,
amanhã, entre os criadores de uma ordem nova''
\end{quote}

Manuel Rui tem o atrevimento de não aceitar uma ``ordem nova'' que sai,
já pronta e perfeita, da Revolução e da Independência meramente política
de Angola. É surpreendente que ele se refira com sátira e clareza às
divisões tribais que perduram mesmo depois que os regimes coloniais que
as aguçavam desapareceram. Manuel Rui tem -- terá ainda hoje? -- a
temeridade de introduzir diálogos que ironizam tudo: a inépcia do
governo, o vazio dos slogans, a panacéia da aliança operário-camponesa
como sara-tudo dos males angolanos. É no conto de abertura da sua
coletânea \emph{Sim Camarada!} (Edições 70) que mais se evidencia essa
sua vigilância inteligente e zombeteira: o tom é característico desde as
primeiras linhas do conto intitulado ``O Conselho'':

\begin{quote}
``Lá fora estava tudo na mesma. Pior ainda que antes porque agora o povo
olhava sempre o Palácio com grande confusão. É que naquele primeiro dia
do tal governo angolano, maior que qualquer outro no mundo porquanto
usava nada mais nada menos que três primeiros-ministros, um ministro,
desses novos, apareceu na varanda a bocar que o Palácio agora era do
povo. E falava mais: que o Palácio que fôra dos colonos passava para os
legítimos donos. E o povo aplaudiu a afirmação que o ministro fez com os
olhos esbugalhados de independência, braços agitados em maneira de
alguns pensarem que a página da história estava virada. Era só pôr cuspo
no dedo, agarrar aí a página e pronto! -- Meu! Este governo não liga.
Veja só: os nossos com umas fatiotas''poder popular'', os outros, de
Mobutu uns e o resto de fato e gravata! Quer dizer: se nisto já começa a
diferença este governo não vai chegar ao fim. Corto-os rentes! --
Sentenciava de dia para si um camarada encostado em uma árvore ao que um
recém-chegado de Lisboa não perdeu a oportunidade de acrescentar:

\begin{itemize}
\item
  O que é preciso é que as contradições se agudizem e a aliança
  operário-camponesa tome de assalto este Palácio o mais depressa
  possível para o salto qualitativo.
\item
  Chiça! Com salto e tudo? O camarada almoçou dicionário e se não é
  doutor herdou biblioteca. Vamos com calma!
\end{itemize}

E o outro escapava-se no meio da multidão''
\end{quote}

Mais adiante, sempre cortante, Manuel Rui alude ao contraste, notável
também no Brasil entre a carestia da vida devida à inflação que devora
os salários dos trabalhadores e a fartura indecente das mordomias
governamentais, como o ministro que ``mastigava-se pela manhã em dez
pães, cinco quilos de presunto e seus queijos vindos do Huambo,
tribalismo à parte.''

Para descarregar o retrato que Manuel Rui faz dos pedantes que exalam
fumaças de conhecimentos de francês, colocando em português galicismos
pernósticos e inúteis:

\begin{quote}
``Tempo de grandes neologismos a enriquecer o léxico nacional muito para
além do `luso-tropicalismo'. Suas celências então que ministros
ressortisantes do Zaire, Suiça ou Alemanha , entregavam-se a grandes
aumentações, propiciadas pelos afluchos das proceduras legislativas
autenticamente importadas, sentindo-se a cada passo catastrofados sempre
que viesse à baila o poder popular''
\end{quote}

O desenho de uma ``excelência asnática'' que está a votar uma lei merece
ser transcrito quase na íntegra, pela sua graça absurda saída de um
\emph{Ubu-rei} africano:

\begin{quote}
``Após vários oradores terem pedido a palavra para apoiarem as normas
propostas, incluindo os próceres da linha de Tavares, sua celência pediu
também a sua:

\begin{itemize}
\tightlist
\item
  Em que concerne o artigo primeiro discordo. Não obstante, o artigo
  segundo também discordo; não é? E o terceiro voto contra; não é? E
  relevo para o quarto que também discordo. E rejeito o quinto. Não é? E
  o sexto idem.
\end{itemize}

E nesta sapientíssima oração o supra-sumo Tavares foi reprovando tudo
até que chegou ao último artigo já com o Alto-Comissário a dormir um
sono profundo com um sonho de levar Cabinda num barco para uma aldeia do
Minho, no fim da transição.

\begin{itemize}
\tightlist
\item
  Concordo com o artigo cinquenta e seis. Aliás, nem percebo por que é
  cinquenta e seis quando podia ser o artigo final.
\end{itemize}

O artigo rezava assim: este decreto entra imediatamente em vigor.

E até o Alto-Comissário, ainda com o desencanto de há pouco ter sido
sócio de americanos nos sonhos dos poços de petróleo, ficou a pensar
franzindo o sobrolho.

\begin{itemize}
\tightlist
\item
  Desculpe. Mas se desaprovou todo o decreto como é que concorda com o
  último artigo que é apenas uma formalidade? Uma praxe. Uma regra --
  comentou um ministro da parte portuguesa. O leitor que não sabe fica a
  saber que este governo era internacional. Tinha partes: a angolana, a
  portuguesa, a americana, a zairota, a alemã e a etecétera. Mas como a
  portuguesa andava com a as calças na mão por vias da indigestão
  spinolar, quer dizer que a imperial mandava na portuguesa e eis que
  podíamos falar de duas partes: a angolana e a imperial.
\end{itemize}

Tavares ripostou sem que arregalasse os olhos:

\begin{itemize}
\item
  Mas então entra imediatamente em vigor?
\item
  Claro -- disse o ministro português.
\item
  Então concordo! Porque entra imediatamente em vigor. Não é?
\item
  Então qual a sua dúvida? -- Interferiu um ministro angolano,
  baralhado.
\item
  \emph{Mon frére}, é que eu pensava que entrava imediatamente em vigor.
\item
  Mas está de acordo? -- voltou à carga o ministro português enquanto os
  angolanos riam e trocavam bilhetinhos.
\item
  Estou e não estou. Ser ou não ser eis a questão da procedura. Por isso
  é que não deve entrar imediatamente em vigor. Não é?
\item
  Embora eu não devesse intervir nos debates, acho que o Secretário de
  Estado do Comércio deu uma achega muito positiva e importante -- falou
  o Alto-Comissário, levando a mão à testa pelo esforço daquela
  conclusão.
\item
  Muito obrigado -- disse o Tavares de sua desgraça.
\item
  Se é assim -- explicou o presidente em exercício -- o irmão Tavares
  está de acordo.
\item
  Evidentemente -- reafirmou o tribuno com um sorriso de vitória.''
\end{itemize}
\end{quote}

Essa enriquecedora importação cultural angolana que chega ao Brasil
traz, qualitativamente, os nomes de José Luandino Vieira, Uanhenga Xitu
e Manuel Rui. Quantitativamente, há um inútil desfile de outros nomes:
Jofre Rocha, Antônio Abreu, Manuel Ferreira, Manuel Lopes, Fernando
Monteiro, Arnaldo Santos, Jorge Macedo, Antônio Jacinto. Mas adentrar-se
e seus livros de prosa ou poesia é pura perda de tempo. Como também soa
artificial e sem convicção o livrelho do vice-ministro atual da Educação
em Angola, Artur Carlos Maurício Pestana dos Santos, que usa como
pseudônimo a palavra \emph{Pepetela}, que significa pestana em um dos
idiomas africanos, provavelmente o quimbundo. \emph{As Aventuras de
Ngunga} de sua lavra, digamos, é irmão xifópago das rosinhas minhas
canoas e dos pés de laranjas limas ácidas que José Mauro Vasconcelos
teima em querer nos infligir como literatura deste lado do Atlântico.
Esse menino pioneiro nada tem a ver com a vibração de outro menino
pioneiro, crianças que ajudavam ativamente e precocemente na libertação
de Angola, escrito por Manuel Rui no mesmo livro e denominado ``Cinco
Dias Depois da Independência''.

Em visita a São Paulo, Pepetela-Pestana declarou ao \emph{Estado de São
Paulo} ou fez tal afirmação durante uma palestra na USP (Universidade de
São Paulo) que

\begin{quote}
``O escritor está muito empenhado na criação de seu país e sobra pouco
tempo para escrever, a não ser que ele se supere''.
\end{quote}

Frase que coloca a litertura como uma argamassa nada urgente, na
construção penosa de um país com problemas típicos dos países
subdesenvolvidos como o Brasil e Angola.

Não: ao contrário. Como já demonstra fartamente a literatura recente de
Angola, mesmo antes da Independência, em 1975, existem autores que
paralelamente à sua participação política ativa nunca relegaram a
literatura a um segundo plano como se fosse uma etapa sucessiva, quase
utópica, de um Plano Quinquenal que se reduzisse a construir barragens
dominar a tecnologia da extração do petróleo e minérios, a alfabetizar,
construir estradas, hospitais, casas.

Enfaticamente: não. Angola se faz TAMBÉM com a inteligência, a
sensibilidade, o arrojo de seus melhores escritores, tão decisivos na
construção ética e estética de seu país quanto os cientistas, os
engenheiros, os operários e até alguns ministros que não sejam do tipo
do Tavares maravilhosamente ironizado por Manuel Rui. Se a população
negra, mestiça ou branca de Angola quiser ser protagonista e não títere
de seu tempo, ela por certo compreenderá que a literatura é o exercício
supremo da Liberdade e da Democracia que nenhum regime de partido único,
de imprensa amordaçada pelo Estado monolítico, pode jamais sufocar. É um
testemunho dessa profundidade que os grandes autores angolanos nos
trazem, fazendo-nos crer, com novo alento, no futuro dessa literatura
tão forte e importante que nos vem de Angola, uma expressão original,
bela, autônoma, de um povo hoje liberto dos grilhões do colonialismo e
prestes a dar sua importante contribuição para a formação daquele
autêntico humanismo a que referia lúcida e lapidarmente Alioune Diop
linhas atrás.

Sem a contribuição original, livre de tutelas estrangeiras adiposas, a
África Negra insuflará, na civilização maquinal que nos massacra, aquela
alma que dela se alijou à força dentro do maniqueísmo asfixiante do
capitalismo selvagem e do marxismo arcaizante. Afinal, a literatura
permanece como testemunho da imaginação e como documento de uma época:
são os tiranos que perecem e mal deixam rastros de sua passagem no
efêmero quebradiço da História.

\chapter{A África e a liberdade. Entrevista com Luandino
Vieira}\label{a-uxe1frica-e-a-liberdade.-entrevista-com-luandino-vieira}

Jornal da Tarde, 1987/01/03. Aguardando revisão.

\hfill\break

À primeira vista, bronzeado, com seu ar franco, ele parece um fazendeiro
do Centro-Oeste brasileiro. Depois, o sotaque angolano, uma harmoniosa
mistura de português lisboeta com a doçura da pronúncia vagarosa das
sílabas, à moda africana ou brasileira, revela a sua verdadeira
nacionalidade.

José Luandino Vieira, já muito conhecido no Brasil como possivelmente o
mais extraordinário talento literário de Angola é presidente da União
(Sindicato) de Escritores Angolanos e autor de obras-primas como
\emph{Luuanda}, \emph{A Cidade e a Infância}, \emph{A Vida Verdadeira de
Domingos Xavier}, \emph{Macandumba} etc.

Durante sua recente estada no Brasil, esse escritor realmente
excepcional no panorama da literatura contemporânea mundial concedeu
esta entrevista ao \emph{Jornal da Tarde}.

Luandino Vieira, você parece não fugir à regra de que a prisão forja
grandes escritores. Afinal, você passou doze anos no cárcere e
prisioneiros estiveram Graciliano Ramos, Solzhenitsyn, Dostoievski,
Tchekhov, Oscar Wilde e tantos mais. O que o encarceramento representou
para você, tão longo que foi?

\begin{quote}
``Eu penso que essa capacidade de o escritor escrever na prisão está
talvez intimamente ligada à condição do escritor que é sempre, em última
instância, uma condição de solitário. Escrever é sempre, em última
instância, um ato de muita solidão. Quando o escritor está escrevendo,
penso que ele está absolutamente só com ele próprio e está só
acompanhado de toda a Humanidade, que é o que ele tem dentro de si. A
sua experiência, a sua vivência, suor, leituras que cimentam a
realidade. Para certos tipos de personalidade, digamos de organizações
psíquicas, a prisão pode ser um momento propício a essa solidão
necessária à criação''.
\end{quote}

Você se refere aos mais introspectivos?

\begin{quote}
``Pois, os mais introspectivos, os mais contemplativos. Agora a dureza
da situação da cadeia pode sempre sobrepor-se a essa necessidade de
solidão e pode também aniquilar qualquer capacidade de criação''.
\end{quote}

Não foi a sua experiência\ldots{}

\begin{quote}
``Eu tive de certo modo a felicidade de estar continuamente, a partir do
3º, 4º ano de prisão, com muita gente. Não eram prisões solitárias: eram
casernas, compartilhadas por muitos companheiros. Isso nos dava a
sociabilidade fundamental para ir recarregando as baterias de
humanidade. E depois no recreio, no pátio, no momento de apanhar sol eu
teria, se quisesse, o meu momento de solidão, que eu aproveitava para
escrever''.
\end{quote}

Que papel você usava para escrever, que papel lhe davam?

\begin{quote}
``Nós vivíamos no campo de concentração, a 7 km de uma vila, e toda
semana saía um guarda com uma lista para fazer compras e invariavelmente
eu era comprador de 4 u 5 cadernos do tipo escolar''.
\end{quote}

Os cadernos eram permitidos?

\begin{quote}
``Sim, porque como nós tínhamos uma escola de instrução primária dentro
de nossa caserna, com os companheiros analfabetos, todo esse material
era tido como material para ensinar as primeiras letras''.
\end{quote}

E você é que lecionava?

\begin{quote}
``Eu era o professor de adultos analfabetos e assim sempre e pensou que
esses cadernos me serviam não para escrever, mas corrigir as redações
dos alunos''.
\end{quote}

E, como você ressaltou, você ficou fascinado com o vocabulário dessas
pessoas?

\begin{quote}
``Claro, encontrei aí camponeses, pequenos funcionários, e tinham
vocabulários derivados de suas línguas maternas, corrompidas pelo
português originalíssimo que, de fato, me fascinava. Estavam todos esses
falares reunidos ali e eu só tinha que ser sensível ao valor estético,
literário, dessas linguagens''.
\end{quote}

As cadeias mantinham um \emph{apartheid} no seu interior?

\begin{quote}
``Não, as forças colonialistas nos colocavam todos juntos. Aliás,
diziam: `Bom, vocês não estão todos juntos na luta de libertação? Então
estão também todos juntos nas consequências das lutas de libertação!' O
que nós achávamos correto, pois entre nós nunca houve esse tipo de
divisões, não é?''
\end{quote}

Mas que tipo de contato você mantinha com a Literatura, por exemplo?

\begin{quote}
``Bom, isso é um caso muito difícil, porque enquanto estivemos presos em
Luanda, a presença das famílias, a pressão das famílias faziam com que
sempre tivéssemos visitas. E nos traziam livros, jornais, embora
censurados, mas o princípio da leitura mantinha-se. Agora, quando
passamos para Cabo Verde, perdidos no Atlântico, esse princípio acabou.
Nesse arquipélago, estávamos na Ilha de Santiago e, ali, na prisão de
Tarrafal. Então, ali não era permitida leitura de nenhuma espécie,
durante muitos anos. Depois, pouco a pouco, foi abrindo, com o
\emph{Jornal Desportivo}, um que outro livro. Mas o objetivo daquele
campo era o de destruir psicologicamente as pessoas. Materialmente não
podíamos trabalhar, nem trabalhos forçados nos eram permitidos''.
\end{quote}

Por quê?

\begin{quote}
``Porque o arquipélago era tão pobre, tão miserável que se utilizassem
presos nas obras públicas a população ia reclamar, pois íamos tirar-lhes
o emprego''.
\end{quote}

No entanto, não foram 12 anos estéreis para você.

\begin{quote}
``Nem para mim nem para muitos de meus companheiros. A grande maioria
aproveitou. Quando mais não fosse, os analfabetos aprenderam a ler, a
escrever; os que tinham o 1º ciclo do liceu passaram ao segundo ciclo,
outros estudaram até contabilidade e outras matérias que lhes foram
úteis quando conquistaram a liberdade''.
\end{quote}

Ali, a prisão era um microuniverso.

\begin{quote}
``É isso, era um microuniverso, mas um microuniverso que prefigurava o
país que íamos ter''.
\end{quote}

E de onde se originou a sua vontade de escrever livros? Em prol de uma
ação política? A favor da justiça social?

\begin{quote}
``Minha infância explica muita coisa. Eu tive uma infância de menino
pobre nas favelas de Luanda chamadas \emph{mussegues}, com todos os
meninos de minha idade, branco, preto, mestiço, português, angolano.
Isso deu o caldo cultural que me fez uma criança irrequieta, com um
determinado tônus cultural diferente do dos filhos da burguesia
colonial. Aí havia um companheiro, Antônio Jacinto, que tinha uma
biblioteca muito boa e ele, quando nós tínhamos dez, onze anos, nos
meteu um vírus que era o de fazermos jornais manuscritos. Então, todos
começamos como jornalistas-mirins. Quando eu me convenci mesmo que tinha
jeito para escrever e tinha escrito, como dizem os brasileiros, uma
série de bobagens, aí ele começou a me dar livros de Steinbeck, Machado
de Assis, Jorge Amado, Graciliano Ramos, Zé Lins, Rachel de
Queiróz\ldots{} Porque nesse tempo os mais velhos recebiam muitos livros
brasileiros..''
\end{quote}

Que não eram censurados?

\begin{quote}
``Iam assim de forma semiclandestina, eram marinheiros que os
levavam\ldots{} Então, a partir daí ele me dizia: em vez de escrever
bobagens é melhor você escrever sobre o que está à sua volta, sua
cidade, sua infância. Então a literatura foi servindo para meu
conhecimento da realidade social e a realidade social valendo à minha
literatura. Portanto, a consciência política foi acompanhando a
consciência literária e vice-versa. Até se dar uma fusão das duas que é
hoje difícil para eu separar''.
\end{quote}

Mas você nunca permitiu que a parte política interferisse na parte
estética?

\begin{quote}
``Não, ela interfere só na medida em que o modo como eu vejo o mundo e
os problemas do mundo enfoca o modo como eu vejo literariamente os
problemas, quer dizer: é interior. Agora, no momento de escrever, quando
eu escrevo, eu não tenho nenhuma preocupação política de maneira
expressa''.
\end{quote}

Você não faz panfletarismo ideológico?

\begin{quote}
``Não, não! A mim o que me interessa é o homem total, a complexidade do
homem. E, no caso concreto, o homem angolano, e este homem angolano não
pode ser compreendido sem a dimensão política, porque a política é um
dos fatores fundamentais em nosso país, que é a destruição de uma
imposição colonial e a criação de um país novo. Uma pátria nova, feita
dos bocados das nações antigas que ficaram pelo caminho em cinco séculos
de História''.
\end{quote}

Além do retalhamento, pelas nações europeias, de território
africano\ldots{} Agora, se você permite a pergunta: você é marxista?

\begin{quote}
``A minha formação é marxista. Quer dizer: li tudo, li muita coisa e sou
ainda hoje um leitor assíduo da Bíblia. Agora, a minha visão é uma visão
do mundo que assenta fundamentalmente no pressuposto de que a vida
material, portanto transformação radical material, o aspecto material de
uma sociedade é que pode criar condições para uma nova vida espiritual.
Isto, de uma maneira muito esquemática, porque sabemos que a vida
espiritual reage sobre a base material e também atua sobre ela''.
\end{quote}

O próprio Marx reconhecia a existência do espírito.

\begin{quote}
``Exatamente! Não, nunca negar a essência do espírito! O espírito é o
que o homem tem de mais elevado e desenvolvido na sua materialidade''.
\end{quote}

Nós sabemos que Angola se defronta com as forças da Unita, de Jonas
Savimbi, e não temos nenhuma bola de cristal para prever o futuro de
Angola. Mas como é o presente de Angola e da sua literatura? Além de
você e Manuel Rui -- e evidentemente o presidente morto, Agostinho Neto
-- pouco se conhece aqui dos autores angolanos. Quais são os outros ou
os mais importantes, a seu ver?

\begin{quote}
``Nossa União de Escritores tem inscritos 63 escritores, muitos são
ainda muito jovens para podermos falar, outros não podem ter para o
público brasileiro um grande significado. Mas dezenas de escritores,
pelo menos, teriam muita ressonância junto ao público brasileiro, se
seus livros pudessem chegar ao público brasileiro. Umas das
características da nossa literatura é um profundo compromisso com a
realidade; é uma literatura que para além do valor ou desvalor estético
tem uma grande carga de informação''.
\end{quote}

Incluindo depoimentos de recuperação histórica?

\begin{quote}
``Sim, informação e depoimentos históricos de um passado recente e do
presente. O presente angolano é um presente muito difícil por causa da
guerra de agressão da África do Sul. Bom, fundamentalmente é a questão
da Namíbia, nós sabemos que é ela que determina todo esse posicionamento
da África do Sul em relação a Angola. Incluindo a invasão, a ocupação do
território e o armamento, o financiamento du grupo de Jonas Savimbi para
desestabilizar totalmente a nossa vida. Basta dizer que o nosso Estado,
que não é um Estado rico..''
\end{quote}

Potencialmente, sim.

\begin{quote}
``Ah, sim, mas falta desenvolver as potencialidades. E é isso que não
nos permitem. Angola é um Estado que saiu de uma situação colonial,
portanto não podia ter muita riqueza, gastara a quase totalidade dos
recursos na defesa e na segurança do país. O que resta para a saúde
pública, a educação, cultura, é muito pouco. Então, nessa situação, a
literatura se desenvolve mais devagar do que poderia. Porque veja: um
livro de poemas em Angola tem que ser publicado com uma tiragem de pelo
menos dez mil exemplares, senão não chega para fazer a primeira
distribuição. Nós só podemos publicar 2, 3, 4 livros por ano. Quando, na
realidade, há muitos mais livros e muitos leitores ávidos de leitura.
Esses recursos que vão para a guerra condicionam também o
desenvolvimento de nossa literatura.

Depois há outros fatores: os autores estão muito empenhados no trabalho
da vida social, não podem dedicar-se à literatura, têm pouco tempo de
ócio, que é necessário para a criação da literatura. Não há
possibilidade de edição de gazetas literárias, jornais literários. O
nosso próprio jornal diário, de circulação nacional, só tem uma página
para as artes e as letras, aos domingos. Só agora começa a aparecer uma
coluna cultural dia sim, dia não, no jornal''.
\end{quote}

Trata-se de um jornal único? Isso não elimina uma pluralidade de visões?

\begin{quote}
``Não, esse é o jornal nacional, que se publica em Luanda, mas há vários
jornais em várias províncias''.
\end{quote}

Com várias tendências?

\begin{quote}
``Várias tendências, não, várias correntes de opinião dentro da nossa
unidade, que têm sido fundamentais para enfrentar a situação de guerra.
Porque, se não houver uma unidade nacional muito grande e uma unidade de
princípios e uma unidade de pontos de vista, nós como país recente e que
desde 1961 não deixa de estar em guerra será muito difícil sobreviver. E
o primeiro direito e o primeiro dever que a República de Angola tem para
com o seu povo é este: o de sobreviver''.
\end{quote}

Mas os escritores angolanos teriam total liberdade de expressão nessas
circunstâncias?

\begin{quote}
``Os escritores angolanos têm total liberdade de expressão e representam
várias correntes de desenvolvimento de uma literatura. E até hoje não há
memória, desde 1975, de qualquer livro que tenha sido ou rejeitado, a
não ser por motivos literários, ou mesmo, se asperamente criticado,
retirado do acesso público. Isso não existe entre nós. E somente o
Conselho Editorial da União de Escritores Angolanos é o único órgão que
aprecia os originais antes de publicar, constituído exclusivamente por
escritores membros da União''.
\end{quote}

Não há, por assim dizer, um \emph{imprimatur}?

\begin{quote}
``Não há \emph{imprimatur} nenhum. A União, funcionando como editora,
tem as suas limitações nas verbas e na sua política editorial. Quer
dizer: se houver possibilidade de editar apenas um livro, a escolher
entre um livro de prosa, um de poesia e um ensaio, dá-se preferência ao
ensaio, porque consideramos que esse estudo vai contribuir para o
alargamento do conhecimento ou, se um romance, um livro de ficção for de
muito alta qualidade, joga em primeiro lugar qualidade literária e em
segundo a estratégia do desenvolvimento da nossa literatura''.
\end{quote}

No entanto, o Manuel Rui, não me lembro em qual livro, talvez
\emph{Regresso Adiado}, deixa claro que faz críticas à presença cubana,
à presença soviética em Angola.

\begin{quote}
``Sim, bem, eu não diria isso. O livro dele mais polêmico é \emph{Quem
me dera ser Onda}. Este é um livro que critica, faz uma crítica social a
uma série de tendências pequeno-burguesas de oportunismo, de covardia
social, aproveitamento de recursos sociais em favor do indivíduo e esse
livro já tem duas, três edições neste momento. E foi premiado. Outro
livro também controverso saiu há pouco tempo, do Pepetela, \emph{O Cão e
os Calouros}, é a reprodução linguística de comportamentos anti-sociais,
comportamentos pequeno-burgueses, de reminiscências de erros do passado
que estão na nossa sociedade e a emergência de erros presentes. E tudo
isso está em letra de fôrma, o livro vendeu quinze mil exemplares, está
esgotado, estamos preparando a reedição. Portanto, nesse aspecto, os
escritores angolanos não têm do que se queixar''.
\end{quote}

Eles leem a imprensa ocidental, por exemplo?

\begin{quote}
``O problema da imprensa que entra em Angola é que nossas divisas para
compra de coisas exteriores são muito poucas. Então nós temos carência
de bibliografia. Isso é verdade. Muitas revistas chegam em um, dois
exemplares, que um amigo manda a outro amigo, que este por sua vez faz
circular. Não é porque não possa circular massivamente, é porque não
pode ser comprada massivamente. Porque a prioridade é para o livro
técnico, escolar didático''.
\end{quote}

E com relação ao Brasil, vocês têm contato?

\begin{quote}
``O contato com o Brasil, que no tempo do colonialismo já foi muito
forte, está agora a ser reaberto em novos moldes. O contato econômico
com o Brasil é que aumenta dia a dia, a olhos vistos: Braspetro,
Petrobrás, Odebrecht etc. Isso tira proveito do capital cultural
brasileiro acumulado durante séculos, mas ao mesmo tempo leva atrás de
si o novo, a nova dimensão cultural do Brasil, que está chegando a todos
nós. A relação entre a literatura brasileira e a literatura angolana vai
ampliar-se muito rapidamente e penso eu que com muita vantagem para
ambas as partes. Pelo menos eu, como escritor, sinto que sempre aprendo
alguma coisa com a experiência literária de meus colegas brasileiros''.
\end{quote}

E no tocante ao idioma, na sua opinião, a língua portuguesa predominará
em Angola diante das outras línguas nacionais?

\begin{quote}
``Na minha opinião Angola continuará a usar a língua portuguesa como
língua oficial, porque eu não vejo outra língua que se possa desenvolver
com rapidez para se comunicar internacionalmente''.
\end{quote}

As demais línguas autóctones são orais?

\begin{quote}
``Já fixamos, com o alfabeto, seis línguas, com gramática, léxico e com
manuais de alfabetização escritos nessas línguas''.
\end{quote}

Começando com o quimbundo?

\begin{quote}
``O quimbundo, umbundo, quicongo, xoquê, bunda e pahá. Falo mal o
quimbundo, eu o escrevo melhor''.
\end{quote}

O português tem mais ou menos o mesmo papel que o inglês na Índia para
intercomunicação de povos de línguas diferentes entre si?

\begin{quote}
``Sim, desempenha esse papel. O português é a língua oficial, é a língua
da escolaridade e é a língua materna de uma grande quantidade de
angolanos, embora não tenhamos inquéritos (pesquisas) a esse respeito. O
propósito é que cada angolano preserve sua herança cultural; o português
viverá em plano de igualdade com as línguas nacionais''.
\end{quote}

Há ainda muitas rivalidades tribais? Há racismo?

\begin{quote}
``Racismo não existe, existe preconceito racial: a mentalidade das
pessoas, mesmo que desapareçam as condições objetivas (propícias ao
racismo), mudança de mentalidade se dá de forma mais devagar. O aspecto
de tribalismo foi muito reavivado durante o colonialismo, pois como o
colonialismo português se baseava no trabalho forçado ele
propositalmente misturou todas as populações e profissões: havia
plantadores de café tentando ser pescadores e pescadores a colher café.
Na realidade, o colonialismo destribalizou o país, desenraizou as
pessoas de suas próprias culturas, de suas terras. Em certo sentido foi
bom porque hoje o tribalismo só existe secundariamente: as pessoas são
em primeiro lugar angolanas, depois é que são de origem quimbundo etc. O
Exército joga um papel muito importante nisso porque ali se vão
verificando as diferenças e as semelhanças. Além do Exército, o futebol
também é um fator importante: equipes do Norte e do Sul se defrontam e
populações que se desconheciam mutuamente se dão conta de que `afinal,
somos todos iguais'\,''.
\end{quote}

E as culturas autóctones estão sendo, então, preservadas?

\begin{quote}
``Ah, sim: a música, a dança, as artes plásticas fundamentalmente, além
da culinária, certas formas de vestir, de decorar a casa, que se vão
mantendo enquanto a industrialização e a urbanização não chegarem, não
é? Eu mesmo sou um exemplo dessa pluralidade de culturas que existe em
Angola. Há a influências europeia sobre populações africanas e
vice-versa. Eu, filho de portugueses, educado em escola de feição
nitidamente portuguesa, tinha porém a rua, os meninos do bairro, e a
influência do quimbundo se fez sentir fortemente em mim. Angola, variam
os casos de aculturação, é estranhamente isto. A convivência fecunda de
etnias e culturas diversas, sem eurocentrismos nem predominâncias
culturais obsoletas''.
\end{quote}

\chapter{Angola, num momento de reflexões e mudanças - Entrevista com o
escritor angolano Agostinho
Mendes}\label{angola-num-momento-de-reflexuxf5es-e-mudanuxe7as---entrevista-com-o-escritor-angolano-agostinho-mendes}

Jornal da Tarde, Sem data. Aguardando revisão.

\hfill\break

Como na antiga -- e vastíssima -- colônia da minúscula Bélgica, o
ex-Congo belga, as odiosas forças coloniais só tinham ``permitido'' que
catorze ``nativos'' se formassem pela Universidade. Em Angola,
igualmente, embora o número da \emph{intelligentsia} fosse bem maior, a
libertação nacional chamou a si, após a independência, todas as forças
mais aptas a liderar a nação emergente. O escritor angolano de passagem
pelo Brasil, Uanhenga Xitu (ou Agostinho Mendes de Carvalho, se usarmos
o seu nome português), também foi ministro da Saúde e governador de uma
província (o que corresponde a um Estado brasileiro), o de Luana.
Fecundo autor de livros como \emph{Maka na Sanzala, Manana} e
\emph{Mestre Tamoda}, além de outro que preparou em pouco tempo, durante
suas férias na Alemanha Oriental, a ser publicado brevemente também no
Brasil, é um autor ``ingênuo'' para as mentalidades complexas europeias,
mas que fixa em seus relatos uma África ancestral em célere
desaparecimento. Vai buscar suas fontes nos rapsodos das regiões rurais,
os anciões que guardam a memória das tribos antes da chegada dos
brancos. Antes dos navios negreiros e seus padres a abençoar as naus
rumo ao Brasil com as palavras de Jesus em seus sermões à beira da praia
e impressas em seus missais: ``Amai-vos uns aos outros''.

Junto aos que carinhosamente chama de ``velhotes'', Uanhenga Xitu,
recolhe a recordação dos rituais animistas, das leis, dos feitiços, das
proibições, do vestuário, das bebidas, dos usos dos antepassados, sem,
no entanto, a imparcialidade de um Chinua Achebe, evocador nigeriano da
barbárie e das humanas imperfeições dos abusos tribais com em seu livro
marcante, \emph{O Mundo se Despedaça}. Não que Uanhenga Xitu não saiba
do lado sinistro da África Negra mesmo antes do advento dos brancos e
seu séquito de estupros culturais, sociais e humanos: ele tem da África,
ou melhor, de Angola, uma visão ao mesmo tempo idealizada e ingênua. Mas
sempre infinitamente mais autêntica do que a de um piegas farsante como
o português Pepetela, irmão siamês -- aí de nós! -- de José Mauro de
Vasconcelos e suas infantilidades presunçosas do lado de lá do oceano
Atlântico.

A experiência administrativa, confessa Uanhenga Xitu, foi-lhe nociva
para a criação literária.

\begin{quote}
``É muito difícil conciliar ao mesmo tempo a política e a literatura.
Como outros escritores, fui chamado à tarefa prioritária de reconstrução
do país pelo governo e pelo partido, e, embora não me tenha furtado a
cumpri-la, durante todo o meu período administrativo parei de escrever.
Escrever exige estar atualizado, ler muito para ganhar experiência com
os outros: não sobra tempo nem sossego para isso..''
\end{quote}

Além desse conflito de esferas de interesse como é o problema da língua
em Angola? Quantas línguas há lá, pelo menos as principais?

\begin{quote}
``Devem ser mais de vinte, as há quatro principais: temos o kimbundu,
que é da minha área, o umbundu, do Sul, o kikongo, da região Norte e
depois o kioo, que já está perto do Zaire. A minha preocupação e captar
as tradições do passado da boca dos `velhotes', porque cada um deles que
morre é uma biblioteca que se perde. Escrevo voltado para essas massas
rurais''
\end{quote}

Enquanto Angola sofre um processo crescente de urbanização?

\begin{quote}
``Exatamente, que cria o problema do êxodo rural rumo às cidades:
favelas, despreparo de muitos para o emprego, a adoção de metas do tipo
ter uma boa casa, um bom emprego, a delinquência e as''casas de
vadias'', sim, a prostituição. O governo quer fixar as populações às
duas regiões de origem, no campo, no interior, mas encontra obstáculo na
crise econômica, no bombardeamento de Angola pela aviação sul-africana.
Muitos desses projetos de interiorização estão parados, à espera de
melhor oportunidade. Nosso plano era criar campos de desportos,
hospitais, escolas, num país onde 90\% da população é analfabeta,
note-se, de criar enfim um mínimo de dignidade mas isso é sustado, como
o sr. pergunta, também pelas incursões de Savimbi e seus guerrilheiros
da \emph{Unita} com apoio da África do Sul e dos americanos''
\end{quote}

A obra literária de Agostinho Neto, de Luandino Vieira, de Manuel Rui,
principalmente, se destacam aqui no Brasil na literatura angolana: que
outros nomes deveríamos conhecer?

\begin{quote}
``Sabe? A literatura angolana, o jornalismo angolano, a atividade
intelectual existem desde 1800, com correntes surgidas em 1820, outras
em 1850, com obras dramáticas etc. escritas nos idiomas originais. Não
que seja uma literatura estruturada: não, são esporádicos gritos de que
nós também queremos nosso lugar ao sol. Foi o fascismo português
colonial que não deixou medrar essas vozes, muitas delas, porém, fizeram
medo aos donos do poder em Portugal. O nome do Brasil, é claro, sempre
esteve nos lábios angolanos: afinal, temos uma herança colonial comum:
foi de Angola que partiram correntes de escravos para cá; isso nem que
se queira, não se pode esquecer. Há vocábulos em dicionários portugueses
que o brasileiro diz que é brasileiro: não é, é angolano. Assim como nos
reconhecemos no canto, no ritmo, na dança, nos discos, nos ritos
religiosos de certas seitas, na capoeira. Ora, parece que o governo
brasileiro sempre andou esquecido de nós, não marcou a presença
brasileira em Angola: por que será?!''
\end{quote}

Além da tarefa de alfabetizar o povo, com as mulheres que se apresentam
como professoras voluntárias para isso, o governo incentiva a criação de
editoras, o aparecimento de talentos novos na literatura?

\begin{quote}
``Temos um instituto de Línguas que esta mapeando o país e estamos
corrigindo a escrita fonética dos idiomas nativos, pois os símbolos de
que dispúnhamos não correspondiam à pronúncia exata, técnicos britânicos
e brasileiros estão a nos ajudar nisto. O Sindicato ou melhor a União de
Escritores Angolanos incentiva jovens a escrever para que possamos
escolher os melhores, chegando mesmo a editá-los.''
\end{quote}

Mas editá-los sem censura?

\begin{quote}
``Nós não temos censura, cada escritor escreve seu livro como crê e
cria. Só não admitimos é uma literatura imoral, pornográfica, porque
isso de nada serve para uma sociedade nova: isso, sim, é proibido, só se
alguns indivíduos quiserem trazer material desse tipo em livros e
videocassetes, não impedimos. Veja: Portugal, ou melhor, seu regime,
durante muito tempo proibiu que falássemos em manifestações públicas ou
nas escolas os nossos idiomas nativos; os batuques, por exemplo, deles
temos notícias apenas pelas anotações feitas pelos missionários
católicos protestantes, quando esses grupos eram apresentados aos
turistas como coisa exótica: em resumo, Angola está-se fazendo dia a
dia.
\end{quote}

\chapter{A última denúncia de
Soromenho}\label{a-uxfaltima-denuxfancia-de-soromenho}

Jornal da Tarde, 1971/02/24. Aguardando revisão.

\hfill\break

\emph{A Chaga}, o último e excelente livro de Soromenho

Na literatura, Angola é um círculo do inferno nos trópicos, complementar
o círculo de campos de concentração soviéticos na Rússia de Soljinitsin.
Uma literatura clandestina, proibida em seu país de origem, ecoa, em
português, as denúncias candentes de uma situação que espezinha o homem
e lhe nega qualquer dignidade essencial. Não é uma literatura da
\emph{négritude}. Nem se insere na literatura portuguesa, a não ser na
categoria de protesto. Mas é um veio secreto, explosivo, de
extraordinária força de linguagem que completa o quadro de discórdia
liberal diante do colonialismo e do racismo que Doris Lessing na África
do Sul, com seus contos reunidos em \emph{The Habit of Loving}, e de
Alan Paton com \emph{Cry, the Beloved Country} iniciavam na literatura
africana em inglês.

Entretanto, seus expoentes mais expressivos são quase totalmente
desconhecidos entre nós: afinal, de que valem os direitos de alguns
milhões de negros diante do peso esmagador do outro e dos diamantes da
África do Sul? Mas como na literatura os pesos são diferentes, para o
leitor vale infinitamente mais um Castro Soromenho do que um
Primeiro-Ministro Verwoerd.

Quem é Castro Soromenho?

Co-fundador do Movimento de Libertação de Angola, esclareceu à imprensa
de Paris -- onde seus romances têm tiragem de 100 a 200 mil exemplares
cada um -- que ``nasci em Angola, sou angolano de raiz mas sou sobretudo
um escritor português''. Incluído como expoente branco numa antologia
dedicada à \emph{Négritude}, irritou-se:

\begin{quote}
\begin{itemize}
\tightlist
\item
  ``Faço parte da''presença africana'', sou uma voz integrante dela, mas
  preferia que me citassem como membro do côro muito mais amplo de uma
  ``presença humana'' na África''.
\end{itemize}
\end{quote}

Herético dentro do pensamento socialista, em seus livros era imparcial o
suficiente para reconhecer o lado ``criador, viril e positivo do
colonialismo branco na África, que para lá levou a tecnologia do século
XX''. Perseguido pelo regime de Salazar, passou no Brasil os últimos
três anos de seus 58 anos de vida, lecionando na Universidade de São
Paulo sociologia africana, embora não tivesse curso superior.

Como recorda um seu ex-discípulo da USP:

\begin{quote}
\begin{itemize}
\tightlist
\item
  Era uma figura estranha, quase disforme. Falava tão baixo que nós,
  alunos, tínhamos dificuldade em acompanhá-lo. Baixo de estatura, os
  olhos claros e sempre inquietos, a cabeleira toda branca, era um
  rebelde que detestava os sectarismos, a limitação rígida dos
  \emph{slogans} do partido, de movimentos organizados, ``todos
  destinados a aprisionar a liberdade original do homem''.
\end{itemize}
\end{quote}

Quatro anos após sua morte em São Paulo, em 1968, é publicado no Brasil
\emph{A Chaga}, o romance amargo, intenso, que o escritor português,
nascido em Luanda, trouxera como manuscrito alinhavado da Europa e
terminara nos intervalos dos trabalhos na redação do jornal \emph{O
Estado de São Paulo} que o acolhera como comentarista político
anti-salazarista.

Integrante de uma literatura subterrânea, eruptiva, vital, Castro
Soromenho é conhecido em alemão, inglês, francês, italiano, russo e
espanhol. É o vértice de um triângulo angolano liberal que se completa
com Luandino Vieira, autor de contos enfeixados em \emph{Luuanda} e
Alexandre Cabral, contista. Mas sua leitura é mais engajada,
inteligentemente, com o ser humano ultrajado em seus direitos, do que
com os métodos externos coercitivos que se proponham a modificar o
comportamento hostil do homem para com seu semelhante por mudanças
meramente político-econômicas.

Acima de tudo, seu compromisso é com a denúncia dos horrores de um
colonialismo que, nutrindo-se literalmente da África, viola seus mais
elementares direitos. Não há perigo, por isso, de se deparar com uma
tendenciosa literatura de propaganda política panfletarista, em que o
talento é substituído pelos \emph{slogans} veementes e ocos.

Opondo-se às pesquisas de estilo de um Aquilino Ribeiro e ao romance
psicológico de Miguel Torga, sua denúncia social nada tem do tom emotivo
de um Ferreira de Castro. Considerado pelo sociólogo e profundo
conhecedor do Brasil e da África Roger Bastide ``o maior ficcionista do
mundo africano'', Castro Soromenho reproduz, com admirável fidelidade,
as duas Áfricas bipartidas pelo colonialismo extorsivo. De um lado, há
os negros, tangidos pela ignorância, pela violência, pela forme. De
outro, a minoria branca, com seus inumeráveis matizes sociais.

Para o Se. Administrador-Geral da Província Ultra-Marina de Angola --
personagem que representa a filosofia do colonialismo do século XIX
ainda sobrevivente nas colônias portuguêsas -- ``os negros admitem a
autoridade, a violência justa''. Os negros, desde cedo, aprendem pela
cartilha ensinada pelos jesuítas e que tão bons efeitos surtiu no Brasil
até 1822: a cartilha dos três Ps: pau, pão e pano. Pau para os rebeldes,
``os atrevidotes das cidades influenciadas por ideias estrangeiras'';
pão para manter a vida escrava e dar lucro contínuo ao branco; pano para
``cobrir as vergonhas'', mas não a ponto de impedir a violação de
``meras negrinhas'' de 12 anos, cobiçadas por soldados solitários peões
anônimos nesse sinistro jogo de xadrez desenhado por Soromenho no
tabuleiro africano. Os brancos são, na maioria, os aldeões como o João
porqueiro, evocado por um soldado:

\begin{quote}
``Estava fora da terra havia muitos anos, mas o meu pai lembrava-se
dele. Ninguém sabia ao certo, or onde andara o João porqueiro. Lá pelas
Áfricas, diziam. Depois, deixou-se de falar dele. Um dia o homem chega
cheio de notas e com um automóvel que nem um ministro! O João porqueiro
descobriu uma mina nos matos da África, disse o meu pai. Uma mina de
quê? Quis saber o me padrinho. O pai não sabia de que era a mina. Mas à
noite, quando estávamos a jantar, o padrinho veio com a novidade: o
Dr.Anacleto, que também andou pelas Áfricas, diz que a mina não é mina
nenhuma, ou melhor, é uma mina, sim, mas é uma mina de pretos.''Os
pretos nascem nas minas'', perguntou a Mariquinhas, minha irmã mais
nova. Nós largamos todos a rir.''
\end{quote}

Ou representantes do pequeno comércio, da incipiente indústria, até o
topo limitadíssimo da pirâmide: a Administração de Além-Mar, uma
Administração obviamente branca, patriótica, civilizadora. Como
esclarece um personagem, funcionário adulador dos poderosos: - ``Somos
nós os portugueses mestres em matéria de colonização, até os ingleses
reconhecem nossa superioridade''.

Entre esse verniz tênue, branco, e o oceano negro estão os
\emph{sipaios}, na maioria mulatos da guarda indígena nacional, que
aspira aos privilégios da minoria europeia. A maioria silenciosa, negra,
os cinco milhões que constituem a base são o motor da economia que supre
a metrópole. Analfabetos, mascando drogas que diminuem o embrutecimento
do trabalho duríssimo nas minas, nas usinas hidrelétricas, nas fazendas,
nas fábricas, os negros refugiam-se nas tradições tribais, invocam os
primitivos e surgem com imagens poderosas no romance:

\begin{quote}
``Vinte homens, com tangas de pele de leopardo, dorsos nus reluzentes de
barro vermelho de antílope ou de penas de papagaio cinzento, deram um
passo avante e apresentaram armas''
\end{quote}

Os negros são os personagens trágicos deste romance que fotografa, com
sensibilidade extraordinária, a Natureza verdejante e com secura de
relato doloroso, pessoal, a humilhação diária de não-ser num mundo
fantoche. Como reconhece um personagem lúcido, autobiográfico, que
retrata em palavras candentes esta clamorosa opressão humana:

\begin{quote}
``É esse negro que por aí anda com ar de medo é como a raiz de uma terra
queimada. Sob a humildade, a resignação, o medo, ele vive com desespero
e ódio. Para a sua vida o colonialismo é uma queimada, uma chaga, mas
eles são as raízes vivas dentro desta terra queimada''.
\end{quote}

Para o leitor deste terrível mural, sóbrio e de paixão contida por um
estilo de magistral disciplina, Angola nunca mais será um ponto verde,
amplo, no mapa da África.

Como para Castro Soromenho, será, como para a consciência do próprio
homem, a chaga na pele de um Continente, admiravelmente captada por uma
sensibilidade compassiva e que não compactua com sua gangrena.

\chapter{Senghor, o Orfeu negro}\label{senghor-o-orfeu-negro}

Veja, 1969/12/31. Aguardando revisão.

\hfill\break

Pelo menos no Senegal, a poesia está no poder. Amigos desde os bancos do
ginásio Louis-le-Grand, em Paris, Georges Pompidou e Léopold Senghor
desde cedo sentiram-se unidos pelas paixões comuns da poesia e da
política. No caso do atual presidente da França, a poesia passou a um
\emph{hobby} cultivado nos intervalos da administração governamental. No
de seu amigo africano, a política não eclipsou a criação poética desde
sua eleição unânime, em 1960, para a presidência da antiga colônia
francesa. Pouco depois de assumir a suprema magistratura senegalesa,
Senghor conquistava fama mundial em outro campo: o da literatura, com
seus \emph{Poèmes}, publicados em 1964. Seus poemas formulavam de forma
concreta a teoria da \emph{négritude} que ele cimentara, expandindo o
pensamento inicial de Aimé Césaire, estruturado filosoficamente por
Jean-Paul Sartre. A \emph{négritude} é porém só um dos aspectos da
grandeza deste Orfeu negro. É o que compravam seus \emph{Poemas}
traduzidos agora no Brasil (Edições Grifo, 1969). A África Negra, como é
natural, inspira a maior parte de seus versos, mas sua versatilidade
estende-se à participação política, à exaltação poética de jardins da
França e da neve caindo sobre Paris. Ou às elegias que catam a doce
saudade portuguesa na voz de Amália Rodrigues.

Sua definição da \emph{négritude} dada durante uma entrevista ao jornal
\emph{Le Monde} aplica-se predominantemente à sua própria poesia: ``Uma
força emotiva que leva à assimilação intuitiva do mundo exterior (e não
racional) aliada ao dom do símbolo, da imagem e do ritmo e uma noção de
excepcional de comunhão, de solidariedade coletiva''. O painel
deslumbrante da sua inventividade poética abre-se com a evocação da
África, uma evocação em que a aldeia natal ressurge docemente: ``Eu não
sei em que tempo se deu, confundo sempre a infância com o Éden''. Uma
infância passada em Joal-la-Portugaise: ``Joal/ Eu me lembro/ Lembro-me
das signares à sombra verde das varandas./ Das signares de olhos
surreais como um luar na praia''. Nesta África com vestígios do Brasil
(signare é uma corruptela de senhora, a sinhá brasileira, que no Senegal
designa as mulheres de cor casadas com homens brancos), a natureza
funde-se com os dialetos e a celebração da missa em latim, palmeiras e
trombetas tribais. A exaltação da pele negra antecede de vários anos o
movimento atual \emph{black is beautiful} (a cor negra é linda) dos
artistas negros americanos. No admirável poema intitulado ``Mulher
Negra'', ele celebra dois de seus temas constantes: a beleza africana e
o amor. Em ``Mensagem'' são os ancestrais dos negros africanos modernos,
diplomados na Europa (como o próprio Senghor), que falam, exortando-os a
não se iludirem pelos diplomas (``amontoais folhas de papel'') e a
voltarem aos ritos de sua raça e de suas civilizações antigas: ``Ide a
Mbissel à Fa'ou; rezai o terço de santuário que balizaram a Grande Via./
Segui de novo a Estrada Real e meditai esse caminho de cruz e glória./
Vossos Grandes Sacerdotes responder-vos-ão: Voz do Sangue!''

Mas a poesia de Senghor não se restringe ao exótico. Suas ``Elegias''
ultrapassam qualquer referência local para adquirir uma grandeza solene
e majestosa. Particularmente na ``Elegia da Meia-Noite'', que celebra o
``Verão, esplêndido Verão, que nutres o Poeta como leite da tua luz/ A
mim, que brotava como trigo primaveril, que me inebriava/ com a verdura
da água, com o verde escorrer no ouro do tempo''. Na ``Elegia das
Saudades'' é a ``gota de sangue português que se perdeu no mar da minha
\emph{Négritude}'', é a origem do nome Senghor (do português Senhor) e
onde ``reencontrei meu sangue, descobri meu nome outro ano em Coimbra/
sob a misturada dos livros/ Mundo selado de caracteres estritos e
misteriosos, ó noite das verdes florestas, aurora de plagas
inauditas..'' Nos seus cânticos para mulatas, com acompanhamento de
instrumentos indígenas como ``Congo (\emph{Woi} para três \emph{korás} e
um \emph{balafong})'', ou ouvindo num ``calmo e grave jardim da França''
o apelo ``do tam-tam retumbante, veemente, lancinante'', Senghor -- como
documentam mesmo essas traduções literais e sem sopro poético -- é um
dos grandes poetas do mundo atual. Delicado e pujante Orfeu negro, em
sua lira se alinham as paisagens paradisíacas da África, o requinte
estilístico do francês com seus recursos de ritmo, cor e melodia, e uma
contribuição inédita de força e sensibilidade à grande poesia deste
século.

\chapter{Resenha sobre Poemas de
Senghor}\label{resenha-sobre-poemas-de-senghor}

Jornal da Tarde, 1970/1/29. Aguardando revisão.

\hfill\break

Poucas vozes poéticas surgem com tanta pujança neste final de século.
Senghor, presidente do Senegal, um dos formuladores da teoria da
\emph{Négritude}, alinha-se ao lado de um Saint-John Perse, de um T. S.
Eliot, de um Gottfried Benn, de um Carlos Drummond de Andrade.

Sua poesia ultrapassa o exotismo pitoresco de uma ``poesia negra''. As
palmeiras, as tribos, os desertos não entram gratuitamente como cor
local em sua poesia. Brotando da união de inspiração africana e forma
francesa, sua poesia é universal pela qualidade altíssima de seu
conteúdo e pela beleza de seu estilo. Requintados, polidos, seus
destilam uma África interior desvendada em toda a sua beleza: cantam a
beleza da mulher negra ``sazonado fruto de carne firme, êxtase de negro
vinho, boca que liriza os meus lábios''; cantam a lembrança na aldeia
natal: ``Joal!/ Lembro-me das pompas do Poente/ Donde Kumba N'Dofene
queria tirar o seu manto régio. / Lembro-me dos banquetes fúnebres,
recendendo sangue dos rebanhos imolados, / Rumor de querelas, rapsódias
de menestréis''; cantam os deuses antigos das civilizações destruídas
pela violência do homem branco: ``Enviaram-se um correio veloz,/ E ele
atravessou a violência dos rios; nos arrozais baixos, mergulhava até a
cintura./ Porque sua mensagem era urgente, /Deixei a refeição fumegando
e o cuidado de muitos litígios. / Uma tanga, nada mais levei para as
manhas orvalhadas./ Por viático, palavras de paz, brancas, abrindo-me
caminho..''

Versátil, porém, Senghor não limita seus temas às imagens vigorosas da
paisagem africana. Seus poemas agradecem a uma enfermeira branca que
cuida de indígenas doentes do hospital: ``Ema Payeleville:/ Teu nome
apagará as poeirentas figuras dos governadores/ Tu, donzela tão frágil e
delicada/ Derrubas as muralhas decretadas entre ti e nós, humildes
indígenas''. Ou de Verlaine, descreve a neve que cai sobre Paris com
sutileza: ``Meu coração, Senhor, derreteu-se como a neve sobre os
telhados de Paris,/ Ao sol da tua doçura..''

O tom majestoso e solene de suas elegias não exclui uma referência aos
fados de Amália Rodrigues, por exemplo, na sua magnífica exortação da
Saudade e da gota de sangue português que se perdeu em meio à sua
\emph{négritude} e que persiste na origem portuguesa de seu sobrenome,
uma corruptela africana de Senhor.

Poeta lírico do amor, poeta religioso que mescla o latim ritual da missa
com os dialetos e instrumentos musicais africanos, poeta social que
transforma o ódio dos negros contra os racistas numa profunda meditação
filosófica sobre a condição humana ou no espírito de paz e boa vontade
do Natal, Senghor não perde sua vitalidade nem mesmo nesta tradução um
pouco prosaica e sem voo poético feita agora no Brasil. Afinal, seria
preciso ser poeta para traduzir bem um grande poeta. O leitor porém
apreciaria ter explicações ao pé da página de termos africanos
abundantes e que são citados sem qualquer esclarecimento:
\emph{Kor-Sanou}! \emph{Paragnessês}. \emph{Bayetê Babá}! \emph{Bayetê ó
Zoulou}! \emph{KhalamI. Balafong} -- o que significam exatamente? São
lugares? Invocações? Instrumentos?

Estes senões são menores. O importante é o encontro do leitor brasileiro
com um dos grandes poetas do século. Senghor, como em sua própria
``Oração às Máscaras'', sem saber, definiu exemplarmente a majestosa
contribuição que sua poesia trouxe à renovação poética mundial: ``Que
respondemos `presente' ao renascer do Mundo/ Qual fermento necessário à
farinha branca./ Pois quem ensinaria o ritmo ao falecido mundo das
máquinas e dos canhões? Quem daria o grito de alegria para despertar
mortos e órfãos à aurora?/ Dizei, quem poderia restituir a memória da
vida ao homem desesperançado?\ldots{} Somos os homens da dança, cujos
pés se revigoram ferindo o rude chão''.

\chapter{A poesia não deve morrer. Senão, que esperança restaria ao
mundo?}\label{a-poesia-nuxe3o-deve-morrer.-senuxe3o-que-esperanuxe7a-restaria-ao-mundo}

Jornal da Tarde, 1977/11/5. Aguardando revisão.

\hfill\break

``Não sei em que tempo foi, confundo sempre a infância e o Éden''

L. Senghor

Ao visitar Coimbra, o poeta do Senegal Léopold Senghor evoca a gota
perdida de sangue português que deu origem a seu nome -- Senghor é uma
corruptela de Senhor -- quando no século XV os portugueses e árabes
começaram o comércio de escravos para as Américas. Hoje presidente de
seu país, Léopold Sedar (da tribo dos \emph{serere}) Senghor é a ponta
de lança mais universal do movimento iniciado na década de 30 em Paris.
Em 1932, um ano antes da ascensão de Hitler ao poder na Alemanha e do
seu \emph{Reich} da ``raça superior'', um antilhano negro, Aimé Césaire,
fundava o movimento \emph{Légitime Défense}. Era o estopim da rebelião
negra contra a canga da cultura branca. Revolução de Copérnico: a Europa
não era o centro do planeta, assim como a Terra não era o centro do
universo. O sol se deslocava para a zona de sombra da raça cativa, da
raça vendida para os algodoais do Sul dos Estados Unidos e para os
canaviais do Nordeste brasileiro, a raça tripudiada e que nunca
inventara nenhum dos instrumentos de servidão do homem branco: nem as
armas de fogo nem a dinamite, nem o campo de concentração nem a
contabilidade, nem o lucro nem o juro e a linha de montagem industrial.

Aimé Césaire, amigo fraternal de Senghor, era a vertente revolucionária,
de uma revolta que escarnecia da assimilação do negro à gramática do
branco, que esticava os cabelos e evitava o sol, as mulheres espalhando
pó de arroz no rosto para se aproximar do ``branqueamento desejado''. O
essencial era ser ``os macacos da civilização branca'', misturar-se com
os brancos para ter filhos mulatos, abandonar gradualmente a situação de
seres ``há pouco descidos das árvores'' e que se distinguiam por sua
``robustez em resistir às chicotadas'' de ingleses, portugueses,
espanhóis, holandeses, franceses. Desiludido de um Cristianismo que
terminava quando o missal se fechava depois da bendição do padre aos
fiéis, desiludido com o marxismo que ``procura amoldar os negros à sua
doutrina em vez de fazer o contrário, que seria o justo'', ele inicia a
centelha de um movimento a que Senghor emprestará logo sua sabedoria
profunda e seus horizontes mais amplos: a \emph{négritude}. O que era a
\emph{négritude}? A repulsa dos padrões brancos, o reconhecimento de que
os homens e mulheres de cor não podiam escrever como Camões, Cervantes,
Shakespeare e Racine, não podiam sentir o que uma cultura estranha lhes
impunha como norma. A \emph{négritude} era saber-se negro, sentir-se
consciente e orgulhosamente preso às características da alegria negra,
do colorido negro, da espontaneidade negra, da cordialidade e meiguice
negras. Aprofundando essa redescoberta que os negros norte-americanos
como Richard Wright tinham feito através do ódio e da amargura, Senghor
trazia a esse soerguimento de uma raça a altivez do pertencer a uma
pátria de riquíssimas tradições artísticas, culturais, humanas: a África
ancestral, a raiz das civilizações do Benin, do Mali e os laços
familiares, tribais que a escravidão viera violentar. As pesquisas dos
antropólogos franceses na África Negra e sobretudo do sábio alemão
Frobenius não deixavam razões para qualquer dúvida: A Europa
desenvolvera somente uma \emph{técnica} superior. Em todos os outros
domínios da presença humana o africano tinha um passado soberbo, com
seus vestígios potentes nas máscaras que agora influenciavam a pintura
cubista de Picasso no quadro famoso \emph{Les Demoiselles d'Avignon},
nas religiões que em vez de assimilarem o africano criavam uma síntese
entre o vodu e o Cristianismo, São Jorge ao lado de Xangô, Iansã
irmanada com Santa Bárbara da Bahia ao Haiti. Revelava-se a beleza do
jazz, do samba, do espiritual, do \emph{blues}. O que o Tiradentes
haitiano, Toussaint L'Ouverture, tentara fazer pela sua pátria negra da
América, pisada pelo colonialismo francês, a \emph{négritude} fazia
agora em termo intelectuais: liberava o negro do seu complexo de
inferioridade, só deu ``exotismo'' de criança grande, de adulto
retardado e que só era aplaudido quando se comportava ``como um branco
tal e qual'': agora \emph{black is beautiful}, o negro é lindo.

Senghor cantava a raiz da religião de curandeiros, de soberanos negros
do passado, cantava a natureza do Kilimandjaro coroado de neve, os rios
coleantes as pirogas, as árvores como o baobá, as girafas e os
elefantes, a poesia oral e sobretudo a doçura da mulher negra: ``mulher
nua, mulher negra/ vestida de tua cor que é vida, de tua forma que é
beleza!\ldots/ Fruto maduro de carnes firmes, sombrios êxtases do vinho
escuro/ boca que torna lírica a minha boca/ Savana de horizontes
límpidos/ e que freme sob as carícias/ ardentes do Vento Leste''.

Senghor, católico, companheiro de ginásio na França do ex-presidente
francês Georges Pompidou, membro da Resistência francesa na França
ocupada pelos alemães em 1940, engloba \emph{todas} as raças num retorno
espiritual e etimológico à palavra \emph{católico}, que significa
universal: de nada adiantava lançar os negros contra os brancos como no
passado os brancos tinham lançado tribo contra tribo ou como agora, na
Europa dilacerada pelo nazismo, indivíduos da mesma pigmentação se
bombardeavam, se exterminavam em campos de concentração e na defesa,
palmo a palmo, de Stalingrado. Não era a guerra que ele pregava, mas o
amor e a congregação da humanidade numa ampla família que abrangia
diversos tipos de viver: o modelo europeu não era o único mas apenas um
a mais a somar-se às propostas da Ásia, dos indígenas das Américas, da
sabedoria ancestral das tribos africanas.

Nos soldados negros norte-americanos que vêm combater pela democracia no
solo europeu ele sente pulsar sob o uniforme o tam-tam da África-mãe
inicial, na infantaria senegalesa que colabora com os aliados para
arrasar catedrais góticas ele se sente cindido entre seu amor pela
igualdade dos homens e a luta bárbara que se trava destruindo os
monumentos que não são herança nacional de nenhum país mas patrimônio
cultural da humanidade inteira.

Professor de gramática, \emph{agregé} negro aceito segundo os padrões
europeus na metrópole, ele é acolhido de volta a seu país natal com
honras de Estado e eleito presidente com um filho pródigo que trazia de
Paris a semente da libertação política também para o Senegal. Foi Paris
que lhe devolveu seu rosto africano: Senghor quer que a África, livre da
colonização europeia, se torne um irmão ativo na construção de um mundo
de paz, de reconciliação, segundo a doutrina do Cristo, de ``amar o
próximo como a ti mesmo''. É a raça negra, com seu instinto espontâneo
para a música, para a dança, para o gesto fraternal, para a
generosidade, parece-lhe a mensageira ideal dessa nova proposição contra
o seco racionalismo do ``Eu penso, portanto eu sou'', de Descartes.

Não: eu danço, portanto eu me irmano com o próximo, retruca a África
alegre. E mesmo entre guerras, durante a carnificina que hoje dilacera o
continente com tropas cubanas em Angola, grupos etíopes contra soldados
somalis, ambos armados pela União Soviética, com a ameaça do regime
racista da monstruosa África do Sul do \emph{apartheid} de apossar-se da
bomba atômica, Senghor, continua crendo na fraternidade do homem e na
perenidade da poesia:

``Já é tempo de sustar o processo de desagregação do mundo moderno e em
primeiro lugar o da poesia. É preciso restituí-la às suas origens, aos
tempos em que ela era cantada e dançada. Como na Grécia, em Israel
sobretudo no Egito dos faraós. Como hoje em dia na África Negra. `Toda
casa dividida por lutas internas', toda arte voltada contra si mesma não
pode deixar de perecer. A poesia não deve morrer. Senão, que esperança
restaria ao mundo?''

\chapter{Sebène Ousmane do Senegal. Um importante momento da literatura
africana. Mas a
tradução}\label{sebuxe8ne-ousmane-do-senegal.-um-importante-momento-da-literatura-africana.-mas-a-traduuxe7uxe3o}

Jornal da Tarde, 1984. Aguardando revisão.

\hfill\break

Literatura negra africana: pró ou contra?

Essa escolha absurda se impôs durante muito tempo: um escritor (ou uma
escritora) dos países da África Negra deve só glorificar os heróis e
grandes feitos da raça e da cultura negras? Ou deve mostrar os homens e
mulheres negros como seres humanos iguais a todos os outros, com suas
qualidades e defeitos idênticos aos de todo mundo?

Sembène Ousmane corta esse nó com uma argumentação sumamente inteligente
e convincente:

\begin{quote}
``Durante anos mantive contato com alguns de vocês: AFRICANOS. (Nota:
maiúsculas do autor). As razões, razões dadas por vocês, não me
convenceram. Estavam, claro, de acordo sobre um ponto: `Você não deve
escrever esta história'. Argumentavam que seria lançar opróbrio sobre
NÓS, A RAÇA NEGRA. Ainda mais, acrescentavam vocês, os detratores da
CIVILIZAÇÃO NEGRO-AFRICANA (sempre maiúsculas do autor) iriam
aproveitar-se dela para\ldots{} para\ldots{} para lançar-nos o opróbrio.

Para não parecer pedante, recuso-me a analisar a reação de vocês diante
deste caso. Mas quando é que vamos deixar de aceitar, de aprovar nossa
conduta em função da cor dos outros e não da do nosso EU HUMANO?''
\end{quote}

Seria demasiado longo citar toda a apresentação que precede um de seus
sóbrios mas eloquentes relatos, ``Branca Gênese'', a respeito de um
incesto. A Editora Ática com o lançamento, no Brasil, desta história e
de outra que a precede, ``A Ordem de Pagamento'', resgata para o leitor
brasileiro um romance que é famoso na Europa há quase 20 anos.
Infelizmente, a Editora confiou esse fino prosador a um tradutor, Jayme
Villa-Lobos, que precisa voltar urgentemente à Aliança Francesa ou tomar
aulas de português para não reduzir livros importantes com este a um
dialeto frantuguês irreconhecível tanto em francês quanto em português.

A grandeza e a força narrativa de Sembène Ousmane resistem à tradução e
às dezenas de galicismos incompreensíveis para quem não souber francês:
``clientes de marca como você'', ``o que eu tive de `molhar' a mão das
pessoas para conseguir esse arroz'', ``assim que Mbarka está ao
corrente'', ``ela falou sem ênfase, com seu acento contrastado'' etc.

\emph{A Ordem do Pagamento} desenha uma África humaníssima, perdida nos
labirintos da burocracia da Administração legada pelos franceses. A
solidariedade dos africanos como grupo desaparece na luta já sem
escrúpulos pela sobrevivência do mais sórdido.Todos se entredevoram em
busca de dinheiro, de alimentos, de gorjetas e subornos: o ingênuo
personagem principal, Ibrahim Dieng, é acossado de todos os lados logo
que es espalha a notícia de que seu sobrinho Abdou, que trabalha em
Paris, lhe mandou uma ordem de pagamento. Sem saber francês (isto é: um
pouco na situação em que ficará o leitor brasileiro que ler esta
tradução sem auxílio de um dicionário \emph{français-portugais}), Dieng
tem de pagar a alguém para que lhe leia a carta do sobrinho, pois é
analfabeto até mesmo numa das línguas locais, o wolof, bem como em todas
as outras (seis ou sete) faladas no Senegal. Muçulmano devoto, Ibrahim
Dieng, invoca os versículos protetores do Corão quando se vê assediado
pela chusma de pedintes, de ``colaboradores'' e parentes improvisados,
próximos ou distantes, que souberam da informação transmitida logo em
mexerico empolgante: ``Ibrahim recebeu dinheiro grosso da França!''

O letor tem acesso a visões ora humorísticas ora trágicas da
desumanidade do homem para com o homem, independentemente de sua raça.
Com seu estilo conciso, com curtas frases de funda comoção humana, ele é
testemunha de uma metamorfose kafkiana dos seus semelhantes,
transformados em animais em animais pela ganância, pela empáfia, pela
adulação, pela miséria, pelo desemprego, plea fome. Seu itinerário,
aparentemente simples, é o mais tortuoso possível: conseguir uma
carteira de identidade, sem a qual não pode retirar a ``ordem de
pagamento'' enviada pelo sobrinho para manter a mãe, indigente, que vive
à míngua na zona rural.

Em certos momentos, a mestria de Sembène Ousmane nos faz esquecer o
nível de dublagem da televisão a que seu romance lamentavelmente foi
exposto na tradução brasileira. Aí nos alçamos acima dos termos não
traduzidos, prece de \emph{Tacousane} (?), \emph{veudieu} (?),
\emph{nidiyea} e varios outros. Nestes instantes, percebe-se o talento
admirável do autor africano para retratar aquele exército de famintos,
de maltrapilhos, sem manchar-lhes o retrato de nenhum traço piegas nem
ideológico:

\begin{quote}
``O ar tórrido misturado ao cheiro sufocante dos canos de descarga
tornava a atmosfera viciada, o cruzamento formigava de gente mal
vestida, em andrajos, aleijados, leprosos, crianças em farrapos,
perdidas naquele oceano'' e em outro trecho: ``Diante do correio
estendia-se a fila de mendigos, dispostos como vasos de flores murchas,
uns estendendo a mão, outros o prato, todos emitindo suas queixas''.

Ressalta imediatamente também a acuidade psicológica do romancista:
capta com poucas frases as intenções velhacas do dono do armazém, a
patética declamação de elogios do adulador que vive disso, ganância
inicial das duas esposas de Ibrahim, logo transformada em desprendimento
e fatalismo. Sembène Ousmane lamenta a África perdida, em que a
solidariedade humana entre as famílias, os conhecidos, servia de muro
contra a adversidade, mas revela uma objetividade que esse espírito
comunitário foi insuficiente, porém, para ``impedir os assassinatos, as
prisões ilegais, as detenções políticas das dinaistias que reinam hoje
na África Negra''.
\end{quote}

Não há um juízo final, terminante, emitido pelo autor. Demasiado sutil,
ele deixa ele deixa em suspenso qualquer conclusão: despojado da ordem
de pagamento por um parente ladrão e untuoso, Ibrahim Dieng, hesita
entre ``tornar-se uma hiena'' igual aos outros e conformar-se com a
vontade de Alá: Alá determinara que aquele dinheiro nunca chegaria às
suas mãos. Parece-me haver um equilíbrio entre a crença religiosa e a
possibilidade de um retorno à honestidade pré-colonial, desvirtuada pela
presença europeia, pelos seus critérios de usura, de mentira, de egoísmo
instituído, de adoração materialista. Possivelmente será esta a diagnose
final que o autor faz desta aflitiva e tocante tragédia humana. Imbuída
a vítima de uma fé e uma esperança no absoluto poder de Alá, o único
Justo, Sábio e Misericordioso, muito além das limitações humanas, ele
não descarta a aparição de uma nova justiça, como que ``dando a César o
que é de César'', segundo o ensinamento do Cristo:

\begin{quote}
``Há que compreender Ibrahim Dieng. Condicionado por anos de surda
submissão inconsciente, ele evitava qualquer ato que pudesse trazer-lhe
prejuízo, tanto físico quanto moral. O soco recebido no nariz era um
\emph{atte Yalla}: a vontade de Deus. O dinheiro perdido também. Estava
escrito que não era ele quem o gastaria, pensou. Se, segundo todas as
aparências, a desonestidade parecia ter levado a melhor, era obra da
época e não de Alá. Aquela época que recusava conformar-se à antiga
tradição. Dieng, para minorar a sua humilhação, invocava o poder
absoluto de Alá: era também um refúgio, aquele Alá. No mais profundo de
seu desespero pela afronta sofrida, sustentava-o a firme convicção que
tinha de sua Fé, que descongelava ma torrente subterrânea de esperança;
mas essa torrente também trazia à tona vastas zonas de dúvida. A certeza
de que amanhã seria melhor do que hoje era ponto pacífico para ele.
Lástima, entretanto! Ibrahim Dieng não sabia quem seria o artífice
daquele melhor amanhã, aquele amanhã que era ponto pacífico para ele.''
\end{quote}

Sembène Ousmane deixa mais claro ainda essa distinção entre o religioso
e o leigo na dedicatória que faz ao velho companheiro de luta de quem se
separou depois que o amigo ``acreditou no Deus do Lucro, na felicidade
com Dinheiro''. (Será confiável esta tradução?).

O autor que a Editora Ática -- a par de sua monumental \emph{História
Geral da África} cujo 2º volume já publicou -- revela ao público leitor
brasileiro é um combatente ativo em prol da democracia: lutou na
libertação da Europa do nazismo durante a Segunda Guerra Mundial, foi
estivador nas docas de Marselha durante toda a vida autoditada. É
preciso que suas obras anteriores, \emph{Le Docker noir} (Paris, 1956),
\emph{O pays, Mon Beau Peuple}! (Paris, 1959),~\emph{Les Bouts de Bois
de Dieu}~(Paris,1960) e \emph{Voltaïque} (Paris, 1962) venham
complementar a sua incisiva presença na literatura africana
contemporânea. E quando será traduzido o senegalês Cheikh Hamidou Kane?
Afinal já começa a haver um justificado cansaço causado aqui por parte
da literatura hispano-americana que abusa do ``realismo fantástico''
como pano de fundo para marionetes sem vida do ``socialismo soviético'':
o militar despótico, os camponeses revoltados, os juízes corruptos, os
índios boníssimos e perfeitos, os brancos todos vendidos e sanguinários,
do tipo que a Sra. Isabel Allende comete, centenas de páginas a fio. E
Gabriel Garcia Márquez é um só, as cópias xerox que dele se fazem do
Peru ao México, são enfadonhas e ilegais.

Da literatura africana de hoje é que se pode esperar aquele renascimento
que a Europa literariamente agonizante com um espinho de nomes de rosa
atravessado inutilmente na garganta não pode mais oferecer ao mundo,
apenas alimentar a lista de \emph{best-sellers} e bobagens
pseudo-eruditas que faz tilintar as máquinas registradoras e faz a gente
pensar que está ``consumindo literatura''\ldots{}

Da literatura africana é que nos virão temas novos, a libertação do jugo
do materialismo politizado, dela virá o despertar de um humanismo
contagiante, desvinculado tanto do relógio de ponto da linha de montagem
quanto de utopias de ditaduras de classes ou de partidos únicos.

Sembène Ousmane é o outro lado da África: sem fanatismos, sem
despotismos, sem terrorismos, sem obsessão pela epiderme. Ele lança um
olhar penetrante, convincente sobre os seus semelhantes e não duvida da
criação consciente de um mundo melhor. Um mundo sem usurpadores, mesmo
quando estes empunham as bandeiras mais falsamente ``democráticas'' e
``igualitárias'' que podem desfraldar. A literatura africana claramente
comprova, se for necessário comprovar o óbvio, que a literatura é o
território mais livre, mais democrático, menos maleável pela propaganda
-- comercial ou política -- de todas as atividades criativas do ser
humano. Desta liberdade ela deriva a sua perenidade e a sua
solidariedade com todos os seres humanos da Terra: ela é que acena com o
verdadeiro e abrangente humanismo redescoberto.

\chapter{A África, hoje. Em dois bons livros - LGR comenta as obras de
Chinua Achebe e Cyprian
Ekwensi}\label{a-uxe1frica-hoje.-em-dois-bons-livros---lgr-comenta-as-obras-de-chinua-achebe-e-cyprian-ekwensi}

Jornal da Tarde, 1983/05/28. Aguardando revisão.

\hfill\break

Com a preponderância avassaladora da economia e da política como únicos
critérios da nossa época, ficam relegados a segundo plano os aspectos
culturais de tal modo que a Nigéria, para o brasileiro medianamente bem
informado, se reduz a um esqueleto: petróleo, população imensa e guerra
do Biafra. Ignora-se que, desde a sua libertação do domínio inglês, há
mais de duas décadas, e até mesmo antes de sua independência política, a
Nigéria já se distinguia de todos os outros países da África Negra pela
sua rica efervescência cultural. Com cinco universidades, entre elas a
mais importante ao sul do Saara, a de Ibadan, com a revista \emph{Black
Orpheus} (Orfeu Negro), que já antes da libertação divulgava os temas e
conquistas da poesia da \emph{négritude} de um Léopold Senghor ou de um
Aimé Césaire, a Nigéria celebra o passado faustoso das esculturas de
bronze de Benin e se afirma no presente com a poesia de nível
internacional, de John Pepper Clark, na sua pesquisa de música popular,
desde o \emph{high life} de Gana até o \emph{dixieland} e o \emph{jazz}
dos Estados Unidos e os sons do Caribe.

Se, através de editoras portuguesas e uma ou outra brasileira, o Brasil
trava conhecimento, assombrado, com o vigor criativo e expressivo de
romancistas, como José Luandino Vieira e Manuel Rui, angolanos, a
surpresa não será menor ao descobrir agora os talentos nigerianos que a
\emph{Editora Ática} em sua oportuna \emph{Coleção de Autores
Africanos}, lança pioneiramente: Chinua Achebe, com \emph{O Mundo se
Despedaça} e Cyprian Ekwensi com \emph{Gente da Cidade}. É pena que a
\emph{Ática} siga, na revelação dessa literatura, , um zigue-zague
estonteante. Ao lado de um romancista importante, original, como José
Luandino Vieira, imprimir-se um farsante desprovido de talento -- como o
português grotescamente autodenominado de Pepetela e sua imitação
fracassada da narrativa oral africana naquele infanticídio literário
chamado \emph{As Aventuras de Ngunga} -- é dar mostras de uma
versatilidade de critérios levada à esquizofrenia ou de uma ausência de
critérios capazes de distinguir o essencial do que lhe é oposto,
adiposo, supérfluo, prejudicial, inútil. Basta recordar outro aborto
pseudoliterário, o insuportável \emph{Portagem} do moçambicano Orlando
Mendes, que não se consegue colocar na pele de um mulato e sufoca o
leitor desprevenido numa atmosfera lacrimosa, evocativa das piores
radionovelas brasileiras da década de 40.

\emph{O Mundo de Despedaça} (no original inglês: \emph{Things Fall
Apart}) tem como epígrafe versos do grande poeta irlandês William Butler
Yeats:

\begin{quote}
``O falcão, a voar num giro que se amplia,~

Não pode mais ouvir o falcoeiro;~

O mundo se despedaça; nada mais o sustenta;~

A simples anarquia se desata no mundo'' (W. B. Yeats, ``O Segundo
Advento'')
\end{quote}

Simbolicamente, Chinua Achebe alude ao desmoronamento das tradições
tribais autóctones com o aparecimento do homem branco -- o missionário,
os comerciantes, as autoridades. No entanto, a capa que a editora dedica
ao livro é excessivamente teatral e dramática, sugerindo que uma África
Negra pura e sem mácula foi enforcada pelo opressor europeu, o que é uma
simplificação de uma situação muito mais complexa e que o próprio livro
desmente. Originário dos orgulhosos e diligentes \emph{Ibos}, Achebe tem
a extraordinária qualidade de não ler o \emph{Pravda} em sua tradução
para o inglês, o \emph{yoruba} ou qualquer outra língua africana que lhe
seja acessível. \emph{O Mundo se Despedaça} é, possivelmente, o livro
mais equilibrado, mais justo e sereno de quantos já se escreveram a
respeito da inserção violenta do Continente Negro nas correntes da
interdependência política, econômica, cultural do nosso planeta, hoje
transformado na aldeia eletrônica prevista por McLuhan. Como este
romance corajoso comprova, a sociedade aborígene da Nigéria -- e haverá
exceção para esta regra para qualquer sociedade humana? -- não vivia num
paraíso, do qual o colonizador a desalojou bruscamente. Com grande
equanimidade, Chinua Achebe ousa mostrar as falhas e injustiças das
organizações tribais intocadas pela civilização ocidental europeia.
Assim, as comunidades se baseavam numa forma monárquica de clãs
oligárquicos. Quem mais rico fosse, mais títulos e poder possuiria. À
semelhança das castas da Índia hindu, os \emph{esus} eram párias,
intocáveis. Sociedade marcadamente machista, nela à mulher fica
reservada apenas um punhado de posições nitidamente subalternas: lides
domésticas, ventres reprodutores e dóceis de pequenas poligamias de
quatro, cinco esposas legítimas para cada herói da tribo. A crueldade
estendia-se também ao abandono, na Floresta Maldita, das crianças
nascidas gêmeas e consideradas maléficas, dos doentes acometidos de
moléstias, como o inchaço. E as lutas físicas consagravam o líder e eram
a forma de aferir o ``valor'' de um homem naquela comunidade que punia
severamente qualquer transgressão de seus mitos: os pobres, os
descrentes da ética do trabalho, da riqueza e da força eram mantidos à
margem dos demais e desprezados inclementemente por todos. Com grande
sensibilidade, Chinua Achebe capta esses instantâneos de um passado em
que a paz se conseguia através das guerras de conquista, da submissão
total aos chefes do momento e a normas nunca postas em dúvida antes.
Okonkwo, o grande lutador de ambição desmesurada, jamais se conforma por
não ter recebido herança alguma de seu pai indolente. Sua ascensão não
conhece os limites do escrúpulo nem da afeição nem da doçura. Todos que
se interpuserem entre seus objetivos obsessivos e ele serão abatidos sem
piedade. Buscando o apoio das divindades que tão mão o aquinhoaram, ele
chafurda no sangue, na violência, na tirania:

\begin{quote}
``Mas essa noite especial estava escura e silenciosa. E em todas as nove
aldeias de Umuófia, um pregoeiro com seu agogô pedia a cada um de seus
habitantes que estivesse presente ao encontro, na manhã seguinte.
Okonkwo, no leito de bambu, tentava imaginar qual seria a natureza da
crise -- guerra contra um clã vizinho? Essa parecia ser a hipótese mais
provável, e ele não tinha medo da guerra. Era homem de ação, homem de
guerra. Ao contrário do pai, era perfeitamente capaz de ver sangue.
Durante a última guerra de Umuófia, fora o primeiro a trazer para casa
uma cabeça humana. Essa era a sua quinta cabeça; e ele ainda não era
velho. Nas grandes ocasiões, como o funeral de alguma celebridade da
aldeia, bebia o vinho de palma no primeiro crânio que cortara''
\end{quote}

Antes da intrusão do mundo da opressão colonialista branca, no entanto,
Chinua Achebe mescla, em dosagens esplêndidas, os mitos e provérbios
africanos com sua aura de colorido e poesia impressionantes. Há o
espírito pessoal (uma espécie de anjo da guarda) que pode influenciar
mal ou bem o destino de cada um: \emph{chi}. Há as deliciosas citações
de ditados africanos: ``\emph{Eneke}, o pássaro, diz que desde que o
homem aprendeu a atirar sem errar a pontaria, ele, o pássaro, aprendeu a
voar sem pousar''. O onipresente senso de humor, o riso espontâneo e
irreverente do africano pontilha também estas páginas: ``Todos riram
gostosamente, exceto Okonkwo, que deu um riso meio sem graça, porque,
como diz o ditado, mulher velha fica sempre um pouco sem graça quando se
faz menção de ossos secos num provérbio''. O Festival da Colheita do
Inhame, a planta da virilidade e alimento principal da tribo, enseja
evocações de grande beleza plástica por ocasião de seus preparativos:

\begin{quote}
``Faltavam apenas três dias para o Festival. As mulheres de Okonkwo
tinham esfregado as paredes das choças com barro vermelho, até que
rebrillhassem. Depois, tinham desenhado nelas motivos decorativos em
branco, amarelo e verde-escuro. Em seguida, pintaram seus próprios
corpos de vermelho e desenharam arabescos, com tinta preta, no estômago
e nas costas. As crianças também foram enfeitadas, os cabelos
parcialmente raspados a formarem bonitos desenhos..''
\end{quote}

O sensível e perspicaz autor nigeriano não omite, porém, as vozes que se
insurgem contra a barbárie de certos costumes irracionais: as mulheres
que surdamente se rebelam contra a lei de atirar os gêmeos
recém-nascidos, sem enterro, na Floresta Maldita; o ancião sábio que
recrimina Okonkwo pela crueldade de matar o adolescente, cuja vida lhe
tinha sido entregue e que devia considerar como seu próprio filho,
deixando o sacrifício do rapaz, decidido pelos oráculos dos deuses, a
outra pessoa. Uma dessas críticas é contra a proibição, absurda, de que
alguém morra no decurso da celebração da Semana de Paz anual:

\begin{quote}
``- Contaram-me ontem -- disse um dos visitantes mais moços -- que, em
certos clãs, se considera uma abominação que um homem morra durante a
Semana da Paz.~

\begin{itemize}
\tightlist
\item
  E realmente é verdade -- falou Ogbuefi Exeudu -- Existe essa crença em
  Obodoani. Se um homem falecer nessa semana, não é enterrado. Jogam-no
  na Floresta Maldita. É um mau costume o que essa gente segue, um mau
  costume o que essa gente segue, porque lhe falta compreensão. Atiram
  na floresta uma grande quantidade de homens e mulheres, sem enterro. E
  qual é o resultado? Seu clã vive cheio de espíritos mais desses
  mortos, sem tumba, ávidos de causar danos aos vivos.''
\end{itemize}
\end{quote}

Para o leitor deslumbrado com a poesia e a situação comovedora das mães,
cujos filhos, crianças perversas, morrem pouco depois de nascer para
voltar a surgir de seus ventres e morrer prematuramente, são eloquentes
as cenas que evocam o lamento dessa perda e os esforços, em vão, das
mães a querer esconjurar a morte que lhes arrebata os filhos ainda
pequenos:

\begin{quote}
``Ekwefi já sofrera muito na vida. Dez vezes tivera filhos e nove deles
tinham morrido na primeira infância, quase todos antes dos três anos. À
medida que ela ia enterrando um filho atrás do outro, sua dor foi sendo
substituída pelo desespero e, mais tarde, por uma terrível resignação. O
nascimento de um filho, que para qualquer mulher era a coroação de sua
glória, para Ekwefi tornara-se simplesmente motivo de agonia física,
destituída por completo de promessa. A cerimônia do nome, passadas sete
semanas de mercado, tornara-se um ritual vazio. Seu desespero, cada vez
mais profundo, encontrava válvula de escape os nomes que dava aos
filhos. Um deles fora um grito patético: Onwmbiko, isto é: `Morte, eu te
imploro'. Mas a morte não prestou ouvidos à súplica e Onwunbiko morreu
no décimo quinto mês de vida. A seguinte, uma menina -- Ozoemena: `Que
jamais isso venha acontecer de novo' -- morreu no décimo primeiro mês, e
mais dois se foram depois dela. Ekwefi, então, tornou-se desafiadora e
chamou o próximo filho de Onwuma: `Que a morte se satisfaça'. E a morte
assim o fez''
\end{quote}

Chinua Achebe deplora, evidentemente, que a África Negra tenha trocado
os males de suas comunidades tribais pelos males do mundo tecnológico,
prosaico, utilitarista, do europeu e do branco norte-americano. Uma
igreja que se diz cristã e abençoa a escravidão de milhões de africanos
é superior eticamente às divindades africanas às vezes caprichosas, às
vezes cruéis, às vezes indevassáveis, em termos de compreensão humana do
passado? As fábricas e favelas substituem com vantagem a era das lutas,
das colheitas, dos inhames, do infanticídio? O sangue derramado pelo
Império Britânico vale mais que o sangue das lutas entre tribos em
guerra?

O romancista nigeriano (apesar dos percalços da tradução brasileira, sem
estilo e demasiado aderente ao original inglês) responde com sarcasmo:
para os invasores brancos aquele ``incidente'' do suicídio do chefe
Okomkwo e os tabus dos ``nativos'' de não tocaar na corda do enforcado
serão apenas um apêndice do livro que o comissário inglês já esboçou.
Seu título não poderia ser mais sarcástico e arrogante: ``A pacificação
das tribos primitivas do Baixo Níger''.

Não, sublinhe-se bem que Chinua Achebe seja um saudosista de um Èden
inexistente: ele entoa, isso sim, um lamento pela imposição de um
\textbf{único} padrão social, cultural, político , econômico: o do
lucro, da concorrência, da imitação servil dos códigos trazidos pela
civilização branca.

Já em \emph{Gente da Cidade}, Cyprian Ekwensi avocará para si a tarefa
de retratar a Nigéria de hoje, com seu tumulto urbano, sua transição
repentina de um estágio cultural para a proletarização citadina, para a
burocratização e uniformização das metrópoles do século XX. A
violentação das sociedades africanas é também, sem nenhuma metáfora, a
destruição de suas estruturas éticas, a perda de sua identidade
psíquica. É tarde, agora, porém, parece concluir Ekwensi: seus
personagens são repórteres, músicos, moças que vendem sua beleza exótica
a brancos milionários, favelados vindos de Gana ou do Alto Volta todos
indistintos naquele caldeirão governamental de Lagos, a capital e seu
burburinho cacofônico.

A África tornou-se igual aos outros continentes? A extirpação da sua
personalidade foi total? Não parece ser a resposta de autores tão
diferentes e, no entanto, semelhantes em sua visão da Nigéria atual: a
África milenar sutilmente se insinua em meio a uma civilização que lhe é
antípoda e nela reconstrói o lado positivo da herança africana -- a
doçura, a humanidade, o riso, o abandono do relógio de ponto em prol de
uma cooperação comunitária cheia de solidariedade e reconquistada
alegria. Apesar dos preconceitos e da opressão, a África vive.

\chapter{O negro e a cultura de Moçambique não merecem romance tão ruim
quanto
este}\label{o-negro-e-a-cultura-de-mouxe7ambique-nuxe3o-merecem-romance-tuxe3o-ruim-quanto-este}

Jornal da Tarde, 1982/02/27. Aguardando revisão.

\hfill\break

A África Negra -- e demograficamente nela se inclui também a racista
África do Sul do ignominioso \emph{apartheid} -- era, até há pouco,
objeto apenas de retratos feitos por autores brancos: Alan Paton em seu
comovente \emph{Cry, the Beloved Country} denunciando a desumanidade
nazista do regime da sua África do Sul natal; Doris Lessing rememorando
no ambiente da antiga Rodésia, hoje Zimbabue, uma luta que lhe parecia
urgente, cega porém para a luta em prol da libertação do negro: a
conquista da emancipação feminina, no quadro de uma ``normal'' opressão
imperialista dos brancos ingleses em Salisbury. Eram, com exceções,
instantâneos da savana africana tirados pela esplêndida escritora
dinamarquesa, baronesa e fazendeira na África, Isak Dinesen, ou
incursões rápidas, superficiais como um cenário de \emph{papier-mâché}
feito em Hollywood: os romances de Ernest Hemingway no Quênia que,
infantis, jamais tiveram a África e sua problemática plural e ingente
como foco central.

Por outro lado, a mudez sobre a cultura negra da África atingia níveis
cômicos, de tão absurdos: o silêncio só se rompia nas salas empoeiradas
dos arqueólogos, das fábulas orais tradicionais das tribos mantidas
secularmente e recolhidas por antropólogos ou etnólogos curiosos\ldots{}
Ou então uma ``excêntrica'', como a estudiosa do Congo Lilyan Kasteloot
(quando essa vasta região africana era ainda dominada pela minúscula
Bélgica), escreve uma tese doutoral sobre uma cultura emergente, os
escritores negros, dos EUA às Antilhas, passando por vários países
africanos, inclusive as partes ainda não libertadas, àquela época, do
colonialismo português. É sua \emph{Anthologie Africaine} em que Castro
Soromenho aparece ao lado de Noêmia de Souza e Antônio Jacinto. É
pouquíssimo, embora fosse já um início quando de sua publicação em 60.

Paradoxalmente, a difusão da literatura negra deve-se à conscientização
dos intelectuais e artistas negros dos EUA quanto à sua condição de
párias sociais, desde o hoje longínquo Langston Hughes até a publicação
de \emph{The Invisible Man} em nossos dias. As lutas pelos direitos
civis lideradas por Martin Luther King e retomadas por James Baldwin em
seu ensaio \emph{Da próxima vez, fogo!} Repercutem nas universidades, na
criação de estudos africanos especiais, na modificação do nome do
boxeador campeão mundial Cassius Clay para Muhammad Ali, depois de sua
conversão ao islamismo (esquecido de que, historicamente, foram os
árabes que primeiro escravizaram em massa os negros africanos), no uso
esnobe do \emph{suahili} e de roupas africanas por parte dos estudantes
negros das universidades da costa Leste à Califórnia.

A incursão de Picasso, no setor das artes plásticas, com sua inspiração
na estatuária do Benin para suas máscaras \emph{à la façon africaine} e
seu famoso quadro \emph{Les Demoiselles D'Avignon} não causam grande
impacto. Os círculos artísticos que convergem em Paris acham que se
trata de outra ``extravagância'' do \emph{grand maître}, algo assim como
sua homenagem à indefesa e pacífica aldeia de Guernica, bombardeada
sadicamente pelos aviões alemães como ``ensaio'' da guerra mundial nº 2.
A África era, como o nazismo para os alemães contemporâneos, um pedaço
do passado que não tinha sido vivenciado nem vencido. Passado em branco,
amorfo, quem sabe inexistente?

Muito antes do \emph{black power} das minorias negras norte-americanas,
no entanto, o prestígio e primeiro reconhecimento relativamente sério da
arte africana, em plano internacional, surge com Aimé Césaire e Léopold
Senghor e sua temática (nada nova a não ser para os brancos) da
\emph{négritude}, da celebração daquilo que precocemente Lima Barreto no
Brasil já queria inaugurar, uma literatura dedicada à ``negrice''. Mas
Lima Barreto?, indagavam os literatos brancos no Brasil: não era um
alcoólatra, um esquivo, um talento voltado para as sarjetas? Nem, por
outro lado, jamais Machado de Assis, com toda a sua grandeza clássica,
admitiu sua origem étnica, preferindo não militar em prol dos direitos
espezinhados dos negros e mulatos do Brasil, para, em vez disso, imitar
a criação grotesca da já em si ridícula Academia Francesa, à qual Lima
Barreto não teve acesso em vida\ldots{}

A \emph{Négritude} vinha sanar um hiato de bocejo, de falta de
inspiração europeia antes da importação dos autores hispano-americanos
e, de forma breve e perfunctória, lembrou aos que não são negros nem
mestiços que a sensibilidade negra, a expressão negra, a criatividade
negra não precisam expressamente seguir, servilmente, os padrões
hegemônicos da Europa, umbigo do mundo. Mesmo nos poemas possivelmente
superiores em originalidade aos do próprio Senghor, Césaire já planteava
a temática básica do exílio depois da diáspora africana: desprovido da
sua cultura original, de suas tradições, de suas linguagens autóctones,
mesmo depois de abolida a escravidão nos países americanos, o negro
sentia-se como ``o homem invisível'' que dá título ao livro: não existia
para a maioria branca senão como mão-de-obra apta aos serviços mais
duros e mais dignamente remunerados. Um pastor calvinista agitou até uma
celeuma efêmera em torno da questão teológica: teriam os negros almas,
como os brancos? Himmler e seus laboratórios teorizados pelo conde de
Gobineau, louro de olhos intensamente azuis, foi mais longe, já que o
assunto de alma estava de antemão decidido: para o paraíso germânico, o
Walhala, iriam só os arianos, mas aqui na terra se tornara claro,
segundo os ensinamentos da ``genética'' nazista, nisto precursora de
``genética'' stalinista de Lyssenko, que a ``inferioridade'' de subraças
(\emph{Untermensch}), como os ciganos, os judeus, , os negros, era
``cientificamente'' comprováve. Portanto, a \emph{Endlösung} (a solução
final) para os ``problemas'' da existência de tais detritos subumanos
incluía, obviamente, uma segunda alimentação dos campos de concentração:
a carne negra e mestiça, assim que os foguetes V-2 de \emph{Herr} von
Braun tivessem dizimado a Inglaterra. A História, caprichosa, trilhou
outros rumos: von Braun levou o homem à Lua, participando da Nasa
norte-americana, e os negros e mestiços, pelo menos formalmente,
libertaram-se do jugo colonial europeu.

Em nada diminui o mérito da raça negra reconhecer que a descoberta e
devida apreciação de seus valores, que contrastam com seus equivalentes
brancos, se deve ao corajoso pioneirismo de um punhado de brancos não
subjugados pela cretinice, congênita ou adquirida por contágio, do
racismo que impedia aos demais de admitir lhanamente a validez de todo o
\emph{ethos} africano. Claro que a contribuição sentimentalóide de uma
Harriet Beecher Stowe nos Estados Unidos, com sua patética (em termos
literários) \emph{Cabana do Pai Tomás}, teve seu quinhão de
conscientização da posição artificialmente inferior em que as castas de
cor clara tinham colocado os escravos. Mas menos espalhafatosa e mais
profunda é a mensagem de fraternidade de uma Carson Mc Cullers, autora
sulista que em seus romances e contas aponta para a injustiça dessa
situação e alude ao desperdício de talentos utilíssimos à humanidade que
o racismo acarreta como resultado inevitável.

Agora, o Brasil se arvora em nação racialmente democrática, apesar da
lei Afonso Arinos, que demonstra a fragilidade dessa nossa
``democracia'' racial, ao punir quem discrimine qualquer pessoa por
motivos de preconceito de cor\ldots{} Ainda assim, pragmaticamente, o
Brasil se voltou -- já era tempo -- para seus vizinhos
hispano-americanos e para os do lado de lá do Atlântico, não só em busca
do petróleo da Nigéria ou de Angola (com seus milhares de \emph{gurkas}
de Havana), mas, pela primeira vez, já se esboça, num país em que o IBGE
na prova predominarem os ``pardos'', o conhecimento tateante de uma
realidade plural africana.

Um dos últimos lançamentos a chegar às nossas livrarias é da Editora
Ática, com \emph{Portagem} de Orlando Mendes. Seria um preconceito
dizer, francamente, que seu autor, branco, moçambicano, tem dificuldade
em se colocar na pele e desventuras do mulato que é o protagonista de
seu romance? Seria \emph{parti pris} também achar que esse romance nos
chega com quase 20 anos de atraso, publicado em 1966, ou 30, se levarmos
em conta que foi escrito nos hoje longínquos anos 50? Nada disso
importaria se se tratasse de um talento realmente vigoroso como o dos
angolanos Manuel Rui e Luandino Vieira, já mencionados.

O que perturba então na escritura de Orlando Mendes? O seu
tradicionalismo literário, amarrado a ``modelos'' de estilo já gastos
quando Balzac morria. Tem-se a impressão de uma cópia -- má -- de
trechos antológicos dos ``bons autores'' adotados no ginásio, perdão
liceus lusitanos. São raros os momentos em que seus personagens adquirem
força e credibilidade. Orlando Mendes é um narrador, sem ironia alguma,
delicado. A posse carnal, o defloramento é tratado quase que como a
descrição do acasalamento de abelhar ao som de baladas longínquas de
Chopin. Há o contraste não resolvido de uma legítima indignação pela
discriminação racial a par de uma descrição cediça, quase vitoriana, do
ambiente carregado de luxúria, insatisfação, injustiça. Sucedem-se as
cenas -- pudicas ao extremo -- de prostituição, assassinato, desemprego,
marginalização social, e opressão colonial sem que o leitor vibre
minimamente com esse desfilar de acontecimentos telenovelescos, de
folhetim de capa e espada.

O que se há de fazer? A férrea natureza do imperialismo lusitano e
europeu em geral (hoje substituído pelo imperialismo soviético,
possivelmente pior) não dava aos angolanos e moçambicanos durante 50
anos, quase, margem a estudos, reflexões, aprimoramentos --
obscurantismo de que o Brasil foi vítima, ao ``receber'' o presente de
uma Imprensa Régia tardíssimo, em 1808, e, competindo com o Haiti, só
ter universidades na década de 30, pois desde os tempos de D. Maria I, a
Louca, nos era proibida qualquer atividade intelectual ou industrial ou
tendente ao nascimento de ideias longinquamente ``francesas'' de
emancipação política, direitos do homem, quebra de monopólios etc.

Orlando Mendes, profissionalmente, é fitopatologista: nada demais, há
escritores de todas as origens, até negreiros no sentido estrito do
termo: Rimbaud no final de sua vida e Voltaire durante sua vida toda,
quase. Assim, como há grandes escritores fascistas que trabalhavam em
sapatarias, como Céline, ou toxicômanos, como Baudelaire. O lamentável é
que nos seja mostrado como o grande escritor moçambicano -- o único? --
justamente Orlando Mendes. O autor nem sequer sabe estruturar uma
narração: esquece que um personagem, como o Esteves, é velho demais para
se apoderar da esposa do herói João Xilim, depois resolve
improvisadamente ``inventar'' uma tia inexistente em Portugal para onde
mandar um miúdo (criança), além do estilo maçante, pré-histórico, uma
prova a mais de que as boas intenções cravejam os caminhos que levam ao
inferno. Efetivamente, as excelentes intenções humanitárias do autor
levam o leitor ao inferno literário das frases-feitas, da involuntária
inautenticidade do dramalhão postiço. Restam dois ou três motivos
legítimos para que \emph{Portagem} fosse escrito, publicado e, quem
sabe?, até lido por leitores ingênuos ou que cultuam esse tipo de
literatura, armados -- é indispensável -- de uma dose maciça de
paciência. Porque Orlando Mendes é o período da pedra lascada da
literatura: interessante para os darwinistas que procuram ainda o ``elo
perdido'', o famoso \emph{missing link} da árvore da qual descemos todos
os antropoides até o pomposo e fátuo \emph{homo Sapiens}, assim
autoproclamado. Um dos motivos que realmente ficam desse livro primário
é o da revolta dupla que sentem os mulatos, desprezados igualmente por
brancos e negros, numa espécie de \emph{no man's land} apavorante pela
sua dupla segregação. Depois, o desejo legítimo e hoje tornado novamente
atual pela ameaça de nova servidão por intermédio do imperialismo
soviético: o instinto de libertação, de nacionalismo, de expressão das
forças vitais de Moçambique na busca de sua identidade racialmente
plural e sua projeção cultural, social, política e econômica na África e
no mundo.

Diante de um romance tão capenga quanto este, reminiscente de algum
original guardado em alguma gaveta de Rondônia como ``o maior romance
escrito por um autor do novo Estado'', só nos resta uma saudável
esperança: a de que surjam novos romances moçambicanos. Não poderão ser
piores. A menos que enveredem pela cartilha suicida do ``realismo
socialista'' leninista-stalinista, do qual Deus há de preservar a nós, à
literatura, à inteligência e a Moçambique. Amém.

\chapter{Resenha do livro - Um fuzil na mão um poema no bolso de
Emmanuel
Dongala}\label{resenha-do-livro---um-fuzil-na-muxe3o-um-poema-no-bolso-de-emmanuel-dongala}

Veja, 1974/09/11. Aguardando revisão.

\hfill\break

\textbf{Africano errado\ldots{}}

Lumumba? ``Um relâmpago verbal que iluminou o continente.'' Malcolm X?
``Liberou o negro americano somente com a palavra.'' Uma granada? ``É
uma granada, não um coco.''

Com definições desse tipo, um escritor da África Negra totalmente
desconhecido no Brasil, o congolês Emmanuel Dongala, pretende escrever a
história de um jovem revolucionário, Mayéla dia Mayéla, disposto a
tarefas gigantescas. Hércules de ébano, como diria o autor, Mayéla luta,
ao mesmo tempo, contra: 1) os feiticeiros tribais; 2) contra o
colonialismo, a ``grande besta fascista, parteira de todos os porcos
enlameados do mundo branco''; 3) contra o regime monstruoso da racista
África do Sul; 4) contra as lutas tribais; 5) contra os chineses e os
brancos e 6) contra o negro, afinal ``o inimigo do próprio negro''. É
uma batalha que o escritor e o heroi perdem em todas as frentes. Há
temas demais. E os personagens representam tipos tão variados que
estariam a exigir uma dúzia de volumes e não um simples romance. Surgem:
um feiticeiro que descrê da luta revolucionária; um negro americano que
não compreende o ``mito África escrito com A maiúsculo''; um outro que
``tenta reencontrar a alma no fundo de um copo''; e um quarto que julga
Louis Armstrong um conformista que meramente se lamenta no trumpete das
injustiças do racismo.

É impossível focalizar o regime paranazista de Johanesburgo com frases
do tipo: as vozes de Bessie Smith e Billie Holiday ``eram luminosas como
um sol! Que rebente esse sol e nós, fortalecidos pela verdade bebida por
nossas pupilas abertas, teremos andado e andaremos ainda''. Dongala,
químico e poeta, confundiu o fuzil com a arte. E o tiro, pobre África,
saiu pela culatra.

\textbf{\ldots{} e os esquecidos}

Culturalmente, as editoras brasileiras -- há exceções? -- sempre viveram
alheias à situação geográfica do Brasil. Evitaram os autores
hispano-americanos como quem impede a entrada de um vírus
subdesenvolvido no país. Foi preciso que a ventoinha do Prêmio Nobel,
dado ao guatemalteco Miguel Angel Asturias, lhes mostrasse de onde
soprava o vento da renovação literária dos últimos trinta anos. Agora,
com a África Negra, mais uma vez eles descobrem tardiamente que nas
entrelinhas das manchetes dos jornais existe, além de Angola, Guiné
Bissau e Moçambique, algo mais que o cenário para Tarzan, safáris e o
onisciente general Idi Amin. A Nova Fronteria, de fino faro, que já
pressentira o livro do general Spinola, \emph{Portugal e o Futuro},
antecipou-se a todas, mas publicou apressadamente este romance imberbe,
\emph{Um Fuzil na Mão, um Poema no Bolso}, de Emmanuel Dongala. É uma
escolha bíblica: primeiro se editem os últimos em qualidade literária.
Porque, de outro modo, qual o motivo para não se lançarem algumas das
dezenas de escritores negros importantes surgidos antes e depois da
célebre \emph{négritude} de Léopold Senghor? E que critério justifica
imprimir um romance panfletário em vez de um grande poeta nigeriano como
John Pepper Clark? Formado em literatura na Universidade de Lagos, sua
peça \emph{Song for a Goat} foi talvez a mais marcante apresentada em
seu país.

Influenciado por poetas eruditos, como o americano Ezra Pound, ou
rapsódicos, como o inglês Dylan Thomas, imagista, de cintilantes
metáforas visuais, John Pepper Clark destila com maestria de estilo o
problema da escravidão e do racismo em seu poema célebre, que compara a
tragédia do negro à do gado prestes a ser abatido, sem resistência, no
matadouro: ``Que secreta esperança ou ciência/ Vos insuflam coragem,/
Talvez os tormentos que sofrestes/ foram mis fortes que as tempestades
que fazem transbordar o Níger?/ mas não me concedereis então,/ já que o
facão no final predominará sobre as vossas cabeças/ Ao menos a paciência
que guardais em vossa cauda?''

Haveria muitos outros nomes e títulos que, como uma barragem rompida,
inundariam o mercado literário brasileiro com talentos excepcionais. A
começar com Castro Soromenho, um romancista branco, mas que deu um
testemunho pungente da brutalidade da colonização lusa em terras
africanas com um estilo seco, sem emotividades grandiloquentes.

E por que não revelar em primeira mão o esplêndido e original poeta
congolês Tchicaya U'Tamsi (que significa ``Folha Pequena que Fala em
Nome de seu País''), premiado no Festival Mundial de Artes Negras em
Dascar, em 1966? Apaixonado pela libertação do Congo e seguidor de
Lumumba, teve sua coletânea de poemas \emph{Epítome} prefaciada com
entusiasmo por Senghor. Algumas linhas do seu poema \emph{Christ} valem
mais que todas as páginas de Dongala: ``Cristo, eu rio de tua tristeza/
Ó meu doce Cristo/ Espinho por espinho/ Temos em comum a mesma coroa de
espinhos/ Conto mais de um Judas nos dedos que tu/ Meus olhos mentem à
minha alma/ Em que o mundo é carneiro pascal./ Dize-me em que Egito meu
povo tem os pés acorrentados''.

Opondo às armas reais as do escritor, as palavras que denunciam, o
sul-africano Alan Paton com seu dilacerante \emph{Cry, the Beloved
Country!} constituiriam ainda um roteiro seguro para quem quiser seguir
as trilhas da África fora das rotas da importação do folclórico ou
``exótico''.

A literatura não se faz apenas com ``cor local e pitoresco''. Nem, como
Emmanuel Dongala prova tão claramente, com um fuzil na mão e um poema no
bolso. Principalmente se a pólvora estiver molhada de lacrimejante
sentimentalismo e o poema se perdeu num bolso furado.

\chapter{Um Nobel para a África - Wole
Soyinka}\label{um-nobel-para-a-uxe1frica---wole-soyinka}

Jornal da Tarde, 1986/10/17. Aguardando revisão.

\hfill\break

A roleta caprichosa do Prêmio Nobel de Literatura deste ano parou diante
de um nome que talvez nem 0,0001\% dos brasileiros conheça. Wole
Soyinka, porém, não é apenas um emaranhado de sons, como vários
agraciados pela esclerose tranquila dos acadêmicos de Estocolmo o foram.
Wole Soyinka pertence em primeiro lugar a um dos países africanos mais
ricos do ponto de vista literário, a Nigeria. John Pepper Clark, Okigbo
Cristopher, Gabriel Okara, Amor Tutuola, Chinua Achebe, Onuora Nzeku
para citar apenas os mais conhecidos, representam facetas criativas
diferentes do escritor da África Negra moderna. Talvez o que distingue a
obra -- principalmente teatral -- de Wole Soyinka é a sua profunda
amargura, sua visão trágica e absolutamente infensa a maniqueísmos. Esta
atitude custou-lhe inúmeras polêmicas e ódios. Afirmam seus inimigos:
como pode um dramaturgo e pensador africano -- além disso fino poeta --
investir contra imagens sacralizadas da África anterior ao colonialismo
europeu? Como ele ousa duvidar da existência de uma África primordial
virgem de defeitos, estuprada pela ganância de missionários e
comerciantes, rumo à qual é preciso que autores africanos indiquem o
caminho de volta como aconselha de forma dogmática Ngugi wa Thiongo em
seus ensaios de 1972: \emph{Homecoming: Essays in Africa and Caribbean
Literature, Culture and Politics}.

Soyinka e Chinue Achebe tomam uma atitude de vigorosa oposição a isso
que chamam de visão romântica, idealizada de uma África que, por ser
humana, não é desprovida de defeitos, no passado, agora ou em qualquer
época. ``Como voltar ao ventre originário?'', os autores nigerianos
indagam. E concluem: é uma tarefa impossível. O futuro não pode se
transformar, pela nossa força de vontade, em um passado mítico, talvez
inexistente como concebido dessa forma. A realidade para o escritor
africano, de fato, é múltipla.

De um lado, como vários deles ressaltam, há a dualidade de pertencerem a
uma tradição que a presença do branco destruiu: a tradição comunitária,
a negação do indivíduo como entidade autônoma, a necessidade imperiosa
de que todos se integrem na tribo, na aldeia e seus objetivos. De outro,
há a questão particularmente dolorosa para Soyinka e seu acre
pessimismo: os feiticeiros, os xamãs das tribos ancestrais não eram
farsantes, não eram exploradores da credulidade do grupo? Quando sua
peça \emph{Os Habiantes do Pantanal} foi ou lida ou levada à cena,
imediatamente ele foi tachado de desrespeitoso para com os mais velhos,
de inimigo da religião, pois revelava como o Sacerdoteda Serpente
Oculta, Kadiye, explorava os habitantes da aldeia com falsas promessas,
apoderando-se do dinheiro e das oferendas que lhe tinham sido entregues
em troca de bençãos e proteções divinas. Igwezu, o filho do casal
incapaz de duvidar dos ``antigos'', Makuri e Alu, cobra do representante
religioso tudo que lhe foi negado: tendo partido para uma cidade grande
(Lagos, a capital da Nigéria, neste caso), lá ele perdeu tudo para o
irmão insaciável, até mesmo a esposa. Igwezu está dilacerado, não só por
sua derrota pessoal, mas também por estar dividido entre o campo e a
cidade. Qual é mais maléfico? O campo com suas crendices e imutabilidade
social ou a cidade grande, aquele mundo de negócios, de crescentes
favelas, de corrupção, de lutas e arengas políticas que nunca favorecem
o povo?

Soyinka não desmente, é óbvio, a importância do passado. O passado
existe, colore o presente e até mesmo o futuro com a sua influência. Mas
o passado não é uma escapatória fácil para os acomodados: por mais que o
passado pré-colonial tenha sido importante ou até hipoteticamente
edênico, não podemos voltar a ele: esse paraíso está fechado. Precisamos
construir o presente e o futuro, pois não há volta atrás.

Outro aspecto da criação artística desse rebelde nigeriano irrita
camadas consideráveis dos que o leem. \emph{The Strong Breed}, \emph{The
Lion and the Jewel} e \emph{The Road} são peças que de modo subjacente
ou explícito confirmam suas ruminações desiludidas durante o tempo que
passou na prisão por motivos políticos: \emph{The Man Died (Prison Notes
of Wole Soyinka}, 1972). A África libertada do jugo colonial encheu-se
de ditadores, de partidos únicos de repressão e fraude: desde o
Abi-Ackell africano das joias caríssimas, o ``Imperador'' Bokassa, até o
massacre perpetrado por Idi Amin Dada em Uganda. Em Gana, no Quênia o
terror dos negros imposto aos negros substituiu o terror dos brancos.

Ironicamente, ele parece subentender a pergunta ácida: deveríamos voltar
ao passado colonial, quando os brancos não tinham estabelecido uma
ditadura política? Esta sua denúncia de regimes tirânicos na África
Negra recorda bastante o desalento de um escritor angolano, Manuel Rui,
que vê seu país libertado de Portugal, é verdade, mas amordaçado pelas
armas soviéticas e por suas sentinelas cubanas. Angola em ruínas e
desmoronando sob o impacto de uma guerra civil movida pela Unita de
Jonas Savimbi. Valeu a pena trocar de senhores? A África poderá
afirmar-se de forma equidistante das zonas de influência norte-americana
ou russa?

Soyinka -- e não só ele -- aborda outro tema de quase impossível
solução: existe \emph{uma} África ou o imenso continente retalhado a
bel-prazer pelas potências europeias não forma um conjunto homogêneo? Os
ideais do panafricanismo serão realizáveis, como o deseja Julius
Nyerere, desejoso de abolir até o nome do país que governa, a Tanzânia,
em prol de uma visão mítica da África?

Ele não responde propriamente a esta pergunta com receitas dogmáticas
nem aceita o que considera meros \emph{slogans} como -- sacrilégio! -- à
\emph{négritude}. A \emph{négritude} nada significa mesmo, ele rebate,
destruindo o que considera falsos ídolos, uma tendência muito difundida
entre os escritores africanos de língua inglesa, em contraste com os de
expressão francesa como Senghor ou Césaire. Ele prefere um termo mais
genérico como ``personalidade africana'', reencontrando uma unidade
dificilmente delimitável na imensa variedade de expressões africanas.

Se não bastassem essas suas facetas iconoclastas, Wole Soyinka pouco se
importa em aderir a um partido político específico, embora reconheça uma
tarefa social a cargo do escritor, mas duvida que o escritor tenha uma
fórmula miraculosa para os problemas socioeconômicos de qualquer regime
político. Soyinka parece insistir nos valores individuais e não quer
perder tempo com as teorias de racistas segundo os quais ``a África
nunca contribuiu em nada para a civilização mundial'', ou ``o africano é
supersticioso, sujo, atrasado mentalmente, não suscetível de
progredir'': para que refutá-las se há pouco tempo para se escrever
sobre a África e buscar, penosamente, um papel que os africanos possam
desempenhar no século XXI? As antigas civilizações do Benin, de Daomé e
outras regiões africanas são um dado inegável: para que nos determos
apenas no passado se o presente é tão urgente e exige toda a nossa
meditação, toda a nossa participação? Parar nos reinos de outrora é tão
inútil quanto lamentar-se séculos a fio sobre os horrores da escravidão
e do imperialismo. Ora, evidentemente, Soyinka é, até mesmo na Nigéria,
um marginalizado, um \emph{outsider} ocupado, na Universidade de Ibadan,
em criar uma dramaturgia nigeriana e em dirigir um grupo de atores
itinerantes que percorre as principais cidades do país com esse
propósito.

Às vezes, o Prêmio Nobel gosta de colocar suas coroas suecas e seus
galardões os bolsos e nas cabeças de dissidentes: Pasternak, proibido na
URSS, Thomas Mann, exilado da Alemanha pelo nazismo, André Gide,
execrado por suas preferências sexuais pela classe bem-pensante
francesa. Ou às vezes o grupo geriátrico de Estocolmo prefere pescar
autores absolutamente desconhecidos a não ser de exóticos perseguidores
do obscuro: o australiano Patrick White, quando não teima em premiar a
mediocridade registrada em cartório: Gabriele Mistral, Roger Martin du
Gard\ldots{} Ao selecionar, este ano, o dramaturgo nigeriano, de certa
maneira os acadêmicos suecos confirmaram a seleção para o prêmio da Paz
atribuído a Elie Wiesel pelo menos num ponto alto: ambos são
``pessimistas sorridentes'', sacudidos pelos dramas de seus povos e pela
diáspora imposta aos judeus e aos negros. Na universidade sueca de
Uppsala, o Instituto Sueco de Estudos Africanos publicara, já no
longínquo ano de 1968, uma das raras afirmações de Wole Soyinka:

``O artista tem sempre assumido, na sociedade africana, o papel de
testemunha dos hábitos e da experiência da sua sociedade e
\emph{simultaneamente} (sublinhado pelo próprio autor) o de voz lucida
de usa própria época. É chegado o tempo de ele corresponder a essa
essência de si mesmo''

Como ressalta com extrema clareza o africanista James Olney em
\emph{Tell me Africa, An Approach to African Literature} (Editora
Princeton, 1973), não cabe ao leitor que não for originário da África
Negra e tiver crescido em meio a seus hábitos compará-la com a França, a
Inglaterra ou qualquer outro país. Por que exigir de um artista africano
a adoção de gêneros como o romance, o soneto, a rima? Por acaso, seria
legítimo esperar que um poeta chinês criasse o equivalente à
\emph{Divina Comédia} ou que um novelista como Kawabata reproduzisse em
japonês a saga do \emph{Ulisses} de Joyce? Esta parece ser a dificuldade
fundamental de todos nós, cada vez que deparamos com uma cultura que não
seja eurocêntrica. Que não tenha forçosamente passado pelos movimentos
culturais, políticos, econômicos do Renascimento, da Reforma, da
Contrarreforma, da Revolução Industrial. Se já há evidentes dificuldades
em apreendermos a realidade autóctone quíchua do Peru e da Bolívia, por
exemplo, que obstáculos não surgirão para que captemos a mentalidade
africana ou do Extremo Oriente, inevitavelmente diferentes da nossa?

É necessário enfatizar, porém, que para Soyinka, como para a maioria dos
autores africanos, cultivar a própria individualidade não é um fim em si
mesmo: é um princípio visando a integrá-lo melhor na comunidade. Uma
surpresa, neste artista profundo, lúcido na sua análise da trágica
condição humana, é o toque de ironia e de humor, que por vezes salta de
seus textos, tanto de poesia quanto de teatro. Seu poema
\emph{``Conversa Telefônica''} ilustra bem essa sátira amarga. Quem
queria alugar um quarto em Londres, presume-se, acha o preço aceitável,
a localização indiferente, só faltava a ``confissão''. E ele a faz à sua
futura, hipotética senhoria: ``Sou africano''. A voz feminina mantém-se
educada e provém, de uma boca que fuma através de uma piteira, os lábios
cobertos de batom. Segue-se um diálogo não se sabe se mais trágico ou
cômico em que ela procura saber se ele, seu futuro, hipotético
inquilino, é ``muito escuro ou mais para preto claro''. Cor de chocolate
amargo ou de leite? À resposta ``Sou da cor sépia da África Oriental''
não satisfaz: será \emph{muito} escuro? Até o final francamente
hilariante: ``Madame, a Senhora devia ver não só a morenice de meus pés
e as palmas de minhas mãos de um louro água oxigenada! O atrito
provocado por eu tanto me sentar, Madame, é que deixou meu traseiro
negro como um corvo''. E quando ela desliga, enfurecida, a pergunta
final, o pedido derradeiro: ``Será que a senhora não quer dar uma
espiada com seus próprios olhos, Madame?''.

Um momento de rara distensão de Wole Soyinka, pouco inclinado a
brincadeiras, mas sem a solenidade fúnebre de um profissional qualquer
do pessimismo. Porque a sua posição, felizmente, não é daquelas
rotuláveis por uma síntese crítica comoda e apressada. Soyinka, com suas
contradições e sua paradoxal descrença esperançosa no ser humano em suas
interpelações dramáticas, tateia, como todo legítimo artista, os
próprios instrumentos de sua busca.

Se Virgínia Woolf, assustadoramente, perguntava: ``Para que viver?'' o
autor nigeriano perscruta o porquê da arte, se é que ela tem uma função
específica. Numa época em que as liberdades de imprensa, de expressão,
de dissenção estendem um arco macabro do Chile a Vladivostock, na União
Soviética, na era dos Gulags, das 41 guerras contemporâneas, das
ditaduras da África, na Ásia, na Europa do Leste, na América Latina,
deve ter ressoado fortemente na consciência da Academia Sueca de
Literatura o testemunho de um intelectual que escreve da prisão um dos
relatos mais comovedores em prol de uma coerência ética que não se dobra
diante de regimes despóticos.

Da prisão Wole Soyinka estabeleceu um princípio que antes dele todos os
prisioneiros de consciência do mundo reafirmaram com a própria vida: não
pode haver uma sociedade livre se a liberdade for extinta por um decreto
governamental apoiado no poder de fogo das armas, de Manágua a Varsóvia.
Deve ter sido este inabalável princípio ético de todos os tempos que
levou o Prêmio Nobel a distinguir, este ano, afortunadamente, o nome
íntegro e fecundo de Wole Soyinka.

\chapter{Nota sobre o livro A Arma da Casa de Nadine
Gordimer}\label{nota-sobre-o-livro-a-arma-da-casa-de-nadine-gordimer}

Caros Amigos, n.39, 2000/06. Aguardando revisão.

\hfill\break

Há muitos anos não leio um livro tão perfeito: \emph{A Arma da Casa}, da
magistral autora sul-africana Nadine Gordimer (Coleção Prêmio Nobel,
Companhia das Letras). Depois de terminar a leitura de \emph{Memórias de
uma Sobrevivente}, da também sul-africana Doris Lessing, pelos temas que
aborda, julguei apressadamente que nada mais se poderia escrever de tão
profundo sobre o esfacelamento de uma sociedade. Engano. Nadine Gordimer
focaliza, destemidamente, a África do Sul de hoje, com uma igualdade de
raças pelo menos no papel e um multiculturalismo efervescente, afastado
o monstruoso \emph{apartheid} que separava brancos e negros numa forma
de campo de concentração ara todos os que não fossem brancos.

E atualmente? A mestria da autora permite-lhe visualizar todos os
personagens sob vários pontos de vista: o crime cometido --
inesperadamente -- numa família branca, de tendência liberal, detona o
romance inteiro. Os pais do jovem assassino são uma caricatura do ``bom
comportamento'' dos brancos depois da nova bandeira, da mudança radical
havida no país? O advogado negro, imbuído de idealismo, será aceito,
mesmo sendo \emph{gay}?

Nadine Gordimer, sem exagero, desenha-se como uma Dostoiévski de nossos
dias, buscando as raízes do crime, do preconceito, do sofrimento, não na
aparência da mudança política e cultural de uma sociedade, mas como
efeito de uma origem remota, entranhada na própria alma humana desde
tempos da pedra lascada. Se não fosse uma aposta precipitada, eu diria,
eu afirmaria que este é o livro do ano, que comprássemos apenas este
para sentir as palpitações do mundo da Bósnia, dos terremotos no Japão,
das guerrilhas em Serra Leoa, do Líbano e também do Brasil, que ainda
busca seu rumo, distante da política, da corrupção, do racismo, do
desleixo governamental, da parte podre -- mas não toda -- do Brasil que
caminha estonteado em meio à ruína de seus problemas e do caos mundial.
Um livro eletrizante que demonstra, cabalmente, que a literatura não só
está viva, como se põe à frente de todas as tecnologias que a desafiam
como consciência mais funda do ser humano trilhando um novo milênio.

\chapter{Nota sobre o livro Desonra de J. M.
Coetzee}\label{nota-sobre-o-livro-desonra-de-j.-m.-coetzee}

Caros Amigos, n.46, 2001/01. Aguardando revisão.

\hfill\break

J. M. Coetzee é um narrador sul-africano como Nadine Gordimer e Doris
Lessing radicada na Rodésia e depois em Londres. Seus romances são a
visão masculina de um país que tem muitas semelhanças com o Brasil:
imensos progressos ao lado de oposições radicais entre a barbárie e as
áreas de civilização branca, tecnicamente avançada, riquíssima e que
também vive um \emph{apartheid} social, longe das favelas e da miséria
circundantes. \emph{Desonra}, de J. Coetzee, faz parte da raça holandesa
que criou na África do Sul castas distintas para os brancos, os negros e
mulatos e para os indianos. Ele também vê, como as duas escritoras, a
desintegração do país, quando a parte branca holandesa ou inglesa) tira
passaportes com a intenção de emigrar para a Nova Zelândia, o Canadá ou
a Austrália. A violência inter-racial, a tentativa de criar uma África
do Sul sem racismo e que mantenha o progresso servem como pano de fundo
aos três escritores. Coetzee não é tão dramático como Gardimer nem sem
esperança como Lessing. São três autores semelhantes, mas diferentes em
seu \emph{approach} dos problemas das muitas raças e tradições na parte
meridional da África: choques contra os fazendeiros ingleses brancos na
Rodésia, atual Zimbábue, multiculturalismo como no Brasil, ou será
melhor abandonar tudo rumo a um país de língua inglesa, sem negros e já
organizado?



\end{document}
